\chapter{Introduction}
The study of homotopy theoretic phenomena in the language of type theory \cite{hottbook} is 
sometimes loosely called `synthetic homotopy theory' \cite{Brunerie16}. 
Homotopy theory in type theory \cite{Awodey12} is only one of the
many aspects of homotopy type theory, which also includes the study of the
set theoretic semantics (models of homotopy type theory and univalence in a
meta-theory of sets or categories \cite{Awodey14,AwodeyWarren,BezemCoquandHuber,KapulkinLeFanuLumsdaine,Shulman15,Voevodsky15}), type theoretic semantics (internal models of homotopy type
theory), and computational semantics \cite{AngiuliHarperWilson}, as well as the study of various questions
in the internal language of homotopy type theory which are not necessarily 
motivated by homotopy theory, or questions related to the development of
formalized libraries of mathematics based on homotopy type theory.
This thesis concerns the development of synthetic homotopy theory.

Homotopy type theory is based on Martin-L\"of's theory of dependent types \cite{MartinLof1984}, which
was developed during the 1970's and 1980's.
The novel additions of homotopy type theory are Voevodsky's univalence axiom \cite{Voevodsky06,Voevodsky10},
and higher inductive types \cite{Lumsdaine11Blog,Shulman11Blog,hottbook}. The univalence axiom characterizes the identity
type on the universe, and establishes the universe as an object classifier \cite{RijkeSpitters}.
Higher inductive types are a generalization of inductive types, in which both
point constructors (generators) and path constructors (relations) may be specified.
A simple class of higher inductive types, which includes most known higher inductive
types, are the homotopy pushouts. When the universe is assumed to be closed
under homotopy pushouts, it is also closed under homotopy coequalizers \cite{hottbook}, 
sequential colimits \cite{hottbook}, and propositional truncation \cite{vanDoorn2016}.
For instance, one way of obtaining the $n$-spheres \cite{Lumsdaine12Blog} using homotopy pushouts is by setting $\sphere{-1}\defeq\emptyt$ and
by inductively defining the $(n+1)$-sphere $\sphere{n+1}$ to be the pushout
of the span $\unit \leftarrow\sphere{n}\rightarrow\unit$. 
Then we can attach $(n+1)$-cells to a type $X$.
Let $f:(\sm{a:A}B(a))\to X$ be a family of attaching maps
for some $\sphere{n}$-bundle $B:A\to\mathrm{BAut}(\sphere{n})$. We
attach the $(n+1)$-disks indexed by $A$ to $X$ by taking the homotopy pushout
\begin{equation*}
\begin{tikzcd}
\sm{a:A}B(a) \arrow[r,"f"] \arrow[d,swap,"\pi_1"]  & X \arrow[d,"\inr"] \\
A \arrow[r,swap,"\inl"] & P. \arrow[ul,phantom,very near start,"\ulcorner"]
\end{tikzcd}
\end{equation*}

Unless otherwise specified, we shall assume a univalent universe that is closed
under the usual type constructors, including a natural numbers object, and
homotopy pushouts. The model in cubical sets by Coquand et al. \cite{BezemCoquandHuber} is a
constructive model for this setup, although the fact that it is closed under
homotopy pushouts is currently unpublished.

We will mostly follow the notation of \cite{hottbook}.
Also, it is worth pointing out that when we give two equivalent descriptions
of the same thing, we mean that their types are homotopy equivalent
(rather than just logically equivalent), and usually
we have a canonical equivalence in mind. A basic
example: dependent types over $A$ may be described equivalently as maps into
$A$. By this assertion we mean that the types $A\to\UU$ and $\sm{B:\UU}B\to A$ are 
homotopy equivalent, where the equivalence sends a type family $P:A\to\UU$ to the projection map
$\pi_1:(\sm{a:A}P(a))\to A$.

\paragraph{Acknowledgements\footnote{I gratefully acknowledge the support of the Air Force Office of Scientific Research through MURI grant FA9550-15-1-0053.}}
First and foremost, it is my pleasure to thank my advisor Steve Awodey. I could not have done this PhD without his advice, support, inspiration, and friendship.

Then I would like to thank the members of my committee: Jeremy Avigad for the wisdom he shared with me and his unparallelled kindness, Ulrik Buchholz for his gift of making many advanced topics accessible and for his friendship, and Michael Shulman for his inspiration, being able to answer virtually all of my (and anyones) questions, and for occasionally catching my mistakes.

I would like to thank my other collaborators, the people I had projects with:
Simon Boulier, Ulrik Buchholtz, Dan Christensen, Floris van Doorn, Jonas Frey, Morgan Opie, Luis Scoccola, Kristina Sojakova, Bas Spitters, Nicolas Tabareau, Felix Wellen, and Alexandra Yarosh. I thank you for all your inspiration, insights, energy, and creativity.

I would also like to thank all the people who have generously invited me for a visit. I am deeply indebted to Joachim Kock, who has hosted me for half a year in 2013-2014 at the \textit{Departament de Matemàtiques} of the Universitat Autònoma de Barcelona and helped me finding my PhD position at CMU; I am grateful to Vladimir Voevodsky for inviting me to the Institute for Advanced Study in March 2015; Nicolas Tabareau and his students Kevin Quirin and Simon Boulier for hosting me at INRIA Nantes during the summers of 2015 and 2016; Andrej Bauer for hosting me at the \textit{Fakulteta za Matematiko in Fiziko} of the University of Ljubljana in January 2016; Lars Birkedal and Bas Spitters for hosting me at the Department of Computer Science of Aarhus University in February 2016; Marie-Françoise Roy for inviting me to speak in the Effective Geometry and Algebra seminar at the \textit{Institut de Recherche Mathématiques de Rennes} in June 2016; Dan Christensen and his student Luis Scoccola for hosting me at the University of Western Ontario in London, Ontario in October 2017; Guillaume Brunerie for hosting me at the Institute for Advanced Study in November 2017; Charles Rezk and his student Nima Rasekh for inviting me to speak in the Topology Seminar at the University of Illinois at Urbana-Champaign in December 2017; Tom Hales for inviting me to speak at the Algebra, Combinatorics, and Geometry seminar at the University of Pittsburgh in April 2018; and Pieter Hofstra for inviting me to speak for the Canadian Mathematical Society in Fredericton, New Brunswick, in June 2018. Tak, dankjewel, thank you, merci, danke sch\"on, spasibo, \v dankujem, hvala, gracias.

Finally, I thank my wonderful parents for their unconditional support, and my brother and two sisters, whom I didn't get to see as much as I would have liked because I chose to study overseas. I miss you and I think of you.
