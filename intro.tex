\chapter{Introduction}
The study of homotopy theoretic phenomena in the language of type theory \cite{hottbook} is 
sometimes loosely called `synthetic homotopy theory' \cite{Brunerie16}. 
Homotopy theory in type theory \cite{Awodey12} is only one of the
many aspects of homotopy type theory, which also includes the study of the
set theoretic semantics (models of homotopy type theory and univalence in a
meta-theory of sets or categories \cite{Awodey14,AwodeyWarren,BezemCoquandHuber,KapulkinLeFanuLumsdaine,Shulman15,Voevodsky15}), type theoretic semantics (internal models of homotopy type
theory), and computational semantics \cite{AngiuliHarperWilson}, as well as the study of various questions
in the internal language of homotopy type theory which are not necessarily 
motivated by homotopy theory, or questions related to the development of
formalized libraries of mathematics based on homotopy type theory.
This thesis concerns the development of synthetic homotopy theory.

Homotopy type theory is based on Martin-L\"of's theory of dependent types \cite{MartinLof84}, which
was developed during the 1970's and 1980's.
The novel additions of homotopy type theory are Voevodsky's univalence axiom \cite{Voevodsky06,Voevodsky10},
and higher inductive types \cite{Lumsdaine11Blog,Shulman11Blog,hottbook}. The univalence axiom characterizes the identity
type on the universe, and thereby establishes the universe as an object classifier \cite{RijkeSpitters}.
Higher inductive types are a generalization of inductive types, in which both
point constructors (generators) and path constructors (relations) may be specified.
A simple class of higher inductive types, which includes most known higher inductive
types, are the homotopy coequalizers. When the universe is assumed to be closed
under homotopy coequalizers, it is also closed under pushouts \cite{hottbook}, 
sequential colimits \cite{hottbook}, and propositional truncation truncation \cite{VanDoorn15}.
For instance, we get the $n$-spheres \cite{Lumsdaine12Blog} by setting $\sphere{-1}\defeq\emptyt$ and
by inductively defining the $(n+1)$-sphere $\sphere{n+1}$ to be the pushout
of the span $\unit \leftarrow\sphere{n}\rightarrow\unit$. 
Then we can attach $(n+1)$-cells to a type $X$.
Let $f:(\sm{a:A}B(a))\to X$ be a family of attaching maps
for some $\sphere{n}$-bundle $B:A\to\sm{S:\UU}\brck{\sphere{n}=S}$. We
attach the $(n+1)$-disks indexed by $A$ to $X$ by taking the homotopy pushout
\begin{equation*}
\begin{tikzcd}
\sm{a:A}B(a) \arrow[r,"f"] \arrow[d,swap,"\pi_1"]  & X \arrow[d,"\inr"] \\
A \arrow[r,swap,"\inl"] & P. \arrow[ul,phantom,very near start,"\ulcorner"]
\end{tikzcd}
\end{equation*}

Unless otherwise specified, we shall assume a univalent universe that is closed
under the usual type constructors, including a natural numbers object, and
homotopy coequalizers. The model in cubical sets by Coquand et al. \cite{BezemCoquandHuber} is a
constructive model for this setup, although the fact that it is closed under
homotopy coequalizers is currently unpublished.

We will follow the notation of \cite{hottbook}.
Also, it is worth pointing out that when we give two equivalent descriptions
of the same thing, we mean that their types are homotopy equivalent
(rather than just logically equivalent), and usually
we have a canonical equivalence in mind. A basic
example: dependent types over $A$ may be described equivalently as maps into
$A$. By this we mean that the types $A\to\UU$ and $\sm{B:\UU}B\to A$ are 
homotopy equivalent, where the equivalence sends $P:A\to\UU$ to the map
$\pi_1:(\sm{a:A}\P(a))\to A$.

\paragraph{Acknowledgements}
I gratefully acknowledge the support of the Air Force Office of Scientific Research through MURI grant FA9550-15-1-0053. Any opinions, findings and conclusions or recommendations expressed in this material are those of the authors and do not necessarily reflect the views of the AFOSR.
