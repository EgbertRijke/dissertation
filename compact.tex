\chapter{Compact types}

\section{Compact types}

\begin{defn}
Call a type $X$ $\omega$-compact, if for any
\begin{equation*}
\begin{tikzcd}
A_0 \arrow[r] & A_1 \arrow[r] & A_2 \arrow[r] & \cdots
\end{tikzcd}
\end{equation*}
the map
\begin{equation*}
\xi_A\defeq\mathsf{rec}(\lam{f} \iota_n\circ f,\lam{f} \kappa_n\circ f)
  : \tfcolim(X\to A_n)\to(X\to A_\infty)
\end{equation*}
is an equivalence.
\end{defn}

\begin{rmk}
In the literature, an object $X$ of a category $\cat{C}$ is said to be $\omega$-compact if for any
functor $F:P\to\cat{C}$ from a directed countable partially ordered set $P$, there
is a natural isomorphism
\begin{equation*}
\mathrm{hom}(X,\mathrm{colim}(F(p))) \cong \mathrm{colim}(\mathrm{hom}(X,p))
\end{equation*}
The notion of $\omega$-compact type $X$ is equivalent to the classical notion,
if one takes a directed poset $P$ to be a Rezk-complete category $P$ in which
each hom-set is a mere proposition, for which there merely is a term of type
\begin{equation*}
\prd{p,q:P}\sm{r:P} (p\leq r)\land (q\leq r).
\end{equation*}
and if one takes a countable set to be a set $X$ for which there merely exists
a function $f:\nat\to X$ such that $\prd{x:X}\hfib{f}{x}$.

In other settings, it might be more useful to replace $\Pi$ and $\Sigma$ in the
notion of directed poset, with the truncated versions $\forall$ and $\exists$. 
\end{rmk}

In this section we will prove that the $\omega$-compact types are closed under
various type theoretical constructions. Most of these proofs decompose the
function $\xi_A$ as a chain of canonical equivalences. In a rigorous proofs,
such chains should be accompanied by the verification that such chains of
canonical equivalences can indeed be identified with the functions $\xi_A$, but
we leave this to the computer verification. 

\begin{lem}
The empty type and the unit type are $\omega$-compact and the $\omega$-compact types are closed under
disjoint sums. Thus, all sub-finite types are $\omega$-compact.
\end{lem}

\begin{proof}
The fact that the empty type and the unit type are $\omega$-compact is trivial.

Thus it remains to show that $\omega$-compact types are closed under disjoint
sums. Suppose that $X$ and $Y$ are $\omega$-compact. Then we have
\begin{align*}
(X+Y)\to A_\infty & \eqvsym (X\to A_\infty)\times(Y\to A_\infty) \\
& \eqvsym \tfcolim((X\to A_n)\to (Y\to A_n)) \tag{By \autoref{thm:colim_sm}} \\
& \eqvsym \tfcolim((X+Y)\to A_n).
\end{align*}
\end{proof}

\begin{lem}
Let $X$ be an $\omega$-compact type and consider
\begin{equation*}
\begin{tikzcd}
A_{0} \arrow[r] \arrow[d,->>] & A_{1} \arrow[r] \arrow[d,->>] & A_{2} \arrow[r] \arrow[d,->>] & \cdots \\
X \arrow[r,equals] & X \arrow[r,equals] & X \arrow[r,equals] & \cdots
\end{tikzcd}
\end{equation*}
Then there is an equivalence
\begin{equation*}
\eqv{(\prd{x:X} A_\infty(x))}{\tfcolim(\prd{x:X}A_n(x))}.
\end{equation*}
\end{lem}

\begin{proof}
The proof is a simple calculation:
\begin{align*}
(\prd{x:X} A_\infty(x)) & \eqvsym \sm{f:X\to\sm{x:X}A_\infty(x)}\id{\fst\circ f}{\catid{X}} \\
& \eqvsym \sm{f:X\to\tfcolim(\sm{x:X} A_n(x))}\id{\mathsf{rec}(\fst,\blank)\circ f}{\catid{X}} \\
& \eqvsym \tfcolim(\sm{f:X\to\sm{x:X}A_n(x)} \id{\fst\circ f}{\catid{X}}) \\
& \eqvsym \tfcolim(\prd{x:X}A_n(x)).\qedhere
\end{align*}
\end{proof}

\begin{defn}
A graph is said to be $\omega$-small if its type of vertices and each type of
edges is $\omega$-compact. 
\end{defn}

\begin{thm}
The colimit of an $\omega$-small graph is $\omega$-compact. 
\end{thm}

\begin{proof}
Let $\Gamma$ be an $\omega$-small graph and consider a sequence
\begin{equation*}
\begin{tikzcd}
A_0 \arrow[r] & A_1 \arrow[r] & A_2 \arrow[r] & \cdots
\end{tikzcd}
\end{equation*}
Then we have
\begin{align*}
\tfcolim(\Gamma)\to A_\infty
& \eqvsym \sm{f:\pts{\Gamma}\to A_\infty}\prd{i,j:\pts{\Gamma}}\edg{\Gamma}(i,j)\to(\id{f(i)}{f(j)}) \\
& \eqvsym \tfcolim(\sm{f:\pts{\Gamma}\to A_n}\prd{i,j:\pts{\Gamma}}\edg{\Gamma}(i,j)\to(\id{\iota_n(f(i))}{\iota_n(f(j))})) \\
& \eqvsym \tfcolim(\sm{f:\pts{\Gamma}\to A_n}\prd{i,j:\pts{\Gamma}}\edg{\Gamma}(i,j)\to\tfcolim(\id{\iota_{n+k}(a^k_n(f(i)))}{\iota_{n+k}(a^k_n(f(i)))}))\\
& \eqvsym \tfcolim(\sm{f:\pts{\Gamma}\to A_n}\prd{i,j:\pts{\Gamma}} \tfcolim(\edg{\Gamma}(i,j)\to\id{\iota_{n+k}(a^k_n(f(i)))}{\iota_{n+k}(a^k_n(f(i)))}))\\
& \eqvsym \tfcolim(\sm{f:\pts{\Gamma}\to A_n}\tfcolim(\prd{i,j:\pts{\Gamma}} \edg{\Gamma}(i,j)\to\id{\iota_{n+k}(a^k_n(f(i)))}{\iota_{n+k}(a^k_n(f(i)))}))\\
& \eqvsym \tfcolim(\sm{f:\pts{\Gamma}\to A_n}\prd{i,j:\pts{\Gamma}} \edg{\Gamma}(i,j)\to\id{\iota_{n}(a_n(f(i)))}{\iota_{n}(a_n(f(i)))})) \\
& \eqvsym \tfcolim(\tfcolim(\Gamma)\to A_n).\qedhere
\end{align*}
\end{proof}

\begin{cor}
The $\omega$-compact types are closed under dependent pair types, subtypes,
cartesian products, suspension and pushouts.
\end{cor}

\begin{rmk}
The $\omega$-compact types are not closed under identity types, because $\Z$ is not $\omega$-compact
and $\eqv{\Z}{\Omega(\Sn^1)}$. To see that $\Z$ is not $\omega$-compact, consider
the sequence $A_n\defeq\nat_{-n}$, where $\nat_{-n}$ is a version of the natural
numbers of starting at $-n$ and $A_n\to A_{n+1}$ is the inclusion map. The colimit
of $A_n$ is $\Z$ and the image of the map $\tfcolim(\Z\to A_n)\to (\Z\to\Z)$ is
the set of maps that are bounded from below. Thus, it is not surjective.

Also, the $\omega$-compact types are not closed under exponentiation. We have
the equivalence $\eqv{(\Sn^1\to\Sn^1)}{\Sn^1\times\Z}$, which is not
$\omega$-compact. It is not know to me whether or not the type $X=Y$ is $\omega$-compact
for any two $\omega$-compact types $X$ and $Y$.
\end{rmk}

\begin{thm}
The $\omega$-compact types are closed under retracts.
\end{thm}

\begin{proof}
Let $Y$ be a retract of an $\omega$-compact type $X$, and consider
the infinite sequence
\begin{equation*}
\begin{tikzcd}
A_0 \arrow[r] & A_1 \arrow[r] & A_2 \arrow[r] & \cdots
\end{tikzcd}
\end{equation*}
Then we hvar $i:Y\to X$ and $r:X\to Y$ such that $r\circ i=\catid{Y}$. 
So we also have $\lam{f}f\circ r:(Y\to A_\infty)\to(X\to A_\infty)$ and
$\lam{g}g\circ i:(X\to A_\infty)\to (Y\to A_\infty)$, and since
$f\circ r\circ i= f$ for any $f:Y\to A_\infty$, we see that $Y\to A_\infty$ is
a retract of $X\to A_\infty$. Similarly, $\tfcolim_n(Y\to A_n)$ is a retract of
$\tfcolim_n(X\to A_n)$.
Now we have the following diagram
\begin{equation*}
\begin{tikzcd}
\tfcolim_n(Y\to A_n) \arrow[r] \arrow[d] & \tfcolim_n(X\to A_n) \arrow[r] & \tfcolim_n(Y\to A_n) \arrow[d]\\
(Y\to A_\infty) \arrow[r] & (X\to A_\infty) \arrow[u,equals] \arrow[r] & (Y\to A_\infty)
\end{tikzcd}
\end{equation*}
and the two squares commute. Thus, we see that the canonical map 
$\tfcolim_n(Y\to A_n)\to (Y\to A_\infty)$ is a retract of an equivalence, hence
it is an equivalence. 
\end{proof}

\begin{conj}
Consider a function $f:X\to Y$ in which the domain $X$ is $\omega$-compact. If
each $\id{f(x)}{f(x')}$ is $\omega$-compact, then $\im{f}$ is $\omega$-compact.
\end{conj}

\begin{conj}
Is it the case that $\trunc{n}{X}$ is $\omega$-compact for any $\omega$-compact
type $X$? Does any modality preserve $\omega$-compactness?
\end{conj}

\begin{conj}
Let $\sequence{A}{a}$ be a type sequence. If each $A_n$ is $\omega$-compact, and
each $a_n$ is surjective, is it the case that $A_\infty$ is $\omega$-compact? And
if just $A_0$ is $\omega$-compact and each $a_n$ is surjective?
\end{conj}

\begin{comment}
\begin{conj}
Suppose $X$ and $Y$ are $\omega$-compact. Then $X=Y$ is $\omega$-compact.
\end{conj}

\begin{proof}[Non-proof]
Consider the infinite sequence
\begin{equation*}
\begin{tikzcd}
A_0 \arrow[r] & A_1 \arrow[r] & A_2 \arrow[r] & \cdots
\end{tikzcd}
\end{equation*}
Since $\isequiv(f)$ is a mere proposition (and hence $\omega$-compact) for
any $f:X\to Y$, we have
\begin{align*}
(X=Y)\to A_\infty 
& \eqvsym
(\eqv{X}{Y})\to A_\infty \\
& \eqvsym 
(\sm{f:X\to Y}\isequiv(f))\to A_\infty \\
& \eqvsym
\prd{f:X\to Y} \isequiv(f)\to A_\infty \\
& \eqvsym
\prd{f:X\to Y} \tfcolim_n(\isequiv(f)\to A_n)
\end{align*}
This reduction is not going to help, because $X\to Y$ does not need to be
$\omega$-compact.
\end{proof}

\begin{conj}
If $X$ is $\omega$-compact, then so is $\mathrm{Aut}(X)$.
\end{conj}

\begin{proof}

\end{proof}
\end{comment}

\section{Construction of localizations}
In this section, we assume to have $P,Q:A\to\type$ and $F:\prd{a:A} P(a)\to Q(a)$.
We will mostly suppress reference to $F$.

\begin{defn}
A type $X$ is said to
be $F$-local, if for all $a:A$, the map $\psi_{X}(a):\lam{g} g\circ F(a):(Q(a)\to X)\to (P(a)\to X)$
is a an equivalence. 
\end{defn}

The idea is to approximate the $F$-localization of $X$ by iterated `quasi $F$-local extensions' of $X$. Those will be maps $l:X\to Y$ with certain extra structure, so when we have iterated extensions, we will be working with type sequences. Thus, we need the following technical lemma.

\begin{lem}\label{lem}
Suppose that $F$ is a family of maps between $\omega$-compact types. Consider a type sequence
\begin{equation*}
\begin{tikzcd}
X_0 \arrow[r,"\alpha_0"] & X_1 \arrow[r,"\alpha_1"] & X_2 \arrow[r,"\alpha_2"] & \cdots
\end{tikzcd}
\end{equation*}
and let $f:P(a)\to X_0$. Then we have an equivalence
\begin{equation*}
\eqv{\hfib{\psi_{X_\infty}}{\iota_0\circ f}}{\tfcolim_n(\hfib{\psi_{X_n}}{\alpha^n\circ f})}.
\end{equation*}
\end{lem}

\begin{proof}
This follows from a simple calculation using the facts about $\omega$-compact
types of the previous section:
\begin{align*}
\hfib{\psi_{X_\infty}}{\iota_0\circ f}
  & \jdeq
\sm{g:Q(a)\to X_\infty} \id{g\circ F(a)}{\iota_0\circ f} \\
& \eqvsym \sm{g:\tfcolim_n(Q(a)\to X_n} \id{\xi_{Q(a)}(g)\circ F(a)}{\iota_0\circ f}) \\
& \eqvsym \tfcolim_n(\sm{g:Q(a)\to X_n} \id{\iota_n\circ g\circ F(a)}{\iota_0\circ f}) \\
& \eqvsym \tfcolim_n(\sm{g:Q(a)\to X_n} \prd{u:P(a)}\id{\iota_n(g(F(a,u)))}{\iota_0(f(u))} \\
& \eqvsym \tfcolim_n(\sm{g:Q(a)\to X_n} \prd{u:P(a)}\tfcolim_m(\id{\iota_{n+m}(g(F(a,u)))}{\iota_{n+m}(\alpha^{n+m}(f(u)))}) \\
& \eqvsym \tfcolim_n(\sm{g:Q(a)\to X_n} \tfcolim_m(\prd{u:P(a)}\id{\iota_{n+m}(g(F(a,u)))}{\iota_{n+m}(\alpha^{n+m}(f(u)))}) \\
& \eqvsym \tfcolim_n(\sm{g:Q(a)\to X_n} \id{g\circ F(a)}{\alpha^n\circ f}) \\
& \jdeq \tfcolim_n(\hfib{\psi_{X_n}}{\alpha^n\circ f}).\qedhere
\end{align*}
\end{proof}

\begin{comment}
\begin{rmk}
We have an equivalence
\begin{align*}
\hfib{\psi_{X}(a)}{f}
& = \sm{g:Q(a)\to X} g\circ F(a) = f
\end{align*}
This is contractible if and only if there is a term
\begin{align*}
j(f) & : Q(a)\to X \\
J(f) & : j(f)\circ F(a) = f \\
k(f) & : \prd{g:Q(a)\to X}{H:g\circ F(a)=f} g = j(f) \\
K(f) & : \prd{g:Q(a)\to X}{H:g\circ F(a)=f} \trans{k(f)}{H} = J(f).
\end{align*}
\end{rmk}
\end{comment}

\begin{defn}
A \define{quasi $F$-local extension} of $X$ consists of a map $l:X\to Y$ and terms
\begin{align*}
\alpha & : \prd{a:A}{f:P(a)\to X}\hfib{\psi_Y}{l\circ f} \\
\beta & : \prd{a:A}{f:P(a)\to X}{(j,H):\hfib{\psi_X}{f}}\pairr{l\circ j,\apfunc{l}\circ H}=\alpha(f).
\end{align*}
\end{defn}

\begin{rmk}
A quasi-local extension $l:X\to Y$ is therefore a map for which the map defined by the universal property
\begin{equation*}
\begin{tikzcd}
\hfib{\psi_X}{f} \arrow[drr,bend left=15] \arrow[ddr, bend right=15,swap,"\proj1(\blank)\circ l"] \arrow[dr,densely dotted,"u"] \\
& \hfib{\psi_Y}{l\circ f} \arrow[r] \arrow[d] & \unit \arrow[d,"l\circ f"] \\
& (Q(a)\to Y) \arrow[r,swap,"\psi_Y"] & (P(a)\to Y)
\end{tikzcd}
\end{equation*}
factors through the unit type.
\end{rmk}

\begin{thm}
Suppose $F$ is a family of maps between $\omega$-compact types, and let
\begin{equation*}
\begin{tikzcd}
X_0 \arrow[r,"l"] & X_1 \arrow[r,"l"] & X_2 \arrow[r,"l"] & \cdots
\end{tikzcd}
\end{equation*}
be a sequence of types in which each $X_{n+1}$ is a quasi $F$-local extension of $X_n$. Then the sequential colimit $X_\infty$ is $F$-local.
\end{thm}

\begin{proof}
We have already seen that there is an equivalence of type
\begin{equation*}
\eqv{\hfib{\psi_{X_\infty}}{\iota_0\circ f}}{\tfcolim_n(\hfib{\psi_{X_n}}{l^n\circ f})}.
\end{equation*}
for any $f:P(a)\to X_0$. Since the maps
\begin{equation*}
\hfib{\psi_{X_n}}{l^n\circ f}\to \hfib{\psi_{X_{n+1}}}{l^{n+1}\circ f}
\end{equation*}
all factor through the unit type, the colimit is contractible. It follows (by generalization), that
\begin{equation*}
\hfib{\psi_{X_\infty}}{\iota_n\circ f}
\end{equation*}
is contractible for any $f:P(a)\to X_n$. Now, by $\omega$-compactness of $P(a)$, it follows that $X_\infty$ is $F$-local.
\end{proof}

Note that the initial quasi $F$-local extension of $X$ can be specified as a higher inductive type (with a $2$-path constructor).

\begin{defn}
For any type $X$, we define the \emph{initial quasi $F$-local extension of $X$} as
the higher inductive type $L(X)$ with constructors
\begin{align*}
l & : X\to L(X) \\
j & : (P(a)\to X) \to (Q(a)\to L(X)) \\
J & : \prd{f:P(a)\to X} j(f)\circ F(a) = l\circ f \\
k & : \prd{f:P(a)\to X}{g:Q(a)\to X}{H:g\circ F(a)=f} l\circ g = j(f) \\
K & : \prd{f:P(a)\to X}{g:Q(a)\to X}{H:g\circ F(a)=f} \trans{k(f,g,H)}{l\circ H} = J(f).
\end{align*}
\end{defn}

\begin{comment}
\begin{lem}
Let $X$ be a type. For any $f:P(a)\to X$, the homotopy fiber 
$\hfib{\psi_{L(X)}(a)}{l\circ f}$ is equivalent to the higher inductive type 
with type constructors
\begin{align*}
i & : \hfib{\psi_X(a)}{f}\to\hfib{\psi_{L(X)}(a)}{l\circ f} \\
c & : \hfib{\psi_{L(X)}(a)}{l\circ f} \\
p & : \prd{h:\hfib{\psi_X(a)}{f}}\id{i(h)}{c}.
\end{align*}
\end{lem}

\begin{proof}
We define $c\defeq\pairr{j(f),J(f)}$ and $p\defeq\pairr{k(f),K(f)}$.
\end{proof}

\begin{cor}\label{cor}
For any type $X$ and $f:P(a)\to X$, we have
\begin{equation*}
\iscontr(\hfib{\psi_{L(X)}(a)}{l\circ f}).
\end{equation*}
\end{cor}

\begin{proof}
Let $X$ be a type and let $f:P(a)\to X$. Then $\hfib{\psi_{L(X)}(a)}{l\circ f}$
fits in the pushout diagram
\begin{equation*}
\begin{tikzcd}
\hfib{\psi_X(a)}{f} \arrow[r] \arrow[d,equals] & \unit \arrow[d] \\
\hfib{\psi_X(a)}{f} \arrow[r] & \hfib{\psi_{L(X)}(a)}{l\circ f}
\end{tikzcd}
\end{equation*}
so the arrow going down on the right is an equivalence.
\end{proof}
\end{comment}

\begin{comment}
\begin{defn}
For any type $X$, we define the \emph{one-step localization of $X$ at $F$} as
the higher inductive type $L(X)$ with constructors
\begin{align*}
l & : X\to L(X) \\
i,j & : (P(a)\to X) \to (Q(a)\to L(X)) \\
p & : \prd{g:Q(a)\to X}{q:Q(a)} \id{i(g\circ F(a),q)}{l(g(q))} \\
q & : \prd{f:P(a)\to X}{p:P(a)} \id{j(f,F(a,p))}{l(f(p))}
\end{align*} 
\end{defn}

\begin{lem}
Let $X$ be a type and consider the short sequence
\begin{equation*}
\begin{tikzcd}
X \arrow[r] & L(X) \arrow[r] & L^2(X).
\end{tikzcd}
\end{equation*}
There is a term of type
\begin{equation*}
\prd{f:P(a)\to X}{q:Q(a)} \id[L^2(X)]{l(k(f,q))}{k(l\circ f,q)}
\end{equation*}
for both $k\jdeq i$ and $k\jdeq j$.
\end{lem}

\begin{proof}
We first prove the assertion for $k\jdeq i$. Let $f:P(a)\to X$ and $q:Q(a)$.
Then we have to find an identification $l(i(f,q)) = i(l\circ f,q)$.
\begin{align*}
l\circ i(f) & = i(i(f)\circ F(a)) \\
& = \cdots \\
& = l\circ j(f) \\
& = i(j(f)\circ F(a))\\
& = i( l\circ f)
\end{align*}
\end{proof}
\end{comment}

\begin{defn}
Consider the diagram
\begin{equation*}
\begin{tikzcd}
X \arrow[r] & L(X) \arrow[r] & L^2(X) \arrow[r] & \cdots
\end{tikzcd}
\end{equation*}
When the situation demands it, we annotate the constructors of $L^n(X)$ with
the number $n$. We define the colimit of the above diagram to be $L^\infty(X)$. 
\end{defn}

\begin{rmk}
Since by definition, $l:X\to L(X)$ is a quasi $F$-local extension of $X$, it follows that $L^\infty(X)$ is $F$-local.
\end{rmk}

\begin{comment}
\begin{lem}
Let $X$ be a type and let $f:P(a)\to X$. Then we have an equivalence
\begin{equation*}
\eqv{\hfib{\psi_{L^\infty(X)}}{\iota_0\circ f}}{\tfcolim_n(\hfib{\psi_{L^n(X)}}{l^n\circ f})}.
\end{equation*}
\end{lem}

\begin{proof}
This follows from a simple calculation using the facts about $\omega$-compact
types of the previous section:
\begin{align*}
\hfib{\psi_{L^\infty(X)}}{\iota_0\circ f}
  & \jdeq
\sm{g:Q(a)\to L^\infty(X)} \id{g\circ F(a)}{\iota_0\circ f} \\
& \eqvsym \sm{g:\tfcolim_n(Q(a)\to L^n(X)} \id{\xi_{Q(a)}(g)\circ F(a)}{\iota_0\circ f}) \\
& \eqvsym \tfcolim_n(\sm{g:Q(a)\to L^n(X)} \id{\iota_n\circ g\circ F(a)}{\iota_0\circ f}) \\
& \eqvsym \tfcolim_n(\sm{g:Q(a)\to L^n(X)} \prd{u:P(a)}\id{\iota_n(g(F(a,u)))}{\iota_0(f(u))} \\
& \eqvsym \tfcolim_n(\sm{g:Q(a)\to L^n(X)} \prd{u:P(a)}\tfcolim_m(\id{\iota_{n+m}(g(F(a,u)))}{\iota_{n+m}(l^{n+m}(f(u)))}) \\
& \eqvsym \tfcolim_n(\sm{g:Q(a)\to L^n(X)} \tfcolim_m(\prd{u:P(a)}\id{\iota_{n+m}(g(F(a,u)))}{\iota_{n+m}(l^{n+m}(f(u)))}) \\
& \eqvsym \tfcolim_n(\sm{g:Q(a)\to L^n(X)} \id{g\circ F(a)}{l^n\circ f}) \\
& \jdeq \tfcolim_n(\hfib{\psi_{L^n(X)}}{l^n\circ f}).\qedhere
\end{align*}
\end{proof}
\end{comment}

\begin{comment}
\begin{cor}
For any $m:\nat$ and $f:P(a)\to L^m(X)$, we have an equivalence
\begin{equation*}
\eqv{\hfib{\psi_{L^\infty(X)}}{\iota_n\circ f}}{\tfcolim_{n\geq m}(\hfib{\psi_{L^n(X)}}{l^{n-m}\circ f})}.
\end{equation*}
\end{cor}
\end{comment}

\begin{comment}
\begin{lem}
For any type $X$, and any $F:\prd{a:A}P(a)\to Q(a)$ where each $P(a)$ and $Q(a)$
is $\omega$-compact, the type $L^\infty(X)$ is $F$-local.
\end{lem}

\begin{proof}
We have to show that
\begin{equation*}
\prd{a:A}{f:P(a)\to L^\infty(X)}\iscontr(\hfib{\psi_{L^\infty(X)}(a)}{f})
\end{equation*}
By our assumption of $\omega$-compactness, this type is equivalent to
\begin{equation*}
\prd{a:A}{f:\tfcolim(P(a)\to L^n(X))}\iscontr(\hfib{\psi_{L^\infty(X)}(a)}{\xi_A(f)})
\end{equation*}
We prove this by induction on $\tfcolim(P(a)\to L^n(X))$. Since we are proving
a mere proposition, it suffices to perform the induction argument on the point
constructors of $\tfcolim(P(a)\to L^n(X))$ only. So, we need to find a term of
type
\begin{equation*}
\prd{a:A}{f:P(a)\to L^n(X)}\iscontr(\hfib{\psi_{L^\infty(X)}(a)}{\iota_n\circ f})
\end{equation*}
for all $n:\nat$. Since we show this fact for all types $X$, it suffices to
show that
\begin{equation*}
\prd{a:A}{f:P(a)\to X}\iscontr(\hfib{\psi_{L^\infty(X)}(a)}{\iota_0\circ f})
\end{equation*}
By \autoref{lem}, this type is equivalent to
\begin{equation*}
\prd{a:A}{f:P(a)\to X}\iscontr(\tfcolim_{n}(\hfib{\psi_{L^n(X)}(a)}{l^{n}\circ f}))
\end{equation*}
The result now follows from \autoref{cor}.
\end{proof}
\end{comment}

\begin{comment}
\begin{proof}
To show that the function $\varphi\defeq \lam{g} g\circ F(a) : (Q(a)\to L^\infty(X))\to
(P(a)\to L^\infty(X))$ is an equivalence, we show that it has both a left 
inverse $\psi_l$ and a right inverse $\psi_r$. Note that we have the equivalence
\begin{align*}
(P(a)\to L^\infty(X))\to (Q(a)\to L^\infty(X))
  & \eqvsym
\tfcolim_n(P(a)\to L^n(X)) \to (Q(a)\to L^\infty(X))
\end{align*}
so we may construct $\psi_l$ and $\psi_r$ by induction on $n$. 
To define the functions $\psi_l$ and $\psi_r$, note that it suffices
to construct functions of type
\begin{equation*}
\psi_k^n: (P(a)\to L^n(X))\to (Q(a)\to L^{n+1}(X))
\end{equation*}
for each $n:\nat$ and $k\jdeq l,r$, and a term of type
\begin{equation*}
\prd{f:P(a)\to L^n(X)}{q:Q(a)} \iota_{n+1}(\psi_{k}^n(f,q))=\iota_{n+2}(\psi_{k}^{n+1}(l\circ f,q))
\end{equation*}

and a term of type
\begin{equation*}
\prd{f:P(a)\to L^n(X)}{q:Q(a)} l(\psi_{k}^n(f,q))=\psi_{k}^{n+1}(l\circ f,q)
\end{equation*}
We first define the left inverse $\psi_l$.  
\end{proof}
\end{comment}

\begin{comment}
\begin{proof}
We have to show that
\begin{equation*}
\prd{a:A}{f:P(a)\to L^\infty(X)}\iscontr(\sm{g:Q(a)\to L^\infty(X)} \id{g\circ F(a)}{f})
\end{equation*}
By our assumption of $\omega$-compactness, this type is equivalent to
\begin{equation*}
\prd{a:A}{f:\tfcolim(P(a)\to L^n(X))}\iscontr(\sm{g:Q(a)\to L^\infty(X)} \id{g\circ F(a)}{\xi_A(f)})
\end{equation*}
We prove this by induction on $\tfcolim(P(a)\to L^n(X))$. Since we are proving
a mere proposition, it suffices to perform the induction argument on the point
constructors of $\tfcolim(P(a)\to L^n(X))$ only. So, we need to find a term of
type
\begin{equation*}
\prd{a:A}{f:P(a)\to L^n(X)}\iscontr(\sm{g:Q(a)\to L^\infty(X)} \id{g\circ F(a)}{i_n\circ f})
\end{equation*}
Let $a:A$ and $f:P(a)\to L^n(X)$. Then we have the equivalences
\begin{align*}
& \sm{g:Q(a)\to L^\infty(X)} \id{g\circ F(a)}{\iota_n\circ f} \\
& \quad \eqvsym \sm{g:\tfcolim_{m\geq n}(Q(a)\to L^m(X)} \id{\xi_{Q(a)}(g)\circ F(a)}{\iota_n\circ f}) \\
& \quad \eqvsym \tfcolim_{m\geq n}(\sm{g:Q(a)\to L^m(X)} \id{i_m\circ g\circ F(a)}{\iota_n\circ f}) \\
& \quad \eqvsym \tfcolim_{m\geq n}(\sm{g:Q(a)\to L^m(X)} \prd{u:P(a)}\id{i_m(g(F(a,u)))}{\iota_n(f(u))} \\
& \quad \eqvsym \tfcolim_{m\geq n}(\sm{g:Q(a)\to L^m(X)} \prd{u:P(a)}\tfcolim_k(\id{i_{m+k}(g(F(a,u)))}{\iota_{m+k}(l^{k+m-n}(f(u)))}) \\
& \quad \eqvsym \tfcolim_{m\geq n}(\sm{g:Q(a)\to L^m(X)} \tfcolim_k(\prd{u:P(a)}\id{i_{m+k}(g(F(a,u)))}{\iota_{m+k}(l^{k+m-n}(f(u)))}) \\
& \quad \eqvsym \tfcolim_{m\geq n}(\sm{g:Q(a)\to L^m(X)} \id{g\circ F(a)}{l^{m-n}\circ f}) \\
& \quad \eqvsym \tfcolim_{m\geq n+1}(\sm{g:Q(a)\to L^m(X)} \id{g\circ F(a)}{l^{m-n}\circ f}) %\\
%& \quad \eqvsym \tfcolim_{m\geq n+1}(\sm{g:Q(a)\to L^m(X)} \id{l\circ g\circ F(a)}{l\circ l^{m-n}\circ f}) \\
%& \quad \eqvsym \tfcolim_{m\geq n+1}(\sm{g:Q(a)\to L^m(X)} \id{i(l\circ g)}{l\circ l^{m-n}\circ f}) \\
\end{align*}
Therefore, it suffices to show that
\begin{equation*}
\iscontr\big(\tfcolim_{m\geq n}(\sm{g:Q(a)\to L^m(X)} \id{g\circ F(a)}{l^{m-n}\circ f})\big).
\end{equation*}
We define the center of contraction to be
\begin{equation*}
c\defeq \iota_{n+1}(j(f),q).
\end{equation*}
It remains to show that every $d:\tfcolim_{m\geq n+1}(\sm{g:Q(a)\to L^m(X)} \id{g\circ F(a)}{l^{m-n}\circ f})$ is
equal to $c$. Denote the graph $\{\sm{g:Q(a)\to L^m(X)} \id{g\circ F(a)}{l^{m-n}\circ f}\}_{m\geq n+1}$ by
$\Gamma$. Then our goal is equivalent to finding a $\Gamma$-structure of the type
family $\id{d}{c}$ over $\tfcolim(\Gamma)$. 
\begin{align*}
\Gamma\mathrm{-struct}(\lam{d}\id{d}{c}) 
& \jdeq \sm{\pts{d}:\prd{i:\pts{\Gamma}} \iota(i)=c}\prd{i,j:\pts{\Gamma}}{e:\edg{\Gamma}(i,j)}\dpath{}{\kappa(e)}{\pts{d}(i)}{\pts{d}(j)}
\end{align*}
For $m\geq n+1$, let $g:Q(a)\to L^m(X)$ and $h:\id{g\circ F(a)}{l^{m-n}\circ f}$.

This is done by induction on $d$. So we first have to find a term of type
\begin{equation*}
\prd{m\geq n+1}{g:Q(a)\to L^m(X)}{h:\id{g\circ F(a)}{l^{m-n}\circ f}} \id{\iota_m(g,h)}{c}.
\end{equation*}
for $m\geq n+1$, let $g:Q(a)\to L^m(X)$ and $h:\id{g\circ F(a)}{l^{m-n}\circ f}$. Then we have
\begin{align*}
\iota_m(g,h) & = \iota_{m+1}(l\circ g,\apfunc{l}\circ h) \\
& = \iota_{m+1}(i(l\circ g\circ F(a)),\trans{\opp{p}}{(\apfunc{l}\circ h)}) \\
& = \iota_{m+1}(i(l\circ l^{m-n}\circ f),\trans{\apfunc{i}(\apfunc{l}(h))}{\trans{\opp{p}}{(\apfunc{l}\circ h)}}) \\
& \jdeq \iota_{m+1}(i(l^{m-n}\circ l\circ f),\blank) \\
& = \iota_{m+1}(i(l^{m-n}\circ j(f)\circ F(a)),\blank) \\
& = \iota_{m+1}(l^{m-n}\circ j(f),\blank) \\
& = \iota_{n+1}(j(f),q).
\end{align*}
This gives a term $\pts{d}(g,h):\pairr{g,h}=c$.
It remains to show that
\begin{equation*}
\prd{m\geq n+1}{g:Q(a)\to L^m(X)}{h:\id{g\circ F(a)}{l^{m-n}\circ f}} \dpath{}{\kappa}{\pts{d}(g,h)}{\pts{d}(l\circ g,\apfunc{l}\circ h)}.
\end{equation*}
This is done by recognizing that the diagram commutes.
\begin{equation*}
\begin{tikzcd}
\iota_m(g,h) \arrow[rr,equals] \arrow[d,equals] & & \iota_{m+1}(l\circ g,\apfunc{l}\circ h) \arrow[d,equals] \\
\iota_{m+1}(l\circ g,\apfunc{l}\circ h) \arrow[rr,equals] \arrow[d,equals] & & \iota_{m+2}(l^2\circ g,\apfunc{l^2}\circ h) \arrow[d,equals] \\
\iota_{m+1}(i(l^{1+m-n}\circ f),\nameless) \arrow[rr,equals] \arrow[d,equals] & & \iota_{m+1}(i(l^{2+m-n}\circ f),\nameless) \arrow[d,equals] \\
\iota_{m+1}(i(l^{m-n}\circ j(f)\circ F(a)),\nameless) \arrow[rr,equals] \arrow[d,equals] & & \iota_{m+1}(i(l^{1+m-n}\circ j(f)\circ F(a)),\nameless) \arrow[d,equals] \\
\iota_{m+1}(l^{m-n}\circ j(f),\nameless) \arrow[rr,equals] \arrow[dr,equals] & & \iota_{m+1}(l^{m-n}\circ j(f),\nameless) \arrow[dl,equals] \\
& \iota_{n+1}(j(f),q)
\end{tikzcd}
\end{equation*}
\end{proof}
\end{comment}

\begin{thm}
For any type $X$ and any family $F:\prd{a:A}P(a)\to Q(a)$ of maps between
$\omega$-compact types, $L_F^\infty(X)$ is the localization of $X$ at $F$.
\end{thm}

\begin{proof}
We have to show that for any function $h:X\to Y$ into an
$F$-local type $Y$, there is a unique function $\tilde{h}: L^\infty(X)\to
Y$ such that the triangle
\begin{equation*}
\begin{tikzcd}
X \arrow[dr,"h"] \arrow[d,swap,"\iota_0"] \\ 
L^\infty(X) \arrow[r,swap,"\tilde{h}"] & Y
\end{tikzcd}
\end{equation*}
commutes. To define $\tilde{h}$ for any $h:X\to Y$ into an $F$-local type, 
it suffices to define $h^1:L(X)\to Y$. To define $h^1$, we need to construct
\begin{align*}
d & : \prd{f:P(a)\to X} \hfib{\psi_Y(a)}{h\circ f}\\
D & : \prd{f:P(a)\to X}{w:\hfib{\psi_{X}(a)}{f}}\id{i(w)}{d}
\end{align*}
We get these terms immediately from the assumption that $Y$ is $F$-local.
Note also that the type of the pair $\pairr{d,D}$ is contractible, so uniqueness
follows immediately.

The above universal property and the fact that $L^\infty(X)$ is $F$-local
together give that $L^\infty(X)$ is the localization of $X$ at $F$.
\end{proof}

\section{Inductive types specified by families of $\omega$-compact types}
