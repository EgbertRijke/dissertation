\chapter{Projective spaces}

\section{The real projective spaces}
In this section I describe joint work with Ulrik Buchholtz \cite{realprojective}, in which we defined the real projective spaces $\rprojective{n}$ in homotopy type theory. Furthermore, I answer a question posed by André Joyal during his visit of CMU in February 2018, regarding the complement of the tautological bundle on $\rprojective{n}$.

\subsection{The type of \texorpdfstring{$2$}{2}-element sets}\label{sec:UUS0}

We write $\UU_{\sphere{0}}$ for the connected component at $\sphere{0}$ of the universe. 

\begin{lem}\label{lem:isequiv_evunit}
Let $X:\UU_{\sphere{0}}$. Then the map
\begin{equation*}
\equivpt_X : (\eqv{\sphere{0}}{X})\to X.
\end{equation*}
given by $e\mapsto e(\north)$ is an equivalence. We will write 
\begin{equation*}
\ptequiv_X : X\to (\eqv{\sphere{0}}{X})
\end{equation*}
for its inverse.
\end{lem}

\begin{proof}
Since $\isequiv(\evunit_X)$ is a proposition, it suffices to prove the claim for $X\jdeq\sphere{0}$. 

The inverse of $\evunit_{\sphere{0}}$ is defined by case analysis:
\begin{align*}
\north & \mapsto \idfunc \\
\south & \mapsto \bneg.
\end{align*}
It is straightforward to verify that this defines the inverse of $\evunit_{\sphere{0}}$.
\end{proof}

\begin{cor}\label{cor:id_U2}
The canonical map
\begin{equation*}
\mathsf{pt\usc{}eq} : \prd{A:\UU_{\sphere{0}}} ({\sphere{0}}= A)\to A
\end{equation*}
given by $\mathsf{pt\usc{}eq}(\refl{\sphere{0}})\defeq \north$ is an equivalence. We will write
\begin{equation*}
\mathsf{eq\usc{}pt} : \prd{A:\UU_{\sphere{0}}} A \to (\sphere{0}=A)
\end{equation*}
for the inverse fiberwise transformation.
\end{cor}

\begin{proof}
The map $\mathsf{pt\usc{}eq}$ factors as a composite of equivalences
\begin{equation*}
\begin{tikzcd}[column sep=tiny]
(\sphere{0}=X) \arrow[dr,swap,"\equiveq"] \arrow[rr,"\mathsf{pt\usc{}eq}"] & & X \\
& (\eqv{\sphere{0}}{X}) \arrow[ur,swap,"\ptequiv"] & \phantom{(\sphere{0}=X)}
\end{tikzcd}
\end{equation*}
so it is an equivalence.
\end{proof}

\begin{cor}\label{thm:ptd_2elt_sets}
The type
\begin{equation*}
\sm{X:\UU_{\sphere{0}}}X
\end{equation*}
of pointed $2$\nobreakdash-element sets is contractible.
\end{cor}

\begin{proof}
Immediate by \cref{thm:id_fundamental}.
\end{proof}

Another way of stating the following theorem, is by saying that the map
$\unit\to\UU_{\sphere{0}}$ classifies the ${\sphere{0}}$\nobreakdash-bundles.

\begin{cor}\label{lem:classifyer_U2}
Let $B:A\to\UU_{\sphere{0}}$ be a ${\sphere{0}}$\nobreakdash-bundle. Then the square
\begin{equation*}
\begin{tikzcd}
\sm{x:A}B(x) \arrow[r] \arrow[d,swap,"\proj 1"] & \unit \arrow[d,"{\sphere{0}}"] \\
A \arrow[r,swap,"B"] & \UU_{\sphere{0}},
\end{tikzcd}
\end{equation*}
which commutes via the homotopy $\lam{(x,y)}\mathsf{eq\usc{}pt}(y)^{-1}$, is a pullback square. 
\end{cor}

\begin{proof}
The square
\begin{equation*}
\begin{tikzcd}
\sm{x:A}B(x) \arrow[r] \arrow[d,swap,"\proj 1"] & \sm{X:\UU_{\sphere{0}}}X \arrow[d,"{\proj 1}"] \\
A \arrow[r,swap,"B"] & \UU_{\sphere{0}},
\end{tikzcd}
\end{equation*}
is a pullback square by \cref{thm:pb_fibequiv}. The claim follows by the fact that $\sm{X:\UU_{\sphere{0}}}X$ is contractible.
\end{proof}

\begin{prp}
For every $\sphere{0}$-bundle $B$ over $A$, the square
\begin{equation*}
\begin{tikzcd}[column sep=12em]
\Big(\sm{x:A}B(x)\Big)\times\sphere{0} \arrow[r,"{\lam{((x,y),b)}(x,\equivpt_{B(x)}(y,b))}"] \arrow[d,swap,"\proj 1"] & \sm{x:A}B(x) \arrow[d,"\proj 1"] \\
\sm{x:A}B(x) \arrow[r,swap,"\proj 1"] & A
\end{tikzcd}
\end{equation*}
is a pullback square. 
\end{prp}

\subsection{Finite dimensional real projective spaces}
\label{sec:fdrp}

Classically, the $(n+1)$-st real projective space can be obtained by attaching an $(n+1)$-cell to the $n$-th real projective space. This suggests a way of defining the real projective spaces that involves simultaneously defining $\rprojective{n}$ and an attaching map $\alpha_n : \sphere{n}\to\rprojective{n}$. Then we obtain $\rprojective{n+1}$ as the mapping cone of $\alpha_n$, i.e., as a pushout
\begin{equation*}
\begin{tikzcd}
\sphere{n} \arrow[r,"\alpha_n"] \arrow[d] & \rprojective{n} \arrow[d] \\
\unit \arrow[r] & \rprojective{n+1},
\end{tikzcd}
\end{equation*}
and we have to somehow find a way to define the attaching map $\alpha_{n+1}:\sphere{n+1}\to\rprojective{n}$ to continue the inductive procedure.
However, it is somewhat tricky to obtain these attaching maps directly, and we have chosen to follow a closely related path towards the definition of the real projective spaces that takes advantage of the machinery of dependent type theory. 

Observe that the attaching map $\alpha_n:\sphere{n}\to\rprojective{n}$ is just the tautological bundle (or the quotient map that identifies the antipodal points). This suggests that we may proceed by defining simultaneously the real projective space $\rprojective{n}$ and its tautological bundle $\tautfam[\R]{n}$. The tautological bundle on $\rprojective{n}$ is an $\sphere{0}$-bundle, so it can be described as a map $\rprojective{n}\to\UU_{\sphere{0}}$. We perform this construction in \cref{defn:realprojective} using the properties of the type of 2-element types developed in \cref{sec:UUS0}, and in \cref{thm:Sn_totalcov} we show that the total space of the tautological bundle on $\rprojective{n}$ is the $n$-sphere. 

\begin{defn}\label{defn:realprojective}
We define simultaneously for each $n:\N_{-1}$, 
the \define{$n$-dimensional real projective space} $\rprojective{n}$, 
and the \define{tautological bundle} $\tautfam[\R]{n}:\rprojective{n}\to \UU_{\sphere{0}}$.
\end{defn}

\begin{proof}[Construction]
The construction is by induction on $n:\N_{-1}$.
For the base case $n\defeq -1$, 
we take $\rprojective{-1}\defeq\emptyt$. 
Then there is a unique map of type $\rprojective{-1}\to \UU_{\sphere{0}}$, which we
take as our definition of $\tautfam[\R]{-1}$.

For the inductive step, suppose $\rprojective{n}$ and $\tautfam[\R]{n}$ are defined. Then we define $\rprojective{n+1}$ to be the pushout
\begin{equation*}
\begin{tikzcd}
{\sm{x:\rprojective{n}}\tautfam[\R]{n}(x)} \arrow[d,swap,"\proj 1"] \arrow[r] & \unit \arrow[d,"\base"] \\
\rprojective{n} \arrow[r,swap,"\inr"] & \rprojective{n+1}
\end{tikzcd}
\end{equation*}
In other words, 
$\rprojective{n+1}$ is the \emph{mapping cone} of the tautological bundle, 
when we view the tautological bundle as the projection 
$\proj 1:(\sm{x:\rprojective{n}}\tautfam[\R]{n}(x))\to\rprojective{n}$. 

To define $\tautfam[\R]{n+1}:\rprojective{n+1}\to \UU_{\sphere{0}}$
we use the universal property of $\rprojective{n+1}$. 
Therefore, it suffices to show that the outer square in the diagram
\begin{equation}\label{eq:diagram}
\begin{tikzcd}
{\sm{x:\rprojective{n}}\tautfam[\R]{n}(x)} \arrow[d,swap,"\proj 1"] \arrow[r] & \unit \arrow[d,swap,"\base"] \arrow[ddr,bend left=15,"\sphere{0}"]\\
\rprojective{n} \arrow[drr,bend right=15,swap,"{\tautfam[\R]{n}}"] \arrow[r,swap,"\inr"] & \rprojective{n+1} \arrow[dr,densely dotted] \\
& & \UU_{\sphere{0}}
\end{tikzcd}
\end{equation}
commutes. Indeed, in \cref{lem:classifyer_U2} we have constructed a homotopy 
\begin{equation*}
R_n\defeq R_{\rprojective{n},\tautfam[\R]{n}}:\prd{x:\rprojective{n}}{y:\tautfam[\R]{n}} \eqv{\tautfam[\R]{n}(x)}{\sphere{0}},
\end{equation*}
and in fact, this square is a pullback.
\end{proof}

\begin{eg}
We have $\rprojective{-1}=\emptyt$, $\rprojective{0}=\unit$, and $\rprojective{1}=\sphere{1}$. 
\end{eg}

\begin{thm}\label{thm:Sn_totalcov}
For each $n:\N_{-1}$, there is an equivalence
\begin{equation*}
e_n:\eqv{\sphere{n}}{\sm{x:\rprojective{n}}\tautfam[\R]{n}(x)}.
\end{equation*}
\end{thm}

In other words, $\rprojective{n+1}$ is obtained from $\rprojective{n}$ by attaching a single $(n+1)$\nobreakdash-disk, i.e., as a pushout
\begin{equation*}
\begin{tikzcd}
\sphere{n} \arrow[r] \arrow[d,swap,"\proj1\circ e_n"] & \unit \arrow[d] \\
\rprojective{n} \arrow[r] & \rprojective{n+1}.
\end{tikzcd}
\end{equation*}

\begin{proof}
For $n\jdeq -1$, we have $\rprojective{-1}\jdeq\emptyt$ and the unique tautological bundle $\tautfam[\R]{-1}$. Therefore the type $\sm{x:\rprojective{-1}}\tautfam[\R]{-1}(x)$ is equivalent to the empty type, which is $\sphere{-1}$ by definition. This gives the base case.

Now assume that we have an equivalence $e_n:\eqv{\sphere{n}}{\sm{x:\rprojective{n}}\tautfam[\R]{n}(x)}$. 
Our goal is to construct the equivalence
\begin{equation*}
e_{n+1}:\eqv{\sphere{n+1}}{\sm{x:\rprojective{n+1}}\tautfam[\R]{n+1}(x)}.
\end{equation*}
such that the square
\begin{equation}\label{eq:Sn_totalcov_natural}
\begin{tikzcd}
\sphere{n} \arrow[d,swap,"e_{n}"] \arrow[r,"\inl"] & \sphere{n+1} \arrow[d,"e_{n+1}"] \\
\sm{x:\rprojective{n}}\tautfam[\R]{n}(x) \arrow[r] & \sm{x:\rprojective{n+1}}\tautfam[\R]{n+1}(x)
\end{tikzcd}
\end{equation}
commutes. By the functoriality of the join (or equivalently, by equivalence induction on $e_n$), it suffices to find an equivalence
\begin{equation*}
\alpha:\eqv{\join{\Big(\sm{x:\rprojective{n}}\tautfam[\R]{n}(x)\Big)}{\sphere{0}}}{\sm{x:\rprojective{n+1}}\tautfam[\R]{n+1}(x)},
\end{equation*}
such that the bottom triangle in the diagram
\begin{equation*}
\begin{tikzcd}
\sphere{n} \arrow[r,"\inl"] \arrow[d,swap,"e_n"] & \sphere{n+1} \arrow[d,"\join{e_n}{\idfunc[\sphere{0}]}"] \\
\sm{x:\rprojective{n}}\tautfam[\R]{n}(x) \arrow[r,"\inl"] \arrow[dr] & \join{\Big(\sm{x:\rprojective{n}}\tautfam[\R]{n}(x)\Big)}{\sphere{0}} \arrow[d,"\alpha"] \\
& \sm{x:\rprojective{n+1}}\tautfam[\R]{n+1}(x)
\end{tikzcd}
\end{equation*}
commutes.
We construct this equivalence using the flattening lemma, \cref{lem:flattening}, from which we get a pushout square:
\begin{equation*}
\begin{tikzcd}[column sep=0.5em]
\sm{x:\rprojective{n}}{y:\tautfam[\R]{n}(x)}\tautfam[\R]{n}(x) \arrow[r] \arrow[d] & \sm{t:\unit}\sphere{0} \arrow[d] \\
\sm{x:\rprojective{n}}\tautfam[\R]{n}(x) \arrow[r] & \sm{x:\rprojective{n+1}}\tautfam[\R]{n+1}(x)
\end{tikzcd}
\end{equation*}
We can calculate this pushout by constructing a natural transformation of spans (diagrams in $\UU$ of the form $\cdot\leftarrow\cdot\rightarrow\cdot$), as indicated by the diagram in Fig.~\ref{fig:sphere-equiv}.
\begin{figure*}
  \centering
\begin{tikzcd}[column sep=6em]
\sm{x:\rprojective{n}}\tautfam[\R]{n}(x) \arrow[d,swap,"\idfunc"]
  & \sm{x:\rprojective{n}}{y:\tautfam[\R]{n}(x)}\tautfam[\R]{n}(x) \arrow[l,swap,"\pairr{x,z}\mapsfrom\pairr{x,y,z}" yshift=1ex] \arrow[d,densely dotted,"u"] \arrow[r,"\pairr{x,y,z}\mapsto\pairr{\ttt,R_n(x,y,z)}" yshift=1ex] 
  & \sm{t:\unit}\sphere{0} \arrow[d,"\proj 2"] \\
\sm{x:\rprojective{n}}\tautfam[\R]{n}(x)
  &
\Big(\sm{x:\rprojective{n}}\tautfam[\R]{n}(x)\Big)\times\sphere{0} \arrow[l,"\pi_1"] \arrow[r,swap,"\pi_2"]
  & \sphere{0}
\end{tikzcd}
\caption{Map of spans used in the proof of Thm~\ref{thm:Sn_totalcov}. The map $u$ is given by $\pairr{x,y,z}\mapsto\pairr{x,z,R_n(x,y,z)}$.}
\label{fig:sphere-equiv}
\end{figure*}
To show that the map $u$ in Fig.~\ref{fig:sphere-equiv} is an equivalence, it suffices to show that $R_n(x,y,z)=R_n(x,y,z)$ for any $x$, $y$, and $z$, because then it follows that $u$ is homotopic to the total map of a fiberwise equivalence. More generally, it suffices to show that $R_{\UU_{\sphere{0}},T}(X,x,y)=R_{\UU_{\sphere{0}},T}(X,y,x)$, where $T$ is the tautological bundle on $\UU_{\sphere{0}}$. Since $\UU_{\sphere{0}}$ is connected and since our goal is a mere proposition, we only need to verify the claim at the base point $\sphere{0}$ of $\UU_{\sphere{0}}$. This boils down to verifying that the group multiplication of $\Zmodtwo$ is indeed commutative.
\end{proof}

\begin{cor}
We obtain the fiber sequence
\begin{equation*}
\begin{tikzcd}
\sphere{0} \arrow[r,hook] & \sphere{n} \arrow[r,->>] & \rprojective{n}.
\end{tikzcd}
\end{equation*}
Hence, for each $k\geq 2$ we have $\pi_k(\sphere{n})=\pi_k(\rprojective{n})$. 
\end{cor}

\begin{proof}
Since we have the double cover $\tautfam[\R]{n}:\rprojective{n}\to\UU_{\sphere{0}}$ with total space $\sphere{n}$, we obtain the long exact sequence
\begin{equation*}
\begin{tikzcd}
  \cdots \arrow[r]
  & \pi_{k+1}(\sphere{n}) \arrow[r] \arrow[d, phantom, ""{coordinate, name=Z}]
  & \pi_{k+1}(\rprojective n) \arrow[dll, rounded corners,
      to path={ -- ([xshift=8.5ex]\tikztostart.center)
                |- (Z) [near end]\tikztonodes
                -| ([xshift=-8ex]\tikztotarget.center) -- (\tikztotarget)}] \\
  \pi_k(\sphere{0}) \arrow[r]
  & \pi_k(\sphere{n}) \arrow[r] \arrow[d, phantom, ""{coordinate, name=W}]
  & \pi_k(\rprojective n) \arrow[dll, rounded corners,
      to path={ -- ([xshift=8.5ex]\tikztostart.center)
                |- (W) [near end]\tikztonodes
                -| ([xshift=-8ex]\tikztotarget.center) -- (\tikztotarget)}] \\
  \pi_{k-1}(\sphere{0}) \arrow[r]
  & \pi_{k-1}(\sphere{n}) \arrow[r]
  & \cdots
\end{tikzcd}
\end{equation*}
Since $\pi_k({\sphere{0}})=0$ for $k\geq 1$, we get the desired isomorphisms.
\end{proof}

\subsection{The infinite dimensional real projective space}
\label{sec:idrp}

Observe that from the definition of $\rprojective{n}$ and its tautological 
cover, we obtain a commutative diagram of the form:
\begin{equation*}
\begin{tikzcd}[row sep=large,column sep=large]
\rprojective{-1} \arrow[r,"\inr"] \arrow[dr,swap,"{\tautfam[\R]{-1}}"] 
& \rprojective{0} \arrow[d,swap,near start,"{\tautfam[\R]{0}}"] \arrow[r,"\inr"] 
& \rprojective{1} \arrow[dl,swap,"{\tautfam[\R]{1}}"] \arrow[r,"\inr"] 
& \cdots \arrow[dll,"{\tautfam[\R]{2}}"]\\
& \UU_{\sphere{0}}
\end{tikzcd}
\end{equation*}
Using this sequence, we define the infinite dimensional real projective space
and its tautological cover:

\begin{defn}
We define the \define{infinite real projective space} $\rprojective{\infty}$ to be the sequential colimit of the finite real projective spaces. The double covers on $\rprojective{n}$ define a cocone on the type sequence of real projective spaces, so we also obtain $\tautfam[\R]{\infty}:\rprojective{\infty}\to \UU_{\sphere{0}}$. 
\end{defn}

\begin{thm}\label{thm:RPoo_US0}
The double cover $\tautfam[\R]{\infty}$ is an equivalence from $\rprojective{\infty}$ to $\UU_{\sphere{0}}$. 
\end{thm}

\begin{proof}
We have to show that the fibers of $\tautfam[\R]{\infty}$ are contractible.
Since being contractible is a mere proposition, and since the type $\UU_{\sphere{0}}$
is connected, it suffices to show that the fiber
\begin{equation*}
\sm{x:\rprojective{\infty}}\sphere{0}=\tautfam[\R]{\infty}(x)
\end{equation*}
of $\tautfam[\R]{\infty}$ at $\sphere{0}:\UU_{\sphere{0}}$ is contractible.
By \cref{cor:id_U2} we have an equivalence of type
\begin{equation*}
\eqv{({\sphere{0}}=\tautfam[\R]{\infty}(x))}{\tautfam[\R]{\infty}(x)},
\end{equation*}
for every $x:\rprojective{\infty}$. 
Therefore it is equivalent to show that the type
\begin{equation*}
\sm{x:\rprojective{\infty}}\tautfam[\R]{\infty}(x)
\end{equation*}
is contractible. The general version of the flattening lemma, as stated in
Lemma 6.12.2 in \cite{hottbook}, can be adapted for sequential colimits, so
we can pull the colimit out: it suffices to prove that
\begin{equation*}
\tfcolim_n\bigl(\sm{x:\rprojective{n}}\tautfam[\R]{n}(x)\bigr)
\end{equation*}
is contractible. 
To do this, observe that the equivalences of \cref{thm:Sn_totalcov} form
a natural equivalence of type sequences as shown in Fig.~\ref{fig:type-sequences}.
\begin{figure*}
  \centering
\begin{tikzcd}
\sm{x:\rprojective{-1}}\tautfam[\R]{-1}(x) \arrow[r] \arrow[d,swap,"\eqvsym"]
& \sm{x:\rprojective{0}}\tautfam[\R]{0}(x) \arrow[r] \arrow[d,swap,"\eqvsym"]
& \sm{x:\rprojective{1}}\tautfam[\R]{1}(x) \arrow[r] \arrow[d,swap,"\eqvsym"]
& \cdots \\
\sphere{-1} \arrow[r] 
& \sphere{0} \arrow[r]
& \sphere{1} \arrow[r]
& \cdots
\end{tikzcd}
\caption{Natural equivalence of type sequences for Thm~\ref{thm:RPoo_US0}.}
\label{fig:type-sequences}
\end{figure*}
Indeed, the naturality follows from \cref{eq:Sn_totalcov_natural}.

Thus, the argument comes down to showing that $\sphere{\infty}\defeq\tfcolim_n(\sphere{n})$
is contractible. This was first shown in homotopy type theory by Brunerie, and
the argument is basically that the sequential colimit of a type sequence of
strongly constant maps (viz., maps factoring through $\unit$) is always contractible.
\end{proof}

\begin{rmk}
Note that by our assumption that the universe is closed under pushouts, it
follows that each $\rprojective{n}$ is in $\UU$. 
Since the universe contains a natural numbers object $\N$, 
it also follows that the universe is closed under sequential colimits,
and therefore we have $\rprojective{\infty}:\UU$. 
Whereas a priori it is not clear that $\UU_{\sphere{0}}$ is equivalent to a 
$\UU$-small type, this fact is contained in \cref{thm:RPoo_US0}.
\end{rmk}

\subsection{The real projective spaces as classifying spaces}
In this subsection I answer a question posed by André Joyal during his visit of CMU in February 2018, regarding the complement of the tautological bundle on $\rprojective{n}$. The question was to construct in homotopy type theory an $\sphere{n}$-bundle $\beta^n$ over $\rprojective{n+1}$, equipped with equivalences $\eqv{\sphere{n+1}}{\join{\beta^n(x)}{\gamma^n(x)}}$ for every $x:\rprojective{n+1}$, and moreover that $\rprojective{n+1}$ classifies types $X$ with an $\sphere{0}$-bundle and an $\sphere{n}$-bundle that become trivial when joined together. 

\begin{defn}
We construct for $n\geq 0$ an $\sphere{n-1}$-bundle $\orthcomp[\R]{n}$ on $\rprojective{n}$ such that
\begin{equation*}
\eqv{\join{\orthcomp[\R]{n}(x)}{\tautfam[\R]{n}(x)}}{\sphere{n}}.
\end{equation*}
\end{defn}

\begin{proof}[Construction]
We will construct for any $n\geq 0$ a term of type
\begin{equation*}
\prd{x:\rprojective{n}} \sm{X:\BAut(\sphere{n-1})} \eqv{\join{X}{\tautfam[\R]{n}(x)}}{\sphere{n}}.
\end{equation*}
For the base case we have to define a term of type
\begin{equation*}
\prd{x:\rprojective{0}} \sm{X:\BAut(\sphere{-1})} \eqv{\join{X}{\sphere{0}}}{\sphere{0}}.
\end{equation*}
We simply choose $X\jdeq \emptyt$, and the canonical equivalence $\eqv{\join{\emptyt}{\sphere{0}}}{\sphere{0}}$. 

For the inductive step suppose we have for every $x:\rprojective{n}$ a type $\orthcomp[\R]{n}(x)$ and an equivalence
\begin{equation*}
\eqv{\join{C(x)}{\tautfam[\R]{n}(x)}}{\sphere{n}}.
\end{equation*}
Our goal is to construct for every $x:\rprojective{n+1}$ a type $\orthcomp[\mathbb{R}]{n+1}(x)$ equipped with an equivalence
\begin{equation*}
\eqv{\join{\orthcomp[\mathbb{R}]{n+1}(x)}{\tautfam[\R]{n+1}(x)}}{\sphere{n+1}}.
\end{equation*}
We do this by the universal property of $\rprojective{n+1}$. Thus, it suffices to construct
\begin{align*}
B_{\pt} & : \sm{X:\BAut(\sphere{n})}\eqv{\join{X}{\sphere{0}}}{\sphere{n}}.\\
B_i & : \prd{x:\rprojective{n}}\sm{X:\BAut(\sphere{n})}\eqv{\join{X}{\tautfam[\R]{n}(x)}}{\sphere{n}} \\
B_p & : \prd{x:\rprojective{n}}{y:\tautfam[\R]{n}(x)} ... 
\end{align*}
\end{proof}

\begin{lem}
For any two $2$-element sets $T$ and $T'$ with $t:T$ and $t':T'$, the square
\begin{equation*}
\begin{tikzcd}
\join{T}{T'} \arrow[r] \arrow[d] & \join{T'}{T} \arrow[d] \\
\join{T}{T'} \arrow[r] & \join{\bool}{\bool}
\end{tikzcd}
\end{equation*}
commutes.
\end{lem}

\begin{proof}
Since $\eqv{T}{(T=\bool)}$ it suffices to show that
\begin{equation*}
\join{\mathsf{neg}}{\idfunc}\htpy \sigma.
\end{equation*}
This is straightforward to verify by the universal property of $\join{\bool}{\bool}$. 
\end{proof}
\section{The complex projective spaces}

\section{The quaternionic Hopf fibration}
\section{The classical Cayley-Dickson construction}
\label{sec:cayley-dickson}

Classically, the $1$-, $3$- and $7$-dimensional spheres are subspaces of
$\mathbb{R}^2$, $\mathbb{R}^4$ and $\mathbb{R}^8$, respectively. Each
of these vector spaces can be given the structure of a normed division
algebra, and we get the complex numbers $\mathbb{C}$, Hamilton's
quaternions $\mathbb{H}$, and the octonions $\mathbb{O}$ of Graves and
Cayley. Since, in each of these algebras, the product preserves norm,
the unit sphere is a subgroup of the multiplicative group.

Cayley's construction of the octonions was later generalized by
Dickson \cite{Dickson1919}, who gave a uniform procedure for
generating each of these algebras from the previous one. The process
can be continued indefinitely, giving for instance the $16$-dimensional
sedenion-algebra after the octonions.

Here we describe one variant of the Cayley-Dickson construction,
following the presentation in \cite{Baez2002}. For this purpose, let
an \emph{algebra} be a vector space $A$ over $\mathbb R$ together with
a bilinear multiplication, which need not be associative, and a unit
element $1$. A \emph{$*$-algebra} is an algebra equipped with a linear involution
$*$ (called the \emph{conjugation}) satisfying $1^*=1$ and $(ab)^*=b^*a^*$.

If $A$ is a $*$-algebra, then $A' \defeq A\oplus A$ is again a
$*$-algebra using the definitions
\begin{equation}
  \label{eq:classical-cd}
  (a,b)(c,d) := (ac - db^*, a^*d + cb),
  \quad 1 := (1,0),
  \quad (a,b)^* := (a^*,-b).
\end{equation}
If $A$ is \emph{nicely normed} in the sense that (i) for all $a$, we have $a+a^*\in \mathbb R$
(i.e., the subspace spanned by $1$), and (ii) $aa^*=a^*a>0$ for
nonzero $a$, then so is $A'$. In the nicely normed case, we get a norm by defining
\[\norm a = aa^*,\]
and we have inverses given by $a^{-1}=a^*/\norm a$.
By applying this construction repeatedly, starting with $\mathbb{R}$, we obtain the
following sequence of algebras, each one having slightly fewer good
properties than the preceding one:
\begin{itemize}
\item $\mathbb R$ is a \emph{real} (i.e., $a^*=a$) commutative associative
  nicely normed $*$-algebra,
\item $\mathbb C$ is a commutative associative nicely normed $*$-algebra,
\item $\mathbb H$ is an associative nicely normed $*$-algebra,
\item $\mathbb O$ is an \emph{alternative} (i.e., any subalgebra generated by
  two elements is associative) nicely normed $*$-algebra,
\item the sedenions and the following algebras are nicely normed
  $*$-algebras, which are neither commutative, nor alternative.
\end{itemize}
Being alternative, the first four are normed division algebras, as
$a,b,a^*,b^*$ are in the subalgebra generated by $a-a^*$ and $b-b^*$,
so we get
\[ \norm{ab}^2 = (ab)(ab)^* = (ab)(b^*a^*) = a(bb^*)a^* = \norm
  a^2\norm b^2.\]
However, starting with the sedenions, this fails and we get nontrivial
zero divisors. In fact, the zero divisors of norm one in the sedenions
form a group homeomorphic to the exceptional Lie group $G_2$.

To sum up the story as it relates to us, we first form the four normed
division algebras $\mathbb R$, $\mathbb C$, $\mathbb H$ and $\mathbb
O$ by applying the Cayley-Dickson construction starting with $\mathbb
R$, and then we carve out the unit spheres and get spaces with
multiplication $\Sn^0$, $\Sn^1$, $\Sn^3$ and $\Sn^7$.

In homotopy type theory, we cannot use this strategy directly. Before we
discuss our alternative construction, let us recall some basics
regarding H-spaces in homotopy type theory.

\section{Spheroids and imaginaroids}
\label{sec:imaginaroids}

We saw in Section~\ref{sec:cayley-dickson} the classical
Cayley-Dickson construction on the level of $*$-algebras. We would obtain nothing
of interest by imitating this directly in homotopy type theory, as any real vector
space is contractible and thus equivalent to the one-point type $1$.

A first idea, which turns out to not quite work, is to give an
analog of the Cayley-Dickson construction on the level of the unit
spheres inside the $*$-algebras, as what we are ultimately after is the
H-space structure on these unit spheres. Thus we propose:
\begin{defn}
  A \define{Cayley-Dickson spheroid}\footnote{We use the term
    ``spheroid'' to emphasize that $S$ is to be thought of as a unit
    sphere, but we do not require $S$ to be an actual sphere.}
  consists of an H-space
  $S$ (we write $1$ for the base point, and concatenation denotes 
  multiplication) with additional operations
\begin{align*}
x &\mapsto x^\ast \tag{\define{conjugation}} \\
x &\mapsto -x \tag{\define{negation}}
\end{align*}
satisfying the further laws
\begin{alignat*}2
  1^* &= 1           &\qquad   (-x)^* &= -x^* \\
  -(-x)&= x = x^{**} &\qquad   x(-y) & = -xy  \\
  (xy)^* &= y^*x^*   &\qquad   x^* x & =1.
\end{alignat*}
\end{defn}

\begin{lem}
For any two points $x$ and $y$ of a Cayley-Dickson spheroid, we have
$xx^\ast=1$ and $(-x)y=-xy$.
\end{lem}

\begin{proof}
  For the first, simply note that $xx^\ast=x^{**}x^*=1$. For the
  second, we have:
  \begin{align*}
    (-x)y & =((-x)y)^{**} & \cdots & =(-y^*x^*)^*\\
          & =(y^*(-x)^*)^* & & =-(y^*x^*)^*\\
          & =(y^*(-x^*))^* & & =-(xy)^{**}\\
          & =\cdots & & =-xy.\qedhere
  \end{align*}
\end{proof}

The hope is now that if $S$ is an \emph{associative} Cayley-Dickson spheroid, then we can give the
join $\join SS$ the structure of a Cayley-Dickson spheroid. This turns out not quite to work, but it is instructive
to see where we get stuck.

We wish to define the multiplication $xy$ for $x,y : \join SS$ by
induction on $x$ and $y$. To do the induction on $x$ we must define
elements $(\inl\, a)y$, $(\inl\, b)y$ and paths
$(\jglue\,a\,b)_*y : (\inl\, a)y = (\inr\, b)y$ for
$a,b:S$. This is of course the same as giving the two bent arrows such
that the outer square commutes in following diagram, where the inner square
is the pushout square defining $\join SS$ pulled back along the
projection $\join SS \to 1$ corresponding to the variable $y$. The
dotted arrow is the desired multiplication map:
\begin{equation*}
\begin{tikzcd}
S\times S\times(\join{S}{S}) \arrow{r} \arrow{d}
\arrow[dr,phantom,"\ulcorner",very near end] &
S\times(\join{S}{S}) \arrow{d} \arrow[bend left=20]{ddr} \\
S\times(\join{S}{S}) \arrow{r} \arrow[bend right=12]{drr} &
(\join{S}{S})\times(\join{S}{S}) \arrow[densely dotted]{dr} \\
& & \join{S}{S}
\end{tikzcd}
\end{equation*}
In each case we do an induction on $y$, giving the following point
constructor problems, which we solve using equation
\eqref{eq:classical-cd}:
\begin{alignat*}2
  (\inl\, a)(\inl\, c) &\defeq \inl(ac) &\quad
  (\inl\, a)(\inr\, d) &\defeq \inr(a^*d) \\
  (\inr\, b)(\inl\, c) &\defeq \inr(cb) &\quad
  (\inr\, b)(\inr\, d) &\defeq \inl(-db^*)
\end{alignat*}
We must define four dependent paths corresponding to the interaction
of a point constructor with a path constructor, and these we all fill
with $\jglue$ (or its inverse). There results a dependent path problem
in an identity type family, which we can think of as the problem of
filling the square on the left, also depicted on the right as a
\define{diamond}:
\begin{equation}\label{eq:cd-diamond}
  \begin{tikzcd}[arrows=equals]
    \inl(ac) \arrow{r}{\jglue}\arrow{d}[swap]{\jglue} &
    \inr(cb) \arrow{d}{\rev{\jglue}} \\
    \inr(a^*d) \arrow{r}[swap]{\rev{\jglue}} &
    \inl(-db^*)
  \end{tikzcd}
  \qquad
  \begin{tikzcd}[every arrow/.append style={-},row sep=tiny,column sep=tiny]
    & cb \arrow{dl}\arrow{dr} & \\
    -db^* \arrow{dr} & & ac\arrow{dl} \\
    & a^*d &
  \end{tikzcd}
\end{equation}
These diamond shapes will play an important role in the
construction. We can define these diamond types as certain square
types sitting in a join, $\join AB$, for any $a,a':A$ and $b,b':B$:
\begin{equation}\label{eq:gen-diamond}
  \begin{tikzcd}[arrows=equals]
    a \arrow{r}{\jglue}\arrow{d}[swap]{\jglue} &
    b \arrow{d}{\rev{\jglue}} \\
    b' \arrow{r}[swap]{\rev{\jglue}} & a'
  \end{tikzcd}
  \qquad
  \begin{tikzcd}[every arrow/.append style={-},row sep=tiny,column sep=tiny]
    & b \arrow{dl}\arrow{dr} & \\
    a' \arrow{dr} & & a\arrow{dl} \\
    & b' &
  \end{tikzcd}
\end{equation}
The geometric intuition behind the shape is that we
picture the join $\join AB$ as $A$ lying on a horizontal line, $B$ on
a vertical line, and $\jglue$-paths connecting every point in $A$ to
every point in $B$.
\begin{defn}\label{defn:vhdiamond}
  Given a diamond problem corresponding to $a,a':A$ and $b,b':B$ as in
  \eqref{eq:gen-diamond}, if we have either a path $p:a=_Aa'$ or a
  path $q:b=_Bb'$, then we can solve it (i.e., fill the square on the
  left).
\end{defn}
\begin{proof}[Construction]
  By path induction on $p$ resp.\ $q$ followed by easy
  2-dimensional box filling.
\end{proof}
\begin{defn}
  Given types $A_1,A_2,B_1,B_2$ and functions $f:A_1\to A_2$ and
  $g:B_1$ to $B_2$, if we have a solution to the diamond problem in
  $\join{A_1}{B_1}$ given by $a,a':A_1$, $b,b':B_1$, then we apply
  the induced function
  $\join fg:\join{A_1}{B_1}\to \join{A_2}{B_2}$ to obtain a
  solution to the diamond problem in $\join{A_2}{B_2}$ given by
  $f\,a,f\,a':A_2$, $g\,b,g\,b':B_2$:
  \begin{equation*}
    \begin{tikzcd}[every arrow/.append style={-},row sep=tiny,column sep=tiny]
      & b \arrow{dl}\arrow{dr} & \\
      a' \arrow{dr} & & a\arrow{dl} \\
      & b' &
    \end{tikzcd}\quad\mapsto\quad
    \begin{tikzcd}[every arrow/.append style={-},row sep=tiny,column sep=tiny]
      & g\,b \arrow{dl}\arrow{dr} & \\
      f\,a' \arrow{dr} & & f\,a\arrow{dl} \\
      & g\,b' &      
    \end{tikzcd}
  \end{equation*}
\end{defn}
\begin{proof}[Construction]
  This is an instance of applying a function to a square.
\end{proof}
Coming back to \eqref{eq:cd-diamond} and fixing $a,b,c,d:S$, consider
the functions $f,g: S \to S$:
\begin{equation*}
  f(x) \defeq -acx, \qquad g(y) \defeq cyb
\end{equation*}
(we are leaving out the parentheses since we are assuming the
multiplication is associative).
\begin{lem}\label{lem:calculations}
  If the multiplication is associative, then we have $f(-1)=ac$,
  $f(c^*a^*db^*)=-db^*$, $g(1)=cb$, and $g(c^*a^*db^*)=a^*d$.
\end{lem}
\begin{proof}
  For example,
  \begin{alignat*}2
    ac(-c^*a^*db^*)
    &= -acc^*a^*db^* & \qquad\cdots
    &= -(aa^*)db^* \\
    &= -a(cc^*)a^*db^* &
    &= -1db^* \\
    &= -a1a^*db^* &
    &= -db^*.
  \end{alignat*}
  \par \vspace{-1.3\baselineskip}
  \qedhere
\end{proof}
Thus, it suffices to solve the diamond problem,
\begin{equation}\label{eq:x-diamond}
  \begin{tikzcd}[every arrow/.append style={-},row sep=tiny,column sep=tiny]
    & 1 \arrow{dl}\arrow{dr} & \\
    c^*a^*db^* \arrow{dr} & & -1\arrow{dl} \\
    & c^*a^*db^* &
  \end{tikzcd}\quad\text{or simply,}\quad
  \begin{tikzcd}[every arrow/.append style={-},row sep=tiny,column sep=tiny]
    & 1 \arrow{dl}\arrow{dr} & \\
    x \arrow{dr} & & -1\arrow{dl} \\
    & x &
  \end{tikzcd}
\end{equation}
with $x=c^*a^*db^*$. Naively, we might hope to solve this problem for every
$x:S$. However, considering the case where $S$ is the unit $0$-sphere
$\{\pm1\}$ in $\mathbb R$, it seems necessary to make a case
distinction on $x$ to do so. This motivates the following revised
strategy.

\subsection{Cayley-Dickson imaginaries}

Instead of just axiomatizing the unit sphere, we shall make use of the
fact that all the unit spheres in the Cayley-Dickson algebras are
suspensions of the unit sphere of imaginaries (the unit $0$-sphere in
$\mathbb R$ is of course the suspension of the $-1$-sphere, i.e., the
empty type, which corresponds to the fact the $\mathbb R$ is a
\emph{real} algebra with no imaginaries).

First we note that both conjugation and negation on a Cayley-Dickson
sphere are determined by the negation acting on the imaginaries. In
fact, we can make the following general constructions:
\begin{defn}
  Suppose $A$ is a type with a negation operation. Then we can define a
  conjugation and a negation on the suspension $\susp A$ of $A$:
  \begin{alignat*}2
    \north^* &\defeq \north &   -\north &\defeq \south \\
    \south^* &\defeq \south &   -\south &\defeq \north \\
    \mathsf{ap}\,(\lambda x. x^*)\,(\merid\,a) &:= \merid(-a) &\quad
    \mathsf{ap}\,(\lambda x. {-x})\,(\merid\,a)    &:= \rev{\merid(-a)}
  \end{alignat*}
  We give $\susp A$ the base point $\north$, which we also write as
  $1$. If the negation on $A$ is involutive, then so is the
  conjugation and negation on $\susp A$.
\end{defn}

\begin{defn}
  A \define{Cayley-Dickson imaginaroid} consists of a type $A$ with an
  involutive negation, together with a binary multiplication operation
  on the suspension $\susp A$, such that $\susp A$ becomes an H-space satisfying
  the \define{imaginaroid laws}
  \begin{align*}
    x(-y) &= -xy \\
    xx^* & =1 \\
    (xy)^* &= y^*x^*
  \end{align*}
  for $x,y:\susp A$.
\end{defn}
Note that if $A$ is a Cayley-Dickson imaginaroid, then $\susp A$
becomes a Cayley-Dickson spheroid.

\begin{defn}\label{def:imag-h-space}
Let $A$ be a Cayley-Dickson imaginaroid where the multiplication on
$\susp A$ is associative. Then $A' \defeq \join{\susp A}{\susp A}$ can
be given the structure of an H-space.
\end{defn}
\begin{proof}[Construction]
We can define the multiplication on $\join{\susp A}{\susp A}$
as in the previous section, leading to the
diamond problem~\eqref{eq:x-diamond}. This we now solve by induction
on $x:\susp A$. The diamonds for the poles are easily filled using
Definition~\ref{defn:vhdiamond}:
\begin{equation*}
  \begin{tikzcd}[every arrow/.append style={-},row sep=tiny,column sep=tiny]
    & \north \arrow{dl}\arrow{dr}\arrow[-, double equal sign distance]{dd} & \\
    \north \arrow{dr} & & \south\arrow{dl} \\
    & \north &
  \end{tikzcd}\qquad
  \begin{tikzcd}[every arrow/.append style={-},row sep=tiny,column sep=tiny]
    & \north \arrow{dl}\arrow{dr} & \\
    \south \arrow{dr}\arrow[-, double equal sign distance]{rr} & & \south\arrow{dl} \\
    & \south &
  \end{tikzcd}
\end{equation*}
These solutions must now be connected by filling, for every $a : A$,
the following hollow cube connecting the diamonds:
\begin{equation*}
  \begin{tikzcd}[every arrow/.append style={-}]
    &   &   &   & \north\arrow{dl}\arrow{dr} & \\
    & \north\arrow{dl}\arrow{dr}\arrow[dashed]{urrr} &
    & \south\arrow{dr}\arrow[-, double equal sign distance]{rr} & & \south\arrow{dl} \\
    \north\arrow{dr}\arrow[dashed]{urrr} & &
    \south\arrow[crossing over]{ul}\arrow{dl}\arrow[dashed,crossing over]{urrr} &
    & \south & \\
    & \north\arrow[-, double equal sign distance,crossing over]{uu}\arrow[dashed]{urrr} & & & &
  \end{tikzcd}
\end{equation*}
Here, the two dashed paths $\north=\north$ and $\south=\south$ are
identities, while the other two are each the meridian,
$\merid\,a:\north=\south$. Generalizing a bit, we see that we can fill
any cube in a symmetric join, $\join BB$, with $p:x=_By$, of this form:
\begin{equation*}
  \begin{tikzcd}[every arrow/.append style={-}]
    &   &   &   & x\arrow{dl}\arrow{dr} & \\
    & x\arrow{dl}\arrow{dr}\arrow[dashed]{urrr} &
    & y\arrow{dr}\arrow[-, double equal sign distance]{rr} & & y\arrow{dl} \\
    x\arrow{dr}\arrow[dashed]{urrr} & &
    y\arrow[crossing over]{ul}\arrow{dl}\arrow[dashed,crossing over]{urrr} &
    & y & \\
    & x\arrow[-, double equal sign distance,crossing over]{uu}\arrow[dashed]{urrr} & & & &
  \end{tikzcd}
\end{equation*}
Indeed, this follows by path induction on $p$ followed by trivial
manipulations.

This multiplication has the virtue that the H-space laws $1x=x1=x$ are
very easy to prove; indeed, for point constructors they follow from
the H-space laws on $\susp A$, and since these point constructors land
in the two different sides of the join, we can glue them together
trivially on path constructors.
\end{proof}

Let us finish this section by stating the result of combining
the Hopf construction (Lemma~\ref{thm:hopf_construction}) and the
H-space structure on $\Sn^3$, which we obtain from
Definition~\ref{def:imag-h-space} using the obvious imaginaroid
structure on $\Sn^0$ and the associativity of the H-space structure on
$\Sn^1=\susp\Sn^0$:
\begin{thm}
  There is a fibration sequence
  \[
    \Sn^3 \to \Sn^7 \to \Sn^4
  \]
  of pointed maps.
\end{thm}
\begin{cor}
  There is an element of infinite order in $\pi_7(\Sn^4)$.
\end{cor}
\begin{proof}
  Consider the long exact sequence of homotopy groups
  \cite[Theorem~8.4.6]{TheBook} corresponding to the above fibration
  sequence. In particular, we get the exactness of
  \[
    \pi_7(\Sn^3) \to \pi_7(\Sn^7) \to \pi_7(\Sn^4).
  \]
  The inclusion of the fiber, $\Sn^3 \hookrightarrow
  \join{\Sn^3}{\Sn^3} = \Sn^7$, is nullhomotopic, so the first map is
  zero. Since $\pi_7(\Sn^7) = \Z$, we get an exact sequence
  \[
    0 \to \Z \to \pi_7(\Sn^4),
  \]
  which gives the desired element of infinite order.
\end{proof}
