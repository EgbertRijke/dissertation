\chapter{Reflective subuniverses}

\section{Properties of reflective subuniverses}
\label{sec:prop-rfsu}

\begin{defn}
A \define{reflective subuniverse} consists of
\begin{enumerate}
\item A subtype $\UU_L\to \UU$ of the universe,
\item A map $L:\UU\to\UU_L$ called \define{localization}, equipped with a transformation
\begin{equation*}
\eta:\prd{X:\UU} X\to LX
\end{equation*}
called the \define{unit} of the localization,
\end{enumerate}
such that for each $X:\UU$ and $Y:\UU_L$, the precomposition map
\begin{equation*}
\blank\circ\eta_X: (LX\to Y)\to (X\to Y)
\end{equation*}
is an equivalence.
\end{defn}

\begin{lem}\label{lem:reflective_uniqueness}
  For any $\mathcal{M}:\UU\to\prop$ and any type $X$, the type of triples $(Y,f,I)$ consisting of $Y:\UU_{\mathcal{M}}$,
  $f:X\to Y$ and $I:\prd{Z:\UU_{\mathcal{M}}}\isequiv(\lam{g}g\circ f:(Y\to Z)\to(X\to Z))$, is a proposition.
\end{lem}

\begin{proof}
Consider $(Y,f,I)$ and $(Y',f',I')$ of the described type. Since $I$ and $I'$
are terms of a proposition, it suffices to show that $(Y,f)=(Y',f')$. In
other words, we have to find an equivalence $g:Y\to Y'$ such that $g\circ f'=f$.

By $I(Y')$, the type of
pairs $(g,h)$ consisting of a function $g:Y\to Y'$ such that $h:g\circ f=f'$ is contractible. By
$I'(Y)$, the type of pairs $(g',h')$ consisting of a function $g':Y'\to Y$
such that $h':g'\circ f'=f$ is contractible.

Now $g'\circ g$ is a function such that $g'\circ g\circ f=g'\circ f'=f$, as
is $\idfunc[Y]$. By contractibility, it follows that $g'\circ g=\idfunc[Y]$.
Similarly, $g\circ g'=\idfunc[Y']$.
\end{proof}

\begin{thm}\label{thm:subuniverse-rs}
The data of any two reflective subuniverses with the same modal types are the same.
\end{thm}
\begin{proof}
  Given the modal types, the rest of the data of a reflective subuniverse consists of, for each type $X$, a triple $(Y,f,I)$ as in \cref{lem:reflective_uniqueness}.
  Thus, by \cref{lem:reflective_uniqueness}, these data form a proposition.
\end{proof}

\begin{lem}\label{lem:subuniv-modal}
  Given a reflective subuniverse, a type $X$ is modal if and only if $\modalunit[X]$ is an equivalence.
\end{lem}
\begin{proof}
  Certainly if $\modalunit[X]$ is an equivalence, then $X$ is modal since it is equivalent to the modal type $\modal X$.
  Conversely, if $X$ is modal then $\idfunc[X]$ has the same universal property of $\modalunit[X]$; so by \cref{lem:reflective_uniqueness} they are equivalent and hence $\modalunit[X]$ is an equivalence.
\end{proof}

\begin{lem}\label{thm:modalunit-retract-equiv}
  Given a reflective subuniverse, if a modal unit $\modalunit[X]$ has a left inverse (i.e.\ a retraction), then it is an equivalence, and hence $X$ is modal.
\end{lem}
\begin{proof}
  Suppose $f$ is a left inverse of $\modalunit[X]$, i.e.\ $f\circ \modalunit[X] = \idfunc[X]$.
  Then $\modalunit[X]\circ f\circ \modalunit[X] = \modalunit[X]$, so $\modalunit[X]\circ f$ is a factorization of $\modalunit[X]$ through itself.
  By uniqueness of such factorizations, $\modalunit[X]\circ f = \idfunc[\modal X]$.
  Thus $f$ is also a right inverse of $\modalunit[X]$, hence $\modalunit[X]$ is an equivalence.
\end{proof}

\begin{lem}
  Any reflective subuniverse is a functor up to homotopy: given $f:A\to B$ we have an induced map $\modal f : \modal A \to \modal B$, preserving identities and composition up to homotopy.
  Moreover, $\modalunit$ is a natural transformation up to homotopy, i.e.\ for any $f$ we have $\modal f \circ \modalunit[A] = \modalunit[B] \circ f$.
\end{lem}
\begin{proof}
  Define $\modal f$ to be the unique function such that $\modal f \circ \modalunit[A] = \modalunit[B] \circ f$, using the universal property of $\modalunit[A]$.
  The rest is easy to check using further universal properties.
\end{proof}

\begin{lem}
  Given a reflective subuniverse and any type $X$, the map $\modal \modalunit[X] : \modal X \to \modal\modal X$ is an equivalence.
\end{lem}
\begin{proof}
  By naturality, we have $\modal \modalunit[X] \circ \modalunit[X] = \modalunit[\modal X] \circ \modalunit[X]$.
  Hence $\modal \modalunit[X] = \modalunit[\modal X]$ by the universal property of $\modalunit[X]$, but $\modalunit[\modal X]$ is an equivalence by \cref{lem:subuniv-modal}.
\end{proof}

\begin{lem}\label{thm:rsu-galois}
  Given a reflective subuniverse, a type $X$ is modal if and only if $(\blank \circ f) : (B\to X) \to (A\to X)$ is an equivalence for any function $f:A\to B$ such that $\modal f$ is an equivalence.
\end{lem}
\begin{proof}
  If $\modal f$ is an equivalence and $X$ is modal, then by the universal property of $\modalunit$, we have a commutative square
  \[
  \begin{tikzcd}
    (B\to X) \ar[r,"\blank\circ f"] & (A\to X) \\
    (\modal B\to X) \ar[r,"\blank\circ\modal f"'] \ar[u,"{\blank\circ \modalunit[B]}"] &
    (\modal A \to X) \ar[u,"{\blank\circ \modalunit[A]}"']
  \end{tikzcd}
  \]
  in which all but the top map are equivalences; thus so is the top map.

  Conversely, since $\modal\modalunit[X]$ is an equivalence, the hypothesis implies that
  $(\blank \circ \modalunit[X]) : (\modal X\to X) \to (X\to X)$
  is an equivalence.
  In particular, its fiber over $\idfunc[X]$ is inhabited, i.e.\ $\modalunit[X]$ has a retraction; hence $X$ is modal.
\end{proof}

\begin{lem}\label{lem:sum_idempotent}
Consider a reflective subuniverse with modal operator $\modal$, and let $P:X\to\UU$ for some type $X:\UU$.
Then the unique map for which the triangle
\begin{equation*}
\begin{tikzcd}
\sm{x:X}P(x) \arrow[d,swap,"\modalunit"] \arrow[dr,"{\lam{\pairr{x,y}}\modalunit(x,\modalunit(y))}"] \\
\modal(\sm{x:X}P(x)) \arrow[r,densely dotted] & \modal(\sm{x:X}\modal(P(x)))
\end{tikzcd}
\end{equation*}
commutes, is an equivalence.
\end{lem}
\begin{proof}
  Since both codomains are modal, it suffices to show that ${\lam{\pairr{x,y}}\modalunit(x,\modalunit(y))}$ has the universal property of $\modalunit[\sm{x:X}P(x)]$, i.e.\ that any map $(\sm{x:X}P(x)) \to Y$, where $Y$ is modal, extends uniquely to $\modal(\sm{x:X}\modal(P(x)))$.
  But we have
  \begin{align*}
    ((\sm{x:X}P(x)) \to Y)
    &\simeq
    \prd{x:X} P(x) \to Y\\
    &\simeq
    \prd{x:X} \modal(P(x)) \to Y\\
    &\simeq
    (\sm{x:X}\modal(P(x))) \to Y\\
    &\simeq
    \modal (\sm{x:X}\modal(P(x))) \to Y
  \end{align*}
  and it is easy to see that this is the desired precomposition map.
\end{proof}

\begin{lem}\label{lem:rs_idstable}
  For any reflective subuniverse, if $X$ is modal, then so is the identity type $x=y$ for any $x,y:X$.
\end{lem}

\begin{proof}
Let $X$ be a modal type, and let $x,y:X$. We have a map
$\modal(x=y)\to\unit$. The outer square in the diagram
\begin{equation*}
\begin{tikzcd}
\modal(x=y) \arrow[ddr,bend right=15] \arrow[drr,bend left=15] \\
& (x=y) \arrow[r] \arrow[d] \arrow[ul,"\modalunit"] \arrow[dr, phantom, "\lrcorner", very near start] & \unit \arrow[d,"x"] \\
& \unit \arrow[r,swap,"y"] & X
\end{tikzcd}
\end{equation*}
commutes, because both maps extend the map $(x=y)\to X$ along $\modalunit$, and
such extensions are unique because $X$ is assumed to be modal.
Hence the universal property of the pullback gives
an inverse of $\modalunit:(x=y)\to\modal(x=y)$.
\end{proof}

\begin{lem}\label{lem:modal-Pi}
Given a reflective subuniverse,
if $P(x)$ is modal for all $x:X$, then so is $\prd{x:X}P(x)$.
\end{lem}

\begin{proof}
By \cref{thm:modalunit-retract-equiv}, it suffices to define a left inverse of the modal unit
$\modalunit:(\prd{x:A}P(x))\to \modal(\prd{x:A}P(x))$. By the universal property
of dependent product, extending
\begin{equation*}
\begin{tikzcd}
\prd{x:A}P(x) \arrow[r,"{\idfunc}"] \arrow[d,"\modalunit"] & \prd{a:A}P(a) \arrow[d,"{\psi\,\defeq\,\lam{f}{a}\modalunit[P(a)](f(a))}"] \\
\modal(\prd{x:A}P(x)) \arrow[r,densely dotted] & \prd{a:A}\modal(P(a))
\end{tikzcd}
\end{equation*}
is equivalent to extending
\begin{equation*}
\begin{tikzcd}[column sep=large]
\prd{x:A}P(x) \arrow[r,"{\mathsf{ev}_a}"] \arrow[d,swap,"{\modalunit}"]
& P(a) \arrow[d,"{\modalunit}"] \\
\modal(\prd{x:A}P(x)) \arrow[r,densely dotted,swap,"{\modal(\mathsf{ev}_a)}"] & \modal(P(a))
\end{tikzcd}
\end{equation*}
for any $a:A$. Thus, we find
\begin{equation*}
f\defeq\lam{m}{a}\modal(\mathsf{ev}_a)(m):\modal(\prd{x:A}P(x))\to\prd{a:A}P(a).
\end{equation*}
as the solution to the first extension problem. In the first extension problem,
the function $\psi$ is an equivalence by the assumption that each $P(a)$ is
modal, so we obtain a retraction of the modal unit.
\end{proof}

\begin{lem}\label{thm:modal-pres-prod}
Given any reflective subuniverse, the modal operator $\modal$ preserves cartesian products.
\end{lem}

\begin{proof}
We have to show that the modal extension
\begin{equation*}
\begin{tikzcd}
X\times Y \arrow[d,swap,"{\modalunit[X\times Y]}"] \arrow[dr,"\lam{\pairr{x,y}}\pairr{\modalunit[X](x),\modalunit[Y](y)}"] \\
\modal(X\times Y) \arrow[r,densely dotted] & \modal X\times\modal Y
\end{tikzcd}
\end{equation*}
is an equivalence.
By \cref{lem:reflective_uniqueness} it suffices to show that $\lam{\pairr{x,y}}\pairr{\modalunit[X](x),\modalunit[Y](y)}$ has the same universal property of $\modalunit[X\times Y]$.
But for any modal type $Z$ we have
\begin{align*}
  (X\times Y \to Z)
  &\eqvsym X\to (Y\to Z)\\
  &\eqvsym X\to (\modal Y\to Z)\\
  &\eqvsym \modal X\to (\modal Y\to Z)\\
  &\eqvsym \modal X\times \modal Y\to Z
\end{align*}
given by precomposition as desired.
Here in the penultimate step we use the fact that $\modal Y\to Z$ is modal since $Z$ is, by \cref{lem:modal-Pi}.
\end{proof}

\begin{lem}\label{lem:modal-pres-prop}
Given any reflective subuniverse, the modal operator preserves propositions.
\end{lem}
\begin{proof}
  A type $P$ is a proposition if and only if the diagonal $P\to P\times P$ is an equivalence.
  The result then follows from \cref{thm:modal-pres-prod}.
\end{proof}

By contrast, even modalities do not generally preserve $n$-types for any $n\ge 0$.
For instance, the ``shape'' modality of~\cite{shulman2015brouwer} takes the topological circle, which is a 0-type, to the homotopical circle, which is a 1-type, and the topological 2-sphere, which is also a 0-type, to the homotopical 2-sphere, which is (conjecturally) not an $n$-type for any finite $n$.
However, we will see in~\autoref{modaln-truncated} that lex modalities do preserve $n$-types for all $n$.

\begin{rmk}
  The basic properties of types and maps in homotopy type theory, such as being contractible, being a proposition, being an $n$-type, being an equivalence, and so on, are all constructed (perhaps inductively) out of identity types and $\Sigma$- and $\Pi$-types.
  Thus, a $\Sigma$-closed reflective subuniverse is closed under them as well.
  That is, if $A$ and $B$ are modal and $f:A\to B$, then the propositions ``$A$ is contractible'', ``$A$ is an $n$-type'', ``$f$ is an equivalence'', and so on, are all modal as well.
\end{rmk}

\section{The reflective subuniverse of separated types}
