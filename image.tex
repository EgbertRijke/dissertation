\chapter{Homotopy images}\label{chap:image}

\marginnote{Cite Floris and Nicolai}\marginnote{Surjectivity}We have observed in \cref{eg:rcoeq} that the reflexive coequalier of the pre-kernel of a map $f:A\to X$ is the fiberwise join $\join[X]{A}{A}$, i.e.~we have a reflexive coequalizer diagram
\begin{equation*}
\begin{tikzcd}
A\times_X A \arrow[r,yshift=1ex] \arrow[r,yshift=-1ex] & A \arrow[l] \arrow[r] & \join[X]{A}{A}.
\end{tikzcd}
\end{equation*}
Furthermore, we have observed in \cref{cor:2-pre-kernel} that the geometric realization of the 2-pre-kernel of a map $f:A\to X$ is the triple fiberwise join $\join[X]{A}{\join[X]{A}{A}}$, i.e.~we have a colimiting 
\begin{equation*}
\begin{tikzcd}
A\times_{X} A \times_{X} A \arrow[r,yshift=2ex] \arrow[r] \arrow[r,yshift=-2ex] & A \times_X A \arrow[l,yshift=1ex] \arrow[l,yshift=-1ex] \arrow[r,yshift=1ex] \arrow[r,yshift=-1ex] & A \arrow[l] \arrow[r] & \join[X]{A}{\join[X]{A}{A}}.
\end{tikzcd}
\end{equation*}
These results suggest that if we were to continue this process ad infinitum, in order to construct the Czech nerve of a map (which requires one to solve the currently open problem of defining in homotopy type theory the type of simplicial types) then one might expect that its geometric realization is the sequential colimit of the type sequence
\begin{equation*}
\begin{tikzcd}
A \arrow[r] & \join[X]{A}{A} \arrow[r] & \join[X]{A}{(\join[X]{A}{A})} \arrow[r] & \cdots
\end{tikzcd}
\end{equation*}
and its sequential colimit. However, we \emph{can} analyze this type sequence and its colimit in homotopy type theory, since it is constructed entirely in terms of known operations. We will show in \cref{thm:image} that for any map $f:A\to X$, the infinite fiberwise join-power $\join[X]{\cdots}{\join[X]{A}{\join[X]{A}{A}}}$ is the image of $f$. We conclude that the image of a map always exists in univalent type theory with homotopy pushouts. This construction of the image of $f$ is called the join construction, and first appeared in \cite{joinconstruction}. Moreover, we show in \cref{thm:image_small} that the image of an essentially small type mapping into a locally small type is again essentially small. In particular, the image of a map from a small type into the universe is essentially small. This corollary should be viewed as a type theoretic replacement axiom. This fact has the important consequence that any connected component of the universe is essentially small.

The join construction leads to new constructions of set quotients and of Rezk completions. Set-quotients have been first constructed in \cite{UniMath2015,UniMath} using propositional resizing, and as higher inductive types in \cite{hottbook} in the case where $A$ is assumed to be a set. The Rezk completion of a pre-category is due to \cite{AhrensKapulkinShulman}, and is also explained in section 9.9 of \cite{hottbook}. In \cref{sec:set-quotients} we elaborate on set quotients, with the aim of generalizing the construction in \cref{chap:giraud} to quotients of type-valued equivalence relations.\marginnote{Cite Sets in HoTT}

\section{The join construction}

\begin{defn}\label{defn:image_up}
Consider a commuting triangle
\begin{equation*}
\begin{tikzcd}[column sep=small]
A \arrow[rr,"i"] \arrow[dr,swap,"f"] & & U \arrow[dl,"m"] \\
& X
\end{tikzcd}
\end{equation*}
with $I:f\htpy m\circ i$, and where $m$ is an embedding\index{embedding}.
We say that $m$ has the \define{universal property of the image of $f$}\index{universal property!of the image|textit} if the map
\begin{equation*}
(i,I)^\ast : \mathrm{hom}_X(m,m')\to\mathrm{hom}_X(f,m')
\end{equation*}
defined by $(i,I)^\ast(h,H)\defeq (h\circ i,\ct{I}{(i\cdot H)})$,
is an equivalence for every embedding $m':U'\to X$. 
\end{defn}

\begin{lem}
For any $f:A\to X$ and any embedding\index{embedding} $m:U\to X$, the type $\mathrm{hom}_X(f,m)$ is a proposition.
\end{lem}

\begin{proof}
From \cref{cor:fib_triangle} we obtain that the type $\mathrm{hom}_X(f,m)$ is equivalent to the type
\begin{equation*}
\prd{x:X}\fib{f}{x}\to\fib{m}{x},
\end{equation*}
so it suffices to show that this is a proposition. 
Recall from \cref{thm:prop_emb} that a map is an embedding if and only if its fibers are propositions.
Thus we see that the type $\prd{x:X}\fib{f}{x}\to\fib{m}{x}$ is a product of propositions, so it is a proposition by \cref{thm:prop_pi}.
\end{proof}

\begin{cor}\label{cor:image_up}
Consider a commuting triangle
\begin{equation*}
\begin{tikzcd}[column sep=small]
A \arrow[rr,"i"] \arrow[dr,swap,"f"] & & U \arrow[dl,"m"] \\
& X
\end{tikzcd}
\end{equation*}
with $I:f\htpy m\circ i$, and where $m$ is an embedding. Then $m$ satisfies the universal property of the image of $f$ if and only if the implication
\begin{equation*}
\mathrm{hom}_X(f,m')\to\mathrm{hom}_X(m,m')
\end{equation*}
holds for every embedding $m':U'\to X$. 
\end{cor}

Recall that embeddings into the unit type are just propositions.
Therefore, the universal property of the image of the map $A\to\unit$ is a proposition $P$ satisfying the universal property of the propositional truncation:

\begin{defn}
Let $A$ be a type, and let $P$ be a proposition that comes equipped with a map $f:A\to P$. We say that $f:A\to P$ satisfies the \define{universal property of propositional truncation}\index{univeral property!of propositional truncation|textit} if for every proposition $Q$, the precomposition map
\begin{equation*}
\blank\circ f:(P\to Q)\to (A\to Q)
\end{equation*}
is an equivalence.
\end{defn}

\subsection{Stage one: constructing the propositional truncation}

\begin{lem}\label{lem:extend_join_prop}
Suppose $f:A\to P$, where $A$ is any type, and $P$ is a proposition.
Then the map
\begin{equation*}
(\join{A}{B}\to P)\to (B\to P)
\end{equation*}
given by $h\mapsto h\circ \inr$ is an equivalence, for any type $B$.
\end{lem}

\begin{proof}
Since both types are propositions by \cref{thm:prop_pi} it suffices to construct a map
\begin{equation*}
(B\to P)\to (\join{A}{B}\to P).
\end{equation*}
Let $g:B\to P$. Then the square
\begin{equation*}
\begin{tikzcd}
A\times B \arrow[r,"\proj 2"] \arrow[d,swap,"\proj 1"] & B \arrow[d,"g"] \\
A \arrow[r,swap,"f"] & P
\end{tikzcd}
\end{equation*}
commutes since $P$ is a proposition. Therefore we obtain a map $\join{A}{B}\to P$ by the universal property of the join.
\end{proof}

The idea of the construction of the propositional truncation is that if we are given a map $f:A\to P$, where $P$ is a proposition, then it extends uniquely along $\inr:A\to \join{A}{A}$ to a map $\join{A}{A}\to P$. This extension again extends uniquely along $\inr:\join{A}{A}\to \join{A}{(\join{A}{A})}$ to a map $\join{A}{(\join{A}{A})}\to P$ and so on, resulting in a diagram of the form
\begin{equation*}
\begin{tikzcd}
A \arrow[dr] \arrow[r,"\inr"] & \join{A}{A} \arrow[d,densely dotted] \arrow[r,"\inr"] & \join{A}{(\join{A}{A})} \arrow[dl,densely dotted] \arrow[r,"\inr"] & \cdots \arrow[dll,densely dotted,bend left=10] \\
& P
\end{tikzcd}
\end{equation*}

\begin{defn}
The \define{join powers} $A^{\ast n}$ of a type $X$ are defined by
\begin{align*}
A^{\ast 0} & \defeq \emptyt \\
A^{\ast 1} & \defeq A \\
A^{\ast (n+1)} & \defeq \join{A}{A^{\ast n}}.
\end{align*}
Furthermore, we define $A^{\ast\infty}$ to be the sequential colimit of the type sequence
\begin{equation*}
\begin{tikzcd}
A^{\ast 0} \arrow[r] & A^{\ast 1} \arrow[r,"\inr"] & A^{\ast 2} \arrow[r,"\inr"] & \cdots.
\end{tikzcd}
\end{equation*}
\end{defn}

Our goal is now to show that $A^{\ast\infty}$ is a proposition and satisfies the universal property of the propositional truncation.

\begin{lem}
Consider a type sequence
\begin{equation*}
\begin{tikzcd}
A_0 \arrow[r,"f_0"] & A_1 \arrow[r,"f_1"] & A_2 \arrow[r,"f_2"] & \cdots
\end{tikzcd}
\end{equation*}
with sequential colimit $A_\infty$, and let $P$ be a proposition. Then the map
\begin{equation*}
\seqin^\ast: (A_\infty\to P)\to \Big(\prd{n:\N}A_n\to P\Big)
\end{equation*}
given by $h\mapsto \lam{n}(h\circ \seqin_n)$ is an equivalence. 
\end{lem}

\begin{proof}
By the universal property of sequential colimits established in \cref{thm:sequential_up} we obtain that $\coconemap$ is an equivalence. Note that we have a commuting triangle
\begin{equation*}
\begin{tikzcd}[column sep=tiny]
\phantom{\Big(\prd{n:\N}A_n\to P\Big).} & P^{A_\infty} \arrow[dl,swap,"\coconemap"] \arrow[dr,"\seqin^\ast"] & \phantom{\cocone(P)} \\
\cocone(P) \arrow[rr,swap,"\proj 1"] & & \Big(\prd{n:\N}A_n\to P\Big)
\end{tikzcd}
\end{equation*}
Note that for any $g:\prd{n:\N}A_n\to P$ the type 
\begin{equation*}
\prd{n:\N} g_n\htpy g_{n+1}\circ f_n
\end{equation*}
is a product of contractible types, since $P$ is a proposition. Therefore it is contractible by \cref{thm:funext_wkfunext}, and it follows that the projection is an equivalence. We conclude by the 3-for-2 property of equivalences that $\seqin^\ast$ is an equivalence.
\end{proof}

\begin{lem}\label{lem:infjp_up}
Let $A$ be a type, and let $P$ be a proposition. Then the function
\begin{equation*}
\blank\circ \seqin_0: (A^{\ast\infty}\to P)\to (A\to P)
\end{equation*}
is an equivalence. 
\end{lem}

\begin{proof}
We have the commuting triangle
\begin{equation*}
\begin{tikzcd}[column sep=0]
& P^{A^{\ast\infty}} \arrow[dl,swap,"\seqin^\ast"] \arrow[dr,"\blank\circ\seqin_0"] & \phantom{\Big(\prd{n:\N}A^{\ast n} \to P\Big)} \\
\Big(\prd{n:\N}A^{\ast n} \to P\Big) \arrow[rr,swap,"\lam{h}h_0"] & & P^A.
\end{tikzcd}
\end{equation*}
Therefore it suffices to show that the bottom map is an equivalence. Since this is a map between propositions, it suffices to construct a map in the converse direction. Let $f:A\to P$. We will construct a term of type
\begin{equation*}
\prd{n:\N}A^{\ast n} \to P
\end{equation*}
by induction on $n:\N$. The base case is trivial. Given a map $g:A^{\ast n}\to P$, we obtain a map $g:A^{\ast(n+1)}\to P$ by \cref{lem:extend_join_prop}.
\end{proof}

\begin{lem}\label{lem:seqcolim_contr}
Consider a type sequence
\begin{equation*}
\begin{tikzcd}
A_0 \arrow[r,"f_0"] & A_1 \arrow[r,"f_1"] & A_2 \arrow[r,"f_2"] & \cdots
\end{tikzcd}
\end{equation*}
and suppose that each $A_n$ is equipped with a base point $a_n:A_n$, and each $f_n$ is equipped with a homotopy $H_n:\mathsf{const}_{a_{n+1}}\htpy f_n$. Then the sequential colimit $A_\infty$ is contractible.\qed
\end{lem}

\begin{lem}\label{lem:isprop_infjp}
The type $A^{\ast\infty}$ is a proposition for any type $A$.
\end{lem}

\begin{proof}
By \cref{lem:prop_char} it suffices to show that $A^{\ast\infty}\to \iscontr(A^{\ast\infty})$, and by \cref{lem:infjp_up} it suffices to show that
\begin{equation*}
A\to \iscontr(A^{\ast\infty}),
\end{equation*}
because $\iscontr(A^{\ast\infty})$ is a proposition. 

Let $x:A$. To see that $A^{\ast\infty}$ is contractible it suffices by \cref{lem:seqcolim_contr} to show that $\inr:A^{\ast n}\to A^{\ast(n+1)}$ is homotopic to the constant function $\const_{\inl(x)}$. However, we get a homotopy $\const_{\inl(x)}\htpy \inr$ immediately from the path constructor $\glue$.  
\end{proof}

\begin{thm}
For any type $A:\UU$ there is a proposition $\brck{A}:\UU$ that comes equipped with a map $\eta:A\to \brck{A}$, and satisfies the universal property of propositional truncation.
\end{thm}

\begin{proof}
Let $A$ be a type. Then we define $\brck{A}\defeq A^{\ast\infty}$, and we define $\eta\defeq \seqin_0:A\to A^{\ast\infty}$. Then $\brck{A}$ is a proposition by \cref{lem:isprop_infjp}, and $\eta:A\to \brck{A}$ satisfies the universal property of propositional truncation by \cref{lem:infjp_up}.
\end{proof}

\subsection{Stage two: constructing the image of a map}\label{sec:join_stage2}
Following Definition 7.6.3 of \cite{hottbook}, we recall that the image of a map $f:A\to X$ can be defined using the propositional truncation:
\begin{defn}
For any map $f:A\to X$ we define the \define{image}\index{image|textbf} of $f$ to be the type
\begin{equation*}
\im(f) \defeq \sm{x:X}\brck{\fib{f}{x}}
\end{equation*}
and we define the \define{image inclusion} to be the projection $\proj 1 :\im(f)\to X$. 
\end{defn}
However, the construction of the fiberwise join in \cref{defn:fib_join} suggests that we can also define the image of $f$ as the infinite join power $f^{\ast\infty}$, where we repeatedly take the fiberwise join of $f$ with itself. Our reason for defining the image in this way is twofold: 
\begin{itemize}
\item We use this construction to show that the image of a map $f:A\to B$ from an essentially small type $A$ into a locally small type $B$ is again essentially small.
\item Some interesting types, such as the real and complex projective spaces, appear in specific instances of this construction.
\end{itemize}

\begin{lem}
Consider a map $f:A\to X$, an embedding $m:U\to X$, and $h:\mathrm{hom}_X(f,m)$. Then the map
\begin{equation*}
\mathrm{hom}_X(\join{f}{g},m)\to \mathrm{hom}_X(g,m)
\end{equation*}
is an equivalence for any $g:B\to X$.
\end{lem}

\begin{proof}
Note that both types are propositions, so any equivalence can be used to prove the claim. Thus, we simply calculate
\begin{align*}
\mathrm{hom}_X(\join{f}{g},m) & \eqvsym \prd{x:X}\fib{\join{f}{g}}{x}\to \fib{m}{x} \\
& \eqvsym \prd{x:X}\join{\fib{f}{x}}{\fib{g}{x}}\to\fib{m}{x} \\
& \eqvsym \prd{x:X}\fib{g}{x}\to\fib{m}{x} \\
& \eqvsym \mathrm{hom}_X(g,m).
\end{align*}
The first equivalence holds by \cref{cor:fib_triangle}; the second equivalence holds by \cref{defn:join-fiber}; the third equivalence holds by \cref{lem:extend_join_prop}; the last equivalence again holds by \cref{cor:fib_triangle}.
\end{proof}

For the construction of the image of $f:A\to X$ we observe that if we are given an embedding $m:U\to X$ and a map $(i,I):\mathrm{hom}_X(f,m)$, then $(i,I)$ extends uniquely along $\inr:A\to \join[X]{A}{A}$ to a map $\mathrm{hom}_X(\join{f}{f},m)$. This extension again extends uniquely along $\inr:\join[X]{A}{A}\to \join[X]{A}{(\join[X]{A}{A})}$ to a map $\mathrm{hom}_X(\join{f}{(\join{f}{f})},m)$ and so on, resulting in a diagram of the form
\begin{equation*}
\begin{tikzcd}
A \arrow[dr] \arrow[r,"\inr"] & \join[X]{A}{A} \arrow[d,densely dotted] \arrow[r,"\inr"] & \join[X]{A}{(\join[X]{A}{A})} \arrow[dl,densely dotted] \arrow[r,"\inr"] & \cdots \arrow[dll,densely dotted,bend left=10] \\
& U
\end{tikzcd}
\end{equation*}

\begin{defn}
Suppose $f:A\to X$ is a map. Then we define the \define{fiberwise join powers} 
\begin{equation*}
f^{\ast n}\defeq A_X^{\ast n}\to X.
\end{equation*}
\end{defn}

\begin{proof}[Construction]
Note that the operation $(B,g)\mapsto (\join[X]{A}{B},\join{f}{g})$ defines an endomorphism on the type
\begin{equation*}
\sm{B:\UU}B\to X.
\end{equation*}
We also have $(\emptyt,\ind{\emptyt})$ and $(A,f)$ of this type. For $n\geq 1$ we define
\begin{align*}
A_X^{\ast (n+1)} & \defeq \join[X]{A}{A_X^{\ast n}} \\
f^{\ast (n+1)} & \defeq \join{f}{f^{\ast n}}.\qedhere
\end{align*}
\end{proof}

\begin{defn}
We define $A_X^{\ast\infty}$ to be the sequential colimit of the type sequence
\begin{equation*}
\begin{tikzcd}
A_X^{\ast 0} \arrow[r] & A_X^{\ast 1} \arrow[r,"\inr"] & A_X^{\ast 2} \arrow[r,"\inr"] & \cdots.
\end{tikzcd}
\end{equation*}
Since we have a cocone
\begin{equation*}
\begin{tikzcd}
A_X^{\ast 0} \arrow[r] \arrow[dr,swap,"f^{\ast 0}" near start] & A_X^{\ast 1} \arrow[r,"\inr"] \arrow[d,swap,"f^{\ast 1}" near start] & A_X^{\ast 2} \arrow[r,"\inr"] \arrow[dl,swap,"f^{\ast 2}" xshift=1ex] & \cdots \arrow[dll,bend left=10] \\
& X
\end{tikzcd}
\end{equation*}
we also obtain a map $f^{\ast\infty}:A_X^{\ast\infty}\to X$ by the universal property of $A_X^{\ast\infty}$. 
\end{defn}

\begin{lem}\label{lem:finfjp_up}
Let $f:A\to X$ be a map, and let $m:U\to X$ be an embedding. Then the function
\begin{equation*}
\blank\circ \seqin_0: \mathrm{hom}_X(f^{\ast\infty},m)\to \mathrm{hom}_X(f,m)
\end{equation*}
is an equivalence. 
\end{lem}

\begin{thm}\label{thm:image}
For any map $f:A\to X$, the map $f^{\ast\infty}:A_X^{\ast\infty}\to X$ is an embedding that satisfies the universal property of the image inclusion of $f$.
\end{thm}

\subsection{Stage three: establishing the smallness of the image}
\begin{lem}
Consider a commuting square
\begin{equation*}
\begin{tikzcd}
A \arrow[r] \arrow[d] & B \arrow[d] \\
C \arrow[r] & D.
\end{tikzcd}
\end{equation*}
\begin{enumerate}
\item If the square is cartesian, $B$ and $C$ are essentially small, and $D$ is locally small, then $A$ is essentially small.
\item If the square is cocartesian, and $A$, $B$, and $C$ are essentially small, then $D$ is essentially small. 
\end{enumerate}
\end{lem}

\begin{cor}
Suppose $f:A\to X$ and $g:B\to X$ are maps from essentially small types $A$ and $B$, respectively, to a locally small type $X$. Then $A\times_X B$ is again essentially small. 
\end{cor}

\begin{lem}
Consider a type sequence
\begin{equation*}
\begin{tikzcd}
A_0 \arrow[r,"f_0"] & A_1 \arrow[r,"f_1"] & A_2 \arrow[r,"f_2"] & \cdots
\end{tikzcd}
\end{equation*}
where each $A_n$ is essentially small. Then its sequential colimit is again essentially small. 
\end{lem}

\begin{thm}\label{thm:image_small}
For any map $f:A\to X$ from a small type $A$ into a locally small type $X$, the image $\im(f)$ is an essentially small type.
\end{thm}

Recall that in set theory, the replacement axiom asserts that for any family of sets $\{X_i\}_{i\in I}$ indexed by a set $I$, there is a set $X[I]$ consisting of precisely those sets $x$ for which there exists an $i\in I$ such that $x\in X_i$. In other words: the image of a set-indexed family of sets is again a set. Without the replacement axiom, $X[I]$ would be a class. In the following corollary we establish a type-theoretic analogue of the replacement axiom: the image of a family of small types indexed by a small type is again (essentially) small.

\begin{cor}\label{cor:im_small}
For any small type family $B:A\to\UU$, where $A$ is small, the image $\im(B)$ is essentially small. We call $\im(B)$ the \define{univalent completion} of $B$. 
\end{cor}

\section{Set quotients}\label{sec:set-quotients}

\subsection{Sets in homotopy type theory}
\begin{defn}
A type $A$ is said to be a \define{set} if there is a term of type
\begin{equation*}
\isset(A)\defeq \prd{x,y:A}\isprop(\id{x}{y}).
\end{equation*}
\end{defn}

\begin{lem}\label{lem:prop_to_id}
Let $A$ be a type, and let $(R,\rho):\mathsf{rRel}(A)$ be a reflexive relation on $A$ such that $R(x,y)$ is a proposition for each $x,y:A$. Then any fiberwise map
\begin{equation*}
\prd{x,y:A}R(x,y)\to (\id{x}{y})
\end{equation*}
is a fiberwise equivalence. Consequently, if there is such a fiberwise map, then $A$ is a set.
\end{lem}

\begin{proof}
Let $f:\prd{x,y:A}R(x,y)\to(\id{x}{y})$. 
Since $R$ is assumed to be reflexive, we also have a fiberwise transformation
\begin{equation*}
\ind{x{=}}(\rho(x)):\prd{y:A}(\id{x}{y})\to R(x,y).
\end{equation*}
Since each $R(x,y)$ is assumed to be a proposition, it therefore follows that each $R(x,y)$ is a retract of $\id{x}{y}$. We conclude by \autoref{cor:id_fundamental_retr} that for each $x,y:A$, the map $f(x,y):R(x,y)\to(\id{x}{y})$ must be an equivalence.

Now it also follows that $A$ is a set, since its identity types are (equivalent to) propositions.
\end{proof}

\begin{eg}
One can apply \cref{lem:prop_to_id} using the \define{observational equality} $\mathrm{Eq}_\N:\N\to (\N\to\UU)$ given by
\begin{align*}
\mathrm{Eq}_\N(0,0) & \defeq \unit & \mathrm{Eq}_\N(\mathsf{succ}(n),0) & \defeq \emptyt \\
\mathrm{Eq}_\N(0,\mathsf{succ}(m)) & \defeq \emptyt & \mathrm{Eq}_\N(\mathsf{succ}(n),\mathsf{succ}(m)) & \defeq \mathrm{Eq}_\N(n,m)
\end{align*}
to show that $\N$ is a set. A routine induction argument shows that $\mathrm{Eq}_\N$ implies identity.
\end{eg}

\subsection{Equivalence relations}

\begin{defn}\label{defn:eq_rel}
Let $R:A\to (A\to\prop)$ be a binary relation valued in the propositions. We say that $R$ is an \define{($0$-)equivalence relation}\index{equivalence relation|textbf}\index{0-equivalence relation|see {equivalence relation}} if $R$ comes equipped with
\begin{align*}
\rho & : \prd{x:A}R(x,x) \\
\sigma & : \prd{x,y:A} R(x,y)\to R(y,x) \\
\tau & : \prd{x,y,z:A} R(x,y)\to (R(y,z)\to R(x,z)).
\end{align*}
Given an equivalence relation $R:A\to (A\to\prop)$, the \define{equivalence class}\index{equivalence class|textbf} $[x]_R:A\to\prop$ of $x:A$ is defined to be
\begin{equation*}
[x]_R\defeq R(x).
\end{equation*}
\end{defn}

\begin{defn}
Let $R:A\to (A\to\prop)$ be a $0$-equivalence relation. 
We define for any $x,y:A$ a map\index{class_eq@{$\mathsf{eq\usc{}class}$}|textbf}
\begin{equation*}
\mathsf{eq\usc{}class}:R(x,y)\to ([x]_R=[y]_R).
\end{equation*}
\end{defn}

\begin{proof}[Construction.]
Let $r:R(x,y)$. By function extensionality, the identity type $R(x)=R(y)$ is equivalent to the type
\begin{equation*}
\prd{z:A}R(x,z)=R(y,z).
\end{equation*}
Let $z:A$. By the univalence axiom, the type $R(x,z)=R(y,z)$ is equivalent to the type
\begin{equation*}
\eqv{R(x,z)}{R(y,z)}.
\end{equation*}
We have the map $\tau_{y,x,z}(\sigma(r)):R(x,z)\to R(y,z)$. Since this is a map between propositions, we only have to construct a map in the converse direction to show that it is an equivalence. The map in the converse direction is just $\tau_{x,y,z}(r):R(y,z)\to R(x,z)$. 
\end{proof}

\begin{prp}\label{thm:equivalence_classes}
Let $R:A\to (A\to\prop)$ be a $0$-equivalence relation. 
Then for any $x,y:A$ the map
\begin{equation*}
\mathsf{eq\usc{}class} : R(x,y)\to ([x]_R=[y]_R)
\end{equation*}
is an equivalence.
\end{prp}

\begin{proof}
By the 3-for-2 property of equivalences, it suffices to show that the map
\begin{equation*}
\lam{r}{z}\tau_{y,x,z}(\sigma(r)) : R(x,y)\to \prd{z:A} \eqv{R(x,z)}{R(y,z)}
\end{equation*}
is an equivalence. Since this is a map between propositions, it suffices to construct a map of type
\begin{equation*}
\Big(\prd{z:A} \eqv{R(x,z)}{R(y,z)}\Big)\to R(x,y).
\end{equation*}
This map is simply $\lam{f} \sigma_{y,x}(f_x(\rho(x)))$. 
\end{proof}

\subsection{The universal property of set quotients}

\begin{defn}
Let $R:A\to (A\to \prop)$ be an equivalence relation\index{equivalence relation|textit}, for $A:\UU$, and consider a map $q:A\to B$ where the type $B$ is a set, for which we have
\begin{equation*}
\prd{x,y:A}R(x,y)\to q(x)=q(y).
\end{equation*}
We will define a map
\begin{equation*}
\quotientrestr:(B\to X) \to \Big(\sm{f:A\to X}\prd{x,y:A}R(x,y)\to (f(x)=f(y))\Big).
\end{equation*}
\end{defn}

\begin{proof}[Construction]
Let $h:B\to X$. Then we have $h\circ q : A\to X$, so it remains to show that
\begin{equation*}
\prd{x,y:A}R(x,y)\to (h(q(x))=h(q(y)))
\end{equation*}
Consider $x,y:A$ which are related by $R$. Then we have an identification $p:q(x)=q(y)$, so it follows that $\ap{h}{p}:h(q(x))=h(q(y))$.  
\end{proof}

\begin{defn}
Let $R:A\to (A\to \prop)$ be an equivalence relation\index{equivalence relation|textit}, for $A:\UU$, and consider a map $q:A\to B$ satisfying
\begin{equation*}
\prd{x,y:A}R(x,y)\to q(x)=q(y),
\end{equation*}
where the type $B$ is a set. We say that the map $q:A\to B$ satisfies the universal property of the \define{set quotient}\index{set quotient}\index{universal property!of set quotients|textit} $A/R$ if for any set $X$ the map
\begin{equation*}
\quotientrestr : (B\to X) \to \Big(\sm{f:A\to X}\prd{x,y:A}R(x,y)\to (f(x)=f(y))\Big)
\end{equation*}
is an equivalence.
\end{defn}

\begin{lem}
Let $R:A\to (A\to \prop)$ be an equivalence relation\index{equivalence relation|textit}, for $A:\UU$, and consider a commuting triangle
\begin{equation*}
\begin{tikzcd}[column sep=tiny]
A \arrow[rr,"q"] \arrow[dr,swap,"R"] & & U \arrow[dl,"m"] \\
& \prop^A
\end{tikzcd}
\end{equation*}
with $H:R\htpy m\circ q$, where $m$ is an embedding. Then we have
\begin{equation*}
\prd{x,y:A}R(x,y)\to (q(x)=q(y)).
\end{equation*}
\end{lem}

\begin{prp}\label{thm:quotient_up}
Let $R:A\to (A\to \prop)$ be an equivalence relation\index{equivalence relation|textit}, for $A:\UU$, and consider a commuting triangle
\begin{equation*}
\begin{tikzcd}[column sep=tiny]
A \arrow[rr,"q"] \arrow[dr,swap,"R"] & & U \arrow[dl,"m"] \\
& \prop^A
\end{tikzcd}
\end{equation*}
with $H:R\htpy m\circ q$, where $m$ is an embedding. Then the following are equivalent:
\begin{enumerate}
\item The embedding $m:U\to \prop^A$ satisfies the universal property of the image of $R$.
\item The map $q:A\to U$ satisfies the universal property of the set quotient $A/R$.
\end{enumerate}
\end{prp}

\begin{proof}
Suppose $m:U\to \prop^A$ satisfies the universal property of the image of $R$. Then it follows by \cref{thm:surjective}\marginnote{Need lemma that states that universal property of image inclusion iff the map into it is surjective} that the map $q:A\to U$ is surjective. Our goal is to prove that $U$ satisfies the universal property of the set quotient $A/R$. \marginnote{Proof incomplete}
\end{proof}

\begin{rmk}
\cref{thm:quotient_up} suggests that we can define the quotient of an equivalence relation $R$ on a type $A$ as the image of a map. However, the type $\prop^A$ of which the quotient is a subtype is not a small type, even if $A$ is a small type.
Therefore it is not clear that the quotient $A/R$ is essentially small\index{essentially small}, as it should be. Luckily, our construction of the image of a map allows us to show that the image is indeed essentially small, using the fact that $\prop^A$ is locally small\index{locally small}.
\end{rmk}

\section{The Rezk completion}

We recall from \cite{AhrensKapulkinShulman} that a pre-category $\mathcal{A}$ consists of a type $A$ of objects, a $\set$-valued binary relation $\mathsf{hom}:A\to A\to\set$ of morphisms, equipped with identity morphisms
\begin{equation*}
\mathsf{id} : \prd{x:A} \mathsf{hom}(x,x)
\end{equation*}
and a composition operation
\begin{equation*}
\mathsf{comp} : \prd{x,y,z:A} \mathsf{hom}(y,z)\to \mathsf{hom}(x,y)\to \mathsf{hom}(x,z),
\end{equation*}
which is associative and satisfies the unit laws. The isomorphisms $x \cong y$ from $x$ to $y$ in a pre-category are defined in the expected way, and the pre-category $\mathcal{A}$ is said to be \define{Rezk-complete} if the canonical map
\begin{equation*}
\prd{x,y:A} (x=y)\to (x\cong y)
\end{equation*}
is an equivalence. Rezk-complete pre-categories are also called \define{categories}. By the univalence axiom, the pre-category $\mathbf{Set}$ of small sets is Rezk-complete. 

A pre-category is said to be \define{(essentially) small} if its types of objects and morphisms are (essentially) small, and a pre-category is said to be \define{locally small} if its hom-types are essentially small. Thus, the category $\mathbf{Set}$ is locally small. 

\begin{defn}
Let $\mathcal{A}$ be a pre-category. A functor $\mathcal{F}:\mathcal{A}\to\mathcal{B}$ into a (Rezk-complete) category $\mathcal{B}$ is said to be a \define{Rezk-completion} if the pre-composition map
\begin{equation*}
\mathsf{Functor}(\mathcal{B},\mathcal{C})\to\mathsf{Functor}(\mathcal{A},\mathcal{C})
\end{equation*}
is an equivalence, for every (Rezk-complete) category $\mathcal{C}$. 
\end{defn}

\begin{thm}\label{cor:rezkcompletion}
The Rezk completion $\hat{A}$ of any small pre-category $A$ can be constructed in any 
univalent universe that is closed under graph quotients,
and $\hat{A}$ is again a small category. 
\end{thm}

\begin{proof}
In the first proof of Theorem 9.9.5 of \cite{hottbook}, the Rezk completion of
a precategory $A$ is constructed as the image of the action on objects of the
Yoneda embedding $\mathbf{y}:A\to\mathbf{Set}^{\op{A}}$.

The hom-set $\mathbf{Set}^{\op{A}}(F,G)$ is the type of natural transformations
from $F$ to $G$. It is clear from Definition 9.9.2 of \cite{hottbook}, that the
type $\mathbf{Set}^{\op{A}}(F,G)$ is in $\UU$ for any two presheaves $F$ and $G$
on $A$. In particular, the type $F\cong G$ of isomorphisms from $F$ to $G$
is small for any two presheaves on $A$.

Since $\mathbf{Set}$ is a category, it follows from Theorem 9.2.5 of
\cite{hottbook} that the presheaf pre-category $\mathbf{Set}^{\op{A}}$ is a category.
Since the type of isomorphisms between any two objects is
small, it follows that the type of objects of
$\mathbf{Set}^{\op{A}}$ is locally small. 

Hence we can use \autoref{thm:modified-join} to construct the image of the
action on objects of the Yoneda-embedding. The image constructed in this way
is of course equivalent to the type $\hat{A}_0$ defined in the first proof of
Theorem 9.9.5. Hence the arguments presented in the rest of that proof apply
as well to our construction of the image. We therefore conclude that the Rezk completion
of any small precategory is a small category.
\end{proof}

\begin{eg}\label{eg:emspaces}
Any group $G$ can be seen as a pre-category $\mathcal{G}$ with a single object and the group $G$ as morphisms. The Rezk-completion of $\mathcal{G}$ is the Eilenberg-Mac Lane space $K(G,1)$. Thus we conclude that any univalent universe closed under pushouts contains the Eilenberg-Mac Lane spaces $K(G,1)$ for any small group $G$. 
\end{eg}
