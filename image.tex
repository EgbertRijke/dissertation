\chapter{Homotopy images}

We have observed in ???? that the reflexive coequalier of the pre-kernel of a map $f:A\to X$ is the fiberwise join $\join[X]{A}{A}$, i.e.~we have a reflexive coequalizer diagram
\begin{equation*}
\begin{tikzcd}
A\times A \arrow[r,yshift=1ex] \arrow[r,yshift=-1ex] & A \arrow[l] \arrow[r] & \join{A}{A}.
\end{tikzcd}
\end{equation*}
Furthermore, we have observed in ???? that the geometric realization of the 2-pre-kernel of a map $f:A\to X$ is the triple fiberwise join $\join[X]{A}{\join[X]{A}{A}}$, i.e.~we have a colimiting 
\begin{equation*}
\begin{tikzcd}
A\times A \times A \arrow[r,yshift=2ex] \arrow[r] \arrow[r,yshift=-2ex] & A \times A \arrow[l,yshift=1ex] \arrow[l,yshift=-1ex] \arrow[r,yshift=1ex] \arrow[r,yshift=-1ex] & A \arrow[l] \arrow[r] & \join{A}{\join{A}{A}}.
\end{tikzcd}
\end{equation*}
The propositional truncation $\brck{A}$ of $A$ is approximated by the type sequence
\begin{equation*}
\begin{tikzcd}
A \arrow[r] & \join{A}{A} \arrow[r] & \join{A}{\join{A}{A}} \arrow[r] & \cdots
\end{tikzcd}
\end{equation*}

The challenge here is to first give a specification of what we mean by a diagram of the form
\begin{equation*}
\begin{tikzcd}
A_2 \arrow[r,yshift=2ex] \arrow[r] \arrow[r,yshift=-2ex] & A_1 \arrow[l,yshift=1ex] \arrow[l,yshift=-1ex] \arrow[r,yshift=1ex] \arrow[r,yshift=-1ex] & A_0. \arrow[l]
\end{tikzcd}
\end{equation*}
Of course, we do so in homotopy type theory, so the task is to write down the correct \emph{homotopy type} of all such diagrams. Then we introduce their homotopy colimits, and show that in the case of the join construction this colimit coincides with the triple join.

Set-quotients have been first constructed in \cite{UniMath2015,UniMath} using propositional resizing, and as higher inductive types in \cite{UFP} in the case where $A$ is assumed to be a set.

\section{The join construction}

\begin{defn}\label{defn:image_up}
Consider a commuting triangle
\begin{equation*}
\begin{tikzcd}[column sep=small]
A \arrow[rr,"i"] \arrow[dr,swap,"f"] & & B \arrow[dl,"m"] \\
& X
\end{tikzcd}
\end{equation*}
with $I:f\htpy m\circ i$, and where $m$ is an embedding\index{embedding}.
We say that $m$ has the \define{universal property of the image of $f$}\index{universal property!of the image|textit} if the map
\begin{equation*}
(i,I)^\ast : \mathrm{hom}_X(m,m')\to\mathrm{hom}_X(f,m')
\end{equation*}
defined by $(i,I)^\ast(h,H)\defeq (h\circ i,\ct{I}{(i\cdot H)})$,
is an equivalence for every embedding $m':B'\to X$. 
\end{defn}

\begin{lem}
For any $f:A\to X$ and any embedding\index{embedding} $m:B\to X$, the type $\mathrm{hom}_X(f,m)$ is a proposition.
\end{lem}

\begin{proof}
From \cref{cor:fib_triangle} we obtain that the type $\mathrm{hom}_X(f,m)$ is equivalent to the type
\begin{equation*}
\prd{x:X}\fib{f}{x}\to\fib{m}{x},
\end{equation*}
so it suffices to show that this is a proposition. 
Recall from \cref{thm:prop_emb} that a map is an embedding if and only if its fibers are propositions.
Thus we see that the type $\prd{x:X}\fib{f}{x}\to\fib{m}{x}$ is a product of propositions, so it is a proposition by \cref{thm:trunc_pi}.
\end{proof}

\begin{cor}\label{cor:image_up}
Consider a commuting triangle
\begin{equation*}
\begin{tikzcd}[column sep=small]
A \arrow[rr,"i"] \arrow[dr,swap,"f"] & & B \arrow[dl,"m"] \\
& X
\end{tikzcd}
\end{equation*}
with $I:f\htpy m\circ i$, and where $m$ is an embedding. Then $m$ satisfies the universal property of the image of $f$ if and only if the implication
\begin{equation*}
\mathrm{hom}_X(f,m')\to\mathrm{hom}_X(m,m')
\end{equation*}
holds for every embedding $m':B'\to X$. 
\end{cor}

Note that embeddings into the unit type are just propositions. To see this, note that
\begin{align*}
\sm{A:\UU}{f:A\to\unit}\isemb(f)
& \eqvsym \sm{A:\UU}\isemb(\const_\ttt) \\
& \eqvsym \sm{A:\UU}\prd{x:\unit}\isprop(\fib{\const_\ttt}{x}) \\
& \eqvsym \sm{A:\UU}\isprop(\fib{\const_\ttt}{\ttt}) \\
& \eqvsym \sm{A:\UU}\isprop(A).
\end{align*}
Therefore, the universal property of the image of the map $A\to\unit$ is a proposition $P$ satisfying the universal property of the propositional truncation:

\begin{defn}
Let $A$ be a type, and let $P$ be a proposition that comes equipped with a map $\eta:A\to P$. We say that $\eta:A\to P$ satisfies the \define{universal property of propositional truncation}\index{univeral property!of propositional truncation|textit} if for every proposition $Q$, the precomposition map
\begin{equation*}
\blank\circ\eta:(P\to Q)\to (A\to Q)
\end{equation*}
is an equivalence.
\end{defn}

\subsection{Stage one: constructing the propositional truncation}

\begin{lem}\label{lem:extend_join_prop}
Suppose $f:A\to P$, where $A$ is any type, and $P$ is a proposition.
Then the map
\begin{equation*}
(\join{A}{B}\to P)\to (B\to P)
\end{equation*}
given by $h\mapsto h\circ \inr$ is an equivalence, for any type $B$.
\end{lem}

\begin{proof}
Since both types are propositions by \cref{thm:trunc_pi} it suffices to construct a map
\begin{equation*}
(B\to P)\to (\join{A}{B}\to P).
\end{equation*}
Let $g:B\to P$. Then the square
\begin{equation*}
\begin{tikzcd}
A\times B \arrow[r,"\proj 2"] \arrow[d,swap,"\proj 1"] & B \arrow[d,"g"] \\
A \arrow[r,swap,"f"] & P
\end{tikzcd}
\end{equation*}
commutes since $P$ is a proposition. Therefore we obtain a map $\join{A}{B}\to P$ by the universal property of the join.
\end{proof}

The idea of the construction of the propositional truncation is that if we are given a map $f:A\to P$, where $P$ is a proposition, then it extends uniquely along $\inr:A\to \join{A}{A}$ to a map $\join{A}{A}\to P$. This extension again extends uniquely along $\inr:\join{A}{A}\to \join{A}{(\join{A}{A})}$ to a map $\join{A}{(\join{A}{A})}\to P$ and so on, resulting in a diagram of the form
\begin{equation*}
\begin{tikzcd}
A \arrow[dr] \arrow[r,"\inr"] & \join{A}{A} \arrow[d,densely dotted] \arrow[r,"\inr"] & \join{A}{(\join{A}{A})} \arrow[dl,densely dotted] \arrow[r,"\inr"] & \cdots \arrow[dll,densely dotted,bend left=10] \\
& P
\end{tikzcd}
\end{equation*}

\begin{defn}
The \define{join powers} $A^{\ast n}$ of a type $X$ are defined by
\begin{align*}
A^{\ast 0} & \defeq \emptyt \\
A^{\ast 1} & \defeq A \\
A^{\ast (n+1)} & \defeq \join{A}{A^{\ast n}}.
\end{align*}
Furthermore, we define $A^{\ast\infty}$ to be the sequential colimit of the type sequence
\begin{equation*}
\begin{tikzcd}
A^{\ast 0} \arrow[r] & A^{\ast 1} \arrow[r,"\inr"] & A^{\ast 2} \arrow[r,"\inr"] & \cdots.
\end{tikzcd}
\end{equation*}
\end{defn}

Our goal is now to show that $A^{\ast\infty}$ is a proposition and satisfies the universal property of the propositional truncation.

\begin{lem}
Consider a type sequence
\begin{equation*}
\begin{tikzcd}
A_0 \arrow[r,"f_0"] & A_1 \arrow[r,"f_1"] & A_2 \arrow[r,"f_2"] & \cdots
\end{tikzcd}
\end{equation*}
with sequential colimit $A_\infty$, and let $P$ be a proposition. Then the map
\begin{equation*}
\seqin^\ast: (A_\infty\to P)\to \Big(\prd{n:\N}A_n\to P\Big)
\end{equation*}
given by $h\mapsto \lam{n}(h\circ \seqin_n)$ is an equivalence. 
\end{lem}

\begin{proof}
By the universal property of sequential colimits established in \cref{thm:sequential_up} we obtain that $\coconemap$ is an equivalence. Note that we have a commuting triangle
\begin{equation*}
\begin{tikzcd}
& P^{A_\infty} \arrow[dl,swap,"\coconemap"] \arrow[dr,"\seqin^\ast"] \\
\cocone(P) \arrow[rr,swap,"\proj 1"] & & \Big(\prd{n:\N}A_n\to P\Big).
\end{tikzcd}
\end{equation*}
Note that for any $g:\prd{n:\N}A_n\to P$ the type 
\begin{equation*}
\prd{n:\N} g_n\htpy g_{n+1}\circ f_n
\end{equation*}
is a product of contractible types, since $P$ is a proposition. Therefore it is contractible by \cref{thm:funext_wkfunext}, and it follows by \cref{ex:proj_fiber} that the projection is an equivalence. We conclude by the 3-for-2 property of equivalences (\cref{ex:3_for_2}) that $\seqin^\ast$ is an equivalence.
\end{proof}

\begin{lem}\label{lem:infjp_up}
Let $A$ be a type, and let $P$ be a proposition. Then the function
\begin{equation*}
\blank\circ \seqin_0: (A^{\ast\infty}\to P)\to (A\to P)
\end{equation*}
is an equivalence. 
\end{lem}

\begin{proof}
We have the commuting triangle
\begin{equation*}
\begin{tikzcd}
&[-3em] P^{A^{\ast\infty}} \arrow[dl,swap,"\seqin^\ast"] \arrow[dr,"\blank\circ\seqin_0"] \\
\Big(\prd{n:\N}A^{\ast n} \to P\Big) \arrow[rr,swap,"\lam{h}h_0"] & & P^A.
\end{tikzcd}
\end{equation*}
Therefore it suffices to show that the bottom map is an equivalence. Since this is a map between propositions, it suffices to construct a map in the converse direction. Let $f:A\to P$. We will construct a term of type
\begin{equation*}
\prd{n:\N}A^{\ast n} \to P
\end{equation*}
by induction on $n:\N$. The base case is trivial. Given a map $g:A^{\ast n}\to P$, we obtain a map $g:A^{\ast(n+1)}\to P$ by \cref{lem:extend_join_prop}.
\end{proof}

\begin{lem}\label{lem:isprop_infjp}
The type $A^{\ast\infty}$ is a proposition for any type $A$.
\end{lem}

\begin{proof}
By \cref{cor:contr_prop} it suffices to show that $A^{\ast\infty}\to \iscontr(A^{\ast\infty})$, and by \cref{lem:infjp_up} it suffices to show that
\begin{equation*}
A\to \iscontr(A^{\ast\infty}),
\end{equation*}
because $\iscontr(A^{\ast\infty})$ is a proposition by \cref{ex:isprop_istrunc}. 

Let $x:A$. To see that $A^{\ast\infty}$ is contractible it suffices by \cref{ex:seqcolim_contr} to show that $\inr:A^{\ast n}\to A^{\ast(n+1)}$ is homotopic to the constant function $\const_{\inl(x)}$. However, we get a homotopy $\const_{\inl(x)}\htpy \inr$ immediately from the path constructor $\glue$.  
\end{proof}

\begin{thm}
For any type $A$ there is a type $\brck{A}$ that comes equipped with a map $\eta:A\to \brck{A}$, and satisfies the universal property of propositional truncation.
\end{thm}

\begin{proof}
Let $A$ be a type. Then we define $\brck{A}\defeq A^{\ast\infty}$, and we define $\eta\defeq \seqin_0:A\to A^{\ast\infty}$. Then $\brck{A}$ is a proposition by \cref{lem:isprop_infjp}, and $\eta:A\to \brck{A}$ satisfies the universal property of propositional truncation by \cref{lem:infjp_up}.
\end{proof}

\subsection{Stage two: constructing the image of a map}\label{sec:join_stage2}
The image of a map $f:A\to X$ can be defined using the propositional truncation:
\begin{defn}
For any map $f:A\to X$ we define the \define{image}\index{image|textbf} of $f$ to be the type
\begin{equation*}
\im(f) \defeq \sm{x:X}\brck{\fib{f}{x}}
\end{equation*}
and we define the \define{image inclusion} to be the projection $\proj 1 :\im(f)\to X$. 
\end{defn}
However, the construction of the fiberwise join in \cref{ex:fib_join} suggests that we can also define the image of $f$ as the infinite join power $f^{\ast\infty}$, where we repeatedly take the fiberwise join of $f$ with itself. Our reason for defining the image in this way is twofold: 
\begin{itemize}
\item We use this construction to show that the image of a map $f:A\to B$ from an essentially small type $A$ into a locally small type $B$ is again essentially small.
\item Many interesting types, such as the real and complex projective spaces, appear in specific instances of this construction.
\end{itemize}

\begin{lem}
Consider a map $f:A\to X$, an embedding $m:U\to X$, and $h:\mathrm{hom}_X(f,m)$. Then the map
\begin{equation*}
\mathrm{hom}_X(\join{f}{g},m)\to \mathrm{hom}_X(g,m)
\end{equation*}
is an equivalence for any $g:B\to X$.
\end{lem}

\begin{proof}
Note that both types are propositions, so any equivalence can be used to prove the claim. Thus, we simply calculate
\begin{align*}
\mathrm{hom}_X(\join{f}{g},m) & \eqvsym \prd{x:X}\fib{\join{f}{g}}{x}\to \fib{m}{x} \\
& \eqvsym \prd{x:X}\join{\fib{f}{x}}{\fib{g}{x}}\to\fib{m}{x} \\
& \eqvsym \prd{x:X}\fib{g}{x}\to\fib{m}{x} \\
& \eqvsym \mathrm{hom}_X(g,m).
\end{align*}
The first equivalence holds by \cref{ex:triangle_fib}; the second equivalence holds by \cref{ex:fib_join}, also using \cref{ex:equiv_precomp,lem:postcomp_equiv} where we established that that pre- and postcomposing by an equivalence is an equivalence; the third equivalence holds by \cref{lem:extend_join_prop,lem:postcomp_equiv}; the last equivalence again holds by \cref{ex:triangle_fib}.
\end{proof}

For the construction of the image of $f:A\to X$ we observe that if we are given an embedding $m:U\to X$ and a map $(i,I):\mathrm{hom}_X(f,m)$, then $(i,I)$ extends uniquely along $\inr:A\to \join[X]{A}{A}$ to a map $\mathrm{hom}_X(\join{f}{f},m)$. This extension again extends uniquely along $\inr:\join[X]{A}{A}\to \join[X]{A}{(\join[X]{A}{A})}$ to a map $\mathrm{hom}_X(\join{f}{(\join{f}{f})},m)$ and so on, resulting in a diagram of the form
\begin{equation*}
\begin{tikzcd}
A \arrow[dr] \arrow[r,"\inr"] & \join[X]{A}{A} \arrow[d,densely dotted] \arrow[r,"\inr"] & \join[X]{A}{(\join[X]{A}{A})} \arrow[dl,densely dotted] \arrow[r,"\inr"] & \cdots \arrow[dll,densely dotted,bend left=10] \\
& U
\end{tikzcd}
\end{equation*}

\begin{defn}
Suppose $f:A\to X$ is a map. Then we define the \define{fiberwise join powers} 
\begin{equation*}
f^{\ast n}:A_X^{\ast n} X.
\end{equation*}
\end{defn}

\begin{proof}[Construction]
Note that the operation $(B,g)\mapsto (\join[X]{A}{B},\join{f}{g})$ defines an endomorphism on the type
\begin{equation*}
\sm{B:\UU}B\to X.
\end{equation*}
We also have $(\emptyt,\ind{\emptyt})$ and $(A,f)$ of this type. For $n\geq 1$ we define
\begin{align*}
A_X^{\ast (n+1)} & \defeq \join[X]{A}{A_X^{\ast n}} \\
f^{\ast (n+1)} & \defeq \join{f}{f^{\ast n}}.\qedhere
\end{align*}
\end{proof}

\begin{defn}
We define $A_X^{\ast\infty}$ to be the sequential colimit of the type sequence
\begin{equation*}
\begin{tikzcd}
A_X^{\ast 0} \arrow[r] & A_X^{\ast 1} \arrow[r,"\inr"] & A_X^{\ast 2} \arrow[r,"\inr"] & \cdots.
\end{tikzcd}
\end{equation*}
Since we have a cocone
\begin{equation*}
\begin{tikzcd}
A_X^{\ast 0} \arrow[r] \arrow[dr,swap,"f^{\ast 0}" near start] & A_X^{\ast 1} \arrow[r,"\inr"] \arrow[d,swap,"f^{\ast 1}" near start] & A_X^{\ast 2} \arrow[r,"\inr"] \arrow[dl,swap,"f^{\ast 2}" xshift=1ex] & \cdots \arrow[dll,bend left=10] \\
& X
\end{tikzcd}
\end{equation*}
we also obtain a map $f^{\ast\infty}:A_X^{\ast\infty}\to X$ by the universal property of $A_X^{\ast\infty}$. 
\end{defn}

\begin{lem}\label{lem:finfjp_up}
Let $f:A\to X$ be a map, and let $m:U\to X$ be an embedding. Then the function
\begin{equation*}
\blank\circ \seqin_0: \mathrm{hom}_X(f^{\ast\infty},m)\to \mathrm{hom}_X(f,m)
\end{equation*}
is an equivalence. 
\end{lem}

\begin{thm}\label{lem:isprop_infjp}
For any map $f:A\to X$, the map $f^{\ast\infty}:A_X^{\ast\infty}\to X$ is an embedding that satisfies the universal property of the image inclusion of $f$.
\end{thm}

\subsection{Stage three: establishing the smallness of the image}
\begin{lem}
Consider a commuting square
\begin{equation*}
\begin{tikzcd}
A \arrow[r] \arrow[d] & B \arrow[d] \\
C \arrow[r] & D.
\end{tikzcd}
\end{equation*}
\begin{enumerate}
\item If the square is cartesian, $B$ and $C$ are essentially small, and $D$ is locally small, then $A$ is essentially small.
\item If the square is cocartesian, and $A$, $B$, and $C$ are essentially small, then $D$ is essentially small. 
\end{enumerate}
\end{lem}

\begin{cor}
Suppose $f:A\to X$ and $g:B\to X$ are maps from essentially small types $A$ and $B$, respectively, to a locally small type $X$. Then $A\times_X B$ is again essentially small. 
\end{cor}

\begin{lem}
Consider a type sequence
\begin{equation*}
\begin{tikzcd}
A_0 \arrow[r,"f_0"] & A_1 \arrow[r,"f_1"] & A_2 \arrow[r,"f_2"] & \cdots
\end{tikzcd}
\end{equation*}
where each $A_n$ is essentially small. Then its sequential colimit is again essentially small. 
\end{lem}

\begin{thm}
For any map $f:A\to X$ from a small type $A$ into a locally small type $X$, the image $\im(f)$ is an essentially small type.
\end{thm}

Recall that in set theory, the replacement axiom asserts that for any family of sets $\{X_i\}_{i\in I}$ indexed by a set $I$, there is a set $X[I]$ consisting of precisely those sets $x$ for which there exists an $i\in I$ such that $x\in X_i$. In other words: the image of a set-indexed family of sets is again a set. Without the replacement axiom, $X[I]$ would be a class. In the following corollary we establish a type-theoretic analogue of the replacement axiom: the image of a family of small types indexed by a small type is again (essentially) small.

\begin{cor}\label{cor:im_small}
For any small type family $B:A\to\UU$, where $A$ is small, the image $\im(B)$ is essentially small. We call $\im(B)$ the \define{univalent completion} of $B$. 
\end{cor}

\section{Set quotients}

\subsection{Sets in homotopy type theory}
\begin{defn}
A type $A$ is said to be a \define{set} if there is a term of type
\begin{equation*}
\isset(A)\defeq \prd{x,y:A}\isprop(\id{x}{y}).
\end{equation*}
\end{defn}

\begin{lem}\label{lem:prop_to_id}
Let $A$ be a type, and let $R:A\to A\to\UU$ be a binary relation on $A$ satisfying
\begin{enumerate}
\item Each $R(x,y)$ is a proposition,
\item $R$ is reflexive, as witnessed by $\rho:\prd{x:A}R(x,x)$.
\end{enumerate}
Then any fiberwise map
\begin{equation*}
\prd{x,y:A}R(x,y)\to (\id{x}{y})
\end{equation*}
is a fiberwise equivalence. Consequently, if there is such a fiberwise map, then $A$ is a set.
\end{lem}

\begin{proof}
Let $f:\prd{x,y:A}R(x,y)\to(\id{x}{y})$. 
Since $R$ is assumed to be reflexive, we also have a fiberwise transformation
\begin{equation*}
\ind{x{=}}(\rho(x)):\prd{y:A}(\id{x}{y})\to R(x,y).
\end{equation*}
Since each $R(x,y)$ is assumed to be a proposition, it therefore follows that each $R(x,y)$ is a retract of $\id{x}{y}$. We conclude by \autoref{ex:id_fundamental_retr} that for each $x,y:A$, the map $f(x,y):R(x,y)\to(\id{x}{y})$ must be an equivalence.

Now it also follows that $A$ is a set, since its identity types are (equivalent to) propositions.
\end{proof}

\begin{eg}
One can apply \cref{lem:prop_to_id} using the \define{observational equality} $\mathrm{Eq}_\N:\N\to (\N\to\UU)$ given by
\begin{align*}
\mathrm{Eq}_\N(0,0) & \defeq \unit & \mathrm{Eq}_\N(\mathsf{succ}(n),0) & \defeq \emptyt \\
\mathrm{Eq}_\N(0,\mathsf{succ}(m)) & \defeq \emptyt & \mathrm{Eq}_\N(\mathsf{succ}(n),\mathsf{succ}(m)) & \defeq \mathrm{Eq}_\N(n,m)
\end{align*}
to show that $\N$ is a set. To see that $\mathrm{Eq}_\N$ implies identity, note that $\mathrm{Eq}_\N$ implies any reflexive relation $R:\N\to\N\to\UU$.
\end{eg}

\subsection{Equivalence relations}

\begin{defn}\label{defn:eq_rel}
Let $R:A\to (A\to\prop)$ be a binary relation valued in the propositions. We say that $R$ is an \define{($0$-)equivalence relation}\index{equivalence relation|textbf}\index{0-equivalence relation|see {equivalence relation}} if $R$ comes equipped with
\begin{align*}
\rho & : \prd{x:A}R(x,x) \\
\sigma & : \prd{x,y:A} R(x,y)\to R(y,x) \\
\tau & : \prd{x,y,z:A} R(x,y)\to (R(y,z)\to R(x,z)).
\end{align*}
Given an equivalence relation $R:A\to (A\to\prop)$, the \define{equivalence class}\index{equivalence class|textbf} $[x]_R$ of $x:A$ is defined to be
\begin{equation*}
[x]_R\defeq R(x).
\end{equation*}
\end{defn}

\begin{defn}
Let $R:A\to (A\to\prop)$ be a $0$-equivalence relation. 
We define for any $x,y:A$ a map\index{class_eq@{$\mathsf{class\usc{}eq}$}|textbf}
\begin{equation*}
\mathsf{class\usc{}eq}:R(x,y)\to ([x]_R=[y]_R).
\end{equation*}
\end{defn}

\begin{proof}[Construction.]
Let $r:R(x,y)$. By function extensionality, the identity type $R(x)=R(y)$ is equivalent to the type
\begin{equation*}
\prd{z:A}R(x,z)=R(y,z).
\end{equation*}
Let $z:A$. By the univalence axiom, the type $R(x,z)=R(y,z)$ is equivalent to the type
\begin{equation*}
\eqv{R(x,z)}{R(y,z)}.
\end{equation*}
We have the map $\tau_{y,x,z}(\sigma(r)):R(x,z)\to R(y,z)$. Since this is a map between propositions, we only have to construct a map in the converse direction to show that it is an equivalence. The map in the converse direction is just $\tau_{x,y,z}(r):R(y,z)\to R(x,z)$. 
\end{proof}

\begin{thm}\label{thm:equivalence_classes}
Let $R:A\to (A\to\prop)$ be a $0$-equivalence relation. 
Then for any $x,y:A$ the map
\begin{equation*}
\mathsf{class\usc{}eq} : R(x,y)\to ([x]_R=[y]_R)
\end{equation*}
is an equivalence.
\end{thm}

\begin{proof}
By the 3-for-2 property of equivalences, it suffices to show that the map
\begin{equation*}
\lam{r}{z}\tau_{y,x,z}(\sigma(r)) : R(x,y)\to \prd{z:A} \eqv{R(x,z)}{R(y,z)}
\end{equation*}
is an equivalence. Since this is a map between propositions, it suffices to construct a map of type
\begin{equation*}
\Big(\prd{z:A} \eqv{R(x,z)}{R(y,z)}\Big)\to R(x,y).
\end{equation*}
This map is simply $\lam{f} \sigma_{y,x}(f_x(\rho(x)))$. 
\end{proof}

\begin{rmk}
By \cref{thm:equivalence_classes} we can begin to think of the \emph{quotient}\index{quotient} $A/R$ of a type $A$ by an equivalence relation $R$. Classically, the quotient is described as the set of equivalence classes, and \cref{thm:equivalence_classes} establishes that two equivalence classes $[x]_R$ and $[y]_R$ are equal precisely when $x$ and $y$ are related by $R$.

However, the type $A\to\prop$ may contain many more terms than just the $R$-equivalence classes. Therefore we are facing the task of finding a type theoretic description of the smallest subtype of $A\to\prop$ containing the equivalence classes.
Another to think about this is as the \emph{image}\index{image} of $R$ in $A\to \prop$. 
The construction of the (homotopy) image of a map can be done with \emph{higher inductive types}\index{higher inductive type}, which we will do in \cref{chap:image}.
\end{rmk}

\subsection{The universal property of set quotients}

\begin{defn}
Let $R:A\to (A\to \prop)$ be an equivalence relation\index{equivalence relation|textit}, for $A:\UU$, and consider a map $q:A\to B$ where the type $B$ is a set, for which we have
\begin{equation*}
\prd{x,y:A}R(x,y)\to q(x)=q(y).
\end{equation*}
We will define a map
\begin{equation*}
\quotientrestr:(B\to X) \to \Big(\sm{f:A\to X}\prd{x,y:A}R(x,y)\to (f(x)=f(y))\Big).
\end{equation*}
\end{defn}

\begin{proof}[Construction]
Let $h:B\to X$. Then we have $h\circ q : A\to X$, so it remains to show that
\begin{equation*}
\prd{x,y:A}R(x,y)\to (h(q(x))=h(q(y)))
\end{equation*}
Consider $x,y:A$ which are related by $R$. Then we have an identification $p:q(x)=q(y)$, so it follows that $\ap{h}{p}:h(q(x))=h(q(y))$.  
\end{proof}

\begin{defn}
Let $R:A\to (A\to \prop)$ be an equivalence relation\index{equivalence relation|textit}, for $A:\UU$, and consider a map $q:A\to B$ satisfying
\begin{equation*}
\prd{x,y:A}R(x,y)\to q(x)=q(y),
\end{equation*}
where the type $B$ is a set. We say that the map $q:A\to B$ satisfies the universal property of the \define{set quotient}\index{set quotient}\index{universal property!of set quotients|textit} $A/R$ if for any set $X$ the map
\begin{equation*}
\quotientrestr : (B\to X) \to \Big(\sm{f:A\to X}\prd{x,y:A}R(x,y)\to (f(x)=f(y))\Big)
\end{equation*}
is an equivalence.
\end{defn}

\begin{lem}
Let $R:A\to (A\to \prop)$ be an equivalence relation\index{equivalence relation|textit}, for $A:\UU$, and consider a commuting triangle
\begin{equation*}
\begin{tikzcd}[column sep=tiny]
A \arrow[rr,"q"] \arrow[dr,swap,"R"] & & U \arrow[dl,"m"] \\
& \prop^A
\end{tikzcd}
\end{equation*}
with $H:R\htpy m\circ q$, where $m$ is an embedding. Then we have
\begin{equation*}
\prd{x,y:A}R(x,y)\to (q(x)=q(y)).
\end{equation*}
\end{lem}

\begin{thm}\label{thm:quotient_up}
Let $R:A\to (A\to \prop)$ be an equivalence relation\index{equivalence relation|textit}, for $A:\UU$, and consider a commuting triangle
\begin{equation*}
\begin{tikzcd}[column sep=tiny]
A \arrow[rr,"q"] \arrow[dr,swap,"R"] & & U \arrow[dl,"m"] \\
& \prop^A
\end{tikzcd}
\end{equation*}
with $H:R\htpy m\circ q$, where $m$ is an embedding. Then the following are equivalent:
\begin{enumerate}
\item The embedding $m:U\to \prop^A$ satisfies the universal property of the image of $R$.
\item The map $q:A\to U$ satisfies the universal property of the set quotient $A/R$.
\end{enumerate}
\end{thm}

\begin{proof}
Suppose $m:U\to \prop^A$ satisfies the universal property of the image of $R$. Then it follows by \cref{thm:surjective} that the map $q:A\to U$ is surjective. Our goal is to prove that $U$ satisfies the universal property of the set quotient $A/R$. 
\end{proof}

\begin{rmk}
\cref{thm:quotient_up} suggests that we can define the quotient of an equivalence relation $R$ on a type $A$ as the image of a map. However, the type $\prop^A$ of which the quotient is a subtype is not a small type, even if $A$ is a small type.
Therefore it is not clear that the quotient $A/R$ is essentially small\index{essentially small}, as it should be. Luckily, our construction of the image of a map allows us to show that the image is indeed essentially small, using the fact that $\prop^A$ is locally small\index{locally small}.
\end{rmk}

\subsection{The construction of set quotients}
\begin{lem}
Consider a commuting square
\begin{equation*}
\begin{tikzcd}
A \arrow[r] \arrow[d] & B \arrow[d] \\
C \arrow[r] & D.
\end{tikzcd}
\end{equation*}
\begin{enumerate}
\item If the square is cartesian, $B$ and $C$ are essentially small, and $D$ is locally small, then $A$ is essentially small.
\item If the square is cocartesian, and $A$, $B$, and $C$ are essentially small, then $D$ is essentially small. 
\end{enumerate}
\end{lem}

\begin{cor}
Suppose $f:A\to X$ and $g:B\to X$ are maps from essentially small types $A$ and $B$, respectively, to a locally small type $X$. Then $A\times_X B$ is again essentially small. 
\end{cor}

\begin{lem}
Consider a type sequence
\begin{equation*}
\begin{tikzcd}
A_0 \arrow[r,"f_0"] & A_1 \arrow[r,"f_1"] & A_2 \arrow[r,"f_2"] & \cdots
\end{tikzcd}
\end{equation*}
where each $A_n$ is essentially small. Then its sequential colimit is again essentially small. 
\end{lem}

\begin{thm}
For any map $f:A\to X$ from a small type $A$ into a locally small type $X$, the image $\im(f)$ is an essentially small type.
\end{thm}

Recall that in set theory, the replacement axiom asserts that for any family of sets $\{X_i\}_{i\in I}$ indexed by a set $I$, there is a set $X[I]$ consisting of precisely those sets $x$ for which there exists an $i\in I$ such that $x\in X_i$. In other words: the image of a set-indexed family of sets is again a set. Without the replacement axiom, $X[I]$ would be a class. In the following corollary we establish a type-theoretic analogue of the replacement axiom: the image of a family of small types indexed by a small type is again (essentially) small.

\begin{cor}\label{cor:im_small}
For any small type family $B:A\to\UU$, where $A$ is small, the image $\im(B)$ is essentially small. We call $\im(B)$ the \define{univalent completion} of $B$. 
\end{cor}

\subsection{Set truncation}

\begin{lem}
For each type $A$, the relation $I_{(-1)}:A\to (A\to\prop)$ given by
\begin{equation*}
I_{(-1)}(x,y)\defeq\brck{x=y}
\end{equation*}
is an equivalence relation.
\end{lem}

\begin{proof}
For every $x:A$ we have $\bproj{\refl{x}}:\brck{x=x}$, so the relation is reflexive. To see that the relation is symmetric note that by the universal property of propositional truncation there is a unique map $\brck{\invfunc}:\brck{x=y}\to\brck{y=x}$ for which the square
\begin{equation*}
\begin{tikzcd}
(x=y) \arrow[r,"\invfunc"] \arrow[d,swap,"\bproj{\blank}"] & (y=x) \arrow[d,"\bproj{\blank}"] \\
\brck{x=y} \arrow[r,densely dotted,swap,"\brck{\invfunc}"] & \brck{y=x}
\end{tikzcd}
\end{equation*}
commutes. This shows that the relation is symmetric. Similarly, we show by the universal property of propositional truncation that the relation is transitive.
\end{proof}

\begin{defn}
For each type $A$ we define the \define{set truncation}
\begin{equation*}
\trunc{0}{A}\defeq A/I_{(-1)},
\end{equation*}
and the unit of the set truncation is defined to be the quotient map.
\end{defn}

\begin{thm}
For each type $A$, the set truncation satisfies the universal property of the set truncation.
\end{thm}

\section{Elementary construction of the $k$-truncations}

\subsection{Truncated types}
\begin{defn}
We define $\istrunc{} : \Z_{\geq-2}\to\UU\to\UU$ by induction on $k:\Z_{\geq -2}$, taking
\begin{align*}
\istrunc{-2}(A) & \defeq \iscontr(A) \\
\istrunc{k+1}(A) & \defeq \prd{x,y:A}\istrunc{k}(\id{x}{y}).\qedhere
\end{align*}
For any type $A$, we say that $A$ is \define{$k$-truncated}, or a \define{$k$-type}, if there is a term of type $\istrunc{k}(A)$. We say that a map $f:A\to B$ is $k$-truncated if its fibers are $k$-truncated.
\end{defn}

\begin{lem}\label{thm:istrunc_next}
If $A$ is a $k$-type, then $A$ is also a $(k+1)$-type.
\end{lem}

\begin{thm}\label{thm:ktype_eqv}
If $e:\eqv{A}{B}$ is an equivalence, and $B$ is a $k$-type, then so is $A$.
\end{thm}

\begin{proof}
We have seen in \autoref{ex:contr_equiv} that if $B$ is contractible and $e:\eqv{A}{B}$ is an equivalence, then $A$ is also contractible. This proves the base case.

For the inductive step, assume that the $k$-types are stable under equivalences, and consider $e:\eqv{A}{B}$ where $B$ is a $(k+1)$-type. In \autoref{cor:emb_equiv} we have seen that
\begin{equation*}
\apfunc{e}:(\id{x}{y})\to(\id{e(x)}{e(y)})
\end{equation*}
is an equivalence for any $x,y$. Note that $\id{e(x)}{e(y)}$ is a $k$-type, so by the induction hypothesis it follows that $\id{x}{y}$ is a $k$-type. This proves that $A$ is a $(k+1)$-type.
\end{proof}

\begin{cor}\label{cor:emb_into_ktype}
If $f:A\to B$ is an embedding, and $B$ is a $(k+1)$-type, then so is $A$.
\end{cor}

\begin{proof}
By the assumption that $f$ is an embedding, the action on paths
\begin{equation*}
\apfunc{f}:(\id{x}{y})\to (\id{f(x)}{f(y)})
\end{equation*}
is an equivalence for every $x,y:A$. Since $B$ is assumed to be a $(k+1)$-type, it follows that $f(x)=f(y)$ is a $k$-type for every $x,y:A$. Therefore we conclude by \cref{thm:ktype_eqv} that $\id{x}{y}$ is a $k$-type for every $x,y:A$. In other words, $A$ is a $(k+1)$-type.
\end{proof}

\begin{thm}\label{thm:trunc_pi}
Assume function extensionality. Then for any type family $B$ over $A$ one has\index{truncated}
\begin{equation*}
\Big(\prd{x:A}\istrunc{k}(B(x))\Big)\to \istrunc{k}\Big(\prd{x:A}B(x)\Big).
\end{equation*}
\end{thm}

\begin{proof}
The theorem is proven by induction on $k\geq -2$, where the base case is just the weak function extensionality principle, which was shown to follow from function extensionality in \autoref{thm:funext_wkfunext}.
\end{proof}

\begin{cor}\label{cor:funtype_trunc}
Suppose $B$ is a $k$-type. Then $A\to B$ is also a $k$-type, for any type $A$.
\end{cor}

\begin{thm}
Let $B$ be a type family over $A$. Then the following are equivalent:
\begin{enumerate}
\item For each $x:A$ the type $B(x)$ is $k$-truncated.
\item The projection map
\begin{equation*}
\proj 1 : \Big(\sm{x:A}B(x)\Big)\to A
\end{equation*}
is a $k$-truncated map.
\end{enumerate}
\end{thm}

\begin{proof}
By \cref{ex:fib_replacement,ex:fiber_trans} we obtain equivalences
\begin{equation*}
\eqv{B(x)}{\fib{\proj 1}{x}}
\end{equation*}
for every $x:A$. Therefore the claim follows from \cref{thm:ktype_eqv}.
\end{proof}

\begin{thm}\label{thm:trunc_ap}
Let $f:A\to B$ be a map. The following are equivalent:
\begin{enumerate}
\item The map $f$ is $(k+1)$-truncated.
\item For each $x,y:A$, the map
\begin{equation*}
\apfunc{f} : (x=y)\to (f(x)=f(y))
\end{equation*}
is $k$-truncated. 
\end{enumerate}
\end{thm}

\begin{proof}
First we show that for any $s,t:\fib{f}{b}$ there is an equivalence
\begin{equation*}
\eqv{(s=t)}{\fib{\apfunc{f}}{\ct{\proj 2(s)}{\proj 2(t)^{-1}}}}
\end{equation*}
We do this by $\Sigma$-induction on $s$ and $t$, and then we calculate using \cref{ex:trans_ap} and basic manipulations of identifications that
\begin{align*}
(\pairr{x,p}=\pairr{y,q}) & \eqvsym \sm{r:x=y} \mathsf{tr}_{f(\blank)=b}(r,p)=q \\
& \eqvsym \sm{r:x=y} \ct{\ap{f}{r}^{-1}}{p}=q \\
& \eqvsym \sm{r:x=y} \ap{f}{r}=\ct{p}{q^{-1}} \\
& \jdeq \fib{\apfunc{f}}{\ct{p}{q^{-1}}}.
\end{align*}
By these equivalences, it follows that if $\apfunc{f}$ is $k$-truncated, then for each $s,t:\fib{f}{b}$ the identity type $s=t$ is equivalent to a $k$-truncated type, and therefore we obtain by \cref{thm:ktype_eqv} that $f$ is $(k+1)$-truncated.

For the converse, note that we have equivalences
\begin{align*}
\fib{\apfunc{f}}{p} & \eqvsym ((x,p)=(y,\refl{f(y)})).
\end{align*}
Therefore it follows that if $f$ is $(k+1)$-truncated, then the identity type $(x,p)=(y,\refl{f(y)})$ in $\fib{f}{f(y)}$ is $k$-truncated for any $p:f(x)=f(y)$, and therefore $\fib{\apfunc{f}}{p}$ is $k$-truncated by \cref{thm:ktype_eqv}. 
\end{proof}

\begin{thm}
Let $f:\prd{x:A}B(x)\to C(x)$ be a fiberwise transformation. Then the following are equivalent:
\begin{enumerate}
\item For each $x:A$ the map $f(x)$ is $k$-truncated.
\item The induced map 
\begin{equation*}
\total{f}:\Big(\sm{x:A}B(x)\Big)\to\Big(\sm{x:A}C(x)\Big)
\end{equation*}
is $k$-truncated.
\end{enumerate}
\end{thm}

\begin{proof}
This follows directly from \cref{lem:fib_total,thm:ktype_eqv}.
\end{proof}

\subsection{The universal property of the $k$-truncations}

\subsection{The join extension and connectivity theorems}

Some basic results about the join of maps include a generalization of Lemma 8.6.1 of
\cite{hottbook}, which we call the join extension theorem (\autoref{thm:join-extension}), and a closely
related theorem which we call the join connectivity theorem (\autoref{thm:join-connectivity}).
The idea of the join connectivity theorem came from Proposition 8.15 in 
Rezk's notes on homotopy toposes \cite{Rezk2010toposes}.
We use the join connectivity theorem in 
\autoref{thm:joinconstruction-connectivity} to conclude that the connectivity 
of the approximations of the image inclusion increases.
In this sense, our approximating sequence of the image is very nice:
after $n$ steps of the approximation, only stuff of homotopy level higher than
$n$ is added.

Lemma 8.6.1 of \cite{hottbook} states that if $f:A\to B$ is an $l$-connected map,
and if $P:B\to\UU$ is a family of $(k+l+2)$-truncated types,
then precomposing by $f$ gives an $k$-truncated map of type
\begin{equation*}
\Big(\prd{b:B}P(b)\Big)\to\Big(\prd{a:A}P(f(a))\Big).
\end{equation*}
The general join extension theorem states that if $f:A\to B$ is an $M$-connected
map for some type $M$, and $P:B\to\UU$ is a family of $(\join{M}{N})$-local 
types, then the mentioned precomposition is an $N$-local map 
(we recall the terminology shortly). 
Note that, if one takes spheres $\Sn^l$ and $\Sn^k$ for $M$ and $N$, 
one retrieves Lemma 8.6.1 of \cite{hottbook} as a simple corollary.

We conclude this section with \autoref{thm:joinconstruction-connectivity},
asserting that $f^{\ast n}$ factors through an $(n-2)$-connected map to
$\im(f)$, for each $n:\N$.

\begin{defn}\label{defn:local}
For a given type $M$, a type $A$ is said to be \define{$M$-local} if the map
\begin{equation*}
\lam{a}{m}a : A \to (M \to A)  
\end{equation*}
is an equivalence.
\end{defn}

In other words, the type $A$ is $M$-local if each $f:M\to A$ has a unique extension along the
map $M\to\unit$, as indicated in the diagram
\begin{equation*}
\begin{tikzcd}
M \arrow[r,"f"] \arrow[d] & A \\
\unit. \arrow[ur,densely dotted]
\end{tikzcd}
\end{equation*}
Note that being $M$-local in the above sense is a proposition, so that the
type of all $M$-local types in $\UU$ is a subuniverse of $\UU$%
\footnote{When $\UU$ is assumed to be closed under recursive higher inductive
types, there exists an operation $\modal_M : \UU\to\UU$, 
which maps a type $A$ to the universal $M$-local type $\modal_M(A)$
with a map of type $A\to\modal_M(A)$. This operation is called 
\define{localization at $M$}, and it is a modality. 
This is just a special case of localization. There is a more general
notion of localization at a family of maps, see%
~\cite{RijkeShulmanSpitters}, for which the localization operation
is a reflective subuniverse, but \emph{not generally} a modality.
The survey article \cite{RijkeShulmanSpitters} contains much
more information about local types and the operation of localization.%
}. 
Recall that a type is $\sphere{n+1}$-local precisely when it is $k$-truncated,
for each $k\geq -2$ (taking the $(-1)$-sphere to be the empty type).

Dually, a type $A$ is said to be \define{$M$-connected} if every $M$-local
type is $A$-local. That is, if for every $M$-local type $B$, the map
\begin{equation*}
\lam{b}{a}{b} : B \to (A \to B)
\end{equation*}
is an equivalence. A map is said to be $M$-connected if its fibers are $M$-connected.
Thus in particular, $M$ itself is $M$-connected, and the unit type $\unit$ is $M$-connected for every $M$. 
Usually, a type $A$ is said to be $M$-connected if its localization
$\modal_M(A)$ is contractible. 
Since we have not assumed that the universe is closed under a general class of recursive higher inductive types, we cannot simply assume that the operation $\modal_M:\UU\to\UU$ of localizing at $M$ is available. For a detailed discussion on localization, see \cite{RijkeShulmanSpitters}.

In the present article, we focus on the interaction
of the join operation with the notions of being local and of connectedness.

\begin{defn}
Let $M$ be a type. We say that a type $X$ has the \define{$M$-extension property}
with respect to a map $F:A\to B$, if the map
\begin{equation*}
\lam{g}{a} g(F(a)) : (B\to X)\to (A\to X)
\end{equation*}
is $M$-local. In the case $M\jdeq\unit$, we say that $X$ is $F$-local.
\end{defn}

\begin{lem}\label{lem:equivalent-extension-problems}
For any three types $A$, $A'$ and $B$, the type $B$ is $(\join{A}{A'})$-local
if and only if for any any $f:A\to B$, the type
\begin{equation*}
\sm{b:B}\prd{a:A}f(a)=b
\end{equation*}
is $A'$-local.
\end{lem}

\begin{proof}
To give $f:A\to B$ and $(f',H):A'\to\sm{b:B}\prd{a:A}f(a)=b$ is equivalent to giving a map $g:\join{A}{A'}\to B$. Concretely, the equivalence is given by substituting in $g:\join{A}{A'}\to B$ the constructors of the join, to obtain $\pairr{g\circ\inl,g\circ\inr,\apfunc{g}\circ\glue}$. 

Now observe that the fiber of precomposing with the unique map $!_{\join{A}{A'}} : \join{A}{A'}\to\unit$ at $g : \join{A}{A'}\to B$, is equivalent to
\begin{equation*}
\sm{b:B}\prd{t:\join{A}{A'}}g(t)=b.
\end{equation*}
Similarly, the fiber of precomposing with the unique map $!_{A'} : A'\to\unit$ at $\pairr{g\circ\inr,\apfunc{g}\circ\glue} : A'\to\sm{b:B}\prd{a:A}f(a)=b$ is equivalent to
\begin{equation*}
\sm{b:B}{h:\prd{a:A}g(\inl(a))=b}\prd{a':A'}\pairr{g(\inr(a')),\apfunc{g}(\glue(a,a'))}=\pairr{b,h}.
\end{equation*}
By the universal property of the join, these types are equivalent.
\end{proof}

\begin{lem}\label{lem:join-local}
Suppose $A$ is an $M$-connected type, and that $B$ is an $(\join{M}{N})$-local type. Then $B$ is $(\join{A}{N})$-local.
\end{lem}

\begin{proof}
Let $B$ be a $(\join{M}{N})$-local type. Our goal of showing that $B$ is
$(\join{A}{N})$-local is equivalent to showing that for any $f:N\to B$, 
the type 
\begin{equation*}
\sm{b:B}\prd{a:A}f(a)=b
\end{equation*}
is $A$-local. 
Since $B$ is assumed to be $(\join{M}{N})$-local, we know that this type is 
$M$-local. Since $A$ is $M$-connected, this type is also $A$-local.
\end{proof}

\begin{lem}\label{lem:N-extension-simple}
Let $A$ be $M$-connected and let $B$ be $(\join{M}{N})$-local. Then the map
\begin{equation*}
\lam{b}{a}b:B\to B^A
\end{equation*}
is $N$-local. 
\end{lem}

\begin{proof}
The fiber of $\lam{b}{a}b$ at a function $f:A\to B$ is equivalent to the type $\sm{b:B}\prd{a:A}f(a)=b$. Therefore, it suffices to show that this type is $N$-local. By \autoref{lem:equivalent-extension-problems}, it is equivalent to show that $B$ is $(\join{A}{N})$-local. This is solved in \autoref{lem:join-local}.
\end{proof}

\begin{thm}[Join extension theorem]\label{thm:join-extension}
Suppose $f:X\to Y$ is $M$-connected, and let $P:Y\to\UU$ be a family of
$(\join{M}{N})$-local types for some type $N$. Then precomposition by $f$, i.e.
\begin{equation*}
\lam{s}s\circ f : \Big(\prd{y:Y}P(y)\Big)\to\Big(\prd{x:X}P(f(x))\Big),
\end{equation*}
is an $N$-local map.
\end{thm}

\begin{proof}
Let $g:\prd{x:X}P(f(x))$. Then we have the equivalences
\begin{align*}
\fib{(\blank\circ f)}{g} 
& \eqvsym \sm{s:\prd{y:Y}P(y)}\prd{x:X}s(f(x))=g(x) \\
& \eqvsym \sm{s:\prd{y:Y}P(y)}\prd{y:Y}{(x,p):\fib{f}{y}} s(y)= \tr({p},{g(x)}) \\
& \eqvsym \prd{y:Y}\sm{z:P(y)}\prd{(x,p):\fib{f}{y}} \tr({p},{g(x)})=z \\
& \eqvsym \prd{y:Y}\fib{\lam{z}{(x,p)}z}{\lam{(x,p)}\tr({p},{g(x)})}.
\end{align*}
Therefore, it suffices to show for every $y:Y$, that $P(y)$ has the $N$-extension property with respect to the unique map of type $\fib{f}{y}\to\unit$. This is a special case of \autoref{lem:N-extension-simple}.
\end{proof}

\begin{thm}\label{thm:simple-join}
Suppose $X$ is an $M$-connected type and $Y$ is an $N$-connected type. Then $\join{X}{Y}$ is an $(\join{M}{N})$-connected type.
\end{thm}

\begin{proof}
It suffices to show that any $(\join{M}{N})$-local type is $(\join{X}{Y})$-local.
Let $Z$ be an $(\join{M}{N})$-local type.
Since $Z$ is assumed to be $(\join{M}{N})$-local, it follows by \autoref{lem:join-local} that $Z$ is $(\join{X}{N})$-local. By symmetry of the join, it also follows that $Z$ is $(\join{X}{Y})$-local.
\end{proof}

\begin{thm}[Join connectivity theorem]\label{thm:join-connectivity}
Consider an $M$-connected map $f:A\to X$ and an $N$-connected map $g:B\to X$. Then $\join{f}{g}$ is $(\join{M}{N})$-connected.
\end{thm}

\begin{proof}
This follows from \autoref{thm:simple-join} and \cref{thm:join-fiber}.
\end{proof}

\begin{thm}\label{thm:joinconstruction-connectivity}
Consider the factorization
\begin{equation*}
\begin{tikzcd}
A_n \arrow[dr,swap,"f^{\ast n}"] \arrow[r,"q_n"] & \im(f) \arrow[d] \\
& X
\end{tikzcd}
\end{equation*}
of $f^{\ast n}$ through the image $\im(f)$. 
Then the map $q_n$ is $(n-2)$-connected, for each $n:\N$.
\end{thm}

\begin{proof}
We first show the assertion that, given a commuting diagram of the form
\begin{equation*}
\begin{tikzcd}
A \arrow[r,"q"] \arrow[dr,swap,"f"] & Y \arrow[d,"m"] & A' \arrow[l,swap,"{q'}"] \arrow[dl,"{f'}"] \\
& X
\end{tikzcd}
\end{equation*}
in which $m$ is an embedding, then $\join{f}{f'}=\join{(m\circ q)}{(m\circ q')}=m\circ (\join{q}{q'})$.
In other words, postcomposition with embeddings distributes over 
the join operation.

Note that, since $m$ is assumed to be an embedding, we have an equivalence of
type $\eqv{f(a)=f'(a)}{q(a)=q'(a)}$, for every $a:A$. Hence the pullback of
$f$ and $f'$ is equivalent to the pullback of $q$ along $q'$. Consequently, the
two pushouts
\begin{equation*}
\begin{tikzcd}
A\times_X A' \arrow[r,"\pi_2"] \arrow[d,swap,"\pi_1"] & A' \arrow[d] \\
A \arrow[r] & \join[X]{A}{A'}
\end{tikzcd}
\qquad\text{and}\qquad
\begin{tikzcd}
A\times_Y A' \arrow[r,"\pi_2"] \arrow[d,swap,"\pi_1"] & A' \arrow[d] \\
A \arrow[r] & \join[Y]{A}{A'}
\end{tikzcd}
\end{equation*}
are equivalent. Hence the claim follows.

As a corollary, we get that $q_n=q_f^{\ast n}$. Note that $q_f$ is surjective,
in the sense that $q_f$ is $\bool$-connected, where $\bool$ is the type of booleans%
\footnote{Recall that the $\bool$-local types are precisely the propositions.}.
Hence it follows that $q_n$ is $\bool^{\ast n}$-connected. 

Now recall that the $n$-th join power of $\bool$ is the $(n-1)$-sphere $\Sn^{n-1}$, and that
a type is $(\Sn^{n-1})$-connected if and only if it is $(n-2)$-connected.
\end{proof}

\subsection{The construction of the $k$-truncation}\label{sec:truncation}

In this section we will construct for any $k:\N$, the $k$-truncation on any univalent universe that contains
a natural numbers object and is closed under graph quotients.
We will do this via the modified join construction of \autoref{thm:modified-join}.
Recall that a $(-2)$-truncated type is simply a contractible type, and that
for $k\geq -2$ an $(n+1)$-truncated type is a type of which the identity types
are $k$-truncated. The $(-2)$-truncation is easy to construct: it sends
every type to the unit type $\unit$. Thus, we shall proceed by induction
on the integers greater or equal to $-2$, and assume that the universe admits
an $k$-truncation operation $\trunc{k}{\blank}:\UU\to\UU$ for a given $n$.

A suggestive way to think of the type $\trunc{k+1}{A}$ is as the quotient of $A$ modulo the
`$(n+1)$-equivalence relation' given by $\trunc{k}{a=b}$. 
Indeed, by Theorem 7.3.12 of \cite{hottbook} we have that the canonical map
\begin{equation*}
\trunc{k}{a=b}\to(\tproj{n+1}{a}=\tproj{n+1}{b})
\end{equation*}
is an equivalence, and the unit $\tproj{k+1}{\blank}:A\to \trunc{k+1}{A}$ is
a surjective map (it is in fact $(k+1)$-connected). 

\begin{thm}\label{thm:truncation}
In Martin-L\"of type theory with a univalent universe $\UU$ that is closed under
graph quotients we can define, for every $k\geq -2$, an $k$-truncation operation
\begin{equation*}
\trunc{k}{\blank} : \UU\to\UU
\end{equation*}
and for every $A:\UU$ a map
\begin{equation*}
\tproj{k}{\blank}:A\to\trunc{k}{A},
\end{equation*}
such that for each $A:\UU$ the type $\trunc{k}{A}$ is an $k$-truncated type satisfying the (dependent) universal property of $k$-truncation, that for every $P:\trunc{k}{A}\to\UU$ such that every $P(x)$ is $k$-truncated,
the canonical map
\begin{equation*}
\blank\circ\tproj{k}{\blank} : \Big(\prd{x:\trunc{k}{A}}P(x)\Big)\to\Big(\prd{a:A}P(\tproj{k}{a})\Big)
\end{equation*}
is an equivalence.
\end{thm}

\begin{proof}[Construction]
As announced, we define the $k$-truncation operation by induction on $k\geq-2$,
with the trivial operation as the base case. Let $k:\N$ and suppose we have
a $k$-truncation operation as described in the statement of the theorem.

We first define the reflexive relation $\mathscr{Y}_k(A) : A \to A \to \UU$ by
\begin{equation*}
\mathscr{Y}_k(A)(a,b) \defeq \trunc{k}{a=b}.
\end{equation*}
Note that the codomain $(A\to\UU)$ of $\mathscr{Y}_k(A)$ is locally small since it is the exponent of
the locally small type $\UU$ by a small type $A$. Hence we we obtain the image
of $\mathscr{Y}_k(A)$ from the modified join construction of \autoref{thm:modified-join}.
This allows us to define
\begin{align*}
\trunc{k+1}{A} & \defeq \im'(\mathscr{Y}_k(A)) \\
\tproj{k+1}{\blank} & \defeq q'_{\mathscr{Y}_k(A)}
\end{align*}
For notational reasons, we shall just write $\im(\mathscr{Y}_k(A))$ for $\im'(\mathscr{Y}_k(A))$. 

We will show that $\trunc{k+1}{A}$ is indeed $(k+1)$-truncated in \autoref{cor:truncated} of \autoref{lem:modal_contr} below. Once this fact is established, it remains to verify the dependent universal property of $(k+1)$-truncation.
By the join extension theorem \autoref{thm:join-extension} (using $N\defeq \emptyt$), it suffices to show that the map $\tproj{k+1}{\blank}:A\to\trunc{k+1}{A}$ is $\sphere{k+2}$-connected. Note that $\tproj{k+1}{\blank}$ is surjective, so the claim that $\tproj{k+1}{\blank}$ is $\sphere{k+2}$-connected follows from \autoref{lem:ap_connectivity}, where we show that for any surjective map $f:A\to X$, if the action on paths is $M$-connected for any two points in $A$, then $f$ is $\susp(M)$-connected. To apply this lemma, we also need to know that $\tproj{k}{\blank}:A\to\trunc{k}{A}$ is $\sphere{k+1}$-connected. This is shown in Corollary 7.5.8 of \cite{hottbook}.
\end{proof}

Before we prove that $\im(\mathscr{Y}_k(A))$ is $(k+1)$-truncated, we prove the stronger claim that $\im(\mathscr{Y}_k(A))$ has the desired identity types:

\begin{lem}\label{lem:modal_contr}
For every $a,b:A$, we have an equivalence
\begin{equation*}
\eqv{\trunc{k}{a=b}}{(\mathscr{Y}_k(A)(a)=\mathscr{Y}_k(A)(b))}.
\end{equation*}
\end{lem}

\begin{proof}
To characterize the identity type of $\im(\mathscr{Y}_k(A))$ we will apply
the encode-decode method of \cite{LicataShulman}. Thus, we need to provide for every $b:A$ a type 
family $Q_b:\im(\mathscr{Y}_k(A))\to\UU$ with a point $q_b:Q_b(\mathscr{Y}_k(A)(b))$,
such that the total space
\begin{equation*}
\sm{P:\im(\mathscr{Y}_k(A))} Q_b(P)
\end{equation*}
is contractible. Moreover, it must be the case that $\eqv{Q_b(\mathscr{Y}_k(A)(a))}{\trunc{k}{a=b}}$ for any $a:A$. 

To construct $Q_b$, note that for any $b:A$, the image inclusion $i:\im(\mathscr{Y}_k(A))\to (A\to\UU)$ defines 
a type family $Q_b:\im(\mathscr{Y}_k(A))\to\UU$ by $Q_b(P)\defeq P(b)$. With this definition for $Q_b$ it follows that $Q_b(\mathscr{Y}_k(A)(a))\jdeq\mathscr{Y}_k(A)(a,b)\jdeq\trunc{k}{a=b}$, as desired. Moreover, we have a reflexivity term $\tproj{n}{\refl{b}}$ in $\trunc{k}{b=b}$, so it remains to prove that the total space 
\begin{equation*}
\sm{P:\im(\mathscr{Y}_k(A))}P(b)
\end{equation*}
of $Q_b$ is contractible. For the center of contraction we take the pair
$\pairr{\mathscr{Y}_k(A)(b),\tproj{k}{\refl{b}}}$.
Now we need to construct a term of type
\begin{equation*}
\prd{P:\im(\mathscr{Y}_k(A))}{y:P(b)} \pairr{\mathscr{Y}_k(A)(b),\tproj{k}{\refl{b}}}=\pairr{P,y}.
\end{equation*}
Since $\mathscr{Y}_k(A)(b,a)\jdeq\trunc{k}{b=a}$, it is equivalent to construct a term of type
\begin{equation*}
\prd{P:\im(\mathscr{Y}_k(A))}{y:P(b)}\sm{\alpha:\prd{a:A} \eqv{\trunc{k}{b=a}}{P(a)}} \alpha_b(\tproj{k}{\refl{b}})=y.
\end{equation*}
Let $P:\im(\mathscr{Y}_k(A))$ and $y:P(b)$. Then $P(a)$ is $k$-truncated for any $a:A$. Therefore, to construct a map
$\alpha(P,y)_a:\trunc{k}{b=a}\to P(a)$, it suffices to construct a map of type $(b=a)\to P(a)$. This may be done by
path induction, using $y:P(b)$. Since it follows that $\alpha(P,y)_b(\tproj{k}{\refl{b}})=y$, it only remains to show that each $\alpha(P,y)_a$ is an equivalence.  

Note that the type of those $P:\im(\mathscr{Y}_k(A))$ such that for all $y:P(b)$ and all $a:A$ the map $\alpha(P,y)_a$ is an equivalence, is a subtype of $\im(\mathscr{Y}_k(A))$, we may use the universal property of the image of $\mathscr{Y}_k(A)$: it suffices to lift
\begin{equation*}
\begin{tikzcd}
& \sm{P:\im(\mathscr{Y}_k(A))}\prd{y:P(b)}{a:A}\isequiv(\alpha(P,y)_a) \arrow[d] \\
A \arrow[ur,densely dotted] \arrow[r,swap,"\mathscr{Y}_k(A)"] & \im(\mathscr{Y}_k(A)).
\end{tikzcd}
\end{equation*}
In other words, it suffices to show that 
\begin{equation*}
\prd{x:A}{y:\mathscr{Y}_k(A)(x,b)}{a:A}\isequiv(\alpha(\mathscr{Y}_k(A)(x),y)_a).
\end{equation*}
Thus, we want to show that for any $y:\trunc{k}{x=b}$, the map $\trunc{k}{a=b}\to\trunc{k}{x=b}$ constructed above is an equivalence.
Since the fibers of this map are $k$-truncated, and $\iscontr(X)$ of an $k$-truncated type $X$ is always $k$-truncated, we may assume that $y$ is of the form $\tproj{k}{p}$ for $p:x=b$. 
Now it is easy to see that our map of type $\trunc{k}{b=a}\to\trunc{k}{x=a}$ is the unique map which
extends the path concatenation $\ct{p}{\blank}$, as indicated in the diagram
\begin{equation*}
\begin{tikzcd}[column sep=8em]
(b=a) \arrow[r,"\ct{p}{\blank}"] \arrow[d] & (x=a) \arrow[d] \\
\trunc{k}{b=a} \arrow[r,densely dotted,swap,"{\alpha(\mathscr{Y}_k(A)(x),y)_a}"] & \trunc{k}{x=a}.
\end{tikzcd}
\end{equation*}
Since the top map is an equivalence, it follows that the map $\alpha(\mathscr{Y}_k(A)(x),y)_a$ is an equivalence.
\end{proof}

\begin{cor}\label{cor:truncated}
The image $\im(\mathscr{Y}_k(A))$ is an $(k+1)$-truncated type. 
\end{cor}

Before we are able to show that for any surjective map $f:A\to X$, if the action on paths is $M$-connected for any two points in $A$, then $f$ is $\susp(M)$-connected, we show that a type is $\susp(M)$-connected precisely when its identity types are $M$-connected.

\begin{lem}\label{lem:local_id}
Let $M$ be a type. Then a type $X$ is $(\join{\bool}{M})$-local
if and only if all of its identity types are $M$-local. 
\end{lem}

\begin{proof}
The map
\begin{equation*}
\lam{p}{m}p : (x=y)\to (M\to (x=y))
\end{equation*}
is an equivalence if and only if the induced map on total spaces
\begin{equation*}
\lam{\pairr{x,y,p}}\pairr{x,y,\lam{m}p} : \Big(\sm{x,y:X}x=y\Big)\to\Big(\sm{x,y:X}M\to (x=y)\Big)
\end{equation*}
is an equivalence. 
Since the map $\lam{x}\pairr{x,x,\refl{x}}:X\to\sm{x,y:X}x=y$ is an equivalence,
the above map is an equivalence if and only if the map
\begin{equation*}
\lam{x}\pairr{x,x,\lam{m}\refl{x}} : X\to\Big(\sm{x,y:X}M\to (x=y)\Big)
\end{equation*}
is an equivalence. For every $x:X$, the triple $\pairr{x,x,\lam{m}\refl{x}}$
induces a map $\susp(M)\to X$. By uniqueness of the universal property,
it follows that this map is the constant map $\lam{m}x$.
Thus we see that $\lam{x}\pairr{x,x,\lam{m}\refl{x}}$ is an equivalence if
and only if the map
\begin{equation*}
\lam{x}{m}x : X \to (\susp(M)\to X)
\end{equation*}
is an equivalence. 
\end{proof}

\begin{lem}\label{lem:ap_connectivity}
Suppose $f:A\to X$ is a surjective map, with the property that for every
$a,b:A$, the map
\begin{equation*}
\mapfunc{f}(a,b):(a=b)\to (f(a)=f(b))
\end{equation*}
is $M$-connected. Then $f$ is $\susp(M)$-connected. 
\end{lem}

\begin{proof}
We have to show that $\fib{f}{x}$ is $\susp(M)$-connected for each $x:X$. 
Since this is a proposition, and we assume that $f$ is surjective, it
is equivalent to show that $\fib{f}{f(a)}$ is $\susp(M)$-connected for each $a:A$. 
Let $Y$ be a $\susp(M)$-local type. 
For every $g:\fib{f}{f(a)}\to Y$ be a map we have the point $\theta(g)\defeq g(a,\refl{f(a)})$ in $Y$,
so we obtain a map
\begin{equation*}
\theta : (\fib{f}{f(a)}\to Y)\to Y
\end{equation*}
It is clear that $\theta(\lam{\pairr{b,p}}y)=y$, so it remains to show that
for every $g:\fib{f}{f(a)}\to Y$ we have $\lam{\pairr{b,p}}\theta(g)=g$.
That is, we must show that
\begin{equation*}
\prd{b:A}{p:f(a)=f(b)} g(a,\refl{f(a)})=g(b,p).
\end{equation*}
Using the assumption that $Y$ is $\susp(M)$-connected, it follows from
\autoref{lem:local_id} that the type $g(a,\refl{f(a)})=g(b,p)$ is $M$-connected,
for every $b:A$ and $p:f(a)=f(b)$.
Therefore it follows, since the map $\mapfunc{f}(a,b):(a=b)\to(f(a)=f(b))$ is connected, that our goal is equivalent to
\begin{equation*}
\prd{b:A}{p:a=b} g(a,\refl{f(a)})=g(b,\mapfunc{f}(a,b,p)).
\end{equation*}
This follows by path induction. 
\end{proof}
