\chapter{The equifibrant replacement operation}

We begin this chapter with a vast generalization of the descent theorems: modal descent.
Any modality $\modal$ gives rise to a class of maps $f:A\to B$ satisfying the condition that the naturality square
\begin{equation*}
\begin{tikzcd}
A \arrow[r,"f"] \arrow[d,swap,"\eta"] & B \arrow[d,"\eta"] \\
\modal A \arrow[r,swap,"\modal f"] & \modal B
\end{tikzcd}
\end{equation*}
is a pullback square. We call such maps $\modal$-\'etale maps. The modal descent theorem asserts that a $\modal$-\'etale map into $A$ is equivalently described as a modal map into $\modal A$. Every $\modal$-\'etale map is certainly modal, but the conditioin of beign $\modal$-\'etale is slightly stronger than the condition of being modal. 

The difference between the notions of $\modal$-\'etale maps and modal maps becomes perhaps most visible when we look at the left orthogonal classes of the $\modal$-\'etale maps and modal maps. The $\modal$-connected maps are left orthogonal to the modal maps, whereas a map is left orthogonal to the $\modal$-\'etale maps if and only if it is a $\modal$-equivalence (i.e.~a map $f$ such that $\modal f$ is an equivalence). In the case of the $n$-truncation there is a clear difference: a map $f:X\to Y$ is an $n$-equivalence if and only if it induces an isomorphism of homotopy groups $\pi_i(X)\to \pi_i(Y)$ for any $i\leq n$, whereas an $n$-connected map is an $n$-equivalence satisfying the further condition that it $\pi_{n+1}(X)\to\pi_{n+1}(Y)$ is surjective.

In \cref{thm:modal_rofs} we show that the $\modal$-equivalences and the $\modal$-\'etale maps form an orthogonal factorization system. We call this factorization system the \emph{reflective} factorization system of a modality. The reflective factorization system is not stable, so it does not form a new modality. What does follow is that for the unique factorization $f=f_r\circ f_l$ of $f:A\to X$ as a $\modal$-equivalence $f_l$ followed by a $\modal$-\'etale map $f_r$, the canonical map
\begin{equation*}
\mathsf{hom}_X(f_r,e)\to\mathsf{hom}_X(f,e)
\end{equation*}
is an equivalence for any $\modal$-\'etale map $e$ into $X$. 

We then proceed to apply these ideas to the case of reflexive graphs and the reflexive coequalizer. The class of $\Delta$-\'etale maps is introduces as the class of morphisms $f:\mathsf{rGph}(\mathcal{B},\mathcal{A})$ of reflexive graphs satisfying the condition that the square
\begin{equation*}
\begin{tikzcd}[column sep=huge]
\mathcal{B} \arrow[r,"f"] \arrow[d,swap,"\mathsf{constr}"] & \mathcal{A} \arrow[d,"\mathsf{constr}"] \\
\Delta(\mathsf{rcoeq}(\mathcal{B})) \arrow[r,swap,"\Delta(\mathsf{rcoeq}(f))"] & \Delta(\mathsf{rcoeq}(\mathcal{A}))
\end{tikzcd}
\end{equation*}
is a pullback square, and we show that the class of $\Delta$-\'etale morphisms is precisely the class of fibrations of reflexive graphs. 

In the case of sequential colimits, another way of obtaining a cartesian transformation from an arbitrary one is to take the sequential colimit fiberwise. We establish this result in \cref{thm:colim_fib}. It has many important consequences. First of all, we show in \cref{thm:colim_id} that sequential colimits commute with identity types. Second, we show in \cref{thm:colim_fiberseq} that the sequential colimit of a fiber sequence is again a fiber sequence. Third, we show in \cref{thm:colim_hlevel} that truncation levels are closed under sequential colimits, and that sequential colimits commute with truncations. Finally, we show in \cref{thm:colim_hgroup} that sequential colimits commute with homotopy groups. 

The idea of a modal version of the descent theorem first arose in my unpublished work on reflexive graphs with Bas Spitters, in the spring of 2016. However, I only learned about $\modal$-\'etale maps much later from Felix Wellen, and many of the results presented in \cref{seq:modal_descent} came out of a discussion I had with Felix Wellen and Mike Shulman. 
The material in \cref{seq:equifibrant_replacement} on the equifibrant replacement operation on reflexive graphs is joint work with Bas Spitters. 
The material in \cref{sec:seqcolim_eqf} on sequential colimits is joint work with Floris van Doorn and Kristina Sojakova, and all our results are formalized in the proof assistant Lean. 

\section{Modal descent}\label{sec:modal_descent}
\subsection{\texorpdfstring{$\modal$}{○}-\'etale maps}

\begin{defn}
We say that a map $f:A\to B$ is \define{$\modal$-\'etale} if the square
\begin{equation*}
\begin{tikzcd}
A \arrow[r,"f"] \arrow[d,swap,"\eta"] & B \arrow[d,"\eta"] \\
\modal A \arrow[r,swap,"\modal f"] & \modal B
\end{tikzcd}
\end{equation*}
is a pullback square. We will write
\begin{equation*}
\isetale(f)\defeq\ispullback(f,\eta_A,\natunit_\modal(f)).
\end{equation*}
\end{defn}

It is immediate from the definition that any equivalence is $\modal$-\'etale, and that the $\modal$-\'etale maps are closed under composition, and that every equivalence is $\modal$-\'etale.

\begin{eg}\label{eg:etale_prop}
We claim that a map $f:A\to B$ is $\brck{\blank}$-\'etale if and only if $A\to \isequiv(f)$. Examples of maps that satisfy this condition include equivalences, maps between propositions, and any map of the form $\emptyt\to B$.

To see that if $f:A\to B$ is $\modal$-\'etale, then $A\to\isequiv(f)$, consider the pullback square
\begin{equation*}
\begin{tikzcd}
A \arrow[r] \arrow[d,swap,"f"] & \brck{A} \arrow[d,"\brck{f}"] \\
B \arrow[r] & \brck{B},
\end{tikzcd}
\end{equation*}
and let $a:A$. Then both $\brck{A}$ and $\brck{B}$ are contractible, so $\brck{f}:\brck{A}\to\brck{B}$ is an equivalence. Since equivalences are stable under pullback it follows that $f$ is an equivalence.

Now suppose that $A\to \isequiv(f)$. Since $\isequiv(f)$ is a proposition, we also have $\brck{A}\to\isequiv(f)$. To see that the gap map
\begin{equation*}
A \to B\times_{\brck{B}}\brck{A}
\end{equation*}
is an equivalence, we will show that its fibers are contractible. Let $b:B$, $x:\brck{A}$ and $p:\bproj{b}=\brck{f}(x)$. Since $\brck{A}\to\isequiv(f)$, it follows that $f$ is an equivalence. Then $\brck{f}$ is also an equivalence, from which it follows that the naturality square is a pullback square. We conclude that the fibers of the gap map are contractible. 
\end{eg}

\begin{lem}\label{lem:etale_modal}
Any map between $\modal$-modal types is $\modal$-\'etale.
\end{lem}

\begin{proof}
Suppose $f:X\to Y$ is a map between $\modal$-modal types. Then the top and bottom maps in the square
\begin{equation*}
\begin{tikzcd}
X \arrow[r] \arrow[d] & \modal X \arrow[d] \\
Y \arrow[r] & \modal Y
\end{tikzcd}
\end{equation*}
are equivalences. Therefore this square is a pullback square, so $f$ is $\modal$-\'etale.
\end{proof}

In this section our goal is to characterize the $\modal$-\'etale maps for the $n$-truncations. We call such maps \define{$n$-\'etale}. Note that the case of propositional truncation has already been addressed in \cref{eg:etale_prop}.

\begin{lem}\label{lem:etale_char}
Let $\modal$ be a modality for which all propositions are modal, and consider a map $f:A\to B$. The following are equivalent:
\begin{enumerate}
\item $f$ is $\modal$-\'etale.
\item The commuting square
\begin{equation*}
\begin{tikzcd}[column sep=large]
A\times_{\modal A} A \arrow[d,swap,"\pi_1"] \arrow[r,"{f\times_{\modal f} f}"] & B\times_{\modal B} B \arrow[d,"\pi_1"] \\
A \arrow[r,swap,"f"] & B
\end{tikzcd}
\end{equation*}
is a pullback square.
\end{enumerate}
\end{lem}

\begin{rmk}
In the special case of $(-1)$-truncation, the characterization of \cref{lem:etale_char} asserts that a map $f:A\to B$ is $(-1)$-\'etale if and only if the square
\begin{equation*}
\begin{tikzcd}
A\times A \arrow[d,swap,"\pi_1"] \arrow[r,"{f\times f}"] & B\times B \arrow[d,"\pi_1"] \\
A \arrow[r,swap,"f"] & B
\end{tikzcd}
\end{equation*}
is a pullback square.
\end{rmk}

\begin{proof}
Suppose first that $f$ is $\modal$-\'etale, and consider the commuting cube
\begin{equation*}
\begin{tikzcd}
& A\times_{\modal A} A \arrow[dl] \arrow[d] \arrow[dr] \\
A \arrow[d] & B\times_{\modal B} B \arrow[dl] \arrow[dr] & A \arrow[dl,crossing over] \arrow[d] \\
B \arrow[dr] & \modal A \arrow[from=ul,crossing over] \arrow[d] & B \arrow[dl] \\
& \modal B
\end{tikzcd}
\end{equation*}
Since the top, bottom, and both front squares are pullback squares, it follows that both back squares are pullback. This proves that (i) implies (ii).

Now suppose that (ii) holds. Then the map
\begin{equation*}
\fib{\modalunit}{\modalunit(a)}\to \fib{\modalunit}{\modalunit(f(a))}
\end{equation*}
is an equivalence for every $a:A$. Since all propositions are assumed to be modal, it follows that
\begin{equation*}
\fib{\modalunit}{t}\to \fib{\modalunit}{\modal f(t)}
\end{equation*}
is an equivalence for every $t:\modal A$. Thus it follows that the square
\begin{equation*}
\begin{tikzcd}
A \arrow[d,swap,"\modalunit"] \arrow[r] & B \arrow[d,"\modalunit"] \\
\modal A \arrow[r] & \modal B
\end{tikzcd}
\end{equation*}
is a pullback square.
\end{proof}

\begin{cor}
If $f:A\to B$ is $\modal$-\'etale, then the square
\begin{equation*}
\begin{tikzcd}[column sep=large]
A \arrow[d,swap,"\delta_{\modalunit}"] \arrow[r,"f"] & B \arrow[d,"\delta_{\modalunit}"] \\
A\times_{\modal A} A \arrow[r,swap,"f\times_{\modal f}f"] & B\times_{\modal B} B
\end{tikzcd}
\end{equation*}
is a pullback square.
\end{cor}

\begin{proof}
Consider the diagram
\begin{equation*}
\begin{tikzcd}[column sep=large]
A \arrow[d,swap,"\delta_{\modalunit}"] \arrow[r,"f"] & B \arrow[d,"\delta_{\modalunit}"] \\
A\times_{\modal A} A \arrow[d,swap,"\pi_1"] \arrow[r,"{f\times_{\modal f} f}"] & B\times_{\modal B} B \arrow[d,"\pi_1"] \\
A \arrow[r,"f"] & B
\end{tikzcd}
\end{equation*}
The bottom square is a pullback square by \cref{lem:etale_char}, and the outer rectangle is a pullback since both vertical composites are homotopic to the respective identity functions. Therefore the top square is a pullback.
\end{proof}

\begin{thm}
A map $f:A\to B$ is $0$-\'etale if and only if for each $a:A$ the restriction
\begin{equation*}
\begin{tikzcd}
\sm{x:A}\brck{a=x} \arrow[d,swap,"\proj 1"] \arrow[r,densely dotted,"f"] & \sm{y:B}\brck{f(a)=y} \arrow[d,"\proj 1"] \\
A \arrow[r,swap,"f"] & B
\end{tikzcd}
\end{equation*}
of $f$ to the connected component at $a$ of $A$ is an equivalence. 
\end{thm}

\begin{proof}
By \cref{lem:etale_char} and the fact that $\eqv{(\tproj{0}{a}=\tproj{0}{x})}{\brck{a=x}}$, it follows that $f$ is $0$-\'etale if and only if the square
\begin{equation*}
\begin{tikzcd}[column sep=huge]
\sm{a,x:A}\brck{a=x} \arrow[d,swap,"\pi_1"] \arrow[r,"\total{\brck{\apfunc{f}}}"] & \sm{b,y:B}\brck{b=y} \arrow[d,"\pi_1"] \\
A \arrow[r,swap,"f"] & B
\end{tikzcd}
\end{equation*}
is a pullback square. Furhtermore, this square is a pullback if and only if the induced map
\begin{equation*}
\Big(\sm{x:A}\brck{a=x}\Big)\to\Big(\sm{y:B}\brck{f(a)=y}\Big)
\end{equation*}
is an equivalence, for each $a:A$.
\end{proof}

\subsection{Modal descent}

The following theorem can be seen as a `modal flattening lemma'.
\begin{thm}\label{thm:etale_flattening}
Consider a pullback square
\begin{equation*}
\begin{tikzcd}
E' \arrow[d,swap,"{p'}"] \arrow[r,"g"] & E \arrow[d,"p"] \\
B' \arrow[r,swap,"f"] & B
\end{tikzcd}
\end{equation*}
with $H:f\circ p'\htpy p\circ g$, where $X$ and $Y$ are modal types. Then the square
\begin{equation*}
\begin{tikzcd}
\modal E' \arrow[r,"\tilde{g}"] \arrow[d,swap,"{\modal p'}"] & E \arrow[d,"p"] \\
\modal B \arrow[r,swap,"\tilde{f}"] & B
\end{tikzcd}
\end{equation*}
is a pullback square, where $\tilde{f}$ and $\tilde{g}$ are the unique extensions of $f$ and $g$ along the modal units of $B'$ and $E'$, respectively.
\end{thm}

\begin{proof}
Consider the diagram
\begin{equation*}
\begin{tikzcd}
E' \arrow[r,"{\mathsf{gap}(p',g,H)}"] \arrow[d,swap,"{p'}"] &[2.5em] \modal B'\times_{B} E \arrow[r,"\pi_2"] \arrow[d,swap,"\pi_1"] & E \arrow[d,"p"] \\
B' \arrow[r,swap,"\modalunit"] & \modal B' \arrow[r,swap,"\tilde{f}"] & B
\end{tikzcd}
\end{equation*}
In this diagram, the square on the right is a pullback by definition, and the outer rectangle is a pullback by assumption, so the square on the left is also a pullback. Therefore the gap map $E'\to \modal B'\times_B E$ is $\modal$-connected. Moreover, since the modal types are closed under pullbacks it follows that $\modal B'\times_B E$ is modal, and therefore it follows that $\pi_2:\modal B'\times_B E\to E$ is a modal map. Therefore the composite
\begin{equation*}
\begin{tikzcd}
E' \arrow[r,"{\mathsf{gap}(p',g,H)}"] &[2.5em] \modal B'\times_{B} E \arrow[r,"\pi_2"] & E 
\end{tikzcd}
\end{equation*}
factors $g$ as a $\modal$-connected map followed by a $\modal$-modal map. Of course, another such factorization is the composite $g\htpy \tilde{g}\circ\modalunit$. Since factorizations are unique, the claim follows.
\end{proof}

Using modal flattening we establish partial left exactness of the modality.

\begin{cor}\label{cor:etale_lex}
Consider a pullback square
\begin{equation*}
\begin{tikzcd}
A' \arrow[d,swap,"{f'}"] \arrow[r] & A \arrow[d,"f"] \\
B' \arrow[r] & B,
\end{tikzcd}
\end{equation*}
where $f$ is assumed to be $\modal$-\'etale. Then the square
\begin{equation*}
\begin{tikzcd}
\modal A' \arrow[d,swap,"{\modal f'}"] \arrow[r] & \modal A \arrow[d,"\modal f"] \\
\modal B' \arrow[r] & \modal B,
\end{tikzcd}
\end{equation*}
is again a pullback square.
\end{cor}

\begin{proof}
Since $f$ is assumed to be $\modal$-\'etale, the square on the right in the diagram
\begin{equation*}
\begin{tikzcd}
A' \arrow[r] \arrow[d,swap,"{f'}"] & A \arrow[r] \arrow[d,swap,"f"] & \modal A \arrow[d,"\modal f"] \\
B' \arrow[r] & B \arrow[r] & \modal B
\end{tikzcd}
\end{equation*}
is a pullback square. Therefore the outer rectangle is a pullback square by the pullback pasting lemma. Now the claim follows from modal flattening \cref{thm:etale_flattening}, using the outer rectangle.
\end{proof}

\begin{cor}\label{cor:etale_pb}
Consider a pullback square
\begin{equation*}
\begin{tikzcd}
E' \arrow[d,swap,"{p'}"] \arrow[r,"g"] & E \arrow[d,"p"] \\
B' \arrow[r,swap,"f"] & B
\end{tikzcd}
\end{equation*} 
and suppose that $p:E\to B$ is $\modal$-\'etale. Then $p':E'\to B'$ is $\modal$-\'etale.
\end{cor}

\begin{proof}
Consider the commuting cube
\begin{equation*}
\begin{tikzcd}
& E' \arrow[dr] \arrow[d] \arrow[dl] \\
\modal E' \arrow[d] & B' \arrow[dl] \arrow[dr] & E \arrow[dl,crossing over] \arrow[d] \\
\modal B' \arrow[dr] & \modal E \arrow[d] \arrow[from=ul,crossing over] & B \arrow[dl] \\
& \modal B. & \phantom{\modal E'}
\end{tikzcd}
\end{equation*}
The vertical squares on the back right and front right are pullback squares by assumption.
Then it follows from \cref{cor:etale_lex} that the vertical square on the front left is a pullback square.
Therefore the square on the back left is a pullback square by the pullback pasting property.
\end{proof}

\begin{defn}
Let $X$ be a type. We will define an operation
\begin{equation*}
\etmap:\Big(\sm{A:\UU_\modal}A\to\modal X\Big)\to\Big(\sm{Y:\UU}{g:Y\to X}\isetale(g)\Big)
\end{equation*}
\end{defn}

\begin{proof}[Construction]
Given a map $f:A\to \modal X$ we take the pullback
\begin{equation*}
\begin{tikzcd}
X\times_{\modal X}A \arrow[d,swap,"\pi_1"] \arrow[r,"\pi_2"] & A \arrow[d,"f"] \\
X \arrow[r,swap,"\modalunit"] & \modal X.
\end{tikzcd}
\end{equation*}
Then the map $\pi_1:X\times_{\modal X}A\to X$ is $\modal$-\'etale by \cref{lem:etale_modal,cor:etale_pb}.
\end{proof}

The following is a descent theorem for $\modal$-\'etale maps.

\begin{thm}[Modal descent]\label{thm:modal_descent}
For any modality $\modal$, and any type $X$, the operation
\begin{equation*}
\etmap:\Big(\sm{A:\UU_\modal}A\to\modal X\Big)\to\Big(\sm{Y:\UU}{g:Y\to X}\isetale(g)\Big)
\end{equation*}
is an equivalence.
\end{thm}

\begin{proof}
If $g:Y\to X$ is $\modal$-\'etale, then the square
\begin{equation*}
\begin{tikzcd}
Y \arrow[d,swap,"g"] \arrow[r,"\modalunit"] & \modal Y \arrow[d,"\modal g"] \\
X \arrow[r,swap,"\modalunit"] & \modal X
\end{tikzcd}
\end{equation*}
is a pullback square. Therefore $g:Y\to X$ is in the fiber of $\etmap$ at $\modal g : \modal Y\to\modal X$. 

It remains to show that for any map $f:A\to\modal X$ with modal domain, there is an equivalence $\eqv{A}{\modal (X\times_{\modal X} A)}$ such that the triangle
\begin{equation*}
\begin{tikzcd}
A \arrow[dr,swap,"f"] \arrow[rr,"\eqvsym"] & & \modal (X\times_{\modal X} A) \arrow[dl,"\modal(\etmap(f))"] \\
& \modal X
\end{tikzcd}
\end{equation*}
commutes. To see this, note that both $f\circ \pi_2$ and $\modal(\etmap(f))\circ \modalunit$ factor the same map as a $\modal$-connected map followed by a modal map, so the claim follows from uniqueness of factorizations.
\end{proof}

\begin{cor}
Suppose $P:X\to\UU_\modal$ is a family of modal types such that the projection map $\proj 1:\big(\sm{x:X}P(x)\big)\to X$ is $\modal$-\'etale. Then $P$ has a unique extension
\begin{equation*}
\begin{tikzcd}
X \arrow[d,swap,"\modalunit"] \arrow[r,"P"] & \UU_\modal. \\
\modal X \arrow[ur,densely dotted,swap,"\tilde{P}"] 
\end{tikzcd}
\end{equation*}
It follows that the square commuting square
\begin{equation*}
\begin{tikzcd}
\sm{x:X}P(x) \arrow[d,swap,"\proj 1"] \arrow[r] & \sm{t:\modal X}\tilde{P}(t) \arrow[d,"\proj 1"] \\
X \arrow[r,swap,"\modalunit"] & \modal X
\end{tikzcd}
\end{equation*}
is a pullback square. In particular the top map is $\modal$-connected, so this square is in fact a $\modal$-naturality square.
\end{cor}

\subsection{The reflective factorization system of a modality}

In this subsection we investigate the reflective factorization system assiciated to a modality, of which the right class is the class of $\modal$-\'etale maps. The left class is the class of \emph{$\modal$-equivalences}.
\begin{defn}
We say that a map $f:A\to B$ is an \define{$\modal$-equivalence} if $L f:\modal A\to \modal B$ is an equivalence.
\end{defn}

\begin{rmk}
The difference between the notions of $\modal$-equivalences and $\modal$-connected maps is best explained by an example. In the case of $n$-truncation, the $n$-equivalences are precisely the maps that induce isomorphisms on the first $n$ homotopy groups. The $n$-connected maps are the maps that induce isomorphisms on the first $n$ homotopy groups, and moreover induce an epimorphism on the $(n+1)$-st homotopy group. 

We also note that the $n$-equivalences are not stable under pullbacks, whereas the $n$-connected maps are. Consider for instance the pullback square
\begin{equation*}
\begin{tikzcd}
\loopspace {\sphere{n+1}} \arrow[r] \arrow[d] & \unit \arrow[d] \\
\unit\arrow[r] & \sphere{n+1}
\end{tikzcd}
\end{equation*}
Here the map on the right is an $n$-equivalence, since $\sphere{n+1}$ is $n$-connected. However, the map on the left is not an $n$-equivalence, since the $n$-th homotpy group of $\loopspace{\sphere{n+1}}$ is not trivial: it is the $(n+1)$-st homotopy group of $\sphere{n+1}$, which is $\Z$.
\end{rmk}

\begin{defn}
The \define{reflective factorization system} associated to a modality $\modal$ consists of the $\modal$-equivalences as the left class, and the $\modal$-\'etale maps as the right class.
\end{defn}

Our goal in this section is to show that the reflective factorization system associated to a modality is an orthogonal factorization system.

\begin{lem}\label{lem:3for2_mequiv}
The $\modal$-equivalences satisfy the 3-for-2 property: given a commuting triangle
\begin{equation*}
\begin{tikzcd}
A \arrow[rr,"h"] \arrow[dr,swap,"f"] & & B \arrow[dl,"g"] \\
& C,
\end{tikzcd}
\end{equation*}
if any two of $f$, $g$, and $h$ are $\modal$-equivalences, then so is the third.
\end{lem}

\begin{proof}
Apply $\modal$ to the commuting triangle, and use the 3-for-2 property of equivalences.
\end{proof}

\begin{lem}\label{lem:modal_equivalence}
For a map $f : A \to B$ the following are equivalent:
\begin{enumerate}
\item $f$ is an $\modal$-equivalence.
\item For any modal type $X$, the precomposition map
\begin{equation*}
\precomp{f} : (B \to X) \to (A \to X)
\end{equation*}
is an equivalence.
\end{enumerate}
\end{lem}

\begin{proof} 
Suppose first that $f$ is an $\modal$-equivalence, and let $X$ be $\modal$-modal. Then the square
\begin{equation*}
\begin{tikzcd}
X^B \arrow[r,"\precomp{f}"] \arrow[d,swap,"\precomp{\eta}"] & X^A \arrow[d,"\precomp{\eta}"] \\
X^{\modal B} \arrow[r,swap,"\precomp{\modal f}"] & X^{\modal A}
\end{tikzcd}
\end{equation*}
commutes. In this square the two vertical maps are equivalences by the universal property of modalization, and the bottom map is an equivalence since $\modal f$ is an equivalence. Therefore the map $\precomp{f}:X^B\to X^A$ is an equivalence, as desired.

Conversely, assume that $\precomp{f} : X^B \to X^A$ is an equivalence for every $\modal$-modal type $X$. By the square above it follows that $\precomp{\modal f}:X^{\modal B}\to X^{\modal A}$ is an equivalence for every $\modal$-modal type $X$. The fiber of $\modal A^{\modal B}\to \modal A^{\modal A}$ at $\idfunc:\modal A\to \modal A$ is contractible, so we obtain a retraction $g$ of $\modal f$. To see that $g$ is also a section observe that the fiber of $\modal B^{\modal B}\to \modal B^{\modal A}$ at $\modal f$ is contractible. This fiber contains $(\idfunc[\modal B],\refl{\modal f})$. However, we also have an identification $p:\precomp{\modal f}(\modal f\circ g)=\modal f$, since
\begin{equation*}
\precomp{\modal f}(\modal f\circ g)\jdeq (\modal f \circ g)\circ \modal f\jdeq \modal f \circ (g\circ \modal f) = \modal f. 
\end{equation*}
Therefore $(\modal f\circ g,p)$ is in the fiber of $\precomp{\modal f}:\modal B^{\modal B}\to \modal B^{\modal A}$ at $\modal f$. By the contractibility of the fibers it follows that $(\modal f\circ g,p)=(\idfunc[\modal B],\refl{\modal f})$, so it follows that $\modal f\circ g=\idfunc[\modal B]$. In other words, $g$ is both a retraction and a section of $\modal f$, so $\modal f$ is an equivalence.
\end{proof}

\begin{cor}\label{cor:mequiv_mconn}
Every $\modal$-connected map is a $\modal$-equivalence.
\end{cor}

\begin{defn}
Let $f:A\to B$ be a map. We define
\begin{align*}
[A]_f & \defeq B \times_{\modal B}\modal A
\intertext{and we define the maps}
\mathsf{et}(f) & : [A]_f \to B \\
\bar{\eta} : A \to [A]_f
\end{align*}
by the universal property of pullbacks, as indicated in the following diagram
\begin{equation*}
\begin{tikzcd}
A \arrow[ddr,bend right=15,swap,"f"] \arrow[drr,bend left=15,"\modalunit"] \arrow[dr,"\bar{\eta}" description] \\
& {[A]_f} \arrow[d,"\mathsf{et}(f)"] \arrow[r,"\pi_2"] & \modal A \arrow[d,"\modal f"] \\
& B \arrow[r,swap,"\modalunit"] & \modal B.
\end{tikzcd}
\end{equation*}
\end{defn}

\begin{lem}\label{lem:rfs_factor}
For every map $f:A\to B$, the map $\bar{\eta}:A\to[A]$ is a $\modal$-equivalence, and the map $\mathsf{et}(f)$ is $\modal$-\'etale.
\end{lem}

\begin{proof}
The map $\mathsf{et}(f)$ is a pullback of a map between modal types, so it is $\modal$-\'etale by \cref{cor:etale_pb}. Furthermore, the map $\pi_2:[A] \to \modal A$ is a pullback of a $\modal$-connected map, so it is $\modal$-connected. It follows from \cref{cor:mequiv_mconn} that it is a $\modal$-equivalence. Since the modal unit $\modalunit :A\to\modal A$ is also $\modal$-connected, and therefore a $\modal$-equivalence, we obtain by the 3-for-2 property of $\modal$-equivalences established in \cref{lem:3for2_mequiv} that the gap map is also a $\modal$-equivalence.
\end{proof}

\begin{lem}\label{lem:rfs_orthogonal}
The class of $\modal$-equivalences is left orthogonal to the class of $\modal$-\'etale maps.
\end{lem}

\begin{proof}
We have to show that for every $\modal$-equivalence $i:A\to B$, and every $\modal$-\'etale map $f:X\to Y$, the square
\begin{equation*}
\begin{tikzcd}
X^B \arrow[r] \arrow[d] & Y^B \arrow[d] \\
X^A \arrow[r] & Y^A
\end{tikzcd}
\end{equation*}
is a pullback square. Consider the commuting cube
\begin{equation*}
\begin{tikzcd}
& X^B \arrow[dl] \arrow[d] \arrow[dr] \\
(\modal X)^B \arrow[d] & X^A \arrow[dl] \arrow[dr] & Y^B \arrow[d] \arrow[dl,crossing over] \\
(\modal X)^A \arrow[dr] & (\modal Y)^B \arrow[from=ul,crossing over] \arrow[d] & Y^A \arrow[dl] \\
& (\modal Y)^A & \phantom{(\modal X)^B}
\end{tikzcd}
\end{equation*}
In this cube the top and bottom squares are pullback by the assumption that $f$ is $\modal$-\'etale and the fact that exponents of pullback squares are again pullback squares. Furthermore, the square in the front left is pullback, because the two vertical maps are equivalences by the assumption that $i:A\to B$ is a $\modal$-equivalence. Therefore we conclude that the square in the back right is also a pullback square, as desired.
\end{proof}

\begin{cor}
For any map $f:X\to Y$, the type of factorizations into a $\modal$-connected map followed by a $\modal$-\'etale map is contractible.
\end{cor}

The class of $\modal$-\'etale morphisms into a given type $A$, thought of as objects of the slice category $\UU/A$, form a reflective subuniverse in the following sense.

\begin{thm}
Let $f:A\to X$ be a map. Then the pre-composition function
\begin{equation*}
\mathrm{hom}_X(\mathsf{et}(f),e)\to \mathrm{hom}_X(f,e)
\end{equation*}
is an equivalence for every \'etale map $e:B\to X$. 
\end{thm}

\begin{proof}
Let $e:B\to X$ be a $\modal$-\'etale map. Then the square
\begin{equation*}
\begin{tikzcd}
B^{[A]} \arrow[r,"\blank\circ\eta_f"] \arrow[d,swap,"e\circ\blank"] & B^A \arrow[d,"e\circ\blank"] \\
X^{[A]} \arrow[r,swap,"\blank\circ\eta_f"] & X^A
\end{tikzcd}
\end{equation*}
is a pullback square by \cref{lem:rfs_factor,lem:rfs_orthogonal}. Therefore we have a fiberwise equivalence
\begin{equation*}
\prd{i:[A]\to X} \fib{e\circ\blank}{i}\to \fib{e\circ\blank}{i\circ\eta_f}
\end{equation*}
by \cref{cor:pb_fibequiv}. Now the claim follows, since we have a commuting square
\begin{equation*}
\begin{tikzcd}
\fib{e\circ\blank}{i} \arrow[r] \arrow[d,swap,"\eqvsym"] & \fib{e\circ\blank}{i\circ\eta_f} \arrow[d,"\eqvsym"] \\
\mathrm{hom}_X(\mathsf{et}(f),e) \arrow[r] & \mathrm{hom}_X(f,e)
\end{tikzcd}
\end{equation*}
with equivalences on both sides, for each $i:[A]\to X$.
\end{proof}
 
\section{The reflective factorization system for the reflexive coequalizer}

\subsection{\texorpdfstring{$\Delta$}{Δ}-\'etale maps}


\begin{defn}
Let $f:\mathsf{rGph}(\mathcal{A},\mathcal{B})$ be a morphism of reflexive graphs. We say that $f$ is \define{$\Delta$-\'etale} if the square
\begin{equation*}
\begin{tikzcd}
\mathcal{A} \arrow[d,swap,"f"] \arrow[r] & \Delta(\rcoeq(\mathcal{A})) \arrow[d,"\Delta(\rcoeq(f))"] \\
\mathcal{B} \arrow[r] & \Delta(\rcoeq(\mathcal{B}))
\end{tikzcd}
\end{equation*}
is a pullback square of reflexive graphs. We write $\mathsf{is\usc{}etale}_\Delta(f)$ for the proposition that $f$ is $\Delta$-\'etale, and we also write $\mathcal{R}^\Delta$ for the class of $\Delta$-\'etale morphisms of reflexive graphs.
\end{defn}

\begin{thm}\label{thm:etale_fibration}
Consider a morphism $f:\mathsf{rGph}(\mathcal{A},\mathcal{B})$ of reflexive graphs. The following are equivalent:
\begin{enumerate}
\item The morphism $f$ is a fibration in the sense of \cref{defn:graph_fibration}.
\item The morphism $f$ is $\Delta$-\'etale.
\end{enumerate}
\end{thm}

\begin{proof}
Consider the commuting diagram
\begin{equation*}
\begin{tikzcd}
& & \pts{\tilde A} \arrow[d] \arrow[dr] \\
& \edg{\tilde A} \arrow[dl] \arrow[d] \arrow[ur] & \pts{\tilde B} \arrow[dl] \arrow[dr] & \rcoeq(\mathcal{A}) \arrow[d] \\
\pts{\tilde A} \arrow[d] & \edg{\tilde B} \arrow[dl] \arrow[dr] \arrow[ur] & \rcoeq(\mathcal{A}) \arrow[from=ul,crossing over] \arrow[dl,crossing over] \arrow[d] \arrow[ur,crossing over] & \rcoeq(\mathcal{B}) \\
\pts{\tilde B} \arrow[dr] & \rcoeq(\mathcal{A}) \arrow[d] \arrow[from=ul,crossing over] & \rcoeq(\mathcal{B}) \arrow[dl] \arrow[ur] \\
\phantom{\rcoeq(\mathcal{B})} & \rcoeq(\mathcal{B})
\end{tikzcd}
\end{equation*}
If $f$ is $\Delta$-\'etale, then the three parallel vertical squares are pullback squares, hence so are the two squares on the back left side. This shows that (ii) implies (i). 

Now suppose that $f$ is a fibration, or equivalently, that $f$ is cartesian. Then the map $\rcoeq(f)$ is the unique map such that the naturality squares are pullback squares. In particular, $f$ is a $\Delta$-\'etale map.
\end{proof}

\begin{cor}
Let $\mathcal{B}$ be a family of reflexive graphs over $\mathcal{A}$. The following are equivalent:
\begin{enumerate}
\item The family $\mathcal{B}$ is equifibered.
\item The morphism $\proj 1 : \mathsf{rGph}(\msm{\mathcal{A}}{\mathcal{B}},\mathcal{A})$ is $\Delta$-\'etale.
\end{enumerate}
\end{cor}

\begin{prp}
Consider a pullback square
\begin{equation*}
\begin{tikzcd}
\mathcal{A}' \arrow[r] \arrow[d,swap,"{f'}"] & \mathcal{A} \arrow[d,"f"] \\
\mathcal{B}' \arrow[r] & \mathcal{B}
\end{tikzcd}
\end{equation*}
of reflexive graphs, and suppose that $f$ is a fibration. Then the square
\begin{equation*}
\begin{tikzcd}
\rcoeq(\mathcal{A}') \arrow[r] \arrow[d,swap,"\rcoeq({f'})"] & \rcoeq(\mathcal{A}) \arrow[d,"\rcoeq(f)"] \\
\rcoeq(\mathcal{B}') \arrow[r] & \rcoeq(\mathcal{B})
\end{tikzcd}
\end{equation*}
is again a pullback square.
\end{prp}

\begin{proof}
Consider the diagram
\begin{equation*}
\begin{tikzcd}
\mathcal{A}' \arrow[r] \arrow[d,swap,"{f'}"] & \mathcal{A} \arrow[d,"f"] \arrow[r,"\mathsf{constr}"] &[1ex] \Delta(\rcoeq(\mathcal{A})) \arrow[d,"\Delta(\rcoeq(f))"] \\
\mathcal{B}' \arrow[r] & \mathcal{B} \arrow[r,"\mathsf{constr}"'] & \Delta(\rcoeq(\mathcal{B}))
\end{tikzcd}
\end{equation*}
of reflexive graphs. Since $f$ is assumed to be a fibration, we obtain by \cref{thm:etale_fibration} that the square on the right is a pullback square. Furthermore, since the left square is a pullback square by assumption, it follows that the outer rectangle is again a pullback square. Hence the assertion follows from \cref{thm:rcoeq_cartesian}.
\end{proof}

\begin{prp}
Consider a pullback square
\begin{equation*}
\begin{tikzcd}
\mathcal{A}' \arrow[r] \arrow[d,swap,"{f'}"] & \mathcal{A} \arrow[d,"f"] \\
\mathcal{B}' \arrow[r] & \mathcal{B}
\end{tikzcd}
\end{equation*}
of reflexive graphs, and suppose that $f$ is a fibration. Then $f'$ is a fibration.
\end{prp}

\begin{proof}
Consider the cube
\begin{equation*}
\begin{tikzcd}
& \mathcal{A}' \arrow[dr] \arrow[d] \arrow[dl] \\
\Delta(\rcoeq(\mathcal{A}')) \arrow[d] & \mathcal{B}' \arrow[dl] \arrow[dr] & \mathcal{A} \arrow[dl,crossing over] \arrow[d] \\
\Delta(\rcoeq(\mathcal{B}')) \arrow[dr] & \Delta(\rcoeq(\mathcal{A})) \arrow[d] \arrow[from=ul,crossing over] & \mathcal{B} \arrow[dl] \\
& \Delta(\rcoeq(\mathcal{B})). & \phantom{\Delta(\rcoeq(\mathcal{A}'))}
\end{tikzcd}
\end{equation*}
Then the two squares in the front and the square in the back right are pullback squares, so it follows that the square in the back left is a pullback square.
\end{proof}

\subsection{The reflective factorization system of discrete graphs}

\begin{defn}
Let $f:\mathsf{rGph}(\mathcal{B},\mathcal{A})$ be a morphism of reflexive graphs. We say that $f$ is a \define{$\Delta$-equivalence} if the map
\begin{equation*}
\begin{tikzcd}
\rcoeq(f):\rcoeq(\mathcal{B})\to \rcoeq(\mathcal{A})
\end{tikzcd}
\end{equation*}
is an equivalence. We also write $\mathcal{L}^\Delta$ for the class of $\Delta$-equivalences of reflexive graphs.
\end{defn}

\begin{lem}
The $\Delta$-equivalences satisfy the 3-for-2 property.
\end{lem}

\begin{prp}
Let $f:\mathsf{rGph}(\mathcal{B},\mathcal{A})$ be a morphism of reflexive graphs. The following are equivalent:
\begin{enumerate}
\item $f$ is a $\Delta$-equivalence.
\item For any type $X$, the map
\begin{equation*}
\mathsf{rGph}(\mathcal{A},\Delta(X))\to\mathsf{rGph}(\mathcal{B},\Delta(X))
\end{equation*}
is an equivalence.
\end{enumerate}
\end{prp}

\begin{defn}
Let $f:\mathsf{rGph}(\mathcal{B},\mathcal{A})$ be a morphism of reflexive graphs. We define the morphisms
\begin{align*}
\mathsf{et}_\Delta(f) & : \mathsf{rGph}(\mathcal{B}^{\eqvsym},\mathcal{A}) \\
\mathsf{\eta}_f & : \mathsf{rGph}(\mathcal{B},\mathcal{B}^{\eqvsym})
\end{align*}
by the universal property of pullbacks, as indicated in the following diagram
\begin{equation*}
\begin{tikzcd}
\mathcal{B} \arrow[ddr,bend right=15] \arrow[drr,bend left=15] \arrow[dr,densely dotted,"\eta_f" near end] \\
& \mathcal{B}^{\eqvsym} \arrow[dr,phantom,"\lrcorner" {very near start,xshift=-1ex}] \arrow[d,"\mathsf{et}_\Delta(f)"] \arrow[r] & \Delta(\rcoeq(\mathcal{B})) \arrow[d] \\[1ex]
& \mathcal{A} \arrow[r,swap,"\mathsf{constr}"] & \Delta(\rcoeq(\mathcal{A}))
\end{tikzcd}
\end{equation*}
The morphism $\mathsf{et}_\Delta(f)$ is called the \define{equifibrant replacement} of $f$, and the morphism $\eta_f$ is called the \define{unit} of the equifibrant replacement.
\end{defn}

\begin{lem}
For any morphism $f:\mathsf{rGph}(\mathcal{B},\mathcal{A})$, the unit $\eta_f$ of the equifibrant replacement of $f$ is a $\Delta$-equivalence. 
\end{lem}

\begin{proof}
By \cref{lem:rcoeq_flattening}, it follows that the square
\begin{equation*}
\begin{tikzcd}
\rcoeq(\mathcal{B}^{\eqvsym}) \arrow[d] \arrow[r] & \rcoeq(\Delta(\rcoeq(\mathcal{B}))) \arrow[d] \\
\rcoeq(\mathcal{A}) \arrow[r] & \rcoeq(\Delta(\rcoeq(\mathcal{A})))
\end{tikzcd}
\end{equation*}
is a pullback square. Since the bottom map is an equivalence of reflexive graphs, it follows that the top map is an equivalence of reflexive graphs. In other words, we have shown that the map
\begin{equation*}
\begin{tikzcd}
\mathcal{B}^{\eqvsym} \arrow[r] & \Delta(\rcoeq(\mathcal{B}))
\end{tikzcd}
\end{equation*}
is a $\Delta$-equivalence. Of course, the morphism $\mathsf{constr}:\mathcal{B}\to\Delta(\rcoeq(\mathcal{B}))$ is also a $\Delta$-equivalence, so the claim follows by the 3-for-2 property of $\Delta$-equivalences.
\end{proof}

\begin{thm}
The pair $(\mathcal{L}^\Delta,\mathcal{R}^\Delta)$ forms an orthogonal factorization system of reflexive graphs.
\end{thm}

\begin{proof}
It follows from \cref{thm:fibration_pullback,lem:D_equiv_pullback} that every morphism of reflexive graphs factors as a $\Delta$-equivalence followed by a $\Delta$-\'etale map. Therefore it remains to show that the class of $\Delta$-equivalences is left orthogonal to the class of $\Delta$-\'etale maps.

We have to show that for every $\Delta$-equivalence $i:\mathcal{A}\to\mathcal{B}$, and every $\Delta$-\'etale morphism $f:\mathcal{X}\to\mathcal{Y}$, the square
\begin{equation*}
\begin{tikzcd}
\mathcal{X}^{\mathcal{B}} \arrow[r] \arrow[d] & \mathcal{Y}^{\mathcal{B}} \arrow[d] \\
\mathcal{X}^{\mathcal{A}} \arrow[r] & \mathcal{Y}^{\mathcal{A}}
\end{tikzcd}
\end{equation*}
is a pullback square. Consider the commuting cube
\begin{equation*}
\begin{tikzcd}
& \mathcal{X}^{\mathcal{B}} \arrow[dl] \arrow[d] \arrow[dr] \\
\Delta(\rcoeq(\mathcal{X}))^{\mathcal{B}} \arrow[d] & \mathcal{X}^{\mathcal{A}} \arrow[dl] \arrow[dr] & \mathcal{Y}^{\mathcal{B}} \arrow[d] \arrow[dl,crossing over] \\
\Delta(\rcoeq(\mathcal{X}))^{\mathcal{A}} \arrow[dr] & \Delta(\rcoeq(\mathcal{Y}))^{\mathcal{B}} \arrow[from=ul,crossing over] \arrow[d] & \mathcal{Y}^{\mathcal{A}} \arrow[dl] \\
& \Delta(\rcoeq(\mathcal{Y}))^{\mathcal{A}} & \phantom{\Delta(\rcoeq(\mathcal{X}))^{\mathcal{B}}}
\end{tikzcd}
\end{equation*}
In this cube the top and bottom squares are pullback by the assumption that $f$ is $\Delta$-\'etale and the fact that exponents of pullback squares are again pullback squares. Furthermore, the square in the front left is pullback, because the two vertical maps are equivalences by the assumption that $i:\mathcal{A}\to\mathcal{B}$ is a $\Delta$-equivalence. Therefore we conclude that the square in the back right is also a pullback square, as desired.
\end{proof}

\begin{thm}
For any morphism $f:\mathsf{rGph}(\mathcal{B},\mathcal{A})$, the pre-composition map
\begin{equation*}
\mathsf{rGph}_{\mathcal{A}}(\mathsf{et}_\Delta(f),e)\to \mathrm{hom}_{\mathcal{A}}(f,e)
\end{equation*}
is an equivalence for every $\Delta$-\'etale map $e:\mathcal{C}\to\mathcal{A}$.
\end{thm}

\subsection{Equifibrant replacement}

\begin{defn}
Let $\mathcal{B}$ and $\mathcal{C}$ be families of reflexive graphs over $\mathcal{A}$. We define the type
\begin{equation*}
\mathsf{rGph}_{\mathcal{A}}(\mathcal{B},\mathcal{C})
\end{equation*}
of morphisms of reflexive graphs \define{over} $\mathcal{A}$ to consist of triples $(\pts{f},\edg{f},\rfx{f})$ consisting of
\begin{align*}
\pts{f}(x) & : \pts{B}(x)\to \pts{C}(x) \\
\edg{f}(e) & : \prd{u:\pts{B}(x)}{v:\pts{B}(y)} \edg{B}(e,u,v)\to \edg{C}(e,\pts{f}(u),\pts{f}(v)) \\
\rfx{f}(x) & : \prd{u:\pts{B}(x)} \edg{f}(\rfx{\mathcal{B}}(x,u))=\rfx{\mathcal{C}}(x,\pts{f}(u))
\end{align*}
\end{defn}

\begin{lem}
Let $\mathcal{B}$ and $\mathcal{C}$ be families of reflexive graphs over $\mathcal{A}$. Then there is an equivalence
\begin{equation*}
\eqv{\mathsf{rGph}_{\mathcal{A}}(\mathcal{B},\mathcal{C})}{\sm{f:\mathsf{rGph}(\msm{\mathcal{A}}{\mathcal{B}},\msm{\mathcal{A}}{\mathcal{B}})} \proj 1 = \proj 1 \circ f}
\end{equation*}
\end{lem}

\begin{defn}
Let $\mathcal{B}$ be a family of reflexive graphs over $\mathcal{A}$. We define the equifibrant replacement $\mathsf{EqF}(\mathcal{B})$ of $\mathcal{B}$ by
\begin{equation*}
\mathsf{EqF}(\mathcal{B}) \defeq \mathsf{equifib\usc{}fam}(\fibf{\rcoeq(\proj 1)}),
\end{equation*}
and we define the morphism $\eta_{\mathcal{B}}:\mathsf{rGph}_{\mathcal{A}}(\mathcal{B},\mathsf{EqF}(\mathcal{B}))$. 
\end{defn}

\begin{thm}\label{thm:eqf_initial}
Consider a family $\mathcal{B}$ of reflexive graphs over $\mathcal{A}$, and let $\mathcal{E}$ be an equifibered family over $\mathcal{A}$. Then the pre-composition operation
\begin{equation*}
\mathsf{rGph}_{\mathcal{A}}(\mathsf{EqF}(\mathcal{B}),\mathcal{E})\to \mathsf{rGph}_{\mathcal{A}}(\mathcal{B},\mathcal{E})
\end{equation*}
is an equivalence for every equifibered family $\mathcal{E}$ over $\mathcal{A}$.
\end{thm}

\section{Equifibrant replacement for other homotopy colimits}

\subsection{Equifibrant replacement for diagrams over graphs}

\begin{defn}
Let $\tau:\mathcal{D}'\to\mathcal{D}$ be a natural transformation of diagrams over a reflexive graph $\mathcal{A}$. 
\begin{enumerate}
\item We say that $\tau$ is a \define{weak equivalence} if it induces an equivalence
\begin{equation*}
\tfcolim(\tau):\tfcolim(\mathcal{D}')\to\tfcolim(\mathcal{D}).
\end{equation*}
We write $\mathcal{W}$ for the class of weak equivalences.
\item We say that $\tau$ is \define{\'etale} if the square
\begin{equation*}
\begin{tikzcd}
\msm{\mathcal{A}}{\mathcal{D}'} \arrow[d,swap,"\tau"] \arrow[r] & \Delta(\tfcolim(\mathcal{D}')) \arrow[d,"\tfcolim(\tau)"] \\
\msm{\mathcal{A}}{\mathcal{D}} \arrow[r] & \Delta(\tfcolim(\mathcal{D}))
\end{tikzcd}
\end{equation*}
is a pullback square of reflexive graphs. We write $\mathcal{R}$ for the class of \'etale maps.
\end{enumerate}
\end{defn}

\begin{thm}
A natural transformation $\tau:\mathcal{D}'\to\mathcal{D}$ is \'etale if and only if it is cartesian.
\end{thm}

\begin{thm}
The pair $(\mathcal{W},\mathcal{R})$ forms an orthogonal factorization system. 
\end{thm}

\subsection{Identity types of reflexive coequalizers}

\begin{defn}
Consider a reflexive graph $\mathcal{A}$ with a base point $a:A_0$. We define the \define{universal $\Delta$-bundle} $\mathcal{E}(a)$ over $\mathcal{A}$ at $a$ to be the equifibered family over $\mathcal{A}$ corresponding to the equifibrant replacement of the morphism $a:\mathsf{rGph}(\unit,\mathcal{A})$ corresponding to $a:A_0$. 
\end{defn}

\begin{thm}
Consider a reflexive graph $\mathcal{A}$ with a base point $a:A_0$. Then there are equivalences
\begin{equation*}
\pts{\mathcal{E}(a)}(x) \eqvsym (\mathsf{constr}_0(a)=\mathsf{constr}_0(x))
\end{equation*}
In particular, we have an equivalence $\eqv{\pts{\mathcal{E}(a)}(a)}{\loopspace{\rcoeq(\mathcal{A})}}$. 
\end{thm}

In the following theorem we establish the universal property of the identity type of $\rcoeq(\mathcal{A})$ as the initial reflexive relation $I$ on 

\begin{thm}
Consider a reflexive graph $\mathcal{A}$, and let
\begin{align*}
R & : A_0 \to A_0 \to\UU, \\
\rho & : \prd{x:A_0}R(x,x)
\end{align*}
be a reflexive relation on $A_0$ equipped with a `composition' operation
\begin{equation*}
\mu:\prd{a,x,y:\pts{A}} \edg{A}(x,y) \to (\eqv{R(a,x)}{R(a,y)}).
\end{equation*}
satisfying the unit law
\begin{align*}
\mathsf{left\usc{}unit}_\mu & : \prd{x,y:\pts{A}}{r:R(x,y)} \mu(\rfx{\mathcal{A}}(y),r)=r.
\end{align*}
Then there is a uniqe extension
\begin{equation*}
\begin{tikzcd}
\mathcal{A} \arrow[r,"\bar{\mu}"] \arrow[d] & (A_0,R,\rho) \\
k(\mathsf{constr}_0) \arrow[ur,densely dotted]
\end{tikzcd}
\end{equation*}
of the morphism $\bar{\mu}:\mathsf{rGph}(\mathcal{A},(A_0,R,\rho))$ consisting of
\begin{align*}
\pts{\bar{\mu}} & \defeq \mathsf{constr}_0 \\
\edg{\bar{\mu}}(e) & \defeq \mu(e,\rho(x)) \\
\rfx{\bar{\mu}}(x) & \defeq \mathsf{left\usc{}unit}_\mu(\rho(x)),
\end{align*}
where $k(\mathsf{constr}_0)$ is the pre-kernel of the function $\mathsf{constr}_0:A_0\to\rcoeq(\mathcal{A})$.
\end{thm}

\begin{cor}
The loop space of the suspension $\susp X$ of a pointed type $X$ is the initial pointed type $Y$ equipped with a pointed map 
\begin{equation*}
X\to_\ast (\eqv{Y}{Y}).
\end{equation*}
In particular, the loop space of the $(n+1)$-sphere is the initial pointed type $Y$ equipped with a a pointed map $\sphere{n}\to (\eqv{Y}{Y})$, or equivalently, an $(n+1)$-loop $\loopspace[n+1]{\mathrm{BAut}(Y)}$. Even more in particular, the loop space of the $2$-sphere is the initial pointed type $Y$ equipped with a homotopy $\idfunc[Y]\htpy\idfunc[Y]$. 
\end{cor}

\section{Equifibrant replacement for sequential colimits}\label{sec:seqcolim_eqf}

\begin{defn}
Let $\mathcal{B}$ be a sequential family over $\mathcal{A}$, and consider $x:A_n$. We write $\mathcal{B}_n[x]$ for the type sequence
\begin{equation*}
\begin{tikzcd}
B_n(x) \arrow[r] & B_{n+1}(f_n(x)) \arrow[r] & B_{n+2}(f_{n+1}(x)) \arrow[r] & \cdots.
\end{tikzcd}
\end{equation*}
\end{defn}

\begin{defn}
Let $\mathcal{B}$ be a sequential family over $\mathcal{A}$. We will define an equifibered sequential family $\square\mathcal{B}$ over $\mathcal{A}$ equipped with a morphism 
\begin{equation*}
\mathsf{hom}_{\mathcal{A}}(\mathcal{B},\square\mathcal{B})
\end{equation*}
\end{defn}

\begin{proof}[Construction]
We define
\begin{equation*}
\square B : \prd{n:\N}A_n\to\UU
\end{equation*}
by $\square B_n(x)\defeq \tfcolim \mathcal{B}_n[x]$. Next, we have to construct a fiberwise equivalence
\begin{equation*}
g_\infty : \prd{n:\N}{x:A_n} \square B_n \eqvsym \square B_{n+1}
\end{equation*}
\end{proof}

\begin{thm}\label{thm:colim_fib}
The equifibered family $\square\mathcal{B}$ over $\mathcal{A}$ satisfies the universal property that
\begin{equation*}
\mathsf{hom}_{\mathcal{A}}(\square\mathcal{B},\mathcal{E})\to\mathrm{hom}_{\mathcal{A}}(\mathcal{B},\mathcal{E})
\end{equation*}
is an equivalence for any equifibered family $\mathcal{E}$ over $\mathcal{A}$. 
\end{thm}

\begin{proof}
Let $\mathcal{E}$ be an equifibered family over $\mathcal{A}$, and consider a morphism $\mathrm{hom}_{\mathcal{A}}(\mathcal{B},\mathcal{E})$. First we show that the type 
\begin{equation*}
\mathsf{seq\usc{}cocone}_{\mathcal{B}_n[x]}(E_n(x)) \to \mathsf{nat}(\mathcal{B}_n[x],\mathcal{E}_n[x])
\end{equation*}
is an equivalence
\end{proof}

\begin{cor}\label{thm:colim_sigma}
Let $\mathcal{B}$ be a sequential family over $\mathcal{A}$. Then we have a commuting triangle
\begin{equation*}
\begin{tikzcd}[column sep=tiny]
\mathsf{colim}\Big(\msm{\mathcal{A}}{\mathcal{B}}\Big) \arrow[rr] \arrow[dr] & & \sm{x:A_\infty} B_\infty(x) \arrow[dl] \\
\phantom{\sm{x:A_\infty} B_\infty(x)} & A_\infty & \phantom{\mathsf{colim}\Big(\msm{\mathcal{A}}{\mathcal{B}}\Big)}
\end{tikzcd}
\end{equation*}
in which the top map is an equivalence.
\end{cor}

\begin{cor}\label{thm:colim_id}
Consider a type sequence
\begin{equation*}
\begin{tikzcd}
A_0 \arrow[r] & A_1 \arrow[r] & A_2 \arrow[r] & \cdots.
\end{tikzcd}
\end{equation*}
Then the canonical map
\begin{equation*}
\mathsf{colim}(f^n(x)= f^n(y)) \to (\mathsf{seq\usc{}in}_0(x)=\mathsf{seq\usc{}in}_0(y))
\end{equation*}
is an equivalence for every $x,y:A_0$.
\end{cor}

\begin{thm}
Consider a sequence
\begin{equation*}
\begin{tikzcd}[column sep=small,row sep=small]
& C_0 \arrow[dl] \arrow[dd] \arrow[rr] & & C_1 \arrow[dl] \arrow[dd] \arrow[rr] & & C_2 \arrow[dl] \arrow[dd] \arrow[rr] & & \cdots \\
A_0 \arrow[dd] \arrow[rr,crossing over] & & A_1 \arrow[rr,crossing over] & & A_2 \arrow[rr,crossing over] & & \cdots \\
& B_0 \arrow[dl] \arrow[rr] & & B_1 \arrow[dl] \arrow[rr] & & B_2 \arrow[dl] \arrow[rr] & & \cdots \\
X_0 \arrow[rr] & \phantom{B_2} & X_1 \arrow[rr] \arrow[from=uu,crossing over] & \phantom{B_2} & X_2 \arrow[rr] \arrow[from=uu,crossing over] & \phantom{B_2} & \cdots
\end{tikzcd}
\end{equation*}
of pullback squares. Then the sequential colimit
\begin{equation*}
\begin{tikzcd}
C_\infty \arrow[r] \arrow[d] & B_\infty \arrow[d] \\
A_\infty \arrow[r] & X_\infty
\end{tikzcd}
\end{equation*}
is again a pullback square.
\end{thm}

\begin{cor}\label{thm:colim_fiberseq}
Consider a sequence of fiber sequences
\begin{equation*}
\begin{tikzcd}
F_0 \arrow[d] \arrow[r] & F_1 \arrow[d] \arrow[r] & F_2 \arrow[d] \arrow[r] & \cdots \\
E_0 \arrow[d] \arrow[r] & E_1 \arrow[d] \arrow[r] & E_2 \arrow[d] \arrow[r] & \cdots \\
B_0 \arrow[r] & B_1 \arrow[r] & B_2 \arrow[r] & \cdots
\end{tikzcd}
\end{equation*}
in which all maps and homotopies are assumed to be pointed. Then the colimit
\begin{equation*}
\begin{tikzcd}
F_\infty \hookrightarrow E_\infty \twoheadrightarrow B_\infty
\end{tikzcd}
\end{equation*}
is again a fiber sequence.
\end{cor}

\begin{cor}
Consider a type sequence
\begin{equation*}
\begin{tikzcd}
A_0 \arrow[r] & A_1 \arrow[r] & A_2 \arrow[r] & \cdots
\end{tikzcd}
\end{equation*}
of pointed types (and pointed maps between them). Then the canonical map
\begin{equation*}
\mathsf{colim}(\loopspace{A_{n}}) \to \loopspace{A_{\infty}}
\end{equation*}
is an equivalence.
\end{cor}

\begin{prp}\label{thm:colim_hlevel}
Consider a type sequence
\begin{equation*}
\begin{tikzcd}
A_0 \arrow[r] & A_1 \arrow[r] & A_2 \arrow[r] & \cdots
\end{tikzcd}
\end{equation*}
If each $A_n$ is $k$-truncated, then so is the sequential colimit $A_\infty$.
\end{prp}

\begin{thm}
Consider a type sequence
\begin{equation*}
\begin{tikzcd}
A_0 \arrow[r] & A_1 \arrow[r] & A_2 \arrow[r] & \cdots
\end{tikzcd}
\end{equation*}
Then the canonical map
\begin{equation*}
\mathsf{colim}_n \trunc{k}{A_n} \to \trunc{k}{A_\infty}
\end{equation*}
is an equivalence. 
\end{thm}

\begin{thm}\label{thm:colim_hgroup}
Consider a type sequence
\begin{equation*}
\begin{tikzcd}
A_0 \arrow[r] & A_1 \arrow[r] & A_2 \arrow[r] & \cdots
\end{tikzcd}
\end{equation*}
Then the canonical map
\begin{equation*}
\mathsf{colim}_n(\pi_k(A_n))\to \pi_k(A_\infty)
\end{equation*}
is a group isomorphism, for any $k\geq 1$.
\end{thm}
