\chapter{Reflexive coequalizers}

The material of this chapter is joint work with Bas Spitters, which we started in the academic year 2012-2013 while I was a research assistant at the Radboud University of Nijmegen.

We begin this chapter with a proof of the type theoretic Yoneda lemma, \cref{lem:yoneda}. The Yoneda lemma is used to show in \cref{thm:emb_disc} that the discrete functor $\Delta:\UU\to\mathsf{rGph}$ from small types into the small reflexive graphs, is an embedding. In \cref{thm:discrete_rgraph} we show that a reflexive graph $\mathcal{A}$ is discrete if and only if it satisfies any one of the unique extension properties
\begin{equation*}
\begin{tikzcd}
\mathcal{I} \arrow[r] \arrow[d] & \mathcal{A} & \unit \arrow[d,swap,"0"] \arrow[r] & \mathcal{A} & \unit \arrow[d,swap,"1"] \arrow[r] & \mathcal{A}, \\
\unit \arrow[ur,densely dotted] & & \mathcal{I} \arrow[ur,densely dotted] & & \mathcal{I} \arrow[ur,densely dotted]
\end{tikzcd}
\end{equation*}
where $\mathcal{I}$ is the walking arrow. 

The (homotopy) reflexive coequalizer is an operation on reflexive graphs that satisfies the universal property of the left adjoint of $\Delta:\UU\to\mathsf{rGph}$. We show in \cref{thm:rcoeq_is_pushout} that the reflexive coequalizer of a reflexive graph $\mathcal{A}$ is just a pushout
\begin{equation*}
\begin{tikzcd}
\sm{x,y:\pts{A}}\edg{A}(x,y) \arrow[r,"\pi_2"] \arrow[d,swap,"\pi_1"] & \pts{A} \arrow[d] \\
\pts{A} \arrow[r] & \mathsf{rcoeq}(\mathcal{A}).
\end{tikzcd}
\end{equation*}
In particular it follows that if a universe is closed under pushouts (which is our running assumption), then it is also closed under reflexive coequalizers. More practically, this characterization of reflexive coequalizers as pushouts allows us to compute in \cref{eg:rcoeq} many reflexive coequalizers in terms of previously defined operations.

Our next purpose is to study the morphisms $f:\mathcal{B}\to\mathcal{A}$ of reflexive graphs that are \emph{fibrations} in the sense that they satisfy the right orthogonality conditions
\begin{equation*}
\begin{tikzcd}
\mathcal{I} \arrow[r] \arrow[d] & \mathcal{B} \arrow[d,"f"] & \unit \arrow[d,swap,"0"] \arrow[r] & \mathcal{B} \arrow[d,"f"] & \unit \arrow[d,swap,"1"] \arrow[r] & \mathcal{B} \arrow[d,"f"] \\
\unit \arrow[ur,densely dotted] \arrow[r] & \mathcal{A} & \mathcal{I} \arrow[ur,densely dotted] \arrow[r] & \mathcal{A} & \mathcal{I} \arrow[ur,densely dotted] \arrow[r] & \mathcal{A}.
\end{tikzcd}
\end{equation*}
Recall from \cite{AnelBiedermanFinserJoyal} that morphism $f:X\to Y$ is right orthogonal to a map $i:A\to B$ if and only if the square
\begin{equation*}
\begin{tikzcd}
X^B \arrow[r,"\blank\circ i"] \arrow[d,swap,"f\circ \blank"] & X^A \arrow[d,"f\circ \blank"] \\
Y^B \arrow[r,swap,"\blank\circ i"] & Y^A
\end{tikzcd}
\end{equation*}
is a pullback square. We use this way of stating the orthogonality condition in our definition of fibrations of reflexive graphs.

We show in \cref{thm:fib_cart} that the class of fibrations is precisely the class of \emph{cartesian} morphisms of reflexive graphs, i.e.~the morphisms $f:\mathcal{A}\to\mathcal{B}$ for which the naturality squares
\begin{equation*}
\begin{tikzcd}
\pts{B} \arrow[d] & \sm{i,j:\pts{B}}\edg{B}(i,j) \arrow[d] \arrow[r] \arrow[l] & \pts{B} \arrow[d] & \pts{B} \arrow[r] \arrow[d] & \sm{i,j:\pts{B}}\edg{B}(i,j) \arrow[d] \\
\pts{A} & \sm{i,j:\pts{A}}\edg{A}(i,j) \arrow[l] \arrow[r] & \pts{A} & \pts{A} \arrow[r] & \sm{i,j:\pts{A}}\edg{A}(i,j)
\end{tikzcd}
\end{equation*}
are pullback squares. It follows that $f:\mathcal{B}\to\mathcal{A}$ is a fibration whenever it appears as a pullback
\begin{equation*}
\begin{tikzcd}
\mathcal{B} \arrow[d,swap,"f"] \arrow[r] & \Delta Y \arrow[d] \\
\mathcal{A} \arrow[r] & \Delta X
\end{tikzcd}
\end{equation*}
of discrete reflexive graphs. Pulling back along $h:\mathcal{A}\to\Delta X$ therefore provides an operation
\begin{equation*}
h^\ast : \Big(\sm{Y:\UU}(Y\to X)\Big)\to \Big(\sm{B:\mathsf{rGph}}\mathsf{fib}(\mathcal{B},\mathcal{A})\Big).
\end{equation*}
The descent property for reflexive coequalizers, established in \cref{thm:descent_rcoeq}, states that this map is an equivalence whenever $f:\mathcal{A}\to\Delta X$ is a reflexive coequalizer. As a corollary of the descent property we obtain in \cref{thm:rcoeq_cartesian} that for any fibration $f:\mathcal{B}\to\mathcal{A}$ of reflexive graphs, a commuting square of the form
\begin{equation*}
\begin{tikzcd}
\mathcal{B} \arrow[d,->>,swap,"f"] \arrow[r] & \Delta Y \arrow[d] \\
\mathcal{A} \arrow[r] & \Delta X
\end{tikzcd}
\end{equation*}
is a pullback square of reflexive graphs if and only if the square
\begin{equation*}
\begin{tikzcd}
\mathsf{rcoeq}(\mathcal{B}) \arrow[d,swap,"\mathsf{rcoeq}(f)"] \arrow[r] & Y \arrow[d] \\
\mathsf{rcoeq}(\mathcal{A}) \arrow[r] & X
\end{tikzcd}
\end{equation*}
is a pullback square. A minor note about the forward direction is that the hypothesis that $f$ is a fibration is implied by the assumption that the square is a pullback square, so this assumption is unrestrictively superfluous. However, this assumption is necessary for the converse direction.

In the final section of this chapter we apply the previous results to colimits of diagrams indexed by a (reflexive) graph. Most useful to us are the results on sequential colimits, which we will rely on on later chapters.

\section{The Yoneda lemma}
The universal property of identity types is sometimes called the type theoretic Yoneda lemma: families of maps out of the identity type are uniquely determined by their action on the reflexivity identification.

\begin{lem}\label{lem:yoneda}
Let $B$ be a type family over $A$, and let $a:A$. Then the map
\begin{equation*}
\mathsf{ev\usc{}refl}:\Big(\prd{x:A} (a=x)\to B(x)\Big)\to B(a)
\end{equation*}
given by $\lam{f} f(a,\refl{a})$ is an equivalence. 
\end{lem}

\begin{proof}
The inverse of $\mathsf{ev\usc{}refl}$ is $\ind{a=}:B(a)\to \prd{x:A}(a=x)\to B(x)$. We have the homotopy $\lam{b}\refl{b}:\mathsf{ev\usc{}refl}\circ\ind{a=}\htpy \idfunc[B(a)]$ by the computation rule for identity types, so it is indeed the case that $\ind{a=}$ is a section of $\mathsf{ev\usc{}refl}$.

To see that $\ind{a=}\circ \mathsf{ev\usc{}refl}\htpy\idfunc$, let $f:\prd{x:A}(a=x)\to B(x)$. To show that $\ind{a=}(f(a,\refl{a}))=f$ we use function extensionality (twice), so it suffices to show that
\begin{equation*}
\prd{x:A}{p:a=x} \varphi(f(a,\refl{a}),x,p)=f(x,p).
\end{equation*}
This follows by path induction on $p$, since $\ind{a=}(f(a,\refl{a}),a,\refl{a})\jdeq f(a,\refl{a})$.
\end{proof}

\begin{cor}
Let $B$ be a type family over $A$, and let $a:A$. Then there is an equivalence
\begin{equation*}
\mathsf{ev\usc{}refl}:\Big(\prd{x:A} (a=x)\eqvsym B(x)\Big)\eqvsym \Big(B(a)\times\iscontr\Big(\sm{x:A}B(x)\Big)\Big)
\end{equation*}
\end{cor}

\begin{proof}
The type of fiberwise equivalences $\prd{x:A} (a=x)\eqvsym B(x)$ is equivalent to the type
\begin{equation*}
\sm{f:\prd{x:A}(a=x)\to B(x)}\prd{x:A}\isequiv(f)
\end{equation*}
By \cref{thm:id_fundamental}, the type $\prd{x:A}\isequiv(f)$ is equivalent to the contractibility of the total space of $B$. Now the claim follows from \cref{lem:yoneda}.
\end{proof}

\begin{cor}
Let $B$ be a type family over $A$, and let $a:A$. Then there is an equivalence
\begin{equation*}
\eqv{\Big(\idtypevar{A}(a)=B\Big)}{\Big(B(a)\times\iscontr\Big(\sm{x:A}B(x)\Big)\Big)}.
\end{equation*}
Furthermore, for $b:B(a)$ there is an equivalence
\begin{equation*}
\eqv{\Big((\idtypevar{A}(a),\refl{a})=(B,b)\Big)}{\iscontr\Big(\sm{x:A}B(x)\Big)}.
\end{equation*}
\end{cor}

\begin{cor}\label{cor:yoneda_rel}
Let $R:A\to (A\to \UU)$ be a reflexive relation on $A$, in the sense that it comes equipped with
\begin{equation*}
\rho : \prd{x:A}R(x,x).
\end{equation*}
Then there is an equivalence
\begin{equation*}
\eqv{\Big((\idtypevar,\refl{})=(R,\rho)\Big)}{\prd{a:A}\iscontr\Big(\sm{x:A}R(a,x)\Big)}.
\end{equation*}
\end{cor}

The following theorem, which was proven independently by Escardó \cite{Escardo2016} around the same time, shows that the canonical map
\begin{equation*}
(x=y)\to (\idtypevar{A}(x)=\idtypevar{A}(y))
\end{equation*}
is an equivalence, for any $x,y:A$. This will be particularly relevant to us once we study $\infty$-equivalence relations.

\begin{prp}
For any type $A:\UU$, the map
\begin{equation*}
\idtypevar{A}:A\to (A\to\UU)
\end{equation*}
is an embedding.\index{identity type!is an embedding|textit}
\end{prp}

\begin{proof}
Let $a:A$. Then we calculate the fiber of $\idtypevar{A}$ at a type family $B:A\to \UU$ as follows:
\begin{align*}
\fib{\idtypevar{A}}{B} & \jdeq \sm{a:A}\idtypevar{A}(a)=B \\
& \eqvsym \sm{a:A} \Big(B(a)\times\iscontr\Big(\sm{x:A}B(x)\Big)\Big) \\
& \eqvsym \Big(\sm{a:A}B(a)\Big)\times \iscontr\Big(\sm{x:A}B(x)\Big) \\
& \eqvsym \iscontr\Big(\sm{x:A}B(x)\Big).
\end{align*}
The last equivalence follows since $X\times \iscontr(X)\eqvsym \iscontr(X)$ for every type $X$. We conclude that the fibers of $\idtypevar{A}$ are propositions, so $\idtypevar{A}$ is an embedding by \cref{thm:prop_emb}.
\end{proof}

\section{Discrete reflexive graphs}

\begin{defn}\label{defn:graphs_ctx}
A \define{graph} $\mathcal{A}$ in $\UU$ is a pair $\pairr{\pts{A},\edg{A}}$ consisting of
\begin{align*}
\pts{A} & : \UU \\
\edg{A} & : \pts{A}\to\pts{A}\to\UU.
\end{align*}
We write $\mathsf{Gph}$ for the type of all graphs in $\UU$. When $\mathcal{A}$ is a graph, we say that $\pts{A}$ is its type of \define{vertices}, and that $\edg{A}$ is its family of \define{edges}.

A \define{reflexive graph} $\mathcal{A}$ in $\UU$ consists of a graph $(\pts{A},\edg{A})$ equipped with a \define{reflexivity} term
\begin{align*}
\rfx{\mathcal{A}} & : \prd{i:\pts{A}}\edg{A}(i,i).
\end{align*}
We write
\begin{equation*}
\mathsf{rGph}\defeq \sm{V:\UU}{E:V\to (V\to\UU)}\prd{v:V}E(v,v)
\end{equation*}
for the type of reflexive graphs in $\UU$. 
\end{defn}

\begin{eg}
\begin{enumerate}
\item A \define{reflexive pair} consists of types $V$ and $E$, and maps $s$, $t$, and $r$ between $V$ and $E$ as indicated in the diagram
\begin{equation*}
\begin{tikzcd}[column sep=large]
E \arrow[r,yshift=1.5ex,"s"] \arrow[r,yshift=-1.5ex,swap,"t"] & V, \arrow[l,"r" description]
\end{tikzcd}
\end{equation*}
equipped with homotopies $H_{sr}:\idfunc[V]\htpy s\circ r$ and $H_{tr}:\idfunc[V]\htpy t\circ r$.
Given a reflexive pair as above, we obtain a reflexive graph $\mathcal{A}$ by taking
\begin{align*}
\pts{A} & \defeq V \\
\edg{A}(v,w) & \defeq \sm{e:E} (v=s(e))\times (w=t(e)) \\
\rfx{\mathcal{A}}(v) & \defeq \pairr{r(v),H_{sr}(v),H_{tr}(v)}.
\end{align*}
By a routine construction it can be shown that the type of small reflexive pairs is equivalent to the type of small reflexive graphs.
\item \label{eg:disc_codisc} Given a type $X$, we have the \define{discrete graph} $\Delta(X)$ on $X$ consisting of
\begin{align*}
\pts{\Delta(X)} & \defeq X \\
\edg{\Delta(X)} & \defeq \idtypevar{X} \\
\rfx{\Delta(X)} & \defeq \refl{},
\end{align*}
and we have the \define{indiscrete graph} $\nabla(X)$ on $X$ consisting of
\begin{align*}
\pts{\nabla(X)} & \defeq X \\
\edg{\nabla(X)} & \defeq \lam{x}{y}\unit \\
\rfx{\nabla(X)} & \defeq \ttt.
\end{align*}
In particular, we have the \define{unit} graph $\unit\defeq\nabla(\unit)$, which happens to also be $\Delta(\unit)$. 
\item The \define{walking arrow} $\mathcal{I}$ is an example of a reflexive graph. It consists of
\begin{align*}
\pts{\mathcal{I}} & \defeq \bool \\
\edg{\mathcal{I}}(x,y) & \defeq \tau(x)\to \tau(y) \\
\rfx{\mathcal{I}}(x) & \defeq \idfunc[\tau(x)]
\end{align*}
where $\tau:\bool\to\UU$ is the \define{tautological family} on $\bool$ given by $\tau(\bfalse)\defeq\emptyt$ and $\tau(\btrue)\defeq\unit$. We write $a$ for the (unique) arrow from $\bfalse$ to $\btrue$. It may be helpful to think of the walking arrow $\mathcal{I}$ as an interval, hence the choice of notation.
\end{enumerate}
\end{eg}

\begin{defn}
A \define{morphism} of graphs $f$ from $\mathcal{A}$ to $\mathcal{B}$ is a pair
\begin{align*}
\pts{f} & : \pts{A} \to \pts{B} \\
\edg{f} & : \prd{i,j:\pts{A}} \edg{A}(i,j)\to \edg{B}(\pts{f}(i),\pts{f}(j)).
\end{align*}
We write $\mathsf{Gph}(\mathcal{A},\mathcal{B})$ for the type of graph morphisms from $\mathcal{A}$ to $\mathcal{B}$.

A \define{morphism} of reflexive graphs $f$ from $\mathcal{A}$ to $\mathcal{B}$ is a morphism of graphs equipped with a term
\begin{align*}
\rfx{f} & : \prd{i:\pts{A}} \edg{f}(\rfx{\mathcal{A}}(i))= \rfx{\mathcal{B}}(\pts{f}(i))
\end{align*}
witnessing that reflexivity is preserved. We write $\mathsf{rGph}(\mathcal{A},\mathcal{B})$ for the type of reflexive graph morphisms from $\mathcal{A}$ to $\mathcal{B}$.
\end{defn}

For any reflexive graph there is an identity morphism, and for any composable pair of morphisms there is a composite. Furthermore, these operations are associative, and satisfy the unit laws. It should be noted, however, that composition of reflexive graph morphisms is not expected to be associative on the nose, since the reflexivity is only preserved up to higher identification. In other words, preservation of reflexivity is part of the \emph{structure} of a reflexive graph morphism, and is not a property. 

\begin{eg}\label{eg:rgraph_morphism}
\begin{enumerate}
\item Given a function $f:X\to Y$ we obtain a morphism $\Delta f:\mathsf{rGph}(\Delta X,\Delta Y)$ given by
\begin{align*}
\pts{\Delta f} & \defeq f \\
\edg{\Delta f} & \defeq \apfunc{f} \\
\rfx{\Delta f} & \defeq \refl{\refl{f(x)}}
\end{align*}
\item The graph $\nabla(\unit)$ is the terminal reflexive graph in the sense that for any graph $\mathcal{A}$, the type $\mathsf{rGph}(\mathcal{A},\unit)$ is contractible.
\item For any reflexive graph $\mathcal{A}$, the map 
\begin{equation*}
\mathsf{ev\usc{}pt}:\mathsf{rGph}(\unit,\mathcal{A})\to\pts{A}
\end{equation*}
given by $f\mapsto \pts{f}(\ttt)$, is an equivalence. The analogous statement is false for (non-reflexive) graphs.
\item The universal property of the walking arrow $\mathcal{I}$ is that every edge $e$ in a graph $\mathcal{A}$, the type of reflexive graph morphisms that map the edge $a$ of $\mathcal{I}$ to $e$ is contractible. In other words, the map
\begin{equation*}
\mathsf{ev\usc{}arrow} : \mathsf{rGph}(\mathcal{I},\mathcal{A})\to \sm{i,j:\pts{A}}\edg{A}(i,j)
\end{equation*}
given by $f\mapsto (\pts{f}(0),\pts{f}(1),\edg{f}(a))$ is an equivalence.
\item For the walking arrow $\mathcal{I}$ there are reflexive graph morphisms
\begin{equation*}
\begin{tikzcd}
\mathcal{I} \arrow[r] & \unit \arrow[l,yshift=1ex,swap,"1"] \arrow[l,yshift=-1ex,"0"]
\end{tikzcd}
\end{equation*}
This is a \emph{cograph} object of reflexive graphs, since the morphism $\mathcal{I}\to\unit$ is a common \emph{retraction} of the end-point inclusions $0,1:\unit\to \mathcal{I}$, whereas in a reflexive pair (a graph object) the morphism $r$ is a common \emph{section} of the source and target maps.
\item \label{eg:freerfx} Since we do not have the technology available to establish that graphs and reflexive graphs form $\infty$-categories, a comparison between the two structures is limited to what we can say directly using the basic categorical operations such as composition. However, this is just enough to establish an adjunction-like property. 

Given a non-reflexive graph $\mathcal{A}\jdeq\pairr{\pts{A},\edg{A}}$, we can obtain a reflexive graph $F\mathcal{A}$ by freely adjoining reflexivity:
\begin{align*}
\pts{F\mathcal{A}} & \defeq \pts{A} \\
\edg{F\mathcal{A}}(i,j) & \defeq \edg{A}(i,j)+(i=j) \\
\rfx{F\mathcal{A}}(i) & \defeq \inr(\refl{i}).
\end{align*}
On the other hand, there is the projection $U:\mathsf{rGph}\to\mathsf{Gph}$ which forgets the reflexivity structure, and for each graph $\mathcal{A}$ there is a graph morphism $\eta:\mathcal{A}\to UF\mathcal{A}$. Both $F$ and $U$ are functorial in the sense that they act on morphisms, and preserve units and composition in the obvious way, and $\eta$ is natural in $\mathcal{A}$. 

The universal property of the construction $F$ of freely adjoining reflexivity, is that the map
\begin{equation*}
U(\blank)\circ \eta : \mathsf{rGph}(F\mathcal{A},\mathcal{B})\to \mathsf{Gph}(\mathcal{A},U\mathcal{B})
\end{equation*}
is an equivalence. Indeed, this is the universal property that establishes $F$ as a left adjoint to $U$, even though we cannot manifest $F$ and $U$ as functors.

The construction $F$ of freely adjoining reflexivity is not surjective on morphisms. For example, there are no morphisms from $\unit$ to the graph $\mathcal{A}$ with a contractible type of vertices but no edges. However $F\mathcal{A}$ is the terminal reflexive graph.
\end{enumerate}
\end{eg}

\begin{prp}\label{thm:emb_disc}
The operation $\Delta:\UU\to\mathsf{rGph}$ is an embedding in the following sense:
\begin{enumerate}
\item As a function, $\Delta:\UU\to\mathsf{rGph}$ is an embedding.
\item For every two types $X$ and $Y$, the action on morphisms
\begin{equation*}
\Delta : (X\to Y)\to \mathsf{rGph}(\Delta X,\Delta Y)
\end{equation*}
is an equivalence.
\end{enumerate}
\end{prp}

\begin{proof}
Let $\mathcal{A}$ be a reflexive graph. Then the fiber of $\Delta:\UU\to\mathsf{rGph}$ is calculated as follows:
\begin{align*}
\fib{\Delta}{\mathcal{A}} & \defeq \sm{X:\UU} \Delta X = \mathcal{A} \\
& \eqvsym (\idtypevar{\pts{A}},\refl{})=(\edg{A},\rfx{A}) \\
& \eqvsym \prd{x:\pts{A}}\iscontr\Big(\sm{y:\pts{A}}\edg{A}(x,y)\Big),\tag{by \cref{cor:yoneda_rel}}
\end{align*}
which is a proposition.

The map $(X\to Y)\to \mathsf{rGph}(\Delta X,\Delta Y)$ is an equivalence, since the type
\begin{equation*}
\sm{\varphi:\prd{x,y:X}(x=y)\to (f(x)=f(y))}\varphi(\refl{x})=\refl{f(x)}
\end{equation*}
is equivalent to the fiber of the map
\begin{equation*}
\mathsf{ev\usc{}refl} : \Big(\prd{x,y:X}(x=y)\to (f(x)=f(y))\Big)\to \Big(f(x)=f(x)\Big),
\end{equation*}
which is contractible by \cref{lem:yoneda}.
\end{proof}

\begin{thm}\label{thm:discrete_rgraph}
Let $\mathcal{A}$ be a reflexive graph. The following are equivalent:
\begin{enumerate}
\item The canonical map
\begin{equation*}
\prd{i,j:\pts{A}} (i=j)\to \edg{A}(i,j)
\end{equation*}
given by $\refl{i}\mapsto \rfx{\mathcal{A}}(i)$ is a fiberwise equivalence. In particular, $\mathcal{A}$ is in the image of $\Delta:\UU\to\mathsf{rGph}$. 
\item The graph $\mathcal{A}$ is \define{$\mathcal{I}$-null} in the sense that the map
\begin{equation*}
\mathsf{rGph}(\unit,\mathcal{A}) \to \mathsf{rGph}(\mathcal{I},\mathcal{A})
\end{equation*}
given by pre-composition by the unique morphism $\mathsf{rGph}(\mathcal{I},\unit)$, is an equivalence.
\item The map
\begin{equation*}
\mathsf{rGph}(\mathcal{I},\mathcal{A})\to \mathsf{rGph}(\unit,\mathcal{A})
\end{equation*}
given by pre-composing with the end-point inclusion $0:\mathsf{rGph}(\unit,\mathcal{I})$, is an equivalence.
\item The map
\begin{equation*}
\mathsf{rGph}(\mathcal{I},\mathcal{A})\to \mathsf{rGph}(\unit,\mathcal{A})
\end{equation*}
given by pre-composing with the end-point inclusion $1:\mathsf{rGph}(\unit,\mathcal{I})$, is an equivalence.
\end{enumerate}
If any of these conditions hold, we say that $\mathcal{A}$ is \define{discrete}.
\end{thm}

\begin{proof}
The outline of our argument is as follows:
\begin{equation*}
\begin{tikzcd}[row sep=tiny]
& & (iii) \arrow[dr,Rightarrow] \\
(i) \arrow[r,Rightarrow] & (ii) \arrow[ur,Rightarrow] \arrow[dr,Rightarrow] & & (i). \\
& & (iv) \arrow[ur,Rightarrow]
\end{tikzcd}
\end{equation*}
Suppose (i) holds. We have a commuting square
\begin{equation*}
\begin{tikzcd}
\mathsf{rGph}(\unit,\mathcal{A}) \arrow[d,swap,"\mathsf{ev\usc{}pt}"] \arrow[r] & \mathsf{rGph}(\mathcal{I},\mathcal{A}) \arrow[d,"\mathsf{ev\usc{}arrow}"] \\
\pts{A} \arrow[r] & \sm{i,j:\pts{A}}\edg{A}(i,j)
\end{tikzcd}
\end{equation*}
where both vertical maps are equivalences, and the top map is an equivalence by assumption. Therefore the bottom map is an equivalence.

Since the reflexive graph morphism $\mathsf{rGph}(\mathcal{I},\unit)$ is a common retraction of both end-point inclusions, it follows that the pre-composition map
\begin{equation*}
\mathsf{rGph}(\unit,\mathcal{A}) \to \mathsf{rGph}(\mathcal{I},\mathcal{A})
\end{equation*}
is a common section of both pre-composition maps
\begin{equation*}
0^\ast,1^\ast:\mathsf{rGph}(\mathcal{I},\mathcal{A})\to \mathsf{rGph}(\unit,\mathcal{A})
\end{equation*}
However, assuming (ii) it follows that both $0^\ast$ and $1^\ast$ are equivalences, so (ii) implies both (iii) and (iv).

Now suppose that (iii) holds; we will show that (i) follows. We have the commuting square
\begin{equation*}
\begin{tikzcd}
\mathsf{rGph}(\mathcal{I},\mathcal{A}) \arrow[d,swap,"\mathsf{ev\usc{}arrow}"] \arrow[r,"0^\ast"] & \mathsf{rGph}(\unit,\mathcal{A}) \arrow[d,"\mathsf{ev\usc{}pt}"] \\
\sm{i,j:\pts{A}}\edg{A}(i,j) \arrow[r,swap,"\pi_1"] & \pts{A}
\end{tikzcd}
\end{equation*}
in which both vertical maps are equivalences. Therefore the fibers of $\pi_1$ are equivalent to the fibers of $0^\ast$. Note that the fibers of $\pi_1$ are of the form
\begin{equation*}
\sm{j:\pts{A}}\edg{A}(i,j),
\end{equation*}
so it follows from (iii) that these are contractible. Then (i) follows by the fundamental theorem of identity types, \cref{thm:id_fundamental}.

The argument that (i) follows from (iv) is similar, using $\pi_2$ instead of $\pi_1$ in the square.
\end{proof}

\section{Reflexive coequalizers}

\begin{defn}
Consider a reflexive graph $\mathcal{A}$ and a type $X$ equipped with a morphism $f:\mathsf{rGph}(\mathcal{A},\Delta(X))$. We say that $X$ is a \define{reflexive coequalizer} if the map
\begin{equation*}
\Delta(\blank) \circ f: (X\to Y)\to \mathsf{rGph}(\mathcal{A},\Delta(Y))
\end{equation*}
is an equivalence.
\end{defn}

Our goal in this section is to show that reflexive coequalizers can be constructed as pushouts. We will use the following lemma, which was discovered by Simon Boulier and me during his visit to CMU.

\begin{lem}\label{lem:coh_red}
Consider a type $A$ with a type family $B$, and $a:A$. Furthermore, suppose that
\begin{equation*}
\alpha:\prd{x:A}B(x)\to (a=x).
\end{equation*}
Then the \define{coherence reduction} map
\begin{equation*}
\mathsf{coh\usc{}red} : \Big(\sm{y:B(a)}\alpha(y)=\refl{a}\Big)\to\Big(\sm{x:A}B(x)\Big)
\end{equation*}
given by $(y,q)\mapsto (a,y)$ is an equivalence.
\end{lem}

\begin{rmk}
A quick way to see that there is an equivalence
\begin{equation*}
\sm{x:A}B(x) \eqvsym \sm{b:B(a)} \alpha_{a}(b)=\refl{a}.
\end{equation*}
is to use the contractibiliby of the total space of identity types twice:
\begin{align*}
\sm{x:A}B(x) & \eqvsym \sm{x:A}{y:B(x)}{p:a=x}\alpha(y)=p \\
& \eqvsym \sm{y:B(a)}\alpha(y)=\refl{a}.\qedhere
\end{align*}
However, it is not clear at once that the underlying map of this composite of equivalences is indeed the map $\mathsf{coh\usc{}red}$ defined in \cref{lem:coh_red}.
\end{rmk}

\begin{proof}[Proof of \cref{lem:coh_red}]
We show that the fibers are contractible:
\begin{align*}
\fib{\mathsf{coh\usc{}red}}{(x,y)} & \eqvsym \sm{y':B(a)}{q:\alpha(y')=\refl{a}} (a,y')=(x,y) \\
& \eqvsym \sm{y':B(a)}{q:\alpha(y')=\refl{a}}{p:a=x} \mathsf{tr}_B(p,y')=y \\
& \eqvsym \sm{y':B(a)}{q:\alpha(y')=\refl{a}}{p:a=x} y'=\mathsf{tr}_B(p^{-1},y) \\
& \eqvsym \sm{p:a=x}\alpha(\mathsf{tr}_B(p^{-1},y))=\refl{a} \\
& \eqvsym \sm{p:a=x}p=\alpha(y).\qedhere
\end{align*}
\end{proof}

\begin{cor}
Consider a type $A$ with a relation $R:A\to A\to\UU$ such that
\begin{equation*}
\alpha: \prd{x,y:A}R(x,y)\to (x=y).
\end{equation*}
Then the map
\begin{equation*}
\mathsf{coh\usc{}red}:\Big(\sm{x:A}{r:R(x,x)}\alpha(r)=\refl{x}\Big)\to \Big(\sm{x,y:A}R(x,y)\Big).
\end{equation*}
given by $(x,r,c)\mapsto (x,x,r)$ is an equivalence.
\end{cor}

\begin{proof}
Immediate by \cref{thm:fib_equiv}.
\end{proof}

\begin{lem}\label{lem:rcoeq_into_disc}
For any reflexive graph $\mathcal{A}$, and any type $X$, the map
\begin{equation*}
f\mapsto (f_0,f_0,\lam{(x,(y,e))}\edg{f}(e)) : \mathsf{rGph}(\mathcal{A},\Delta X) \to \mathsf{cocone}_{\tilde{\mathcal{A}}}(X),
\end{equation*}
is an equivalence, where $\tilde{\mathcal{A}}$ is the span $(\tilde{A}_1,\pi_1,\pi_2)$ from $A_0$ to $A_0$. 
\end{lem}

\begin{proof}
Write $\varphi$ for the asserted map, and write
\begin{align*}
\mathsf{rGph}'(\mathcal{A},\Delta X) & \defeq \sm{\pts{f}:\pts{A}\to X}{\edg{f}:\prd{x,y:\pts{A}}\edg{A}(x,y)\to f(x)=f(y)} \\
& \phantom{\defeq\sm{\pts{f}:\pts{A}\to X}} \mathsf{eq\usc{}htpy}(\lam{x}\edg{f}(\rfx{\mathcal{A}}(x)))=\refl{\pts{f}} \\
\mathsf{cocone}'_{\tilde{A}}(X) & \defeq \sm{f,g:\pts{A}\to X}\prd{x,y:\pts{A}}\edg{A}(x,y)\to f(x)=f(y).
\end{align*}
Note that we have a commuting square
\begin{equation*}
\begin{tikzcd}[column sep=large]
\mathsf{rGph}(\mathcal{A},\Delta X) \arrow[d] \arrow[r,"\varphi"] & \mathsf{cocone}_{\tilde{\mathcal{A}}}(X) \arrow[d] \\
\mathsf{rGph}'(\mathcal{A},\Delta X) \arrow[r,swap,"\mathsf{coh\usc{}red}"] & \mathsf{cocone}'_{\tilde{\mathcal{A}}}(X).
\end{tikzcd}
\end{equation*}
The coherence reduction map uses the homotopy
\begin{equation*}
(f,g,H)\mapsto \mathsf{eq\usc{}htpy}(\lam{x}H(\rfx{A}(x))).
\end{equation*}
The vertical map on the left is obtained by function extensionality, so it is an equivalence. The vertical map on the right is obtained by evaluating at a pair, which is an equivalence by the dependent universal property of $\Sigma$-types. It follows that the asserted map is an equivalence.
\end{proof}

\begin{prp}\label{thm:rcoeq_is_pushout}
Let $\mathcal{A}$ be a reflexive graph, and let $X$ be a type equipped with $f:\mathsf{rGph}(\mathcal{A},\Delta X)$. Then the following are equivalent:
\begin{enumerate}
\item $X$ is a reflexive coequalizer of $\mathcal{A}$.
\item The square
\begin{equation*}
\begin{tikzcd}
\sm{x,y:\pts{A}}\edg{A}(x,y) \arrow[r,"\pi_2"] \arrow[d,swap,"\pi_1"] & \pts{A} \arrow[d,"\pts{f}"] \\
\pts{A} \arrow[r,swap,"\pts{f}"] & X
\end{tikzcd}
\end{equation*}
which commutes by $\lam{(i,j,e)}\edg{f}(e)$, is a pushout square.
\end{enumerate}
In particular, there is a reflexive coequalizer for every reflexive graph $\mathcal{A}$, for which we write
\begin{equation*}
\mathsf{constr}:\mathsf{rGph}(\mathcal{A},\Delta(\mathsf{rcoeq}(\mathcal{A}))).
\end{equation*}
\end{prp}

\begin{proof}
The triangle
\begin{equation*}
\begin{tikzcd}
& Y^X \arrow[dl,swap,"{\Delta(\blank)\circ f}"] \arrow[dr,"\mathsf{cocone\usc{}map}"] \\
\mathsf{rGph}(\mathcal{A},\Delta Y) \arrow[rr,"\eqvsym"] & & \mathsf{cocone}(Y)
\end{tikzcd}
\end{equation*}
commutes, for any type $Y$. In this triangle, the bottom map is the map defined in \cref{lem:rcoeq_into_disc}, which is an equivalence, so the claim follows by the 3-for-2 property of equivalences.
\end{proof}

\begin{eg}\label{eg:rcoeq} {}~
\begin{enumerate}
\item The reflexive graph quotient of the discrete graph $\Delta(X)$ of a type $X$ is just $X$ itself. It also follows that $\mathsf{constr} : \mathsf{rGph}(\mathcal{A},\Delta(\mathsf{rcoeq}(\mathcal{A})))$ is an equivalence of reflexive graphs if and only if $\mathcal{A}$ is a discrete graph.
\item The reflexive graph quotient of the indiscrete graph $\nabla(X)$ on a type $X$ is the join square $\join{X}{X}$.
\item Let $\mathcal{A}$ be a non-reflexive graph, and let $F(\mathcal{A})$ be the reflexive graph obtained by freely adding a proof of reflexivity, as in \autoref{eg:freerfx} of \autoref{eg:running_graphs}. Then the non-reflexive graph quotient of $\mathcal{A}$ is the reflexive graph quotient of $F(\mathcal{A})$. 
\item Let $X$ be a type with base point $x_0:X$. Define the reflexive graph $SX$ by
\begin{align*}
\pts{SX} & \defeq \unit \\
\edg{SX} & \defeq \lam{\nameless}{\nameless}X \\
\rfx{SX} & \defeq \lam{\nameless}x_0.
\end{align*} 
The reflexive graph quotient of $SX$ is the suspension of $X$.
\end{enumerate}
\item The reflexive coequalizer of the walking arrow $\mathcal{I}$ is the interval, which is contractible.
\end{eg}

\section{Descent for reflexive coequalizers}\label{sec:descent_rcoeq}

\subsection{Fibrations of reflexive graphs}

\begin{defn}
A morphism $f:\mathsf{rGph}(\mathcal{A},\mathcal{B})$ is said to be a \define{left fibration} of reflexive graphs if it is right orthogonal to the morphism $0:\mathsf{rGph}(\unit,\mathcal{I})$ in the sense that the commuting square
\begin{equation*}
\begin{tikzcd}
\mathsf{rGph}(\mathcal{I},\mathcal{A}) \arrow[r] \arrow[d] & \mathsf{rGph}(\unit,\mathcal{A}) \arrow[d] \\
\mathsf{rGph}(\mathcal{I},\mathcal{B}) \arrow[r] & \mathsf{rGph}(\unit,\mathcal{B})
\end{tikzcd}
\end{equation*}
is a pullback square. Similarly, we say that $f$ is a \define{right} fibration of reflexive graphs if it is right orthogonal to the morphism $1:\mathsf{rGph}(\unit,\mathcal{I})$, and we say that $f$ is a \define{fibration} if it is both a left and a right fibration.
\end{defn}

\begin{lem}\label{lem:leftfib_Inull}
Suppose $f:\mathsf{rGph}(\mathcal{B},\mathcal{A})$ is a left or a right fibration. Then $f$ is right orthogonal to the terminal projection $t:\mathsf{rGph}(\mathcal{I},\unit)$.
\end{lem}

\begin{proof}
We prove the assertion assuming that $f$ is a left fibration, the case of a right fibration being similar. Consider the diagram
\begin{equation*}
\begin{tikzcd}
\mathsf{rGph}(\unit,\mathcal{A}) \arrow[d] \arrow[r] & \mathsf{rGph}(\mathcal{I},\mathcal{A}) \arrow[r] \arrow[d] & \mathsf{rGph}(\unit,\mathcal{A}) \arrow[d] \\
\mathsf{rGph}(\unit,\mathcal{B}) \arrow[r] & \mathsf{rGph}(\mathcal{I},\mathcal{B}) \arrow[r] & \mathsf{rGph}(\unit,\mathcal{B})
\end{tikzcd}
\end{equation*}
The square on the right is a pullback. Since the composite $\unit \to \mathcal{I}\to \unit$ is the identity morphism of reflexive graphs, the outer rectangle is also a pullback. Therefore the square on the left is a pullback.
\end{proof}

\begin{defn}
A morphism $f:\mathsf{rGph}(\mathcal{A},\mathcal{B})$ of reflexive graphs is said to be \define{left cartesian} if the naturality square
\begin{equation*}
\begin{tikzcd}
\sm{i,j:\pts{A}}\edg{A}(i,j) \arrow[d] \arrow[r,"\pi_1"] & \pts{A} \arrow[d] \\
\sm{i,j:\pts{B}}\edg{B}(i,j) \arrow[r,swap,"\pi_1"] & \pts{B}
\end{tikzcd}
\end{equation*}
is a pullback square. Similarly we say that $f$ is \define{right cartesian} if the naturality square
\begin{equation*}
\begin{tikzcd}
\sm{i,j:\pts{A}}\edg{A}(i,j) \arrow[d] \arrow[r,"\pi_2"] & \pts{A} \arrow[d] \\
\sm{i,j:\pts{B}}\edg{B}(i,j) \arrow[r,swap,"\pi_2"] & \pts{B}
\end{tikzcd}
\end{equation*}
is a pullback square, and we say that $f$ is \define{cartesian} if it is both left and right cartesian.
\end{defn}

\begin{eg}
A common way to obtain a cartesian morphism of reflexive graphs is via equifibered families.
An \define{equifibered family} $\mathcal{E}$ over $\mathcal{A}$ consists of
\begin{align*}
\pts{E} & : \pts{A}\to\UU \\
\edg{E} & : \prd{i,j:\pts{A}} \edg{A}(i,j)\to (\eqv{\pts{E}(i)}{\pts{E}(j)}) \\
\rfx{E} & : \prd{i:\pts{A}} \edg{E}(\rfx{\mathcal{A}}(i))\htpy \idfunc[\pts{E}(i)]
\end{align*}
Given an equifibered family $\mathcal{E}$, we form the reflexive graph $\msm{\mathcal{A}}{\mathcal{E}}$ by
\begin{align*}
\pts{\msm{\mathcal{A}}{\mathcal{E}}} & \defeq \sm{x:\pts{A}}\pts{E}(x) \\
\edg{\msm{\mathcal{A}}{\mathcal{E}}}((x,u),(y,v)) & \defeq \sm{e:\edg{A}(x,y)} \edg{E}(e,u) = v \\
\rfx{\msm{\mathcal{A}}{\mathcal{E}}}(x,u) & \defeq (\rfx{\mathcal{A}}(x),\rfx{\mathcal{E}}(x,u)).
\end{align*}
There is an obvious projection morphism $\proj 1 : \mathsf{rGph}(\msm{\mathcal{A}}{\mathcal{E}},\mathcal{A})$, which is cartesian.
\end{eg}

\begin{lem}
Suppose that $f:\mathsf{rGph}(\mathcal{B},\mathcal{A})$ is left or right cartesian. Then the square
\begin{equation*}
\begin{tikzcd}
\pts{B} \arrow[r] \arrow[d] & \sm{i,j:\pts{B}}\edg{B}(i,j) \arrow[d] \\
\pts{A} \arrow[r] & \sm{i,j:\pts{A}}\edg{A}(i,j)
\end{tikzcd}
\end{equation*}
is a pullback square.
\end{lem}

\begin{proof}
Analogous to the proof of \cref{lem:leftfib_Inull}.
\end{proof}

\begin{lem}
Consider $f:\mathsf{rGph}(\mathcal{B},\mathcal{A})$ and $g:\mathsf{rGph}(\mathcal{C},\mathcal{B})$, and suppose that $f$ is (left/right) cartesian. Then $g$ is (left/right) cartesian if and only if $f\circ g$ is (left/right) cartesian, respectively.
\end{lem}

\begin{proof}
Immediate by \cref{thm:pb_pasting}.
\end{proof}

\begin{prp}\label{prp:fib_cart}
Let $f:\mathsf{rGph}(\mathcal{B},\mathcal{A})$ be a morphism of reflexive graphs. The following are equivalent:
\begin{enumerate}
\item $f$ is a (left/right) fibration.
\item $f$ is (left/right) cartesian.
\end{enumerate}
\end{prp}

\begin{proof}
We only show that $f$ is a left fibration if and only if $f$ is left cartesian, the right case being similar.

For any morphism $f:\mathsf{rGph}(\mathcal{B},\mathcal{A})$ we have the commuting cube
\begin{equation*}
\begin{tikzcd}
& \mathsf{rGph}(\mathcal{I},\mathcal{B}) \arrow[dl] \arrow[d] \arrow[dr] \\
\mathsf{rGph}(\unit,\mathcal{B}) \arrow[d] & \sm{i,j:\pts{B}}\edg{B}(i,j) \arrow[dl] \arrow[dr] & \mathsf{rGph}(\mathcal{I},\mathcal{A}) \arrow[dl,crossing over] \arrow[d] \\
\pts{B} \arrow[dr] & \mathsf{rGph}(\unit,\mathcal{A}) \arrow[d] \arrow[from=ul,crossing over] & \sm{i,j:\pts{A}}\edg{A}(i,j) \arrow[dl] \\
\phantom{\sm{i,j:\pts{A}}\edg{A}(i,j)} & \pts{A}
\end{tikzcd}
\end{equation*}
in which all vertical maps are equivalences. Therefore the top square is a pullback if and only if the bottom square is a pullback, which proves that (ii) holds if and only if (iii) holds.
\end{proof}

\begin{prp}
Consider a commuting square
\begin{equation*}
\begin{tikzcd}
\mathcal{C} \arrow[r] \arrow[d] & \mathcal{B} \arrow[d] \\
\mathcal{A} \arrow[r] & \mathcal{X}
\end{tikzcd}
\end{equation*}
of reflexive graphs. The following are equivalent:
\begin{enumerate}
\item The square is a pullback square.
\item The squares
\begin{equation*}
\begin{tikzcd}
\tilde{C}_0 \arrow[r] \arrow[d] & \tilde{B}_0 \arrow[d] & \tilde{C}_1 \arrow[r] \arrow[d] & \tilde{B}_1 \arrow[d] \\
\tilde{A}_0 \arrow[r] & \tilde{X}_0 & \tilde{A}_1 \arrow[r] & \tilde{X}_1
\end{tikzcd}
\end{equation*}
are pullback squares.
\end{enumerate}
\end{prp}

\begin{proof}
Straightforward.
\end{proof}

\begin{prp}
Consider a pullback square
\begin{equation*}
\begin{tikzcd}
\mathcal{B}' \arrow[r] \arrow[d,swap,"{f'}"] & \mathcal{B} \arrow[d,"f"] \\
\mathcal{A}' \arrow[r] & \mathcal{A}
\end{tikzcd}
\end{equation*}
and let $h:\mathsf{rGph}(\mathcal{Y},\mathcal{X})$ be a morphism of reflexive graphs. If $f$ is right orthogonal to $h$, then so is $f'$. In particular, if $f$ is a (left/right) fibration, then so is $f'$. 
\end{prp}

\begin{proof}
Consider the commuting cube
\begin{equation*}
\begin{tikzcd}
& \mathsf{rGph}(\mathcal{Y},\mathcal{B}') \arrow[dl] \arrow[d] \arrow[dr] \\
\mathsf{rGph}(\mathcal{Y},\mathcal{A}') \arrow[d] & \mathsf{rGph}(\mathcal{X},\mathcal{B}') \arrow[dl] \arrow[dr] & \mathsf{rGph}(\mathcal{Y},\mathcal{B}) \arrow[dl,crossing over] \arrow[d] \\
\mathsf{rGph}(\mathcal{X},\mathcal{A}') \arrow[dr] & \mathsf{rGph}(\mathcal{Y},\mathcal{A}) \arrow[from=ul,crossing over] \arrow[d] & \mathsf{rGph}(\mathcal{X},\mathcal{B}) \arrow[dl] \\
& \mathsf{rGph}(\mathcal{X},\mathcal{A}). & \phantom{\mathsf{rGph}(\mathcal{X},\mathcal{A}')}
\end{tikzcd}
\end{equation*}
Note $\mathsf{rGph}(\mathcal{X},\blank)$ preserves pullbacks for any reflexive graph $\mathcal{X}$. Therefore it follows that the top and bottom squares are pullback squares. Moreover, the square on the front right is a pullback square by the assumption that $f$ is right orthogonal to $h:\mathsf{rGph}(\mathcal{Y},\mathcal{X})$. Now it follows by the pasting property of pullbacks that the square on the back left is a pullback square. In other words, $f'$ is right orthogonal to $h$. 
\end{proof}

\begin{prp}
For any map $f:X\to Y$, the morphism $\Delta f : \Delta X \to \Delta Y$ is a fibration.
\end{prp}

\begin{proof}
The commuting square
\begin{equation*}
\begin{tikzcd}
\mathsf{rGph}(\mathcal{I},\Delta X) \arrow[d] \arrow[r] & \mathsf{rGph}(\unit,\Delta X) \arrow[d] \\
\mathsf{rGph}(\mathcal{I},\Delta Y) \arrow[r] & \mathsf{rGph}(\unit,\Delta Y)
\end{tikzcd}
\end{equation*}
is a pullback square, since both horizontal maps are equivalences.
\end{proof}

It follows that any pullback of a morphism between discrete reflexive graphs is a fibration. We will see in \cref{thm:etale_fibration} that a morphism of reflexive graphs is a fibration if and only if the square
\begin{equation*}
\begin{tikzcd}
\mathcal{A} \arrow[d,swap,"f"] \arrow[r] & \Delta(\rcoeq(\mathcal{A})) \arrow[d,"\Delta(\rcoeq(f))"] \\
\mathcal{B} \arrow[r] & \Delta(\rcoeq(\mathcal{B}))
\end{tikzcd}
\end{equation*}
is a pullback square.

\begin{lem}
Consider any square (commuting or not)
\begin{equation*}
\begin{tikzcd}
\mathcal{B} \arrow[d,swap,"f"] \arrow[r,"k"] & \Delta Y \arrow[d,"\Delta g"] \\
\mathcal{A} \arrow[r,swap,"h"] & \Delta X.
\end{tikzcd}
\end{equation*}
Then the type of identifications witnessing that the square commutes is equivalent to the type of triples $(H,H',K)$ consisting of two homotopies $H,H': h_0 \circ f_0\htpy g\circ k_0$ and a homotopy witnessing that the cube
\begin{equation*}
\begin{tikzcd}
& \tilde{B}_1 \arrow[dl] \arrow[d] \arrow[dr] \\
\tilde{B}_0 \arrow[d] & \tilde{A}_1 \arrow[dl] \arrow[dr] & \tilde{B}_0 \arrow[dl,crossing over] \arrow[d] \\
\tilde{A}_0 \arrow[dr] & Y \arrow[from=ul,crossing over] \arrow[d] & \tilde{A}_0 \arrow[dl] \\
& X
\end{tikzcd}
\end{equation*}
commutes. In this cube we assign $H$ and $H'$ to the front left and right squares, respectively.
\end{lem}

\begin{thm}\label{thm:descent_rcoeq}
Consider a reflexive graph morphism $h:\mathsf{rGph}(\mathcal{A},\Delta X)$. If $h$ is a reflexive coequalizer, then the map
\begin{equation*}
h^\ast : \Big(\sm{Y:\UU}Y\to X\Big)\to \Big(\sm{\mathcal{B}:\mathsf{rGph}}\mathsf{fib\usc{}rGph}(\mathcal{B},\mathcal{A})\Big)
\end{equation*}
is an equivalence. 
\end{thm}

\begin{cor}\label{cor:descent_rcoeq}
For an cartesian morphism $f:\mathsf{rGph}(\mathcal{B},\mathcal{A})$ of reflexive graphs, the type of maps $g:X\to \mathsf{rcoeq}(\mathcal{A})$ equipped with a morphism $h:\mathsf{rGph}(\mathcal{B},\Delta X)$ and an identification witnessing that the square
\begin{equation*}
\begin{tikzcd}
\mathcal{B} \arrow[d,swap,"f"] \arrow[r,"h"] & \Delta X \arrow[d,"\Delta g"] \\
\mathcal{A} \arrow[r,swap,"\mathsf{constr}"] & \Delta(\mathsf{rcoeq}(\mathcal{A}))
\end{tikzcd}
\end{equation*}
of reflexive graph morphisms commutes, is contractible.
\end{cor}

\begin{thm}\label{thm:rcoeq_cartesian}
Consider a commuting square
\begin{equation*}
\begin{tikzcd}
\mathcal{B} \arrow[r] \arrow[d] & \Delta Y \arrow[d] \\
\mathcal{A} \arrow[r] & \Delta X 
\end{tikzcd}
\end{equation*}
of reflexive graphs. Then the following are equivalent:
\begin{enumerate}
\item The square is a pullback square of reflexive graphs.
\item The square
\begin{equation*}
\begin{tikzcd}
\mathsf{rcoeq}(\mathcal{B}) \arrow[r] \arrow[d] & Y \arrow[d] \\
\mathsf{rcoeq}(\mathcal{A}) \arrow[r] & X 
\end{tikzcd}
\end{equation*}
of types is a pullback square.
\end{enumerate}
\end{thm}

\begin{comment}
\subsection{Equifibered families over reflexive graphs}

In preparation for the full descent theorem we will show in \cref{prp:equifib_fam_equiv} that for any reflexive graph $\mathcal{A}$, the type of (small) type families over the reflexive coequalizer $\mathsf{rcoeq}(\mathcal{A})$ is equivalent to the type of \emph{equifibered families} over $\mathcal{A}$, and in \cref{prp:rcoeq_flattening} we will prove the flattening `lemma' for reflexive coequalizer.

\begin{defn}

\end{defn}

\begin{defn}
We define an operation
\begin{equation*}
\mathsf{equifib\usc{}fam} : (\mathsf{rcoeq}(\mathcal{A})\to\UU)\to \mathsf{equifib}(\mathcal{A})
\end{equation*}
for each reflexive graph $\mathcal{A}$, by
\begin{align*}
\pts{\mathsf{equifib\usc{}fam}(P)} & \defeq P \circ \pts{\mathsf{constr}} \\
\edg{\mathsf{equifib\usc{}fam}(P)}(e) & \defeq \mathsf{tr}_P(\edg{\mathsf{constr}}(e)) \\
\rfx{\mathsf{equifib\usc{}fam}(P)}(i) & \defeq \mathsf{htpy\usc{}eq}(\ap{\mathsf{tr}_P}{\rfx{\mathsf{constr}}(i)}).
\end{align*}
\end{defn}

\begin{prp}\label{prp:equifib_fam_equiv}
The operation
\begin{equation*}
\mathsf{equifib\usc{}fam} : (\mathsf{rcoeq}(\mathcal{A})\to\UU)\to \mathsf{equifib}(\mathcal{A})
\end{equation*}
is an equivalence for each reflexive graph $\mathcal{A}$.
\end{prp}

\begin{proof}
Note that there is a commuting square
\begin{equation*}
\begin{tikzcd}
\big(\mathsf{rcoeq}(\mathcal{A})\to\UU\big) \arrow[r] \arrow[d,swap,"\mathsf{cocone\usc{}map}(\mathsf{constr}(\mathcal{A}))","\eqvsym"'] & \mathsf{equifib}(\mathcal{A}) \arrow[from=d,swap,"\eqvsym"] \\
\mathsf{rGph}(\mathcal{A},\Delta \UU) \arrow[r,swap,"i\circ\blank"] & \mathsf{rGph}(\mathcal{A},\UU^{\eqvsym}),
\end{tikzcd}
\end{equation*}
where $\UU^{\eqvsym}$ is the reflexive graph consisting of
\begin{align*}
{\UU^{\eqvsym}_0} & \defeq \UU \\
{\UU^{\eqvsym}_1}(X,Y) & \defeq (\eqv{X}{Y}) \\
\rfx{(\UU^{\eqvsym})}(X) & \defeq \idfunc[X].
\end{align*}
Since $\UU^{\eqvsym}$ is a discrete graph by the univalence axiom, it follows that the canonical morphism $i:\mathsf{rGph}(\Delta\UU,\UU^{\eqvsym})$ is an equivalence of graphs. Furthermore, the only difference between the type of equifibered families over $\mathcal{A}$ and the type of morphisms $\mathsf{rGph}(\mathcal{A},\UU^{\eqvsym})$ is that in an equifibered family reflexivity is required to be preserved up to homotopy, whereas in a morphism reflexivity is required to be preserved up to identification. Therefore the canonical map $\mathsf{rGph}(\mathcal{A},\UU^{\eqvsym})\to\mathsf{equifib}(\mathcal{A})$ is an equivalence by function extensionality. Thus we see that $\mathsf{equifib\usc{}fam}(\mathcal{A})$ is a composite of equivalences, so the claim follows.
\end{proof}

\begin{rmk}
Using the walking arrow $\mathcal{I}$ it can be shown that if the operation $\mathsf{equifib\usc{}fam}(\mathcal{I})$ is an equivalence, then the univalence axiom holds \cite{RijkeSpittersBarcelona}. Thus: `an interval satisfying descent implies univalence'.
\end{rmk}

\begin{cor}
Let $\mathcal{E}$ be an equifibered family over $\mathcal{A}$. The type of small families $E$ over $\mathsf{rcoeq}(\mathcal{A})$ equipped with
\begin{enumerate}
\item a family of equivalences
\begin{align*}
\varphi_0 & : \prd{x:\pts{A}} E_0(x) \eqvsym E(\mathsf{constr}_0(x)),
\end{align*}
\item for each edge $e:\edg{A}(x,y)$ a commuting square
\begin{equation*}
\begin{tikzcd}
E_0(x) \arrow[d,swap,"E_1(e)"] \arrow[r,"\varphi_0(x)"] & E(\mathsf{constr}_0(x)) \arrow[d,"\mathsf{tr}_{E}(\mathsf{constr}_1(e))"] \\
E_0(y) \arrow[r,swap,"\varphi_0(y)"] & E(\mathsf{constr}_0(y)),
\end{tikzcd}
\end{equation*}
\item for each $x:A_0$ a homotopy witnessing that the triangle
\begin{equation*}
\begin{tikzcd}[column sep=tiny]
\varphi_0(x)\circ E_1(\rfx{\mathcal{A}}(x)) \arrow[rr,"\varphi_1(\rfx{\mathcal{A}}(x))"] \arrow[dr,swap,"\varphi_0(x)\cdot \rfx{\mathcal{E}}(x)"] & & \mathsf{tr}_{E}(\mathsf{constr}_1(\rfx{\mathcal{A}}(x)))\circ \varphi_0(x) \arrow[dl,"\mathsf{htpy\usc{}eq}(\ap{\mathsf{tr}_{E}}{\rfx{\mathsf{constr}}(x)})\cdot\varphi_0(x)"] \\
& \varphi_0(x),
\end{tikzcd}
\end{equation*}
\end{enumerate}
is contractible.
\end{cor}

\begin{proof}
The described type is equivalent to the fiber of $\mathsf{equifib\usc{}fam}$. 
\end{proof}

\begin{defn}
Let $\mathcal{E}$ be an equifibered family over $\mathcal{A}$. 
We define $\rcoeq(\mathcal{E}):\rcoeq (\mathcal{A})\to\UU$ to be the type family
$\mathsf{equifib\usc{}fam}^{-1}(\mathcal{E})$. 
\end{defn}

By \cref{cor:descent_rcoeq} there is a commuting square
\begin{equation*}
\begin{tikzcd}[column sep=huge]
\msm{\mathcal{A}}{\mathcal{E}} \arrow[d,swap,"\proj 1"] \arrow[r,"\mathsf{tot\usc{}constr}"] & \Delta\big(\sm{x:\mathsf{rcoeq}(\mathcal{A})}\mathsf{rcoeq}(\mathcal{E},x)\big) \arrow[d,"\Delta(\proj 1)"] \\
\mathcal{A} \arrow[r,swap,"\mathsf{constr}"] & \Delta(\mathsf{rcoeq}(\mathcal{A}))
\end{tikzcd}
\end{equation*}
of reflexive graph morphisms.

\begin{prp}\label{prp:rcoeq_flattening}
For any equifibered family over $\mathcal{A}$, the reflexive graph morphism
\begin{equation*}
\mathsf{tot\usc{}constr}:\mathsf{rGph}\Big(\msm{\mathcal{A}}{\mathcal{E}},\Delta\big(\sm{x:\mathsf{rcoeq}(\mathcal{A})}\mathsf{rcoeq}(\mathcal{E},x)\big)\Big)
\end{equation*}
is a reflexive coequalizer.
\end{prp}

\subsection{The descent theorem for reflexive graphs}

\begin{defn}\label{defn:graphs_fam}
A \define{family $\mathcal{B}$ of graphs over $\mathcal{A}$} is a triple $\pairr{\pts{B},\edg{B},\rfx{B}}$ consisting of
\begin{align*}
\pts{B} & : \pts{A}\to\UU \\
\edg{B} & : \prd*{i,j:\pts{A}}{q:\edg{A}(i,j)} \pts{B}(i)\to\pts{B}(j)\to\UU \\
\rfx{\mathcal{B}} & : \prd{i:\pts{A}}{x:\pts{B}(i)} \edg{B}(\rfx{\mathcal{A}}(i),x,x).
\end{align*}
We will write $\mathsf{fam}(\mathcal{A})$ for the type of families of (small) graphs over $\mathcal{A}$. 
\end{defn}

\begin{defn}\label{graph:ext}
Suppose that $\mathcal{B}$ is a family of reflexive graphs over $\mathcal{A}$. Then we define the \define{total graph} of $\mathcal{B}$ to be the reflexive graph $\msm{\mathcal{A}}{\mathcal{B}}$ given by
\begin{align*}
\pts{\msm{\mathcal{A}}{\mathcal{B}}} & \defeq \sm{i:\pts{A}}\pts{B}(i)\\
\edg{\msm{\mathcal{A}}{\mathcal{B}}}(\pairr{i,x},\pairr{j,y}) & \defeq \sm{q:\edg{A}(i,j)}\edg{B}(q,x,y)\\
\rfx{\msm{\mathcal{A}}{\mathcal{B}}}(\pairr{i,x}) & \defeq \pairr{\rfx{\mathcal{A}}(i),\rfx{\mathcal{B}}(x)}.
\end{align*}
Furthermore, we define the reflexive graph morphism $\proj 1:\mathsf{rGph}(\msm{\mathcal{A}}{\mathcal{B}},\mathcal{A})$ by
\begin{align*}
\pts{(\proj 1)} & \defeq \proj 1 \\
\edg{(\proj 1)}((i,x),(j,y)) & \defeq \proj 1 \\
\rfx{(\proj 1)}((i,x)) & \defeq \refl{\rfx{\mathcal{A}}(i)}.
\end{align*}
\end{defn}

\begin{prp}
For any small reflexive graph $\mathcal{A}$, the type $\mathsf{fam}_\UU(\mathcal{A})$ is equivalent to the type
\begin{equation*}
\sm{\mathcal{B}:\mathsf{rGph}}\mathsf{rGph}(\mathcal{B},\mathcal{A}).
\end{equation*}
\end{prp}

\begin{eg}
We define the universe $\mathcal{G}$ of reflexive graphs to be the reflexive graph consisting of
\begin{align*}
\pts{\mathcal{G}} & \defeq \mathsf{rGph} \\
\edg{\mathcal{G}}(\mathcal{A},\mathcal{B}) & \defeq \pts{A}\to\pts{B}\to\UU \\
\rfx{\mathcal{G}}(\mathcal{A}) & \defeq \edg{A},
\end{align*}
and we define the \define{universal family of reflexive graphs} $\tilde{\mathcal{G}}$ over $\mathcal{G}$ to consist of
\begin{align*}
\pts{\tilde{\mathcal{G}}}(\mathcal{A}) & \defeq \pts{A} \\
\edg{\tilde{\mathcal{G}}}(S) & \defeq S \\
\rfx{\tilde{\mathcal{G}}}(\mathcal{A}) & \defeq \rfx{\mathcal{A}}.
\end{align*}
This family of graphs is universal in the sense that for any reflexive graphs $\mathcal{A}$, the pre-composition map $\alpha\mapsto \tilde{\mathcal{G}}\cdot\alpha$ of type
\begin{equation*}
\begin{tikzcd}
\mathsf{rGph}({\mathcal{A}},{\mathcal{G}})\to\tffam{\mathcal{A}}
\end{tikzcd}
\end{equation*}
is an equivalence.

To see this, note that a morphism $\chi:\mathcal{A}\to\mathcal{G}$ is a triple $\pairr{\mathcal{G},R,\rho}$ consisting of
\begin{align*}
\mathcal{G} & : \pts{A}\to\mathsf{rGraph} \\
R & : \prd*{i,j:\pts{A}}{q:\edg{A}(i,j)} \pts{\mathcal{G}(i)}\to\pts{\mathcal{G}(j)}\to\UU \\
\rho & : \prd{i:\pts{A}} \edg{\mathcal{G}(i)}=R(\rfx{\mathcal{A}}(i)). 
\end{align*}
The type of such triples $\pairr{\mathcal{G},R,\rho}$ is equivalent to the type of pentuples $\pairr{\pts{G},\edg{G},\rfx{G},R,\rho}$ consisting of
\begin{align*}
\pts{G} & : \pts{A}\to\UU \\
\edg{G} & : \prd*{i:\pts{A}} \pts{G}(i)\to\pts{G}(i)\to\UU \\
\rfx{G} & : \prd{i:\pts{A}}{g:\pts{G}(i)} \edg{G}(g,g) \\
R & : \prd*{i,j:\pts{A}}{q:\edg{A}(i,j)} \pts{G}(i)\to\pts{G}(j)\to\UU \\
\rho & : \prd{i:\pts{A}} \edg{G}(i)=R(\rfx{\mathcal{A}}(i)).
\end{align*}
Note that the type of $R$ does not depend on $\edg{G}$ or $\rfx{G}$, and that the type of $\rho$ does not depend on $\rfx{G}$. Thus, we can use the contractibility of the type of pairs $\pairr{\edg{G},\rho}$ to conclude that the type of pentuples $\pairr{\pts{G},\edg{G},\rfx{G},R,\rho}$ as described above is equivalent to the type of triples $\pairr{\pts{G},R,\rfx{G}}$ consisting of
\begin{align*}
\pts{G} & : \pts{A}\to\UU \\
R & : \prd*{i,j:\pts{A}}{q:\edg{A}(i,j)} \pts{G}(i)\to\pts{G}(j)\to\UU \\
\rfx{G} & : \prd*{i:\pts{A}}{g:\pts{G}(i)} R(\rfx{A}(i),g,g).
\end{align*}
Note that this is precisely the type of families of reflexive graphs over $\mathcal{A}$. Explicitly, the equivalence we have constructed is given by
\begin{equation*}
\pairr{\mathcal{G},R,\rho}\mapsto \pairr{(\lam{i}\pts{\mathcal{G}(i)}),R,(\lam{i}{g}\trans{\rho(i)}{\rfx{\mathcal{G}(i)}(g)})}.
\end{equation*}
By inspection one sees that this map is the map 
\begin{equation*}
\pairr{\mathcal{G},R,\rho}\mapsto \tilde{U}\cdot\pairr{\mathcal{G},R,\rho}
\end{equation*} 
of which the theorem asserts that it is an equivalence.
\end{eg}

\begin{rmk}
There also is a universal family of non-reflexive graphs, which is considerably simpler. However, the universe of non-reflexive graphs is \emph{not} obtained by simply forgetting the reflexivity of $\mathcal{G}$. Instead, the universe $\mathcal{S}$ of non-reflexive graphs is given by
\begin{align*}
\pts{\mathcal{S}} & \defeq \UU \\
\edg{\mathcal{S}}(A, B) & \defeq A \to B \to\UU,
\end{align*}
and its universal family $\tilde{\mathcal{S}}$ over $\mathcal{S}$ is given by
\begin{align*}
\pts{\tilde{\mathcal{S}}}(A) & \defeq A \\
\edg{\tilde{\mathcal{S}}}(R) & \defeq R.
\end{align*}
Then it can be shown that the canonical map
\begin{equation*}
\begin{tikzcd}
\mathsf{Gph}({\mathcal{A}},{\mathcal{S}})\to\tffam{\mathcal{A}}
\end{tikzcd}
\end{equation*}
is an equivalence, where $\tffam{\mathcal{A}}$ is the type of small families of non-reflexive graphs.
\end{rmk}

\begin{defn}\label{defn:functional}
Let $R:A\to (B\to \UU)$ be a binary relation from $A$ to $B$.
\begin{enumerate}
\item $R$ is said to be \define{functional} if there is a term of type
\begin{equation*}
\isfunction (R) \defeq \prd{x:A}\iscontr \big(\sm{y:B}R(x,y)\big).
\end{equation*}
\item The \define{opposite} $R^{op}:B\to (A\to \UU)$ of a relation $R:A\to (B\to \UU)$ is defined by $R^{op}(b,a)\defeq R(a,b)$.
\item $R$ is said to be a \define{correspondence} if both $R$ and $R^{op}$ are functional. We will write $\mathsf{is\usc{}corr}(R)$ for the property that $R$ is a correspondence.
\end{enumerate}
\end{defn}

\begin{lem}[Exercise 4.2 in \cite{hottbook}]\label{lem:fun_rel}
Let $A$ and $B$ be types.
\begin{enumerate}
\item The map $f\mapsto \lam{x}{y}(f(x)=y)$ with the canonical proof that the relation associated to $f$ is functional is an equivalence
\begin{equation*}
(A\to B) \eqvsym {\sm{R:A\to B\to \UU}\isfunction (R)}.
\end{equation*}
\item The map $e\mapsto \lam{x}{y}(e(x)=y)$ with the canonical proof that the relation associated to $e$ is a correspondence is an equivalence
\end{enumerate}
\begin{equation*}
(\eqv{A}{B}) \eqvsym {\sm{R:A\to B\to\UU}\mathsf{is\usc{}corr}(R)}.\qedhere
\end{equation*}
\end{lem}

\begin{defn}
Let $\mathcal{B}$ be a family of reflexive graphs over $\mathcal{A}$. 
\begin{enumerate}
\item We say that $\mathcal{B}$ is \define{diagrammatic} if each relation
\begin{equation*}
\edg{B}(e):\pts{B}(i)\to \pts{B}(j)\to\UU
\end{equation*}
is functional. We write $\mathsf{is\usc{}diagr}(\mathcal{B})$ for the proposition that $\mathcal{B}$ is diagrammatic.
\item We say that $\mathcal{B}$ is \define{equifibered} if each relation
\begin{equation*}
\edg{B}(e):\pts{B}(i)\to \pts{B}(j)\to\UU
\end{equation*}
is a correspondence. We write $\mathsf{is\usc{}eqf}(\mathcal{B})$ for the proposition that $\mathcal{B}$ is equifibered. 
\end{enumerate}
\end{defn}

\begin{lem}
The type of diagrammatic families over $\mathcal{A}$ is equivalent to the type of diagrams over $\mathcal{A}$.
\end{lem}

\begin{lem}
The type of families $\mathcal{B}$ over $\mathcal{A}$ satisfying $\mathsf{is\usc{}eqf}(\mathcal{B})$ is equivalent to the type of equifibered families over $\mathcal{A}$.
\end{lem}

\begin{thm}
Let $\mathcal{B}$ be a family of reflexive graphs over $\mathcal{A}$. The following are equivalent:
\begin{enumerate}
\item $\mathcal{B}$ is diagrammatic.
\item The morphism $\proj 1 : \mathsf{rGph}(\msm{\mathcal{A}}{\mathcal{B}},\mathcal{A})$ is a left fibration.
\item The morphism $\proj 1 : \mathsf{rGph}(\msm{\mathcal{A}}{\mathcal{B}},\mathcal{A})$ is left cartesian.
\end{enumerate}
\end{thm}

\begin{thm}\label{thm:fib_cart}
Let $\mathcal{B}$ be a family of reflexive graphs over $\mathcal{A}$. The following are equivalent:
\begin{enumerate}
\item $A$ is equifibered.
\item The morphism $\proj 1 : \mathsf{rGph}(\msm{\mathcal{A}}{\mathcal{B}},\mathcal{A})$ is a fibration.
\item The morphism $\proj 1 : \mathsf{rGph}(\msm{\mathcal{A}}{\mathcal{B}},\mathcal{A})$ is cartesian.
\end{enumerate}
\end{thm}
\end{comment}

\section{Colimits of diagrams over graphs}

\begin{defn}
Let $\mathcal{A}$ be a reflexive graph. A diagram $\mathcal{D}$ over $\mathcal{A}$ is a triple $(\pts{D},\edg{D},\rfx{\mathcal{D}})$ consisting of
\begin{align*}
\pts{D} & : \pts{A}\to \UU \\
\edg{D} & : \prd{i,j:\pts{A}} \edg{A}(i,j)\to (\pts{D}(i)\to \pts{D}(j)) \\
\rfx{\mathcal{D}} & : \prd{i:\pts{A}} \edg{D}(\rfx{\mathcal{A}}(i))\htpy \idfunc[\pts{D}(i)].
\end{align*}
\end{defn}

\begin{defn}
Let $\mathcal{D}$ be a diagram over $\mathcal{A}$. We form the \define{total graph} $\msm{\mathcal{A}}{\mathcal{D}}$ of $\mathcal{D}$ by
\begin{align*}
\pts{\msm{\mathcal{A}}{\mathcal{D}}} & \defeq \sm{i:\pts{A}}\pts{D}(i) \\
\edg{\msm{\mathcal{A}}{\mathcal{D}}}((i,x),(j,y)) & \defeq \sm{e:\edg{A}(i,j)} \edg{D}(e,x)=y \\
\rfx{\msm{\mathcal{A}}{\mathcal{D}}}((i,x)) & \defeq (\rfx{\mathcal{A}}(i),\rfx{\mathcal{D}}(i,x)).
\end{align*}
There is an obvious projection $\proj 1 : \mathsf{rGph}(\msm{\mathcal{A}}{\mathcal{D}},\mathcal{A})$.
\end{defn}

\begin{lem}
For any diagram $\mathcal{D}$ over $\mathcal{A}$, the projection $\proj 1 : \mathsf{rGph}(\msm{\mathcal{A}}{\mathcal{D}},\mathcal{A})$ is a left fibration.
\end{lem}

\begin{defn}
Let $D$ be a diagram over $\mathcal{A}$, and let $X$ be a type. A $D$-cocone on $X$ is a morphism $f:\mathsf{rGph}(\msm{\mathcal{A}}{\mathcal{D}},\Delta X)$ of reflexive graphs. A $D$-cocone $f$ on $X$ is colimiting if the map
\begin{equation*}
\Delta(\blank)\circ f : (X\to Y)\to \mathsf{rGph}(\msm{\mathcal{A}}{\mathcal{D}},\Delta Y)
\end{equation*}
is an equivalence for every type $Y$. 
\end{defn}

\begin{rmk}
By \cref{thm:rcoeq_is_pushout} it follows that every diagram $D$ over any reflexive graph $\mathcal{A}$ has a colimit. 
\end{rmk}

\begin{defn}
Let $\mathcal{D}$ and $\mathcal{D}'$ be diagrams over $\mathcal{A}$. A \define{natural transformation} $\tau : \mathcal{D}'\to \mathcal{D}$ of diagrams consists of
\begin{align*}
\pts{\tau} & : \prd{i:\pts{A}}\pts{D'}(i)\to\pts{D}(i) \\
\edg{\tau} & : \prd*{i,j:\pts{A}}{e:\edg{A}(i,j)}\prd{x:\pts{D}(i)}\pts{\tau}(\edg{D'}(e,x))=\edg{D}(e,\pts{\tau}(x)) \\
\rfx{\tau} & : \prd{i:\pts{A}}{x:\pts{D}(i)} \dpath{}{}{\edg{\tau}(\rfx{A}(i),x)}{\refl{\pts{\tau}(x)}}
\end{align*}
\end{defn}

\begin{defn}\label{defn:cartesian}
A natural transformation $\tau:\mathcal{D}'\to \mathcal{D}$ of diagrams over $\mathcal{A}$ is said to be \define{cartesian} if the commutative squares
\begin{equation*}
\begin{tikzcd}[column sep=large]
\pts{D'}(i) \arrow[r,"{\edg{D'}(e)}"] \arrow[d,swap,"\pts{\tau}(i)"] & \pts{D'}(j) \arrow[d,"\pts{\tau}(j)"] \\
\pts{D}(i) \arrow[r,swap,"{\edg{D}(e)}"] & \pts{D}(i)
\end{tikzcd}
\end{equation*}
are all pullback squares. 
%We will write $D'\mathbin{\to_\times}D$ for the type of cartesian natural transformations from $D'$ to $D$.
\end{defn}

We need the following general fact about pullbacks.

\begin{prp}\label{lem:pb_total}
Let $I$ be a type, and consider for each $i:I$ a commuting square
\begin{equation*}
\begin{tikzcd}
C_i \arrow[r,"q_i"] \arrow[d,swap,"p_i"] & B_i \arrow[d,"g_i"] \\
A_i \arrow[r,swap,"f_i"] & X_i.
\end{tikzcd}
\end{equation*}
with $H_i:f_i\circ p_i\htpy g_i\circ q_i$. Then the following are equivalent:
\begin{enumerate}
\item For each $i:I$, the square is a pullback square.
\item The induced square on total spaces
\begin{equation*}
\begin{tikzcd}
\sm{i:I}C_i \arrow[r,"\total{q}"] \arrow[d,swap,"\total{p}"] & \sm{i:I}B_i \arrow[d,"\total{g}"] \\
\sm{i:I}A_i \arrow[r,swap,"\total{f}"] & \sm{i:I}X_i.
\end{tikzcd}
\end{equation*}
which commutes via the homotopy $\total{H}:\total{f}\circ\total{p}\htpy\total{g}\circ\total{q}$, is a pullback square.
\end{enumerate}
\end{prp}

\begin{proof}
The gap map of the square in assertion (ii) factors as follows:
\begin{equation*}
\begin{tikzcd}[column sep=small]
\phantom{\Big(\sm{i:I}A_i\Big)\times_{\big(\sm{i:I}X_i\big)} \Big(\sm{i:I}B_i\Big)} & \sm{i:I}C_i \arrow[dl,swap,"\total{\mathsf{gap}(p_i,q_i,H_i)}"] \arrow[dr,"{\mathsf{gap}(\total{p},\total{q},\total{H})}"] \\
\sm{i:I}A_i\times_{X_i}B_i \arrow[rr,swap,"{\lam{(i,a,b,p)}((i,a),(i,b),(\refl{i},p))}"] & & \Big(\sm{i:I}A_i\Big)\times_{\big(\sm{i:I}X_i\big)} \Big(\sm{i:I}B_i\Big)
\end{tikzcd}
\end{equation*}
and the bottom map is an equivalence. Therefore it follows by the 3-for-2 property and an application of \cref{thm:fib_equiv} that $\mathsf{gap}(p_i,q_i,H_i)$ is an equivalence for each $i:I$, if and only if $\mathsf{gap}(\total{p},\total{q},\total{H})$ is an equivalence.
\end{proof}

\begin{prp}
Let $\tau:\mathsf{nat}(\mathcal{D}',\mathcal{D})$ be a natural transformation of diagrams over $\mathcal{A}$. Then $\tau$ is cartesian if and only if the induced morphism of reflexive graphs
\begin{equation*}
\total{\tau} : \mathsf{rGph}(\msm{\mathcal{A}}{\mathcal{D}'},\msm{\mathcal{A}}{\mathcal{D}})
\end{equation*}
is cartesian.
\end{prp}

\begin{proof}
Straightforward consequence of \cref{lem:pb_total}.
\end{proof}

\begin{cor}
Let $\tau:\mathsf{cart}(\mathcal{E},\mathcal{D})$ be a cartesian morphisms of diagrams over $\mathcal{A}$, and consider a commuting square of reflexive graphs of the following form
\begin{equation*}
\begin{tikzcd}
\msm{\mathcal{A}}{\mathcal{E}} \arrow[d,swap,"\tau"] \arrow[r] & \Delta Y \arrow[d] \\
\msm{\mathcal{A}}{\mathcal{D}} \arrow[r] & \Delta X.
\end{tikzcd}
\end{equation*}
Then the following are equivalent:
\begin{enumerate}
\item The square is a pullback square.
\item The square
\begin{equation*}
\begin{tikzcd}
\mathsf{colim}(\mathcal{E}) \arrow[r] \arrow[d,swap,"\mathsf{colim}(\tau)"] & Y \arrow[d] \\
\mathsf{colim}(\mathcal{D}) \arrow[r] & X
\end{tikzcd}
\end{equation*}
is a pullback square.
\end{enumerate}
\end{cor}

\section{Sequential colimits}

\marginnote{Boil this down to the most essential theorems}
\begin{defn}
We define the non-reflexive graph $\N^{\prec}$ by 
\begin{align*}
\pts{(\N^{\prec})} & \defeq \N \\
\edg{(\N^{\prec})}(n,m) & \defeq n+1=m \\
\end{align*}
\end{defn}

Type sequences are diagrams of the following form.
\begin{equation*}
\begin{tikzcd}
A_0 \arrow[r,"f_0"] & A_1 \arrow[r,"f_1"] & A_2 \arrow[r,"f_2"] & \cdots.
\end{tikzcd}
\end{equation*}
Their formal specification is as follows.

\begin{defn}
An \define{(increasing) type sequence} $\mathcal{I}$ consists of
\begin{align*}
A & : \N\to\UU \\
f & : \prd{n:\N} A_n\to A_{n+1}. 
\end{align*}
\end{defn}

In this section we will introduce the sequential colimit of a type sequence.
The sequential colimit includes each of the types $A_n$, but we also identify each $x:A_n$ with its value $f_n(x):A_{n+1}$. 
Imagine that the type sequence $A_0\to A_1\to A_2\to\cdots$ defines a big telescope, with $A_0$ sliding into $A_1$, which slides into $A_2$, and so forth.

As usual, the sequential colimit is characterized by its universal property.

\begin{defn}
\begin{enumerate}
\item A \define{(sequential) cocone} on a type sequence $\mathcal{I}$ with vertex $B$ consists of
\begin{align*}
h & : \prd{n:\N} A_n\to B \\
H & : \prd{n:\N} f_n\htpy f_{n+1}\circ H_n.
\end{align*}
We write $\mathsf{cocone}(B)$ for the type of cones with vertex $X$.
\item Given a cone $(h,H)$ with vertex $B$ on a type sequence $\mathcal{I}$ we define the map
\begin{equation*}
\mathsf{cocone\usc{}map}(h,H) : (B\to C)\to \mathsf{cocone}(B)
\end{equation*}
given by $f\mapsto (f\circ h,\lam{n}{x}\mathsf{ap}_f(H_n(x)))$. 
\item We say that a cone $(h,H)$ with vertex $B$ is \define{colimiting} if $\mathsf{cocone\usc{}map}(h,H)$ is an equivalence for any type $C$. 
\end{enumerate}
\end{defn}

\begin{thm}\label{thm:sequential_up}
Consider a cocone $(h,H)$ with vertex $B$ for a type sequence $\mathcal{I}$. The following are equivalent:
\begin{enumerate}
\item The cocone $(h,H)$ is colimiting.
\item The cocone $(h,H)$ is inductive in the sense that for every type family $P:B\to \UU$, the map
\begin{align*}
\Big(\prd{b:B}P(b)\Big)\to {}& \sm{h:\prd{n:\N}{x:A_n}P(h_n(x))}\\ 
& \qquad \prd{n:\N}{x:A_n} \mathsf{tr}_P(H_n(x),h_n(x))={h_{n+1}(f_n(x))}
\end{align*}
given by
\begin{equation*}
s\mapsto (\lam{n}s\circ h_n,\lam{n}{x} \mathsf{apd}_{s}(H_n(x)))
\end{equation*}
has a section.
\item The map in (ii) is an equivalence.
\end{enumerate}
\end{thm}

\subsection{Type families over type sequences}

\begin{defn}
The type of \define{descent data} on a type sequence $\mathcal{A}\jdeq (A,f)$ is defined to be
\begin{equation*}
\mathsf{Desc}(\mathcal{A}) \defeq \sm{B:\prd{n:\N}A_n\to\UU}\prd{n:\N}{x:A_n}\eqv{B_n(x)}{B_{n+1}(f_n(x))}.
\end{equation*}
\end{defn}

\begin{defn}
We define a map
\begin{equation*}
\mathsf{desc\usc{}fam} : (A_\infty\to\UU)\to\mathsf{Desc}(\mathcal{A})
\end{equation*}
by $B\mapsto (\lam{n}{x}B(\mathsf{seq\usc{}in}(n,x)),\lam{n}{x}\mathsf{tr}_B(\mathsf{seq\usc{}glue}(n,x)))$.
\end{defn}

\begin{thm}
The map 
\begin{equation*}
\mathsf{desc\usc{}fam} : (A_\infty\to\UU)\to\mathsf{Desc}(\mathcal{A})
\end{equation*}
is an equivalence.
\end{thm}

\begin{defn}
A \define{cartesian transformation} of type sequences from $\mathcal{A}$ to $\mathcal{B}$ is a pair $(h,H)$ consisting of
\begin{align*}
h & : \prd{n:\N} A_n\to B_n \\
H & : \prd{n:\N} g_n\circ h_n \htpy h_{n+1}\circ f_n,
\end{align*}
such that each of the squares in the diagram
\begin{equation*}
\begin{tikzcd}
A_0 \arrow[d,swap,"h_0"] \arrow[r,"f_0"] & A_1 \arrow[d,swap,"h_1"] \arrow[r,"f_1"] & A_2 \arrow[d,swap,"h_2"] \arrow[r,"f_2"] & \cdots \\
B_0 \arrow[r,swap,"g_0"] & B_1 \arrow[r,swap,"g_1"] & B_2 \arrow[r,swap,"g_2"] & \cdots
\end{tikzcd}
\end{equation*}
is a pullback square. We define
\begin{align*}
\mathsf{cart}(\mathcal{A},\mathcal{B}) & \defeq\sm{h:\prd{n:\N}A_n\to B_n} \\
& \qquad\qquad \sm{H:\prd{n:\N}g_n\circ h_n\htpy h_{n+1}\circ f_n}\prd{n:\N}\mathsf{is\usc{}pullback}(h_n,f_n,H_n),
\end{align*}
and we write
\begin{equation*}
\mathsf{Cart}(\mathcal{B}) \defeq \sm{\mathcal{A}:\mathsf{Seq}}\mathsf{cart}(\mathcal{A},\mathcal{B}).
\end{equation*}
\end{defn}

\begin{defn}
We define a map
\begin{equation*}
\mathsf{cart\usc{}map}(\mathcal{B}) : \Big(\sm{X':\UU}X'\to X\Big)\to\mathsf{Cart}(\mathcal{B}).
\end{equation*}
which associates to any morphism $h:X'\to X$ a cartesian transformation of type sequences into $\mathcal{B}$.
\end{defn}

\begin{thm}
The operation $\mathsf{cart\usc{}map}(\mathcal{B})$ is an equivalence.
\end{thm}

\subsection{The flattening lemma for sequential colimits}

The flattening lemma for sequential colimits essentially states that sequential colimits commute with $\Sigma$. 

\begin{lem}
Consider
\begin{align*}
B & : \prd{n:\N}A_n\to\UU \\
g & : \prd{n:\N}{x:A_n}\eqv{B_n(x)}{B_{n+1}(f_n(x))}.
\end{align*}
and suppose $P:A_\infty\to\UU$ is the unique family equipped with
\begin{align*}
e & : \prd{n:\N}\eqv{B_n(x)}{P(\mathsf{seq\usc{}in}(n,x))}
\end{align*}
and homotopies $H_n(x)$ witnessing that the square
\begin{equation*}
\begin{tikzcd}[column sep=7em]
B_n(x) \arrow[r,"g_n(x)"] \arrow[d,swap,"e_n(x)"] & B_{n+1}(f_n(x)) \arrow[d,"e_{n+1}(f_n(x))"] \\
P(\mathsf{seq\usc{}in}(n,x)) \arrow[r,swap,"{\mathsf{tr}_P(\mathsf{seq\usc{}glue}(n,x))}"] & P(\mathsf{seq\usc{}in}(n+1,f_n(x)))
\end{tikzcd}
\end{equation*}
commutes. Then $\sm{t:A_\infty}P(t)$ satisfies the universal property of the sequential colimit of the type sequence
\begin{equation*}
\begin{tikzcd}
\sm{x:A_0}B_0(x) \arrow[r,"{\total[f_0]{g_0}}"] & \sm{x:A_1}B_1(x) \arrow[r,"{\total[f_1]{g_1}}"] & \sm{x:A_2}B_2(x) \arrow[r,"{\total[f_2]{g_2}}"] & \cdots.
\end{tikzcd}
\end{equation*}
\end{lem}

In the following theorem we rephrase the flattening lemma in using cartesian transformations of type sequences.

\begin{thm}
Consider a commuting diagram of the form
\begin{equation*}
\begin{tikzcd}[column sep=small,row sep=small]
A_0 \arrow[rr] \arrow[dd] & & A_1 \arrow[rr] \arrow[dr] \arrow[dd] &[-.9em] &[-.9em] A_2 \arrow[dl] \arrow[dd] & & \cdots \\
& & & X \arrow[from=ulll,crossing over] \arrow[from=urrr,crossing over] \arrow[from=ur,to=urrr] \\
B_0 \arrow[rr] \arrow[drrr] & & B_1 \arrow[rr] \arrow[dr] & & B_2 \arrow[rr] \arrow[dl] & & \cdots \arrow[dlll] \\
& & & Y \arrow[from=uu,crossing over] 
\end{tikzcd}
\end{equation*}
If each of the vertical squares is a pullback square, and $Y$ is the sequential colimit of the type sequence $B_n$, then $X$ is the sequential colimit of the type sequence $A_n$. 
\end{thm}

\begin{cor}
Consider a commuting diagram of the form
\begin{equation*}
\begin{tikzcd}[column sep=small,row sep=small]
A_0 \arrow[rr] \arrow[dd] & & A_1 \arrow[rr] \arrow[dr] \arrow[dd] &[-.9em] &[-.9em] A_2 \arrow[dl] \arrow[dd] & & \cdots \\
& & & X \arrow[from=ulll,crossing over] \arrow[from=urrr,crossing over] \arrow[from=ur,to=urrr] \\
B_0 \arrow[rr] \arrow[drrr] & & B_1 \arrow[rr] \arrow[dr] & & B_2 \arrow[rr] \arrow[dl] & & \cdots \arrow[dlll] \\
& & & Y \arrow[from=uu,crossing over] 
\end{tikzcd}
\end{equation*}
If each of the vertical squares is a pullback square, then the square
\begin{equation*}
\begin{tikzcd}
A_\infty \arrow[r] \arrow[d] & X \arrow[d] \\
B_\infty \arrow[r] & Y
\end{tikzcd}
\end{equation*} 
is a pullback square.
\end{cor}

\section{Type sequences}
Infinite sequences of types can be seen as diagrams over the graph of natural
numbers with an edge between each number and its successor. Since diagrams over
a graph model dependent type theory with $\Sigma$ and $\idtypevar{}$, we also
get the structure of type dependency for infinite sequences of types. We make
that structure explicit here.

\begin{defn}
A \define{type sequence} $\sequence{A}{f}$ consists of a diagram of the form
\begin{equation*}
\begin{tikzcd}
A_0 \arrow[r,"f_0"] & A_1 \arrow[r,"f_1"] & A_2 \arrow[r,"f_2"] & \cdots
\end{tikzcd}
\end{equation*}
Thus, the type of all sequences of types is
\begin{equation*}
\mathrm{Seq} \defeq \sm{A:\nat\to\type}\prd{n:\nat} A_n\to A_{n+1}
\end{equation*}
\end{defn}

Recall that the relation $\leq$ on the natural numbers is defined as an inductive family of types $\leq:\N\to\N\to\UU$ with
\begin{align*}
r & : \prd{n:\N} n\leq n \\
s & : \prd{n,m:\N} n\leq m \to n\leq m+1.
\end{align*}
It follows that $n\leq m$ is a a mere proposition for each $n,m:\N$.

\begin{defn}
Let $\sequence{A}{f}$ be a type sequence. For any $n,m:\nat$, we define
\begin{equation*}
f^{n\leq m} : A_n\to A_m.
\end{equation*}
where we leave the proof that $n\leq m$ implicit.
\end{defn}

\begin{proof}[Construction]
We define $f^{n\leq m}$ by induction on the proof that $n\leq m$ by taking
\begin{align*}
f^{n\leq n} & \defeq \idfunc[A_n] \\
f^{n\leq m+1} & \defeq f_m\circ f^{n\leq m} \qedhere
\end{align*}
\end{proof}

\begin{defn}
Let $\sequence{A}{f}$ be a type sequence. For any $n,k:\nat$, we define
$f_n^k:A_n\to A_{n+k}$ to be $f^{n\leq n+k}(p)$, where $p$ is the canonical proof that $n\leq n+k$.
\end{defn}

\begin{defn}
A \define{sequence $\sequence{B}{g}$ of types over $\sequence{A}{f}$} consists of a diagram of the form
\begin{equation*}
\begin{tikzcd}
B_{0} \arrow[r,"g_0"] \arrow[d,->>] & B_{1} \arrow[r,"g_1"] \arrow[d,->>] & B_{2} \arrow[r,"g_2"] \arrow[d,->>] & \cdots \\
A_0 \arrow[r,"f_0"] & A_1 \arrow[r,"f_1"] & A_2 \arrow[r,"f_2"] & \cdots
\end{tikzcd}
\end{equation*}
where each $g_n$ has type $\prd{a:A_n} B_n(x)\to B_{n+1}(f_n(a))$, implicitly rendering the
squares commutative.

We say that a sequence $\sequence{B}{g}$ over $\sequence{A}{f}$ is \define{equifibered} if each $g_n$ is a family of equivalences.
\end{defn}

\begin{comment}
\begin{defn}
A \define{global section} $\sequence{t}{T}$ of a sequence $\sequence{P}{f}$ over
$\sequence{A}{a}$ consists of 
\begin{align*}
t & : \prd{n:\nat}{x:A_n}P_n(x) \\
T & : \prd{n:\nat}{x:A_n}\id{t_{n+1}(a_n(x))}{f_n(t_n(x))}
\end{align*}
\end{defn}

For the following definition, recall that for any two types $X$ and $Y$, we
may consider the constant type family $\ctxwk{X}{Y}\defeq\lam{x}Y:X\to\type$.
This process is called weakening.

\begin{defn}
Let $\sequence{A}{a}$ and $\sequence{B}{b}$ be sequences of types. Then we
define the \define{weakened sequence} $\ctxwk{\sequence{A}{a}}{\sequence{B}{b}}$ over
$\sequence{A}{a}$ to consist of
\begin{equation*}
\begin{tikzcd}[column sep=large]
\ctxwk{A_0}{B_0} \arrow[r,"\lam{x} b_0"] \arrow[d,->>] & \ctxwk{A_1}{B_1} \arrow[r,"\lam{x}b_1"] \arrow[d,->>] & \ctxwk{A_2}{B_2} \arrow[r,"\lam{x}b_2"] \arrow[d,->>] & \cdots \\
A_0 \arrow[r,"a_0"] & A_1 \arrow[r,"a_1"] & A_2 \arrow[r,"a_2"] & \cdots
\end{tikzcd}
\end{equation*}
\end{defn}
\end{comment}

\begin{defn}
Let $\sequence{A}{f}$ and $\sequence{A'}{f'}$ be type sequences.
A \define{natural transformation} $\sequence{A}{f}\to\sequence{A'}{f'}$
is a pair $\sequence{\tau}{H}$ consisting of a family of maps
\begin{equation*}
\tau : \prd{n:\N} A_n \to A'_n
\end{equation*}
and a family $H_n$ of homotopies witnessing that the diagram
\begin{equation*}
\begin{tikzcd}
A_{0} \arrow[r,"f_0"] \arrow[d,"\tau_0"] & A_{1} \arrow[r,"f_1"] \arrow[d,"\tau_1"] & A_{2} \arrow[r,"f_2"] \arrow[d,"\tau_2"] & \cdots \\
A'_0 \arrow[r,"{f'_0}"] & A'_1 \arrow[r,"{f'_1}"] & A'_2 \arrow[r,"{f'_2}"] & \cdots
\end{tikzcd}
\end{equation*}
commutes.
\end{defn}

\begin{defn}
A \define{natural equivalence} is a natural
transformation $(\tau,H)$ such that each $\tau_n$ is an equivalence. 
The type of natural equivalences from $\sequence{A}{f}$ to $\sequence{A'}{f'}$
is called $\mathsf{NatEq}(\sequence{A}{f},\sequence{A'}{f'})$.
\end{defn}

\begin{lem}
The canonical dependent function $\mathsf{idtonateq}$ 
\begin{equation*}
%\prd{\sequence{A}{f},\sequence{A'}{f'}:\mathrm{Seq}} 
(\id{\sequence{A}{f}}{\sequence{A'}{f'}})\to \mathsf{NatEq}(\sequence{A}{f},\sequence{A'}{f'})
\end{equation*}
which sends $\refl{\sequence{A}{f}}$ to the identity natural transformation, is
an equivalence.
\end{lem}

\begin{proof}
Straightforward application of univalence.
\end{proof}

Every type sequence $\sequence{B}{g}$ over $\sequence{A}{f}$ gives rise to a natural transformation, by the following definition. 

\begin{defn}
Let $\sequence{B}{g}$ be a sequence over $\sequence{A}{f}$. Then we define the
sequence $\msm{\sequence{A}{f}}{\sequence{B}{g}}$ to consist of the diagram
\begin{equation*}
\begin{tikzcd}[column sep=large]
\sm{a:A_0}B_0(a) \arrow[r,"\pairr{f_0,g_0}"] & \sm{a:A_1}B_1(a) \arrow[r,"\pairr{f_1,g_1}"]
& \sm{a:A_2}B_2(a) \arrow[r,"\pairr{f_2,g_2}"] & \cdots
\end{tikzcd}
\end{equation*}
where we take the usual definition
\begin{equation*}
\pairr{f_n,g_n} \defeq \lam{\pairr{a,b}}\pairr{f_n(a),g_n(a,b)}.
\end{equation*}
Furthermore, we define a natural transformation 
\begin{equation*}
\sequence{\pi}{\theta}:\msm{\sequence{A}{f}}{\sequence{B}{g}}\to \sequence{A}{f}
\end{equation*}
by taking 
\begin{align*}
\pi_n & \defeq \proj1 & & : (\sm{a:A_n}B_n(a))\to A_n \\
\theta_n(a,b) & \defeq\refl{f_n(a)} & & : f_n(\proj 1(a,b))= \proj 1(f_n(a),g_n(b)).
\end{align*}
\end{defn}

\begin{defn}
A natural transformation is said to be \define{cartesian} if all the naturality squares are pullback squares.
\end{defn}

Of course, every natural equivalence is also cartesian. Recall that for any $f:A\to A'$ and any $g:\prd{a:A}B(a)\to B'(f(a))$, the square
\begin{equation*}
\begin{tikzcd}
\sm{a:A}B(a) \arrow[r,"\pairr{f,g}"] \arrow[d,->>] & \sm{a':A'}B'(a') \arrow[d,->>] \\
A \arrow[r,swap,"f"] & A'
\end{tikzcd}
\end{equation*}
is a pullback square if and only if $g$ is a family of equivalences. Thus, we have the following:

\begin{lem}
A family $\sequence{B}{g}$ of sequences over $\sequence{A}{f}$ is equifibered if and only if the projection transformation $(\pi,\theta):\msm{(A,f)}{(B,g)}\to (A,f)$ is cartesian. \qed
\end{lem}

We end this section with reviewing the shift operation on type sequences, in particular to bring up subtleties that come up in the formalization of mathematics in homotopy type theory. The issue we face is that equality in the natural numbers isn't always strict. For instance, when addition is defined by induction on the second argument, then $n+0$ is judgmentally equal to $n$, while $0+n$ is not. This implies that sometimes we might have to \emph{transport} along the equalities in the natural numbers (such as $n=0+n$), and this complicates the formalization process.

We define the shift operation.
\begin{defn}
For any type sequence $\sequence{A}{f}$ we define the type sequence $(S(A),S(f))$ by taking
\begin{align*}
S(A)_n & \defeq A_{n+1} \\
S(f)_n & \defeq f_{n+1}.
\end{align*}
\end{defn}

Of course we can iterated the shift operation, defining a type sequence $(S^k(A),S^k(f))$ for every $k:\N$. However, while the type $S^k(A)_n$ is $A_{n+k}$, the function $S^k(f)_n$ is some function $A_{n+k}\to A_{(n+1)+k}$ that isn't judgmentally equal to a function of the form $f_m$ for some $m:\N$. Therefore, we make the following alternative definition of the $k$-shift.

\begin{defn}
Given a type sequence $(A,f)$, we define $S_k(A,f)\jdeq(S_k(A),S_k(f))$ to be the type sequence given by
\begin{align*}
S_k(A)_n & \defeq A_{k+n} \\
S_k(f)_n & \defeq f_{k+n}.
\end{align*}
Given a dependent sequence $(B,g)$ over $(A,f)$, we also define $S_k(B,g)\jdeq (S_k(B),S_k(g))$ by 
\begin{align*}
S_k(B)_n & \defeq B_{k+n} \\
S_k(g)_n & \defeq g_{k+n}.
\end{align*}
\end{defn}

The sequence $(S_{k+1}(A),S_{k+1}(f))$ is not judgmentally equal to the sequence $S(S_k(A),S_k(f))$, since in general we do not have $(k+1)+n\jdeq (k+n)+1$. Therefore we have the following lemma.

\begin{lem}\label{lem:iterate_succ}
For any $k,n:\nat$ and $a : A_k$, one has $q_{k,n}(a):\dpath{A}{p(k,n)}{f_k^{n+1}(a)}{f_{k+1}^n(f_k(a))}$ where $p(k,n):(k+n)+1=(k+1)+n$ is the canonical path in $\nat$.
\end{lem}

\begin{proof}
By induction on $n:\N$.
\end{proof}

\begin{cor}
For any type sequence $\sequence{A}{f}$, the type sequence $(S_{k+1}(A),S_{k+1}(f))$ is naturally equivalent to the type sequence $(S(S_k(A)),S(S_k(f)))$. 
\end{cor}

\begin{comment}
\begin{defn}
A \define{dependent natural transformation} $\tau:\sequence{P}{f}\to\sequence{Q}{g}$ of
dependent sequences over $\sequence{A}{a}$ consists of functions
$\tau_n(x):P_n(x)\to Q_n(x)$ for each $x:A_n$, together with a term witnessing 
that any square of the form
\begin{equation*}
\begin{tikzcd}
P_n(x) \arrow[r,"f_n(x)"] \arrow[d,swap,"\tau_n(x)"] & P_{n+1}(a(x)) \arrow[d,"\tau_{n+1}(a(x))"] \\
Q_n(x) \arrow[r,"g_n(x)"] & Q_{n+1}(a(x))
\end{tikzcd}
\end{equation*}
commutes. A \define{dependent natural equivalence} is a dependent natural transformation
$\tau$ for which $\tau_n(x)$ is an equivalence for each $n:\nat$ and $x:A_n$.
\end{defn}

\begin{lem}
For any two dependent sequences $\sequence{P}{f}$ and $\sequence{Q}{g}$ over
$\sequence{A}{a}$, the function
\begin{equation*}
(\id{\sequence{P}{f}}{\sequence{Q}{g}})\to\mathsf{NatEq}_{\sequence{A}{a}}(\sequence{P}{f},\sequence{Q}{g})
\end{equation*}
is an equivalence.
\end{lem}
\end{comment}

\section{Sequential colimits}

\begin{defn}
Consider the sequence $\sequence{A}{f}$ of types
\begin{equation*}
\begin{tikzcd}
A_0 \arrow[r,"f_0"] & A_1 \arrow[r,"f_1"] & A_2 \arrow[r,"f_2"] & \cdots
\end{tikzcd}
\end{equation*}
We define the colimit $A_\infty$ to be the higher inductive type with constructors
\begin{align*}
\iota^{A,f} & : \prd{n:\mathbb{N}} A_n\to A_\infty \\
\kappa^{A,f} & : \prd{n:\mathbb{N}}{x:A_n} \id{\iota_{n+1}(f(x))}{\iota_n(x)}.
\end{align*}
\end{defn}

\begin{rmk}
The induction principle for sequential colimits tells us how to construct a dependent function $f:\prd{a:A_\infty}P(a)$ for a type family $P:A_\infty\to\type$.

Given $s:\prd{a:A_\infty}P(a)$ we get
\begin{align*}
\lam{n}{a} s(\iota_n(a)) & : \prd{n:\N}{a:A_n}P(\iota_n(a)) \\
\lam{n}{a} \apd{s}{\kappa_n(a)} & : \prd{n:\N}{a:A_n} s(\iota_n(a)) =_{\kappa_n(a)}^P s(\iota_{n+1}(f_n(a)))
\end{align*}
In other words, we have a canonical map
\begin{equation*}
\Big(\prd{a:A_\infty}P(a)\Big)\to\Big(\sm{h:\prd{n:\N}{a:A_n}P(\iota_n(a))}\prd{n:\N}{a:A_n} h_n(a) =_{\kappa_n(a)}^P h_{n+1}(f_n(a))\Big)
\end{equation*}
Now we can state the induction principle and computation rule concisely: the canonical map described above comes equipped with a section. We assume that that the computation rule is strict on the point constructors.
\end{rmk}

The universal property of sequential colimits is a straightforward consequence of the induction principle.

\begin{thm}
Let $\sequence{A}{f}$ be a type sequence, and let $X$ be a type. Then the canonical map
\begin{equation*}
(A_\infty\to X)\to \sm{h:\prd{n:\N}A_n\to X}\prd{n:\N} h_n\htpy h_{n+1}\circ f_n
\end{equation*}
is an equivalence.
\end{thm}

The following theorem is a descent theorem for sequential colimits.

\begin{thm}\label{thm:descent}
Consider a sequence $\sequence{A}{f}$. The type $A_\infty\to\UU$ is equivalent to the type of equifibered type sequences over $\sequence{A}{f}$.
\end{thm}

\begin{proof}
By the universal property of $A_\infty$ and by univalence we have
\begin{align*}
(A_\infty\to \UU) & \eqvsym \sm{B:\prd{n:\N}A_n\to \UU}\prd{n:\N} B_n\htpy B_{n+1}\circ f_n \\
& \eqvsym \sm{B:\prd{n:\N}A_n\to \UU}\prd{n:\N}{x:A_n} B_n(x)\eqvsym B_{n+1}(f_n(x))\qedhere
\end{align*}
\end{proof}

\begin{lem}\label{lem:seq_colim_functor}
  Given a natural transformation $(\tau,H):\sequence{A}{f}\to\sequence{A'}{f'}$.
  \begin{enumerate}
    \item\label{part:seq_colim_functor} We get a function $\tfcolim(\tau,H)$ or $\tau_\infty:A_\infty\to A'_\infty$.
    \item\label{part:1_functoriality} The sequential colimit is 1-functorial. This means the following three things. If $(\sigma,K):\sequence{A'}{f'}\to\sequence{A''}{f''}$ then $(\tau\circ\sigma)_\infty\sim\tau_\infty\circ\sigma_\infty$. Moreover, $1_\infty\sim \idfunc$, where $1$ is the identity natural transformation. Lastly, if $(\tau',H'):\sequence{A}{f}\to\sequence{A'}{f'}$ and $q:\prd{n:\N}\tau_n\sim\tau'_n$ and we can fill the following square for all $a:A_n$
    \begin{center}\begin{tikzcd}[column sep=25mm]
      \tau_{n+1}(f_na)
      \ar[r,equal,"{q_{n+1}(f_na)}"] 
      \ar[d,equal,"{H_n(a)}"] &
      \tau'_{n+1}(f_na)
      \ar[d,equal,"{H'_n(a)}"]\\
      f'_n(\tau_n(a))
      \ar[r,equal,"{\apfunc{f'_n}(q_n(a))}"] & 
      f'_n(\tau'_n(a))
    \end{tikzcd}\end{center}
    then $\tau_\infty\sim\tau'_\infty$.
    \item\label{part:functor_equivalence} If $\tau$ is a natural equivalence then $\tau_\infty$ is an equivalence.
  \end{enumerate}
\end{lem}
\begin{proof}\mbox{}
  \begin{enumerate}
    \item 
    We define $\tau_\infty(\iota_n(a))\defeq\iota_n(\tau_n(a))$ and 
    $$\apfunc{\tau_\infty}(\kappa_n(a))\vcentcolon=\apfunc{\iota_{n+1}}(H(a))\cdot\kappa_n(\tau_n(a)):
    \iota_{n+1}(\tau_{n+1}(f_na))=\iota_n(\tau_n(a)).$$
    \item All three parts are by induction on the element of $A_\infty$, and all parts are straightforward.
    % to do? We might want to give an explicit proof
    \item We define $(\tau_\infty)^{-1}\defeq(\tau^{-1})_\infty$ where $\tau^{-1}$ is the natural transformation by inverting $\tau_n$ for each $n$. Now we can check that this is really the inverse by using all three parts of the 1-functoriality.
    $$\tau_\infty^{-1}\circ\tau_\infty\sim (\tau^{-1}\circ\tau)_\infty\sim 1_\infty\sim\idfunc[A_\infty].$$
    For the second homotopy we need to show that we can fill a certain square, which is straightforward.
    The other composite is homotopic to the identity by a similar argument.\qedhere
  \end{enumerate}
\end{proof}

\begin{lem}\label{lem:colim_shift_one}
For any type sequence $(A,f)$, the colimits of $(A,f)$ and $S(A,f)$ are equivalent.
\end{lem}

\begin{proof}
We construct a map $\varphi:A_\infty \to S(A)_\infty$ by induction on $A_\infty$, by taking
\begin{align*}
(x:A_n) & \mapsto \iota^{S(A),S(f)}_n(f_n(x)) \\
(x:A_n) & \mapsto \kappa^{S(A),S(f)}_n(f_n(x)).
\end{align*}

Next, we construct a map $\psi:S(A)_\infty\to A_\infty$ by induction on $S(A)_\infty$, by taking
\begin{align*}
(x:S(A)_n) & \mapsto \iota^{A,f}_{n+1}(x) \\
(x:S(A)_n) & \mapsto \kappa^{A,f}_{n+1}(x).
\end{align*}

Then we prove that $\psi\circ \varphi\htpy \idfunc$ by induction on $A_\infty$, by taking
\begin{align*}
(x:A_n) & \mapsto \kappa^{A,f}_n(x)
\end{align*}
Now we compute
\begin{align*}
\mathsf{ap}_{\psi\circ\varphi}(\kappa^{A,f}_n(x))) & = \mathsf{ap}_\psi(\mathsf{ap}_\varphi(\kappa^{A,f}_n(x))) \\
& = \mathsf{ap}_\psi(\kappa^{S(A),S(f)}_n(f_n(x))) \\
& = \kappa^{A,f}_{n+1}(f_n(x))
\end{align*}
from the computation rules of $A_\infty$ and $S(A)_\infty$.

We construct the homotopy $\varphi\circ\psi\htpy\idfunc$ by induction on $A_\infty$, by taking
\begin{align*}
(x:S(A)_n) & \mapsto \kappa_{n+1}(S(f)_n(x))
\end{align*}
Now we compute
\begin{align*}
\mathsf{ap}_{\varphi\circ\psi}(\kappa^{S(A),S(f)}_n(x)) & = \mathsf{ap}_\varphi(\mathsf{ap}_\psi(\kappa^{S(A),S(f)}_n(x))) \\
& = \mathsf{ap}_\varphi(\kappa^{A,f}_{n+1}(x)) \\
& = \kappa^{S(A),S(f)}_{n+1}(f_{n+1}(x)).\qedhere
\end{align*}
\end{proof}

\begin{lem}\label{lem:colim_shift_k}
For any type sequence $(A,f)$, we have an equivalence
\begin{equation*}
\kshiftequiv_{k} : \tfcolim(A,f)\eqvsym\tfcolim (S_k(A,f)).
\end{equation*}
\end{lem}

The shift operations and the corresponding equivalences on the sequential colimits can be used to turn an arbitrary sequence $\sequence{B}{g}$ over $\sequence{A}{f}$ into an equifibered sequence over $\sequence{A}{f}$. 

\begin{defn}
Given a dependent sequence $(B,g)$ over $(A,f)$ and $x:A_0$, we define a type sequence $(B[x],g[x])$ by
\begin{align*}
B[x]_n & \defeq B_n(f^n(x)) \\
g[x]_n & \defeq g_n(f^n(x),\blank).
\end{align*}
\end{defn}

\begin{defn}
Given any sequence $\sequence{B}{g}$ over $\sequence{A}{f}$, we define an equifibered sequence
$\sequence{\square B}{\square g}$ over the sequence $\sequence{A}{f}$.
\end{defn}

\begin{constr}
For $x:A_n$ we define
\begin{equation*}
(\square B)_n(x) \defeq S_n(B)[x]_{\infty} \jdeq \tfcolim_m(B_{n+m}(f^m(x))).
\end{equation*}
Now note that
\begin{align*}
  (\square B)_{n+1}(f(x)) & \jdeq \tfcolim_m(B_{(n+1)+m}(f^m(f(x))) \\
  & \simeq \tfcolim_m(B_{n+(m+1)}(f^{m+1}(x))\\
  & \simeq \tfcolim_m(B_{n+m}(f^m(x))\\
  & \jdeq (\square B)_n(x)
\end{align*}
The first equivalence $u_{n,m}$ is given by transporting along the dependent path in \autoref{lem:iterate_succ} in the family $B$. This forms a natural equivalence, because $\mathsf{transport}$ is natural. The second equivalence is given by applying \autoref{lem:colim_shift_one}. We call the composite equivalence $F$, which shows that $\square B$ is an equifibered sequence.
\end{constr}

\begin{defn}
Let $\sequence{B}{g}$ be a sequence over $\sequence{A}{f}$. Then we define
\begin{equation*}
B_\infty : A_\infty\to\UU
\end{equation*}
to be the family over $A_\infty$ associated to the equifibered sequence $(\square B,\square g)$ via the equivalence of \autoref{thm:descent}.
\end{defn}

By construction of $B_\infty$ we get the equality
$$r(y):\transfib{B_\infty}{\kappa_n(x)}{y}=F(y)$$
for $y:B_\infty(\iota_{n+1}(f_n(x)))$ witnessing that $B_\infty$ is defined by the equivalence $F$ on the path constructor.

