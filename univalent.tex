\chapter{The object classifier}\label{chap:univalent}

In this chapter we establish notation, and we highlight the basic results concerning fiberwise transformations and fiberwise equivalences, which we will use for the descent theorems \cref{thm:descent,thm:rcoeq_cartesian}. Of particular importance are the following theorems:
\begin{enumerate}
\item The Fundamental Theorem of Identity Types (\cref{thm:id_fundamental}), which establishes that a type family $B$ over $A$ with $b:B(a)$ for a given $a:A$ is fiberwise equivalent to the identity type $a=x$ if and only if its total space is contractible. This result appears in \cite{hottbook} as Theorem 5.8.2, which contains other equivalent conditions as well.
\item \cref{thm:pb_fibequiv}, in which we establish that for any fiberwise map
\begin{equation*}
g:\prd{x:A}P(x)\to Q(f(x)),
\end{equation*}
the commuting square
\begin{equation*}
\begin{tikzcd}[column sep=large]
\sm{x:A}P(x) \arrow[r,"{\total[f]{g}}"] \arrow[d,swap,"\proj 1"] & \sm{y:B}Q(y) \arrow[d,"\proj 1"] \\
A \arrow[r,swap,"f"] & B,
\end{tikzcd}
\end{equation*}
where $\total[f]{g}$ is defined as $\lam{(x,p)}(f(x),g(x,y))$, is a pullback square if and only if $g$ is a fiberwise equivalence\index{fiberwise equivalence}. As a consequence, we obtain that a commuting square
\begin{equation*}
\begin{tikzcd}
A \arrow[d,swap,"f"] \arrow[r] & B \arrow[d,"g"] \\
X \arrow[r,swap,"h"] & Y
\end{tikzcd}
\end{equation*}
is a pullback square if and only if the induced fiberwise transformation
\begin{equation*}
\prd{x:X} \fib{f}{x}\to\fib{g}{h(x)}
\end{equation*}
is a fiberwise equivalence. Our main reference \cite{hottbook} does not present many results of homotopy pullbacks, although the material we present about homotopy pullbacks is surely well-known. The connection between pullbacks and fiberwise equivalences has an important role in the descent theorem\index{descent} in \cref{chap:descent}, which is why we devote a section to this result.
\item \cref{thm:classifier}, in which we establish that the universe is an object classifier. This result appears in \cite{RijkeSpitters} as Theorem 2.31, and in \cite{hottbook} as Theorem 4.8.4
\end{enumerate}

\section{Notation and preliminary results}

We work in Martin-L\"of dependent type theory with $\Pi$-types, $\Sigma$-types and cartesian products, coproducts $A+B$ equipped with $\inl:A\to A+B$ and $\inr:B\to A+B$ for any two types $A$ and $B$, an empty type $\emptyt$, a unit type $\unit$ equipped with $\ttt:\unit$, a type $\bool$ of booleans equipped with $\btrue,\bfalse:\bool$, a type $\N$ of natural numbers equipped with $0:\N$ and $\suc:\N\to\N$, and identity types. 

\begin{rmk}
As usual, we write $\idfunc[A]:A\to A$ for the \define{identity function} $\lam{x}x$ on $A$, and we write $g\circ f:A\to C$ for the \define{composite function} $\lam{x}g(f(x))$ of $f:A\to B$ and $g:B\to C$. For any two types $A$ and $B$, and any $b:B$, we write
\begin{equation*}
\const_b : A\to B
\end{equation*}
for the \define{constant function} $\lam{x}b$. Sometimes we also write $\lam{\nameless}b$ for the constant function.

In the case of $\Sigma$-types, the empty type $\emptyt$, and the unit type, we use the following notation to define functions by pattern-matching:
\begin{align*}
\lam{(x,y)}f(x,y) & : \prd{t:\sm{x:A}B(x)} P(t) \\
\lam{\ttt}y & : \prd{t:\unit}P(t).
\end{align*}
For instance, the first and second projection maps 
\begin{align*}
\proj 1 & : (\sm{x:A}B(x))\to A \\
\proj 2 & : \prd{p:\sm{x:A}B(x)}B(\proj 1)
\end{align*}
are defined as $\proj 1\defeq \lam{(x,y)}x$ and $\proj 2\defeq\lam{(x,y)}y$.
We use similar notation for definitions by iterated pattern-matching. For instance, given a dependent function $f:\prd{x:A}{y:B(x)}{z:C(x,y)} P((x,y),z)$ we obtain the function
\begin{equation*}
\lam{((x,y),z)} f(x,y,z): \prd{t:\sm{s:\sm{x:A}B(x)}C(s)}P(t).
\end{equation*}
\end{rmk}

Given a type $A$ in context $\Gamma$, the \define{identity type} of $A$ at $a:A$ is the inductive type family 
\begin{equation*}
\Gamma,x:A\vdash a =_A x~\mathrm{type}
\end{equation*}
with constructor
\begin{equation*}
\Gamma \vdash \refl{a} : a=_A a.
\end{equation*}
The induction principle for the identity type of $A$ at $a$ asserts that for any type family
\begin{equation*}
\Gamma,x:A,\alpha: a=_A x\vdash P(x,\alpha)~\mathrm{type}
\end{equation*}
there is a term
\begin{equation*}
\ind{a=} : P(a,\refl{a})\to \prd{x:A}{\alpha:a=_A x}P(x,\alpha)
\end{equation*}
in context $\Gamma$, satisfying the computation rule
\begin{equation*}
\ind{a=}(p,a,\refl{a})\jdeq p.
\end{equation*}

A term of type $a=_A x$ is also called an \define{identification} of $a$ with $x$, or a \define{path} from $a$ to $x$.
The induction principle for identity types is sometimes called \define{identification elimination} or \define{path induction}. Occasionally, we also write $\idtypevar{A}$ for the identity type on $A$. 

Moreover, we assume that there is a universe $\UU$ with a universal family $\mathrm{El}$ over $\UU$, that is closed under the type forming operations. For example, there is a map
\begin{equation*}
\check{\idtypevar{}}:\prd{A:\UU}\mathrm{El}(A)\to\mathrm{El}(A)\to\UU
\end{equation*}
satisfying
\begin{equation*}
\mathrm{El}(\check{\idtypevar{}}(A,x,y))\jdeq (x=_{\mathrm{El}(A)} y),
\end{equation*}
establishing that the universe is closed under identity types.

Given a type $A$ the \define{concatenation} operation
\begin{equation*}
\concat : \prd{x,y,z:A} (\id{x}{y})\to(\id{y}{z})\to (\id{x}{z})
\end{equation*}
is defined by $\concat(\refl{x},q)\defeq q$. We will usually write $\ct{p}{q}$ for $\concat(p,q)$. 
The concatenation operation satisfies the unit laws
\begin{align*}
\leftunit(p) & : \ct{\refl{x}}{p}=p \\
\rightunit(p) & : \ct{p}{\refl{y}}=p.
\end{align*}

The \define{inverse operation} 
\begin{equation*}
\invfunc:\prd{x,y:A} (x=y)\to (y=x)
\end{equation*}
is defined by $\invfunc(\refl{x})\defeq\refl{x}$. We will usually write $p^{-1}$ for $\invfunc(p)$.
The inverse operation satisfies the inverse laws
\begin{align*}
\leftinv(p) & : \ct{p^{-1}}{p} = \refl{y} \\
\rightinv(p) & : \ct{p}{p^{-1}} = \refl{x}.
\end{align*}

The \define{associativity operation}, which assigns to each $p:x=y$, $q:y=z$, and $r:z=w$ the \define{associator}
\begin{equation*}
\assoc(p,q,r) : \ct{(\ct{p}{q})}{r}=\ct{p}{(\ct{q}{r})}
\end{equation*}
is defined by $\assoc(\refl{x},q,r)\defeq \refl{\ct{q}{r}}$.

Given a map $f:A\to B$, the \define{action on paths} of $f$ is an operation
\begin{equation*}
\apfunc{f} : \prd{x,y:A} (\id{x}{y})\to(\id{f(x)}{f(y)})
\end{equation*}
defined by $\ap{f}{\refl{x}}\defeq\refl{f(x)}$. 
Moreover, there are operations
\begin{align*}
\apid_A & : \prd{x,y:A}{p:\id{x}{y}} \id{p}{\ap{\idfunc[A]}{p}} \\
\apcomp(f,g) & : \prd{x,y:A}{p:\id{x}{y}} \id{\ap{g}{\ap{f}{p}}}{\ap{g\circ f}{p}}
\end{align*}
defined by $\apid_A(\refl{x})\defeq \refl{\refl{x}}$ and $\apcomp(f,g,\refl{x})\jdeq \refl{\refl{g(f(x))}}$, respectively.
It can be shown easily that the action on paths of a map preserves the groupoid operations, and that the groupoid laws are also preserved.

\begin{defn}
Let $A$ be a type, and let $B$ be a type family over $A$. The \define{transport} operation
\begin{equation*}
\tr_B:\prd{x,y:A} (\id{x}{y})\to (B(x)\to B(y))
\end{equation*}
is defined by $\tr_B(\refl{x}) \defeq \idfunc[B(x)]$. 
\end{defn}

\begin{defn}\label{defn:apd}
Given a dependent function $f:\prd{a:A}B(a)$ and a path $p:\id{x}{y}$ in $A$, the \define{dependent action on paths}
\begin{equation*}
\apdfunc{f} : \prd{x,y:A}{p:x=y}\id{\tr_B(p,f(x))}{f(y)}
\end{equation*}
is defined by $\apd{f}{\refl{x}}\defeq \refl{f(x)}$.
\end{defn}

\begin{defn}
Let $f,g:\prd{x:A}P(x)$ be two dependent functions. The type $f\htpy g$ of \define{homotopies}\index{homotopy|textbf} from $f$ to $g$ is defined as
\begin{equation*}
f\htpy g \defeq \prd{x:A} f(x)=g(x).
\end{equation*}
\end{defn}

Commutativity of diagrams is stated using homotopies. For instance, a triangle
\begin{equation*}
\begin{tikzcd}[column sep=tiny]
A \arrow[dr,swap,"f"] \arrow[rr,"h"] & & B \arrow[dl,"g"] \\
& X
\end{tikzcd}
\end{equation*}
is said to commute if it comes equipped with a homotopy $H:f\htpy g\circ h$, and a square
\begin{equation*}
\begin{tikzcd}
A \arrow[d,"i"'] \arrow[r,"g"] & X \arrow[d,"f"] \\
B \arrow[r,swap,"h"] & Y
\end{tikzcd}
\end{equation*}
is said to commute if it comes equipped with a homotopy $H:h\circ i\htpy f\circ g$. 

The reflexivity, inverse, and concatenation operations on homotopies are defined pointwise.
We will write $H^{-1}$ for $\lam{x}H(x)^{-1}$, and $\ct{H}{K}$ for $\lam{x}\ct{H(x)}{K(x)}$.
These operations satisfy the groupoid laws (phrased appropriately as homotopies). Apart from the groupoid operations and their laws, we will occasionally need \emph{whiskering} operations and the naturality of homotopies.

\begin{defn}\label{defn:htpy_whisering}
We define the following \define{whiskering}\index{homotopy!whiskering operations|textbf}\index{whiskering operations!of homotopies|textbf} operations on homotopies:
\begin{enumerate}
\item Suppose $H:f\htpy g$ for two functions $f,g:A\to B$, and let $h:B\to C$. We define
\begin{equation*}
h\cdot H\defeq \lam{x}\ap{h}{H(x)}:h\circ f\htpy h\circ g.
\end{equation*}
\item Suppose $f:A\to B$ and $H:g\htpy h$ for two functions $g,h:B\to C$. We define
\begin{equation*}
H\cdot f\defeq\lam{x}H(f(x)):h\circ f\htpy g\circ f.
\end{equation*}
\end{enumerate}
\end{defn}

We will frequently make use of commuting cubes. The commutativity of a cube is stated using the whiskering operations on homotopies.

\begin{defn}\label{defn:cube}
A \define{commuting cube}\index{commuting cube|textbf}
\begin{equation*}
\begin{tikzcd}
& A_{111} \arrow[dl] \arrow[dr] \arrow[d] \\
A_{110} \arrow[d] & A_{101} \arrow[dl] \arrow[dr] & A_{011} \arrow[dl,crossing over] \arrow[d] \\
A_{100} \arrow[dr] & A_{010} \arrow[d] \arrow[from=ul,crossing over] & A_{001} \arrow[dl] \\
& A_{000},
\end{tikzcd}
\end{equation*}
consists of 
\begin{enumerate}
\item types
\begin{equation*}
A_{111},A_{110},A_{101},A_{011},A_{100},A_{010},A_{001},A_{000},
\end{equation*}
\item \begin{samepage}%
maps
\begin{align*}
f_{11\check{1}} & : A_{111}\to A_{110} & f_{\check{1}01} & : A_{101}\to A_{001} \\
f_{1\check{1}1} & : A_{111}\to A_{101} & f_{01\check{1}} & : A_{011}\to A_{010} \\
f_{\check{1}11} & : A_{111}\to A_{011} & f_{0\check{1}1} & : A_{011}\to A_{001} \\
f_{1\check{1}0} & : A_{110}\to A_{100} & f_{\check{1}00} & : A_{100}\to A_{000} \\
f_{\check{1}10} & : A_{110}\to A_{010} & f_{0\check{1}0} & : A_{010}\to A_{000} \\
f_{10\check{1}} & : A_{101}\to A_{100} & f_{00\check{1}} & : A_{001}\to A_{000},
\end{align*}
\end{samepage}%
\item homotopies
\begin{align*}
H_{1\check{1}\check{1}} & : f_{1\check{1}0}\circ f_{11\check{1}} \htpy f_{10\check{1}}\circ f_{1\check{1}1} & H_{0\check{1}\check{1}} & : f_{0\check{1}0}\circ f_{01\check{1}} \htpy f_{00\check{1}}\circ f_{0\check{1}1} \\
H_{\check{1}1\check{1}} & : f_{\check{1}10}\circ f_{11\check{1}} \htpy f_{01\check{1}}\circ f_{\check{1}11} & H_{\check{1}0\check{1}} & : f_{\check{1}00}\circ f_{10\check{1}} \htpy f_{00\check{1}}\circ f_{\check{1}01} \\
H_{\check{1}\check{1}1} & : f_{\check{1}01}\circ f_{1\check{1}1} \htpy f_{0\check{1}1}\circ f_{\check{1}11} & H_{\check{1}\check{1}0} & : f_{\check{1}00}\circ f_{1\check{1}0} \htpy f_{0\check{1}0}\circ f_{\check{1}10},
\end{align*}
\item and a homotopy 
\begin{align*}
C & : \ct{(f_{\check{1}00}\cdot H_{1\check{1}\check{1}})}{(\ct{(H_{\check{1}0\check{1}}\cdot f_{1\check{1}1})}{(f_{00\check{1}}\cdot H_{\check{1}\check{1}1})})} \\
& \qquad \htpy \ct{(H_{\check{1}\check{1}0}\cdot f_{11\check{1}})}{(\ct{(f_{0\check{1}0}\cdot H_{\check{1}1\check{1}})}{(H_{0\check{1}\check{1}}\cdot f_{\check{1}11})})}
\end{align*}
filling the cube.
\end{enumerate}
\end{defn}

\begin{defn}
Let $f:A\to B$ be a function. We say that $f$ has a \define{section}\index{section!of a map|textbf} if there is a term of type\index{sec(f)@{$\sections(f)$}|textbf}
\begin{equation*}
\sections(f) \defeq \sm{g:B\to A} f\circ g\htpy \idfunc[B].
\end{equation*}
Dually, we say that $f$ has a \define{retraction}\index{retraction} if there is a term of type\index{retr(f)@{$\retractions(f)$}|textbf}
\begin{equation*}
\retractions(f) \defeq \sm{h:B\to A} h\circ f\htpy \idfunc[A].
\end{equation*}
If $f$ has a retraction, we also say that $A$ is a \define{retract}\index{retract!of a type} of $B$.
\end{defn}

\begin{defn}
We say that a function $f:A\to B$ is an \define{equivalence}\index{equivalence|textbf}\index{bi-invertible map|see {equivalence}} if it has both a section and a retraction, i.e.~if it comes equipped with a term of type\index{is_equiv@{$\isequiv$}|textbf}
\begin{equation*}
\isequiv(f)\defeq\sections(f)\times\retractions(f).
\end{equation*}
We will write $\eqv{A}{B}$\index{equiv@{$\eqv{A}{B}$}|textbf} for the type $\sm{f:A\to B}\isequiv(f)$.
\end{defn}

Clearly, if $f$ is \define{invertible}\index{invertible map} in the sense that it comes equipped with a function $g:B\to A$ such that $f\circ g\htpy\idfunc[B]$ and $g\circ f\htpy\idfunc[A]$, then $f$ is an equivalence. We write\index{has_inverse@{$\hasinverse$}|textbf}
\begin{equation*}
\hasinverse(f)\defeq\sm{g:B\to A} (f\circ g\htpy \idfunc[B])\times (g\circ f\htpy\idfunc[A]).
\end{equation*}
The section of an equivalence is also a retraction (and vice versa), so we define the \define{inverse} of an equivalence to be its section. It follows immediately that the inverse of any equivalence is again an equivalence.\index{equivalence!invertibility of} The identity function $\idfunc[A]$ on a type $A$ is an equivalence since it is its own section and its own retraction.

It is straightforward to show that for any two functions $f,g:A\to B$, we have
\begin{equation*}
(f\htpy g)\to (\isequiv(f)\leftrightarrow\isequiv(g)).
\end{equation*}
Given a commuting triangle
\begin{equation*}
\begin{tikzcd}[column sep=tiny]
A \arrow[rr,"h"] \arrow[dr,swap,"f"] & & B \arrow[dl,"g"] \\
& X.
\end{tikzcd}
\end{equation*}
with $H:f\htpy g\circ h$, we have:
\begin{enumerate}
\item If the map $h$ has a section, then $f$ has a section if and only if $g$ has a section.
\item If the map $g$ has a retraction, then $f$ has a retraction if and only if $h$ has a retraction.
\item (The \define{3-for-2 property} for equivalences.) If any two of the functions
\begin{equation*}
f,\qquad g,\qquad h
\end{equation*}
are equivalences, then so is the third.
\end{enumerate}

In the following theorem we characterize the identity type of a $\Sigma$-type as a $\Sigma$-type of identity types.


\begin{prp}[Theorem 2.7.2 of \cite{hottbook}]\label{thm:eq_sigma}
Let $B$ be a type family over $A$, let $s:\sm{x:A}B(x)$, and consider the dependent function\index{pair_eq@{$\paireq$}|textbf}
\begin{equation*}
\paireq_s:\prd{t:\sm{x:A}B(x)} (s=t)\to \sm{\alpha:\proj 1(s)=\proj 1(t)} \tr_B(\alpha,\proj 2(s))=\proj 2(t)
\end{equation*}
defined by $\paireq_s(\refl{s}) \defeq (\refl{\proj 1(s)},\refl{\proj 2(s)})$. Then $\paireq_{s,t}$ is an equivalence for every $t:\sm{x:A}B(x)$.\index{Sigma type@{$\Sigma$-type}!identity types of|textit}\index{identity type!of a Sigma-type@{of a $\Sigma$-type}|textit}
\end{prp}

We include the proof mainly to introduce some more notation.

\begin{proof}
The maps in the converse direction\index{eq_pair@{$\eqpair$}}
\begin{equation*}
\eqpair_{s,t} : \Big(\sm{p:\proj 1(s)=\proj 1(t)}\id{\tr_B(p,\proj 2(s))}{\proj 2(t)}\Big)\to(\id{s}{t})
\end{equation*}
is defined by
\begin{equation*}
\eqpair_{(x,y),(x',y')}(\refl{x},\refl{y})\defeq \refl{(x,y)}.
\end{equation*}
The proofs that the function $\eqpair_{s,t}$ is indeed an inverse of $\paireq_{s,t}$ are also by induction.
\end{proof}

\begin{defn}
We say that a type $A$ is \define{contractible}\index{contractible!type|textbf} if there is a term of type
\begin{equation*}
\iscontr(A) \defeq \sm{c:A}\prd{x:A}c=x.
\end{equation*}
Given a term $(c,C):\iscontr(A)$, we call $c:A$ the \define{center of contraction}\index{center of contraction|textbf} of $A$, and we call $C:\prd{x:A}a=x$ the \define{contraction}\index{contraction} of $A$.
\end{defn}

Suppose $A$ is a contractible type with center of contraction $c$ and contraction $C$. Then the type of $C$ is (judgmentally) equal to the type
\begin{equation*}
\const_c\htpy\idfunc[A].
\end{equation*}
In other words, the contraction $C$ is a \emph{homotopy} from the constant function to the identity function.

\begin{defn}
Consider a type $A$ with a base point $a:A$. We say that $A$ satisfies \define{singleton induction}\index{singleton induction|textbf} if for every type family $B$ over $A$, the map
\begin{equation*}
\evpt:\Big(\prd{x:A}B(x)\Big)\to B(a)
\end{equation*}
given by $f\mapsto f(a)$ has a section.
\end{defn}

\begin{prp}\label{thm:contractible}
A type $A$ is contractible if and only if it satisfies singleton induction.
\end{prp}

\begin{eg}
By definition the unit type\index{unit type!contractibility} $\unit$ satisfies singleton induction, so it is contractible.
\end{eg}

\begin{rmk}
For any family $P:\Big(\sm{x:A}B(x)\Big)\to\UU$ there is a map
\begin{equation*}
\evpair : \Big(\prd{t:\sm{x:A}B(x)}P(t)\Big)\to \prd{x:A}{y:B(x)}P(x,y)
\end{equation*}
that evaluates $f:\prd{t:\sm{x:A}B(x)}P(t)$ at pairs $(x,y)$. In other words, $\evpair$ is defined by $\lam{f}{x}{y}f(x,y)$. By the induction principle for $\Sigma$-types, this map has a section. It is easy to show that $\evpair$ is in fact an equivalence. 

Similarly, there is a map
\begin{equation*}
\evrefl : \Big(\prd{x:A}{p:a=x}B(x,p)\Big)\to B(a,\refl{a})
\end{equation*}
given by $\lam{f}f(a,\refl{a})$, for any type family $B:\prd{x:A} (a=x)\to\UU$. By path induction, this map has a section, and again it is easy to show that this map is in fact an equivalence. 
\end{rmk}

\begin{prp}[Lemma 3.11.8 in \cite{hottbook}]\label{thm:total_path}
For any $x:A$, the type
\begin{equation*}
\sm{y:A}x=y
\end{equation*}
is contractible.\index{identity type!contractibility of total space|textit}
\end{prp}

\begin{proof}
We have the term $(x,\refl{x}):\sm{y:A}x=y$, and both maps in the composite
\begin{equation*}
\begin{tikzcd}[column sep=large]
\prd{t:\sm{y:A}x=y}B(t) \arrow[r,"\evpair"] & \prd{y:A}{p:x=y}B((y,p)) \arrow[r,"\evrefl"] & B((x,\refl{x}))
\end{tikzcd}
\end{equation*}
have sections, so the composite has a section. The composite is $\evpt$, so we see that the asserted type satisfies singleton induction.
\end{proof}

\begin{defn}
Let $f:A\to B$ be a function, and let $b:B$. The \define{fiber}\index{fiber|textbf}\index{homotopy fiber|see {fiber}} of $f$ at $b$ is defined to be the type
\begin{equation*}
\fib{f}{b}\defeq\sm{a:A}f(a)=b.
\end{equation*}
\end{defn}

\begin{eg}[Lemma 4.8.1 of \cite{hottbook}]\label{eg:fib_proj}
Consider a type family $B$ over $A$. Then the map
\begin{equation*}
B(a)\to \fib{\proj 1}{a}
\end{equation*}
given by $b\mapsto ((a,b),\refl{a})$ is an equivalence. In other words, the fibers of the projection function $\proj 1 : \big(\sm{x:A}B(x)\big)\to A$ are just the fibers of the family $B$.
\end{eg}

\begin{defn}
We say that a function $f:A\to B$ is \define{contractible}\index{contractible!map|textbf} if there is a term of type
\begin{equation*}
\iscontr(f)\defeq\prd{b:B}\iscontr(\fib{f}{b}).
\end{equation*}
\end{defn}

We cite Chapter 4 of \cite{hottbook} for the following result, although it is well-known that it can be proven directly and without the use of function extensionality.

\begin{prp}[Chapter 4 in \cite{hottbook}]\label{thm:contr_equiv}
A function is an equivalence if and only if it is contractible.\index{contractible!map!is an equivalence|textit}
\end{prp}

\section{The Fundamental Theorem of Identity Types}
Consider a family
\begin{equation*}
f : \prd{x:A}B(x)\to C(x)
\end{equation*}
of maps. Such $f$ is also called a \define{fiberwise map} or \define{fiberwise transformation}.

\begin{defn}[Definition 4.7.5 of \cite{hottbook}]
We define the map
\begin{equation*}
\total{f}:\sm{x:A}B(x)\to\sm{x:A}C(x).
\end{equation*}
by $\lam{(x,y)}(x,f(x,y))$.
\end{defn}

\begin{lem}[Theorem 4.7.6 of \cite{hottbook}]\label{lem:fib_total}
For any fiberwise transformation $f:\prd{x:A}B(x)\to C(x)$, and any $a:A$ and $c:C(a)$, there is an equivalence
\begin{equation*}
\eqv{\fib{f(a)}{c}}{\fib{\total{f}}{\pairr{a,c}}}.
\end{equation*}
\end{lem}

\begin{eg}
There are equivalences
\begin{equation*}
\eqv{\fib{(\apfunc{f})_{x,y}}{q}}{\fib{\delta_f}{(x,y,q)}}
\end{equation*}
for any $q:f(x)=f(y)$, because the triangle
\begin{equation*}
\begin{tikzcd}[column sep=-1em]
A \arrow[rr,"{\lam{x}(x,x,\refl{x})}"] \arrow[dr,swap,"\delta_f"] & & \sm{x,y:A}x=y \arrow[dl,"\total{\total{\apfunc{f}}}"] \\
\phantom{\sm{x,y:A}x=y} & \sm{x,y:A}f(x)=f(y)
\end{tikzcd}
\end{equation*}
commutes, and the top map is an equivalence.
\end{eg}

\begin{prp}[Theorem 4.7.7 of \cite{hottbook}]\label{thm:fib_equiv}
Let $f:\prd{x:A}B(x)\to C(x)$ be a fiberwise transformation. The following are logically equivalent:
\begin{enumerate}
\item For each $x:A$, the map $f_x:B(x)\to C(x)$ is an equivalence. In this case we say that $f$ is a \define{fiberwise equivalence}.
\item The map $\total{f}:\sm{x:A}B(x)\to\sm{x:A}C(x)$ is an equivalence.
\end{enumerate}
\end{prp}

The following theorem is the key to many results about identity types, which we will use instead of the \emph{encode-decode method}\index{encode-decode method} of \cite{LicataShulman}. We refer to it as the \define{Fundamental Theorem of Identity Types}.

\begin{thm}[Theorem 5.8.2 of \cite{hottbook}]\label{thm:id_fundamental}
Let $A$ be a type with $a:A$, and let $B$ be a type family over $A$ with $b:B(a)$.
Then  the following are logically equivalent:
\begin{enumerate}
\item The canonical family of maps
\begin{equation*}
\ind{a{=}}(b):\prd{x:A} (a=x)\to B(x)
\end{equation*}
is a fiberwise equivalence.
\item The total space
\begin{equation*}
\sm{x:A}B(x)
\end{equation*}
is contractible.
\end{enumerate}
\end{thm}

\begin{proof}
By \autoref{thm:fib_equiv} it follows that the fiberwise transformation $\ind{a{=}}(b)$ is a fiberwise equivalence if and only if it induces an equivalence
\begin{equation*}
\eqv{\Big(\sm{x:A}a=x\Big)}{\Big(\sm{x:A}B(x)\Big)}
\end{equation*}
on total spaces. We have that $\sm{x:A}a=x$ is contractible. Now it follows by the 3-for-2 property of equivalences, applied in the case
\begin{equation*}
\begin{tikzcd}
\sm{x:A}a=x \arrow[rr,"\total{\ind{a{=}}(b)}"] \arrow[dr,swap,"\eqvsym"] & & \sm{x:A}B(x) \arrow[dl] \\
& \unit & \phantom{\sm{x:A}a=x}
\end{tikzcd}
\end{equation*}
that $\total{\ind{a{=}}(b)}$ is an equivalence if and only if $\sm{x:A}B(x)$ is contractible.
\end{proof}

Observe that in the proof of \cref{thm:id_fundamental} we haven't used the actual definition of the fiberwise transformation. Indeed, for any fiberwise transformation
\begin{equation*}
f:\prd{x:A}(a=x)\to B(x)
\end{equation*}
we have that $f$ is a fiberwise equivalence if and only if the total space of $B$ is contractible.

Since retracts of contractible types are again contractible, it follows that the only retract of the identity type is the identity type itself:

\begin{cor}\label{cor:id_fundamental_retr}
Let $a:A$, and let $B$ be a type family over $A$. If each $B(x)$ is a retract of $\id{a}{x}$, then $B(x)$ is equivalent to $\id{a}{x}$ for every $x:A$.
\end{cor}

As a first application of the fundamental theorem we give a quick new proof that equivalences are embeddings. The proof of the corresponding theorem in \cite{hottbook} is more involved.

\begin{defn}
An \define{embedding}\index{embedding|textbf} is a map $f:A\to B$ satisfying the property that
\begin{equation*}
\apfunc{f}:(\id{x}{y})\to(\id{f(x)}{f(y)})
\end{equation*}
is an equivalence for every $x,y:A$. We write $\isemb(f)$ for the type of witnesses that $f$ is an embedding.
\end{defn}

\begin{prp}[Theorem 2.11.1 in \cite{hottbook}]
\label{cor:emb_equiv} 
Any equivalence is an embedding.\index{embedding!equivalences are embeddings|textit}\index{equivalence!is an embedding|textit}
\end{prp}

\begin{proof}
Let $e:\eqv{A}{B}$ be an equivalence, and let $x:A$. By \autoref{thm:id_fundamental} it follows that
\begin{equation*}
\apfunc{e} : (\id{x}{y})\to (\id{e(x)}{e(y)})
\end{equation*}
is an equivalence for every $y:A$ if and only if the total space
\begin{equation*}
\sm{y:A}e(x)=e(y)
\end{equation*}
is contractible for every $y:A$. Now observe that $\sm{y:A}e(x)=e(y)$ is equivalent to the fiber $\fib{e}{e(x)}$, which is contractible by \cref{thm:contr_equiv}.
\end{proof}

\begin{defn}
A type $A$ is said to be a \define{proposition} if there is a term of type
\begin{equation*}
\isprop(A)\defeq\prd{x,y:A}\iscontr(x=y).
\end{equation*}
Furthermore, we write $\prop\defeq\sm{X:\UU}\isprop(X)$ for the type of all small propositions.
\end{defn}

We will often use either of the following characterizations of propositions.

\begin{lem}[Lemma 3.11.10 and Exercise 3.5 of \cite{hottbook}]\label{lem:prop_char}
For any type $A$ the following are equivalent:
\begin{enumerate}
\item $A$ is a proposition.
\item $A$ is \define{proof irrelevant} in the sense that $\prd{x,y:A}x=y$.
\item $A\to\iscontr(A)$. 
\end{enumerate}
\end{lem}

\begin{eg}\label{eg:prop_contr}
Any contractible type is a proposition. The empty type is a proposition by a direct application of the induction principle of the empty type. Furthermore, any retract of a proposition is again a proposition. In particular, propositions are closed under equivalences.
\end{eg}

\begin{lem}\label{lem:id_fib}
Consider a function $f:A\to B$, and let $(a,p),(a',p'):\fib{f}{b}$ for some $b:B$. Then the canonical map
\begin{equation*}
((a,p)=(a',p'))\to \fib{\apfunc{f}}{\ct{p}{p'^{-1}}}
\end{equation*}
is an equivalence.
\end{lem}

\begin{proof}
By \cref{thm:fib_equiv} it suffices to show that the type
\begin{equation*}
\sm{y:A}{q:f(y)=b}{p:\proj 1(s)=y} \ap{f}{p}=\ct{\proj 2(s)}{q^{-1}}
\end{equation*}
is contractible, which is immediate by two applications of \cref{thm:total_path}.
\end{proof}

\begin{prp}[Lemma 7.6.2 of \cite{hottbook}]\label{thm:prop_emb}
A map is an embedding if and only if its fibers are propositions.
\end{prp}

\begin{proof}
If $f$ is an embedding, then the fibers of $\apfunc{f}$ are contractible by \cref{thm:contr_equiv}. Thus it follows by \cref{lem:id_fib} that the fibers of $f$ are propositions.

Conversely, if the fibers of $f$ are propositions, then we have by \cref{lem:id_fib} an equivalence
\begin{equation*}
\eqv{((x,p)=(y,\refl{f(y)}))}{\fib{\apfunc{f}}{p}}
\end{equation*}
for any $p:f(x)=f(y)$, which shows that the fibers of $\apfunc{f}$ are contractible. Thus $f$ is an embedding by \cref{thm:contr_equiv}.
\end{proof}

\begin{defn}
A type family $B$ over $A$ is said to be a \define{subtype} of $A$ if for each $x:A$ the type $B(x)$ is a proposition.
\end{defn}

\begin{cor}\label{thm:subtype}
A type family $B$ over $A$ is a subtype if and only if the projection map
\begin{equation*}
\proj 1 : \Big(\sm{x:A}B(x)\Big)\to A
\end{equation*}
is an embedding.
\end{cor}

\begin{proof}
Immediate by \cref{eg:fib_proj,thm:prop_emb}.
\end{proof}

\section{Function extensionality}
\begin{prp}[Theorem 4.9.5 of \cite{hottbook}]\label{thm:funext_wkfunext}
The following are equivalent:
\begin{enumerate}
\item The \define{function extensionality principle}\index{function extensionality}: For every type family $B$ over $A$, and any two dependent functions $f,g:\prd{x:A}B(x)$, the canonical map\index{htpy_eq@{$\htpyeq$}|textbf}
\begin{equation*}
\htpyeq(f,g) : (\id{f}{g})\to (f\htpy g)
\end{equation*}
by path induction (sending $\refl{f}$ to $\lam{x}\refl{f(x)}$) is an equivalence. We will write $\eqhtpy$\index{eq_htpy@{$\eqhtpy$}} for its inverse.
\item The \define{weak function extensionality principle}\index{weak function extensionality} holds: For every type family $B$ over $A$ one has\index{contractible!weak function extensionality}
\begin{equation*}
\Big(\prd{x:A}\iscontr(B(x))\Big)\to\iscontr\Big(\prd{x:A}B(x)\Big).
\end{equation*}
\end{enumerate}
\end{prp}

From now on we will assume that function extensionality holds.

\begin{cor}[Theorem 7.1.9 of \cite{hottbook}]\label{thm:prop_pi}
For any type family $B$ over $A$ one has
\begin{equation*}
\Big(\prd{x:A}\isprop(B(x))\Big)\to \isprop\Big(\prd{x:A}B(x)\Big).
\end{equation*}
In particular, if $B$ is a proposition, then $A\to B$ is a proposition for any type $A$.
\end{cor}

We show in this section that a map $f:A\to B$ is an equivalence if and only if for any type family $P$ over $B$, the precomposition map
\begin{equation*}
\blank\circ f: \Big(\prd{y:B}P(y)\Big)\to\Big(\prd{x:A}P(f(x))\Big)
\end{equation*}
is an equivalence. 
In the proof we use the notion of \emph{path-split} maps, which was introduced in \cite{RijkeShulmanSpitters}.

\begin{defn}
We say that a map $f:A\to B$ is \define{path-split}\index{path-split|textbf} if $f$ has a section, and for each $x,y:A$ the map
\begin{equation*}
\apfunc{f}(x,y):(x=y)\to (f(x)=f(y))
\end{equation*}
also has a section. We write $\pathsplit(f)$\index{path_split(f)@{$\pathsplit(f)$}|textbf} for the type
\begin{equation*}
\sections(f)\times\prd{x,y:A}\sections(\apfunc{f}(x,y)).
\end{equation*}
\end{defn}

We will also use the notion of \emph{half-adjoint equivalences}, which were introduced in \cite{hottbook}.

\begin{defn}[Definition 4.2.1 of \cite{hottbook}]
We say that a map $f:A\to B$ is a \define{half-adjoint equivalence}\index{half-adjoint equivalence|textbf}, in the sense that there are
\begin{align*}
g & : B \to A\\
G & : f\circ g \htpy \idfunc[B] \\
H & : g\circ f \htpy \idfunc[A] \\
K & : G\cdot f \htpy f\cdot H.
\end{align*}
We write $\halfadj(f)$\index{half_adj(f)@{$\halfadj(f)$}|textbf} for the type of such quadruples $(g,G,H,K)$.
\end{defn}

Furthermore, we will need `type theoretic choice'. 

\begin{prp}[Theorem 2.15.7 of \cite{hottbook}]\label{thm:choice}
Let $C(x,y)$ be a type in context $\Gamma,x:A,y:B(x)$. Then the map
\begin{equation*}
\varphi:\Big(\prd{x:A}\sm{y:B(x)}C(x,y)\Big)\to \Big(\sm{f:\prd{x:A}B(x)}\prd{x:A}C(x,f(x))\Big)
\end{equation*}
given by $\lam{h}(\lam{x}\proj 1(h(x)),\lam{x}\proj 2(h(x)))$ is an equivalence.
\end{prp}

\begin{cor}
For type $A$ and any type family $C$ over $B$, the map
\begin{equation*}
\Big(\sm{f:A\to B} \prd{x:A}C(f(x))\Big)\to\Big(A\to\sm{y:B}C(x)\Big)
\end{equation*}
given by $\lam{(f,g)}{x}(f(x),g(x))$ is an equivalence.
\end{cor}

\begin{prp}\label{prp:equiv_precomp}
For any map $f:A\to B$, the following are equivalent:
\begin{enumerate}
\item $f$ is an equivalence.
\item $f$ is path-split.
\item $f$ is a half-adjoint equivalence.
\item For any type family $P$ over $B$ the map
\begin{equation*}
\Big(\prd{y:B}P(y)\Big)\to\Big(\prd{x:A}P(f(x))\Big)
\end{equation*}
given by $s\mapsto s\circ f$ is an equivalence.
\item For any type $X$ the map
\begin{equation*}
(B\to X)\to (A\to X)
\end{equation*}
given by $g\mapsto g\circ f$ is an equivalence. 
\end{enumerate}
\end{prp}

\begin{proof}
To see that (i) implies (ii) we note that any equivalence has a section, and its action on paths is an equivalence by \cref{cor:emb_equiv} so again it has a section.

To show that (ii) implies (iii), assume that $f$ is path-split. Thus we have $(g,G):\sections(f)$, and the assumption that $\apfunc{f}:(x=y)\to (f(x)=f(y))$ has a section for every $x,y:A$ gives us a term of type
\begin{equation*}
\prd{x:A}\fib{\apfunc{f}}{G(f(x))}.
\end{equation*}
By \cref{thm:choice} this type is equivalent to
\begin{equation*}
\sm{H:\prd{x:A}g(f(x))=x}\prd{x:A}G(f(x))=\ap{f}{H(x)},
\end{equation*}
so we obtain $H:g\circ f\htpy \idfunc[A]$ and $K:G\cdot f\htpy f\cdot H$, showing that $f$ is a half-adjoint equivalence.

To show that (iii) implies (iv), suppose that $f$ comes equipped with $(g,G,H,K)$ witnessing that $f$ is a half-adjoint equivalence. Then we define the inverse of $\blank\circ f$ to be the map
\begin{equation*}
\varphi:\Big(\prd{x:A}P(f(x))\Big)\to\Big(\prd{y:B}P(y)\Big)
\end{equation*}
given by $s\mapsto \lam{y}\tr_P(G(y),sg(y))$. 

To see that $\varphi$ is a section of $\blank\circ f$, let $s:\prd{x:A}P(f(x))$. By function extensionality it suffices to construct a homotopy $\varphi(s)\circ f\htpy s$. In other words, we have to show that
\begin{equation*}
\tr_P(G(f(x)),s(g(f(x)))=s(x)
\end{equation*}
for any $x:A$. Now we use the additional homotopy $K$ from our assumption that $f$ is a half-adjoint equivalence. Since we have $K(x):G(f(x))=\ap{f}{H(x)}$ it suffices to show that
\begin{equation*}
\tr_P(\ap{f}{H(x)},sgf(x))=s(x).
\end{equation*}
A simple path-induction argument yields that
\begin{equation*}
\tr_P(\ap{f}{p})\htpy \tr_{P\circ f}(p)
\end{equation*}
for any path $p:x=y$ in $A$, so it suffices to construct an identification
\begin{equation*}
\tr_{P\circ f}(H(x),sgf(x))=s(x).
\end{equation*}
We have such an identification by $\apd{H(x)}{s}$.

To see that $\varphi$ is a retraction of $\blank\circ f$, let $s:\prd{y:B}P(y)$. By function extensionality it suffices to construct a homotopy $\varphi(s\circ f)\htpy s$. In other words, we have to show that
\begin{equation*}
\tr_P(G(y),sfg(y))=s(y)
\end{equation*}
for any $y:B$. We have such an identification by $\apd{G(y)}{s}$. This completes the proof that (iii) implies (iv).

Note that (v) is an immediate consequence of (iv), since we can just choose $P$ to be the constant family $X$.

It remains to show that (v) implies (i). Suppose that
\begin{equation*}
\blank\circ f:(B\to X)\to (A\to X)
\end{equation*}
is an equivalence for every type $X$. Then its fibers are contractible by \cref{thm:contr_equiv}. In particular, choosing $X\jdeq A$ we see that the fiber
\begin{equation*}
\fib{\blank\circ f}{\idfunc[A]}\jdeq \sm{h:B\to A}h\circ f=\idfunc[A]
\end{equation*}
is contractible. Thus we obtain a function $h:B\to A$ and a homotopy $H:h\circ f\htpy\idfunc[A]$ showing that $h$ is a retraction of $f$. We will show that $h$ is also a section of $f$. To see this, we use that the fiber
\begin{equation*}
\fib{\blank\circ f}{f}\jdeq \sm{i:B\to B} i\circ f=f
\end{equation*}
is contractible (choosing $X\jdeq B$). 
Of course we have $(\idfunc[B],\refl{f})$ in this fiber. However we claim that there also is an identification $p:(f\circ h)\circ f=f$, showing that $(f\circ h,p)$ is in this fiber, because
\begin{align*}
(f\circ h)\circ f & \jdeq f\circ (h\circ f) \\
& = f\circ \idfunc[A] \\
& \jdeq f
\end{align*}
Now we conclude by the contractibility of the fiber that there is an identification $(\idfunc[B],\refl{f})=(f\circ h,p)$. In particular we obtain that $\idfunc[B]=f\circ h$, showing that $h$ is a section of $f$.
\end{proof}

\section{Homotopy pullbacks}
\subsection{Cartesian squares}

Recall that a square
\begin{equation*}
\begin{tikzcd}
C \arrow[r,"q"] \arrow[d,swap,"p"] & B \arrow[d,"g"] \\
A \arrow[r,swap,"g"] & X
\end{tikzcd}
\end{equation*}
is said to \define{commute}\index{commuting square|textbf} if there is a homotopy $H:f\circ p\htpy g\circ q$. 

\begin{defn}\label{defn:cospan}
A \define{cospan}\index{cospan|textbf} consists of three types $A$, $X$, and $B$, and maps $f:A\to X$ and $g:B\to X$. Given a type $C$, a \define{cone}\index{cone!on a cospan|textbf} on the cospan $A \stackrel{f}{\rightarrow} X \stackrel{g}{\leftarrow} B$ with \define{vertex} $C$\index{vertex!of a cone|textbf} consists of maps $p:C\to A$, $q:C\to B$ and a homotopy $H:f\circ p\htpy g\circ q$ witnessing that the square
\begin{equation*}
\begin{tikzcd}
C \arrow[r,"q"] \arrow[d,swap,"p"] & B \arrow[d,"g"] \\
A \arrow[r,swap,"f"] & X
\end{tikzcd}
\end{equation*}
commutes. We write\index{cone(C)@{$\cone(\blank)$}|textbf}
\begin{equation*}
\cone(C)\defeq \sm{p:C\to A}{q:C\to B}f\circ p\htpy g\circ q
\end{equation*}
for the type of cones with vertex $C$.
\end{defn}

Given a cone with vertex $C$ on a span $A\stackrel{f}{\rightarrow} X \stackrel{g}{\leftarrow} B$ and a map $h:C'\to C$, we construct a new cone with vertex $C'$ in the following definition.

\begin{defn}
For any cone $(p,q,H)$ with vertex $C$ and any type $C'$, we define a map\index{cone map@{$\conemap$}|textbf}
\begin{equation*}
\conemap(p,q,H):(C'\to C)\to\cone(C')
\end{equation*}
by $h\mapsto (p\circ h,q\circ h,H\cdot h)$. 
\end{defn}

\begin{defn}
We say that a commuting square
\begin{equation*}
\begin{tikzcd}
C \arrow[r,"q"] \arrow[d,swap,"p"] & B \arrow[d,"g"] \\
A \arrow[r,swap,"f"] & X
\end{tikzcd}
\end{equation*}
with $H:f\circ p\htpy g\circ q$ is a \define{pullback square}\index{pullback square|textbf}, or that it is \define{cartesian}\index{cartesian square|textbf}, if it satisfies the \define{universal property} of pullbacks\index{universal property!of pullbacks}, which asserts that the map
\begin{equation*}
\conemap(p,q,H):(C'\to C)\to\cone(C')
\end{equation*}
is an equivalence for every type $C'$. 
\end{defn}

We often indicate the universal property with a diagram as follows:
\begin{equation*}
\begin{tikzcd}
C' \arrow[drr,bend left=15,"{q'}"] \arrow[dr,densely dotted,"h"] \arrow[ddr,bend right=15,swap,"{p'}"] \\
& C \arrow[r,"q"] \arrow[d,swap,"p"] & B \arrow[d,"g"] \\
& A \arrow[r,swap,"f"] & X
\end{tikzcd}
\end{equation*}
since the universal property states that for every cone $(p',q',H')$ with vertex $C'$, the type of pairs $(h,\alpha)$ consisting of $h:C'\to C$ equipped with $\alpha:\conemap((p,q,H),h)=(p',q',H')$ is contractible by \cref{thm:contr_equiv}.


\begin{prp}\label{thm:pullback_up}
Consider a commuting square
\begin{equation*}
\begin{tikzcd}
C \arrow[r,"q"] \arrow[d,swap,"p"] & B \arrow[d,"g"] \\
A \arrow[r,swap,"f"] & X
\end{tikzcd}
\end{equation*}
with $H:f\circ p\htpy g\circ q$
Then the following are equivalent:\index{universal property!of pullbacks (characterization)|textit}
\begin{enumerate}
\item The square is a pullback square.
\item For every type $C'$ and every cone $(p',q',H')$ with vertex $C'$, the type of quadruples $(h,K,L,M)$ consisting of
\begin{align*}
h & : C'\to C \\
K & : p\circ h \htpy p' \\
L & : q\circ h \htpy q' \\
M & : \ct{(H\cdot h)}{(g\cdot L)} \htpy \ct{(f\cdot K)}{H'}
\end{align*}
is contractible.
\end{enumerate}
\end{prp}

\begin{rmk}
The homotopy $M$ in \cref{thm:pullback_up} witnesses that the square
\begin{equation*}
\begin{tikzcd}
f\circ p\circ h \arrow[r,"f\cdot K"] \arrow[d,swap,"H\cdot h"] & f\circ p' \arrow[d,"{H'}"] \\
g\circ q\circ h \arrow[r,swap,"g\cdot L"] & g\circ q'
\end{tikzcd}
\end{equation*}
of homotopies commutes.
\end{rmk}

\subsection{The unique existence of pullbacks}

\begin{defn}
Let $f:A\to X$ and $B\to X$ be maps. Then we define
\begin{align*}
A\times_X B & \defeq \sm{x:A}{y:B}f(x)=g(y) \\
\pi_1 & \defeq \proj 1 & & : A\times_X B\to A \\
\pi_2 & \defeq \proj 1\circ\proj 2 & & : A\times_X B\to B\\
\pi_3 & \defeq \proj 2\circ\proj 2 & & : f\circ \pi_1 \htpy g\circ\pi_2.
\end{align*}
The type $A\times_X B$ is called the \define{canonical pullback}\index{canonical pullback|textbf} of $f$ and $g$.
\end{defn}

Note that $A\times_X B$ depends on $f$ and $g$, although this dependency is not visible in the notation.

\begin{prp}[Exercise 2.11 of \cite{hottbook}]
Given maps $f:A\to X$ and $g:B\to X$, the commuting square\index{canonical pullback|textit}
\begin{equation*}
\begin{tikzcd}
A\times_X B \arrow[r,"\pi_2"] \arrow[d,swap,"\pi_1"] & B \arrow[d,"g"] \\
A \arrow[r,swap,"f"] & X,
\end{tikzcd}
\end{equation*}
is a pullback square.
\end{prp}

In the following lemma we establish the uniqueness of pullbacks up to equivalence via a \emph{3-for-2 property} for pullbacks.

\begin{lem}\label{lem:pb_3for2}\index{pullback!3-for-2 property|textit}\index{3-for-2 property!of pullbacks|textit}%
Consider the squares
\begin{equation*}
\begin{tikzcd}
C \arrow[r,"q"] \arrow[d,swap,"p"] & B \arrow[d,"g"] & {C'} \arrow[r,"{q'}"] \arrow[d,swap,"{p'}"] & B \arrow[d,"g"] \\
A \arrow[r,swap,"f"] & X & A \arrow[r,swap,"f"] & X
\end{tikzcd}
\end{equation*}
with homotopies $H:f\circ p \htpy g\circ q$ and $H':f\circ p'\htpy g\circ q'$.
Furthermore, suppose we have a map $h:C'\to C$ equipped with
\begin{align*}
K & : p\circ h \htpy p' \\
L & : q\circ h \htpy q' \\
M & : \ct{(H\cdot h)}{(g\cdot L)} \htpy \ct{(f\cdot K)}{H'}.
\end{align*}
If any two of the following three properties hold, so does the third:
\begin{samepage}%
\begin{enumerate}
\item $C$ is a pullback.
\item $C'$ is a pullback.
\item $h$ is an equivalence.
\end{enumerate}%
\end{samepage}%
\end{lem}

\begin{proof}
The type of triples $(K,L,M)$ is equivalent to the type of identifications
\begin{equation*}
\conemap((p,q,H),h)=(p',q',H').
\end{equation*}
Let $D$ be a type, and let $k:D\to C'$ be a map. We observe that
\begin{align*}
\conemap((p,q,H),(h\circ k)) & \jdeq (p\circ (h\circ k),q\circ (h\circ k),H\circ (h\circ k)) \\
& \jdeq ((p\circ h)\circ k,(q\circ h)\circ k, (H\circ h)\circ k) \\
& \jdeq \conemap(\conemap((p,q,H),h),k) \\
& = \conemap((p',q',H'),k).
\end{align*}
Thus we see that the triangle 
\begin{equation*}
\begin{tikzcd}[column sep=-1em]
(D\to C') \arrow[rr,"{h\circ \blank}"] \arrow[dr,swap,"{\conemap(p',q',H')}"] & & (D\to C) \arrow[dl,"{\conemap(p,q,H)}"] \\
& \cone(D) & \phantom{(D\to C')}
\end{tikzcd}
\end{equation*}
commutes. Therefore it follows from the 3-for-2 property of equivalences that if any two of the following properties hold, then so does the third:
\begin{enumerate}
\item The map $\conemap(p,q,H):(D\to C)\to \cone(D)$ is an equivalence,
\item The map $\conemap(p',q',H'):(D\to C')\to \cone(D)$ is an equivalence,
\item The map $h\circ\blank : (D\to C')\to (D\to C)$ is an equivalence.
\end{enumerate}
Thus the 3-for-2 property for pullbacks follows from the fact that $h$ is an equivalence if and only if $h\circ\blank : (D\to C')\to (D\to C)$ is an equivalence for any type $D$.
\end{proof}


\begin{defn}
Given a commuting square
\begin{equation*}
\begin{tikzcd}
C \arrow[r,"q"] \arrow[d,swap,"p"] & B \arrow[d,"g"] \\
A \arrow[r,swap,"f"] & X
\end{tikzcd}
\end{equation*}
with $H:f\circ p \htpy g \circ q$, we define the \define{gap map}\index{gap map|textbf}\index{pullback!gap map|textbf}
\begin{equation*}
\gap(p,q,H):C \to A\times_X B
\end{equation*}
by $\lam{z}(p(z),q(z),H(z))$. Furthermore, we will write\index{is_pullback@{$\ispullback$}|textbf}
\begin{equation*}
\ispullback(f,g,H)\defeq \isequiv(\gap(p,q,H)).
\end{equation*}
\end{defn}

\begin{prp}\label{thm:is_pullback}
Consider a commuting square
\begin{equation*}
\begin{tikzcd}
C \arrow[r,"q"] \arrow[d,swap,"p"] & B \arrow[d,"g"] \\
A \arrow[r,swap,"f"] & X
\end{tikzcd}
\end{equation*}
with $H:f\circ p \htpy g \circ q$. The following are equivalent:
\begin{enumerate}
\item The square is a pullback square
\item There is a term of type
\begin{equation*}
\ispullback(p,q,H)\defeq \isequiv(\gap(p,q,H)).
\end{equation*}
\end{enumerate}
\end{prp}

\begin{proof}
Note that there are homotopies
\begin{align*}
K & : \pi_1\circ \gap(p,q,H) \htpy p \\
L & : \pi_2\circ \gap(p,q,H) \htpy q \\
M & : \ct{(\pi_3\cdot \gap(p,q,H))}{(g\cdot L)} \htpy \ct{(f\cdot K)}{H}.
\end{align*}
given by 
\begin{align*}
K & \defeq \lam{z}\refl{p(z)} \\
L & \defeq \lam{z}\refl{q(z)} \\
M & \defeq \lam{z}\ct{\rightunit(H(z))}{\leftunit(H(z))^{-1}}.
\end{align*}
Therefore the claim follows by \cref{lem:pb_3for2}.
\end{proof}

\subsection{Fiberwise equivalences}

\begin{prp}\label{thm:pb_fibequiv}
Let $f:A\to B$, and let $g:\prd{a:A}P(a)\to Q(f(a))$ be a fiberwise transformation\index{fiberwise transformation|textit}. The following are equivalent:
\begin{enumerate}
\item The commuting square
\begin{equation*}
\begin{tikzcd}[column sep=large]
\sm{a:A}P(a) \arrow[r,"{\total[f]{g}}"] \arrow[d,swap,"\proj 1"] & \sm{b:B}Q(b) \arrow[d,"\proj 1"] \\
A \arrow[r,swap,"f"] & B
\end{tikzcd}
\end{equation*}
is a pullback square.
\item $g$ is a fiberwise equivalence.\index{fiberwise equivalence|textit}
\end{enumerate}
\end{prp}

\begin{proof}
The gap map factors as follows
\begin{equation*}
\begin{tikzcd}[column sep=-2em]
\sm{x:A}P(x) \arrow[dr,swap,"\total{g}"] \arrow[rr,"\gap"] & & A \times_B \Big(\sm{y:B}Q(y)\Big) \\
\phantom{A \times_B \Big(\sm{y:B}Q(y)\Big)} & \sm{x:A}Q(f(x)) \arrow[ur,swap,"{\gap'\,\defeq\,\lam{(x,q)}(x,(f(x),q),\refl{f(x)})}"]
\end{tikzcd}
\end{equation*}
Since $\gap'$ is an equivalence, it follows by \cref{thm:fib_equiv} that the gap map is an equivalence if and only if $g$ is a fiberwise equivalence.
\end{proof}

\begin{lem}
Consider a commuting square
\begin{equation*}
\begin{tikzcd}
C \arrow[r,"q"] \arrow[d,swap,"p"] & B \arrow[d,"g"] \\
A \arrow[r,swap,"f"] & X
\end{tikzcd}
\end{equation*}
with $H:f\circ p\htpy g\circ q$, and consider the fiberwise transformation
\begin{equation*}
\fibf{(f,q,H)} : \prd{a:A} \fib{p}{a}\to \fib{g}{f(a)}
\end{equation*}
given by $\lam{a}{(c,u)}(q(c),\ct{H(c)^{-1}}{\ap{f}{u}})$. Then there is an equivalence
\begin{equation*}
\eqv{\fib{\gap(p,q,H)}{(a,b,\alpha)}}{\fib{\fibf{(f,q,H)}(a)}{(b,\alpha^{-1})}}
\end{equation*}
\end{lem}

\begin{proof}
To obtain an equivalence of the desired type we simply concatenate known equivalences:
\begin{align*}
\fib{h}{(a,b,\alpha)} & \jdeq \sm{z:C} (p(z),q(z),H(z))=(a,b,\alpha) \\
& \eqvsym \sm{z:C}{u:p(z)=a}{v:q(z)=b}\ct{H(z)}{\ap{g}{v}}=\ct{\ap{f}{u}}{\alpha} \\
& \eqvsym \sm{(z,u):\fib{p}{a}}{v:q(z)=b} \ct{H(z)^{-1}}{\ap{f}{u}}=\ct{\ap{g}{v}}{\alpha^{-1}} \\
& \eqvsym \fib{\varphi(a)}{(b,\alpha^{-1})}\qedhere
\end{align*}
\end{proof}

\begin{cor}\label{cor:pb_fibequiv}
Consider a commuting square
\begin{equation*}
\begin{tikzcd}
C \arrow[r,"q"] \arrow[d,swap,"p"] & B \arrow[d,"g"] \\
A \arrow[r,swap,"f"] & X
\end{tikzcd}
\end{equation*}
with $H:f\circ p\htpy g\circ q$. The following are equivalent:
\begin{enumerate}
\item The square is a pullback square.\index{pullback square!characterized by fiberwise equivalence|textit}
\item The induced map on fibers
\begin{equation*}
\fibf{(p,q,H)}:\prd{x:A}\fib{p}{x}\to \fib{g}{f(x)}
\end{equation*}
is a fiberwise equivalence.
\end{enumerate}
\end{cor}

\begin{cor}\label{cor:pb_equiv}
Consider a commuting square
\begin{equation*}
\begin{tikzcd}
C \arrow[r,"q"] \arrow[d,swap,"p"] & B \arrow[d,"g"] \\
A \arrow[r,swap,"f"] & X.
\end{tikzcd}
\end{equation*}
and suppose that $g$ is an equivalence. Then the following are equivalent:
\begin{enumerate}
\item The square is a pullback square.
\item The map $p:C\to A$ is an equivalence.\index{equivalence!pullback of|textit}
\end{enumerate}
\end{cor}

\begin{proof}
If the square is a pullback square, then by \cref{thm:pb_fibequiv} the fibers of $p$ are equivalent to the fibers of $g$, which are contractible by \cref{thm:contr_equiv}. Thus it follows that $p$ is a contractible map, and hence that $p$ is an equivalence.

If $p$ is an equivalence, then by \cref{thm:contr_equiv} both $\fib{p}{x}$ and $\fib{g}{f(x)}$ are contractible for any $x:X$. It follows that the induced map $\fib{p}{x}\to\fib{g}{f(x)}$ is an equivalence. Thus we apply \cref{cor:pb_fibequiv} to conclude that the square is a pullback.
\end{proof}

\section{The univalence axiom}

The univalence axiom characterizes the identity type of the universe. It is considered to be an \emph{extensionality principle}\index{extensionality principle!types} for types. In the following theorem we introduce the univalence axiom and give two more equivalent ways of stating this.

\begin{prp}\label{thm:univalence}
The following are equivalent:
\begin{enumerate}
\item The \define{univalence axiom}\index{univalence axiom|textbf}: for any $A:\UU$ the map\index{equiv_eq@{$\equiveq$}|textbf}
\begin{equation*}
\equiveq\defeq \ind{A=}(\idfunc[A]) : \prd{B:\UU} (\id{A}{B})\to(\eqv{A}{B}).
\end{equation*}
is a fiberwise equivalence.\index{identity type!universe} If this is the case, we write
$\eqequiv$\index{eq equiv@{$\eqequiv$}|textbf}
for the inverse of $\equiveq$.
\item The type
\begin{equation*}
\sm{B:\UU}\eqv{A}{B}
\end{equation*}
is contractible for each $A:\UU$.
\item The principle of \define{equivalence induction}\index{equivalence induction}\index{induction principle!for equivalences}: for every $A:\UU$ and for every type family
\begin{equation*}
P:\prd{B:\UU} (\eqv{A}{B})\to \type,
\end{equation*}
the map
\begin{equation*}
\Big(\prd{B:\UU}{e:\eqv{A}{B}}P(B,e)\Big)\to P(A,\idfunc[A])
\end{equation*}
given by $f\mapsto f(A,\idfunc[A])$ has a section.\qedhere
\end{enumerate}
\end{prp}

It is a trivial observation, but nevertheless of fundamental importance, that by the univalence axiom the identity types of $\UU$ are equivalent to types in $\UU$, because it provides an equivalence $\eqv{(A=B)}{(\eqv{A}{B})}$, and the type $\eqv{A}{B}$ is in $\UU$ for any $A,B:\UU$. Since the identity types of $\UU$ are equivalent to types in $\UU$, we also say that the universe is \emph{locally small}.

\begin{defn}\label{defn:ess_small}
\begin{enumerate}
\item A type $A$ is said to be \define{essentially small}\index{essentially small!type|textbf} if there is a type $X:\UU$ and an equivalence $\eqv{A}{X}$. We write\index{ess_small(A)@{$\esssmall(A)$}|textbf}
\begin{equation*}
\esssmall(A)\defeq\sm{X:\UU}\eqv{A}{X}.
\end{equation*}
\item A map $f:A\to B$ is said to be \define{essentially small}\index{essentially small!map|textbf} if for each $b:B$ the fiber $\fib{f}{b}$ is essentially small.
We write\index{ess_small(f)@{$\esssmall(f)$}|textbf}
\begin{equation*}
\esssmall(f)\defeq\prd{b:B}\esssmall(\fib{f}{b}).
\end{equation*}
\item A type $A$ is said to be \define{locally small}\index{locally small!type} if for every $x,y:A$ the identity type $x=y$ is essentially small.
We write\index{loc_small(A)@{$\locsmall(A)$}|textbf}
\begin{equation*}
\locsmall(A)\defeq \prd{x,y:A}\esssmall(x=y).
\end{equation*}
\item Similarly, a map $f:A\to X$ is said to be \define{locally small} if $\delta_f:A\to A\times_X A$ is essentially small.
\end{enumerate}
\end{defn}

\begin{lem}\label{lem:isprop_ess_small}
The type $\esssmall(X)$ is a proposition for any type $X$.\index{essentially small!is a proposition|textit}
\end{lem}

\begin{proof}
Let $X$ be a type. Our goal is to show that the type
\begin{equation*}
\sm{Y:\UU}\eqv{X}{Y}
\end{equation*}
is a proposition. Suppose there is a type $X':\UU$ and an equivalence $e:\eqv{X}{X'}$, then the map
\begin{equation*}
(\eqv{X}{Y})\to (\eqv{X'}{Y})
\end{equation*}
given by precomposing with $e^{-1}$ is an equivalence. This induces an equivalence on total spaces
\begin{equation*}
\eqv{\Big(\sm{Y:\UU}\eqv{X}{Y}\Big)}{\Big(\sm{Y:\UU}\eqv{X'}{Y}\Big)}
\end{equation*}
However, the codomain of this equivalence is contractible by \cref{thm:univalence}. Thus it follows that the asserted type is a proposition.
\end{proof}

\begin{cor}
For each function $f:A\to B$, the type $\esssmall(f)$ is a proposition, and for each type $X$ the type $\locsmall(X)$ is a proposition.
\end{cor}

\begin{proof}
This follows from the fact that propositions are closed under dependent products, established in \cref{thm:prop_pi}.
\end{proof}

\begin{rmk}
The property of essentially smallness is preserved by $\Pi$, $\Sigma$, and $\mathsf{Id}$. Of course, any contractible type is essentially small, and so is any small type. The property of essentially smallness is preserved by $\Sigma$ and $\mathsf{Id}$, and the exponent $X^A$ of a locally small type $X$ by an essentially small type $A$ is again locally small. Furthermore, any proposition is locally small, and any universe is locally small with respect to itself. 
\end{rmk}

\begin{defn}
Consider two functions $f:A\to X$ and $g:B\to X$. We define the type 
\begin{equation*}
\hom_X(f,g)\defeq \sm{h:A\to B} f\htpy g\circ h.
\end{equation*}
\end{defn}

In other words, the type $\hom_X(f,g)$ is the type of functions $h:A\to B$ equipped with a homotopy witnessing that the triangle
\begin{equation*}
\begin{tikzcd}[column sep=tiny]
A \arrow[dr,swap,"f"] \arrow[rr,"h"] & & B \arrow[dl,"g"] \\
& X
\end{tikzcd}
\end{equation*}

\begin{lem}
Let $P$ and $Q$ be two type families over $X$, and write $\proj 1^P$ and $\proj 1^Q$ for their first projections, respectively. Then the map
\begin{equation*}
\tottriangle:\Big(\prd{x:X} P(x)\to Q(x)\Big)\to \hom_X(\proj 1^P,\proj 1^Q)
\end{equation*}
given by $\tottriangle(f)\defeq (\total{f},\lam{(x,y)}\refl{x})$, is an equivalence.
\end{lem}

\begin{cor}\label{cor:fib_triangle}
For any two maps $f:A\to X$ and $g:B\to X$, the map
\begin{equation*}
\fibtriangle : \hom_X(f,g) \to \prd{x:X}\fib{f}{x}\to \fib{g}{x}
\end{equation*}
given by $\lam{(h,H)}{x}{(a,p)}(h(a),\ct{H(a)^{-1}}{p})$ is an equivalence.
\end{cor}

\begin{thm}\label{thm:fam_proj}
For any small type $A:\UU$ there is an equivalence
\begin{equation*}
\eqv{(A\to \UU)}{\Big(\sm{X:\UU} X\to A\Big)}.
\end{equation*}
\end{thm}

\begin{proof}
Note that we have the function
\begin{equation*}
\varphi :\lam{B} \Big(\sm{x:A}B(x),\proj 1\Big) : (A\to \UU)\to \Big(\sm{X:\UU}X\to A\Big).
\end{equation*}
The fiber of this map at $(X,f)$ is by univalence and function extensionality equivalent to the type
\begin{equation*}
\sm{B:A\to \UU}{e:\eqv{(\sm{x:A}B(x))}{X}} \proj 1\htpy f\circ e.
\end{equation*}
By \cref{cor:fib_triangle} this type is equivalent to the type
\begin{equation*}
\sm{B:A\to \UU}\prd{a:A} \eqv{B(a)}{\fib{f}{a}},
\end{equation*}
and by `type theoretic choice', which was established in \cref{thm:choice}, this type is equivalent to
\begin{equation*}
\prd{a:A}\sm{X:\UU}\eqv{X}{\fib{f}{a}}.
\end{equation*}
We conclude that the fiber of $\varphi$ at $(X,f)$ is equivalent to the type $\esssmall(f)$. However, since $f:X\to A$ is a map between small types it is essentially small. Moreover, since being essentially small is a proposition by \cref{lem:isprop_ess_small}, it follows that $\fib{\varphi}{(X,f)}$ is contractible for every $f:X\to A$. In other words, $\varphi$ is a contractible map, and therefore it is an equivalence.
\end{proof}

\begin{rmk}
The inverse of the map
\begin{equation*}
\varphi : (A\to \UU)\to \Big(\sm{X:\UU}X\to A\Big).
\end{equation*}
constructed in \cref{thm:fam_proj} is the map $(X,f)\mapsto \fibf{f}$.
\end{rmk}

\section{The object classifier}

\begin{defn}
Let $p:E\to B$ and $p':E'\to B'$ be maps. A morphism from $p'$ to $p$ is a triple $(f,g,H)$ consisting of maps $f:B'\to B$ and $g:E'\to E$ and a homotophy $H:f\circ p'\htpy p\circ g$ witnessing that the square
\begin{equation}\label{eq:morphism_arrow}
\begin{tikzcd}
E' \arrow[r,"g"] \arrow[d,swap,"p'"] & E \arrow[d,"p"] \\
B' \arrow[r,"f"] & B
\end{tikzcd}
\end{equation}
commutes. We write $\mathsf{hom}(p',p)$ for the type of such triples $(f,g,H)$, and sometimes we write $\mathsf{hom}_f(p',p)$ for the type of pairs $(g,H)$. A morphism $(f,g,H)$ is said to be \define{cartesian} if the square in \cref{eq:morphism_arrow} is cartesian. We write $\mathsf{cart}(p',p)$ for the type of cartesian morphisms from $p'$ to $p$, and we write $\mathsf{cart}_f(p',p)$ for the type of triples $(g,H,t)$ for the type of triples, where $t:\ispullback(p',g,H)$.  
\end{defn}

\begin{defn}\label{defn:object_classifier}
A morphism $p:E\to B$ is said to be an \define{object classifier} if the type $\mathsf{cart}(p',p)$ is a proposition for each $p':E'\to B'$. If $p:E\to B$ is an object classifier, we also write 
\begin{equation*}
\mathsf{is\usc{}classified}(p')\defeq \mathsf{hom}(p',p).
\end{equation*}
\end{defn}

Our goal in this section is to show that a univalent universe is an object classifier. 

\begin{prp}\label{thm:pb_fibequiv_complete}
Let $\alpha:I\to J$ be a map, and let $A:I\to\UU$ and $B:J\to\UU$ be type families.
Then the map
\begin{equation*}
\Big(\prd{i:I}A_i\to B_{\alpha(i)}\Big) \to \mathsf{hom}_\alpha(\proj 1^A,\proj 1^B)
\end{equation*}
given by $\lam{f}(\total[\alpha]{f},\lam{(x,y)}\refl{x})$ is an equivalence. Furthermore, the map
\begin{equation*}
\Big(\prd{i:I}A_i\eqvsym B_{\alpha(i)}\Big) \to \mathsf{cart}_\alpha(\proj 1^A,\proj 1^B)
\end{equation*}
given by $\lam{e}(\total[\alpha]{e},\lam{(x,y)}\refl{x},t)$ where $t$ is the term constructed in \cref{thm:pb_fibequiv}, is an equivalence. 
\end{prp}

\begin{proof}
We have the equivalences
\begin{align*}
\prd{i:I}A_i\to B_{\alpha(i)} & \eqvsym \prd{i:I}{a:A_i}\sm{j:J}{\gamma : \alpha(i)=j} B_j \\
& \eqvsym \prd{i:I}{a:A_i}\sm{j:J}{b:B_j}\alpha(i)=j \\
& \eqvsym \prd{(i,a):\sm{i:I}A_i}\sm{(j,b):\sm{j:J}B_j}\alpha(i)=j \\
& \eqvsym \sm{f:\big(\sm{i:I}A_i\big)\to\big(\sm{j:J}B_j\big)}\alpha\circ \proj 1^A\htpy \proj 1^B\circ f \\
& \jdeq \mathsf{hom}_\alpha(\proj 1^A,\proj 1^B).
\end{align*}
It is easy to check that this composite is the asserted map. The second claim follows from \cref{thm:pb_fibequiv}.
\end{proof}

\begin{cor}\label{cor:sq_fib}
Consider a diagram of the form
\begin{equation*}
\begin{tikzcd}
A \arrow[d,swap,"f"] & B \arrow[d,"g"] \\
I \arrow[r,swap,"\alpha"] & J.
\end{tikzcd}
\end{equation*}
Then the map
\begin{equation*}
\mathsf{hom}_\alpha(f,g)\to \Big(\prd{i:I}\fib{f}{i}\to\fib{g}{\alpha(i)}\Big)
\end{equation*}
given by $\lam{(h,H)}{i}{(a,p)}(h(a),\ct{H(a)^{-1}}{\ap{\alpha}{p}})$ is an equivalence.
\end{cor}

\begin{thm}\label{thm:classifier}
Let $f:A\to B$ be a map, and let $\UU$ be a univalent universe with universal family $\mathrm{El}$ over $\UU$. Then there is an equivalence
\begin{equation*}
\eqv{\esssmall(f)}{\mathsf{cart}(f,\proj 1^{\mathrm{El}})}.
\end{equation*}
In particular, the type $\mathsf{cart}(f,\proj 1^{\mathrm{El}})$ is a proposition for each map $f$, so the universe is an object classifier in the sense of \cref{defn:object_classifier}.
\end{thm}

\begin{proof}
From \cref{cor:sq_fib} we obtain that the type of pairs $(\tilde{F},H)$ is equivalent to the type of fiberwise transformations
\begin{equation*}
\prd{b:B}\fib{f}{b}\to F(b).
\end{equation*}
By \cref{cor:pb_fibequiv} the square is a pullback square if and only if the induced map
\begin{equation*}
\prd{b:B}\fib{f}{b}\to F(b)
\end{equation*}
is a fiberwise equivalence. Thus the data $(F,\tilde{F},H,pb)$ is equivalent to the type of pairs $(F,e)$ where $e$ is a fiberwise equivalence from $\fibf{f}$ to $F$. By \cref{thm:choice} the type of pairs $(F,e)$ is equivalent to the type $\esssmall(f)$. 
\end{proof}

\begin{rmk}
For any type $A$ (not necessarily small), and any $B:A\to \UU$, the square\index{Sigma-type@{$\Sigma$-type}!as pullback of universal family|textit}
\begin{equation*}
\begin{tikzcd}[column sep=6em]
\sm{x:A}B(x) \arrow[d,swap,"\proj 1"] \arrow[r,"{\lam{(x,y)}(B(x),y)}"] & \sm{X:\UU}X \arrow[d,"\proj 1"] \\
A \arrow[r,swap,"B"] & \UU
\end{tikzcd}
\end{equation*}
is a pullback square. Therefore it follows that for any family $B:A\to\UU$ of small types, the projection map $\proj 1:\sm{x:A}B(x)\to A$ is an essentially small map.
To see that the claim is a direct consequence of \cref{thm:pb_fibequiv} we write the asserted square in its rudimentary form:
\begin{equation*}
\begin{tikzcd}[column sep=6em]
\sm{x:A}\mathrm{El}(B(x)) \arrow[d,swap,"\proj 1"] \arrow[r,"{\lam{(x,y)}(B(x),y)}"] & \sm{X:\UU}\mathrm{El}(X) \arrow[d,"\proj 1"] \\
A \arrow[r,swap,"B"] & \UU.
\end{tikzcd}
\end{equation*}
\end{rmk}

In the following theorem we show that a type is locally small if and only if its diagonal is classified by $\UU$.

\begin{thm}
Let $A$ be a type. The following are equivalent:
\begin{enumerate}
\item $A$ is locally small.\index{locally small|textit}
\item The diagonal $\delta_A : A\to A\times A$ is classified by $\UU$.\index{diagonal!of a type|textit}
\end{enumerate}
\end{thm}

\begin{proof}
The identity type $x=y$ is the fiber of $\delta_A$ at $(x,y):A\times A$. Therefore it follows that $A$ is locally small if and only if the diagonal $\delta_A$ is essentially small.
Now the result follows from \cref{thm:classifier}.
\end{proof}
