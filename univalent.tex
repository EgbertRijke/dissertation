\chapter{The object classifier}

In this chapter we establish notation, and we highlight the basic results concerning fiberwise transformations and fiberwise equivalences, which we will use for the descent theorems. Of particular importance are the following theorems:
\begin{enumerate}
\item The fundamental theorem of identity types (\cref{thm:id_fundamental}), in which we establish that a type family $B$ over $A$ with $b:B(a)$ for a given $a:A$ is fiberwise equivalent to the identity type $a=x$ if and only if its total space is contractible. This result appears in \cite{hottbook} as Theorem 5.8.2, which contains other equivalent conditions as well.
\item \cref{thm:pb_fibequiv}, in which we establish that a commuting square
\begin{equation*}
\begin{tikzcd}
A \arrow[d,swap,"f"] \arrow[r] & B \arrow[d,"g"] \\
X \arrow[r,swap,"h"] & Y
\end{tikzcd}
\end{equation*}
is a pullback square if and only if the induced fiberwise transformation
\begin{equation*}
\prd{x:X} \fib{f}{x}\to\fib{g}{h(x)}
\end{equation*}
is a fiberwise equivalence. Our main reference \cite{hottbook} does not present many results of homotopy pullbacks, so there is no type theoretical exhibition of this result, although it was surely known to experts.
\item \cref{thm:object_classifier}, in which we establish that the universe is an object classifier. This result appears in \cite{rijkespitters} as Theorem 2.31, and in \cite{hottbook} as Theorem 4.8.4
\end{enumerate}

\section{Notation and preliminary results}

We work in Martin-L\"of dependent type theory with $\Pi$-types, $\Sigma$-types and cartesian products, coproducts $A+B$ equipped with $\inl:A\to A+B$ and $\inr:B\to A+B$ for any two types $A$ and $B$, an empty type $\emptyt$, a unit type $\unit$ equipped with $\ttt:\unit$, a type $\bool$ of booleans equipped with $\btrue,\bfalse:\bool$, a type $\N$ of natural numbers equipped with $0:\N$ and $\mathsf{succ}:\N\to\N$, and identity types. 

\begin{rmk}
As usual, we write $\idfunc[A]:A\to A$ for the identity function $\lam{x}x$ on $A$, and we write $g\circ f:A\to C$ for the composite function $\lam{x}g(f(x))$ of $f:A\to B$ and $g:B\to C$. For any two types $A$ and $B$, and any $b:B$, we write
\begin{equation*}
\mathsf{const}_b : A\to B
\end{equation*}
for the \define{constant function} $\lam{x}b$.

In the case of $\Sigma$-types, the empty type $\emptyt$, and the unit type, we use the following notation to define functions by pattern-matching:
\begin{align*}
\lam{(x,y)}f(x,y) & : \prd{t:\sm{x:A}B(x)} P(t) \\
\lambda & : \prd{t:\emptyt} P(t) \\
\lam{\ttt}y & : \prd{t:\unit}P(t),
\end{align*}
and we use similar notation for definitions by iterated pattern-matching. For instance, given a function $f:\prd{x:A}{y:B(x)}{z:C(x,y)} P((x,y),z)$ we obtain the function
\begin{equation*}
\lam{((x,y),z)} f(x,y,z): \prd{t:\sm{s:\sm{x:A}B(x)}C(s)}P(t).
\end{equation*}
\end{rmk}

Given a type $A$ in context $\Gamma$, the \define{identity type} of $A$ at $a:A$ is the inductive type family 
\begin{equation*}
\Gamma,x:A\vdash a =_A x~\mathrm{type}
\end{equation*}
with constructor
\begin{equation*}
\Gamma \vdash \refl{a} : a=_A a.
\end{equation*}
The induction principle for the identity type of $A$ at $a$ asserts that for any type family
\begin{equation*}
\Gamma,x:A,\alpha: a=_A x\vdash P(x,\alpha)~\mathrm{type}
\end{equation*}
there is a term
\begin{equation*}
\ind{a=} : P(a,\refl{a})\to \prd{x:A}{\alpha:a=_A x}P(x,\alpha)
\end{equation*}
in context $\Gamma$, satisfying the computation rule
\begin{equation*}
\ind{a=}(p,a,\refl{a})\jdeq p.
\end{equation*}
We also use pattern-matching notation $\lam{\,\refl{a}}p$ for $\ind{a=}(p)$.

A term of type $a=_A x$ is also called an \define{identification} of $a$ with $x$, or a \define{path} from $a$ to $x$.
The induction principle for identity types is sometimes called \define{identification elimination} or \define{path induction}. Occasionally, we also write $\idtypevar{A}$ for the identity type on $A$. 

Moreover, we assume that there is a universe $\UU$ with a universal family $\mathrm{El}$ over $\UU$, that is closed under the type forming operations. For example, there is a map
\begin{equation*}
\check{\mathsf{Id}}:\prd{A:\UU}\mathrm{El}(A)\to\mathrm{El}(A)\to\UU
\end{equation*}
satisfying
\begin{equation*}
\mathrm{El}(\check{\mathsf{Id}}(A,x,y))\jdeq x=_{\mathrm{El}(A)} y,
\end{equation*}
establishing that the universe is closed under identity types.

Given a type $A$ the \define{concatenation} operation
\begin{equation*}
\mathsf{concat} : \prd{x,y,z:A} (\id{x}{y})\to(\id{y}{z})\to (\id{x}{z})
\end{equation*}
is defined by $\mathsf{concat}(\refl{x},q)\defeq q$. We will usually write $\ct{p}{q}$ for $\mathsf{concat}(p,q)$. 
The concatenation operation satisfies the unit laws
\begin{align*}
\mathsf{left\usc{}unit}(p) & : \ct{\refl{x}}{p}=p \\
\mathsf{right\usc{}unit}(p) & : \ct{p}{\refl{y}}=p,
\intertext{with a cohererence}
\mathsf{coh\usc{}unit}(x) & : \mathsf{left\usc{}unit}(\refl{x})=\mathsf{right\usc{}unit}(\refl{x})
\end{align*}
that will come into play when we study type valued equivalence relations.

The \define{inverse operation} 
\begin{equation*}
\mathsf{inv}:\prd{x,y:A} (x=y)\to (y=x)
\end{equation*}
is defined by $\mathsf{inv}(\refl{x})\defeq\refl{x}$. We will usually write $p^{-1}$ for $\mathsf{inv}(p)$.
The inverse operation satisfies the inverse laws
\begin{align*}
\mathsf{left\usc{}inv}(p) & : \ct{p^{-1}}{p} = \refl{y} \\
\mathsf{right\usc{}inv}(p) & : \ct{p}{p^{-1}} = \refl{x},
\intertext{with a coherence}
\mathsf{coh\usc{}inv}(x) & \defeq \mathsf{left\usc{}inv}(\refl{x})=\mathsf{right\usc{}inv}(\refl{x}).
\end{align*}

The \define{associativity operation}, which assigns to each $p:x=y$, $q:y=z$, and $r:z=w$ the \define{associator}
\begin{equation*}
\mathsf{assoc}(p,q,r) : \ct{(\ct{p}{q})}{r}=\ct{p}{(\ct{q}{r})}
\end{equation*}
is defined by $\mathsf{assoc}(\refl{x},q,r)\defeq \refl{\ct{q}{r}}$. The associator again satisfies unit laws, which we will encounter when we study type-valued equivalence relations.

Given a map $f:A\to B$, the \define{action on paths} of $f$ is an operation
\begin{equation*}
\apfunc{f} : \prd{x,y:A} (\id{x}{y})\to(\id{f(x)}{f(y)})
\end{equation*}
defined by $\ap{f}{\refl{x}}\defeq\refl{f(x)}$. 
Moreover, there are operations
\begin{align*}
\mathsf{ap\usc{}idfun}_A & : \prd{x,y:A}{p:\id{x}{y}} \id{p}{\ap{\idfunc[A]}{p}} \\
\mathsf{ap\usc{}comp}(f,g) & : \prd{x,y:A}{p:\id{x}{y}} \id{\ap{g}{\ap{f}{p}}}{\ap{g\circ f}{p}}
\end{align*}
defined by $\mathsf{ap\usc{}idfun}_A(\refl{x})\defeq \refl{\refl{x}}$ and $\mathsf{ap\usc{}comp}(f,g,\refl{x})\jdeq \refl{\refl{g(f(x))}}$, respectively.

The action on paths of a map preserves the groupoid operations, i.e.~we can construct
\begin{align*}
\mathsf{ap\usc{}refl}(f,x) & : \id{\ap{f}{\refl{x}}}{\refl{f}(x)} \\
\mathsf{ap\usc{}inv}(f,p) & : \id{\ap{f}{p^{-1}}}{\ap{f}{p}^{-1}} \\
\mathsf{ap\usc{}concat}(f,p,q) & : \id{\ap{f}{\ct{p}{q}}}{\ct{\ap{f}{p}}{\ap{f}{q}}}
\end{align*}
by identification elimination. The groupoid laws are also preserved.

\begin{defn}
Let $A$ be a type, and let $B$ be a type family over $A$. The \define{transport} operation
\begin{equation*}
\mathsf{tr}_B:\prd{x,y:A} (\id{x}{y})\to (B(x)\to B(y))
\end{equation*}
is defined by $\mathsf{tr}_B(\refl{x}) \defeq \idfunc[B(x)]$. 
\end{defn}

\begin{defn}\label{defn:apd}
Given a dependent function $f:\prd{a:A}B(a)$ and a path $p:\id{x}{y}$ in $A$, the \define{dependent action on paths}
\begin{equation*}
\apdfunc{f} : \prd{x,y:A}{p:x=y}\id{\mathsf{tr}_B(p,f(x))}{f(y)}
\end{equation*}
is defined by $\apd{f}{\refl{x}}\defeq \refl{f(x)}$.
\end{defn}

\begin{defn}
Let $f,g:\prd{x:A}P(x)$ be two dependent functions. The type $f\htpy g$ of \define{homotopies}\index{homotopy|textbf} from $f$ to $g$ is defined as
\begin{equation*}
f\htpy g \defeq \prd{x:A} f(x)=g(x).
\end{equation*}
\end{defn}

The reflexivity, inverse, and concatenation operations on homotopies are defined pointwise.
We will write $H^{-1}$ for $\lam{x}H(x)^{-1}$, and $\ct{H}{K}$ for $\lam{x}\ct{H(x)}{K(x)}$.
These operations satisfy the groupoid laws (phrased appropriately as homotopies). Apart from the groupoid operations and their laws, we will occasionally need \emph{whiskering} operations and the naturality of homotopies.

\begin{defn}\label{defn:htpy_whisering}
We define the following \define{whiskering}\index{homotopy!whiskering operations|textbf}\index{whiskering operations!of homotopies|textbf} operations on homotopies:
\begin{enumerate}
\item Suppose $H:f\htpy g$ for two functions $f,g:A\to B$, and let $h:B\to C$. We define
\begin{equation*}
h\cdot H\defeq \lam{x}\ap{h}{H(x)}:h\circ f\htpy h\circ g.
\end{equation*}
\item Suppose $f:A\to B$ and $H:g\htpy h$ for two functions $g,h:B\to C$. We define
\begin{equation*}
H\cdot f\defeq\lam{x}H(f(x)):h\circ f\htpy g\circ f.
\end{equation*}
\end{enumerate}
\end{defn}

\begin{comment}
\begin{defn}\label{defn:htpy_nat}\index{homotopy!naturality|textbf}
Let $f,g:A\to B$ be functions, and consider a homotopy $H:f\htpy g$. We define identification
\begin{align*}
\mathsf{nat\usc{}htpy}(H,p) & : \prd{y:A}{p:x=y}\ct{H(x)}{\ap{g}{p}}=\ct{\ap{f}{p}}{H(y)}
\end{align*}
witnessing that the square
\begin{equation*}
\begin{tikzcd}
f(x) \arrow[r,equals,"H(x)"] \arrow[d,equals,swap,"\ap{f}{p}"] & g(x) \arrow[d,equals,"\ap{g}{p}"] \\
f(y) \arrow[r,equals,swap,"H(y)"] & g(y)
\end{tikzcd}
\end{equation*}
commutes for every $p:x=y$, as $\lam{\,\refl{x}} \mathsf{right\usc{}unit}(H(x))$. This square is also called the \define{naturality square} of the homotopy $H$ at $p$.
\end{defn}
\end{comment}

\begin{defn}
Let $f:A\to B$ be a function. We say that $f$ has a \define{section}\index{section!of a map|textbf} if there is a term of type\index{sec(f)@{$\mathsf{sec}(f)$}|textbf}
\begin{equation*}
\mathsf{sec}(f) \defeq \sm{g:B\to A} f\circ g\htpy \idfunc[B].
\end{equation*}
Dually, we say that $f$ has a \define{retraction}\index{retraction} if there is a term of type\index{retr(f)@{$\mathsf{retr}(f)$}|textbf}
\begin{equation*}
\mathsf{retr}(f) \defeq \sm{h:B\to A} h\circ f\htpy \idfunc[A].
\end{equation*}
If $f$ has a retraction, we also say that $A$ is a \define{retract}\index{retract!of a type} of $B$.
\end{defn}

\begin{eg}
The induction principles for $\Sigma$-types, $\emptyt$, $\unit$, and $\bool$ provide sections for the functions
\begin{align*}
\mathsf{ev\usc{}pair} & : \Big(\prd{t:\sm{x:A}B(x)}P(t)\Big)\to \prd{x:A}{y:B(x)}P((x,y)) \\
\mathsf{const}_\ttt & : \Big(\prd{t:\emptyt}P(t)\Big)\to \unit \\
\mathsf{ev\usc{}pt} & : \Big(\prd{t:\unit}P(t)\Big) \to P(\ttt) \\
\mathsf{ev\usc{}f\usc{}t} & : \Big(\prd{t:\bool}P(t)\Big) \to P(\bfalse)\times P(\btrue) \\
\end{align*}
\end{eg}

\begin{defn}
We say that a function $f:A\to B$ is an \define{equivalence}\index{equivalence|textbf}\index{bi-invertible map|see {equivalence}} if it has both a section and a retraction, i.e.~if it comes equipped with a term of type\index{is_equiv@{$\isequiv$}|textbf}
\begin{equation*}
\isequiv(f)\defeq\mathsf{sec}(f)\times\mathsf{retr}(f).
\end{equation*}
We will write $\eqv{A}{B}$\index{equiv@{$\eqv{A}{B}$}|textbf} for the type $\sm{f:A\to B}\isequiv(f)$.
\end{defn}

Clearly, if $f$ is \define{invertible}\index{invertible map} in the sense that it comes equipped with a function $g:B\to A$ such that $f\circ g\htpy\idfunc[B]$ and $g\circ f\htpy\idfunc[A]$, then $f$ is an equivalence. We write\index{has_inverse@{$\mathsf{has\usc{}inverse}$}|textbf}
\begin{equation*}
\mathsf{has\usc{}inverse}(f)\defeq\sm{g:B\to A} (f\circ g\htpy \idfunc[B])\times (g\circ f\htpy\idfunc[A]).
\end{equation*}
The section of an equivalence is also a retraction (and vice versa), so we define the \define{inverse} of an equivalence to be its section. It follows immediately that the inverse of any equivalence is again an equivalence.\index{equivalence!invertibility of} The identity function $\idfunc[A]$ on a type $A$ is an equivalence since it is its own section and its own retraction.

It is straightforward to show that for any two functions $f,g:A\to B$, we have
\begin{equation*}
(f\htpy g)\to (\isequiv(f)\leftrightarrow\isequiv(g)).
\end{equation*}
Given a commuting triangle
\begin{equation*}
\begin{tikzcd}[column sep=tiny]
A \arrow[rr,"h"] \arrow[dr,swap,"f"] & & B \arrow[dl,"g"] \\
& X.
\end{tikzcd}
\end{equation*}
with $H:f\htpy g\circ h$, we have:
\begin{enumerate}
\item Suppose that the map $h$ has a section, then $f$ has a section if and only if $g$ has a section.
\item Suppose that the map $g$ has a retraction, then $f$ has a retraction if and only if $h$ has a retraction.
\item (The \define{3-for-2 property} for equivalences.) If any two of the functions
\begin{equation*}
f,\qquad g,\qquad h
\end{equation*}
are equivalences, then so is the third.
\end{enumerate}

In the following theorem we characterize the identity type of a $\Sigma$-type as a $\Sigma$-type of identity types.


\begin{prp}[Theorem 2.7.2 of \cite{hottbook}]\label{thm:eq_sigma}
Let $B$ be a type family over $A$, let $s:\sm{x:A}B(x)$, and consider the dependent function\index{pair_eq@{$\mathsf{pair\usc{}eq}$}|textbf}
\begin{equation*}
\mathsf{pair\usc{}eq}_s:\prd{t:\sm{x:A}B(x)} (s=t)\to \sm{\alpha:\proj 1(s)=\proj 1(t)} \mathsf{tr}_B(\alpha,\proj 2(s))=\proj 2(t)
\end{equation*}
defined by $\mathsf{pair\usc{}eq}_s(\refl{s}) \defeq (\refl{\proj 1(s)},\refl{\proj 2(s)})$. Then $\mathsf{pair\usc{}eq}_{s,t}$ is an equivalence for every $t:\sm{x:A}B(x)$.\index{Sigma type@{$\Sigma$-type}!identity types of|textit}\index{identity type!of a Sigma-type@{of a $\Sigma$-type}|textit}
\end{prp}

We include the proof mainly to introduce some more notation.

\begin{proof}
The maps in the converse direction\index{eq_pair@{$\mathsf{eq\usc{}pair}$}}
\begin{equation*}
\mathsf{eq\usc{}pair}_{s,t} : \Big(\sm{p:\proj 1(s)=\proj 1(t)}\id{\mathsf{tr}_B(p,\proj 2(s))}{\proj 2(t)}\Big)\to(\id{s}{t})
\end{equation*}
is defined by
\begin{equation*}
\mathsf{eq\usc{}pair}_{(x,y),(x',y')}(\refl{x},\refl{y})\defeq \refl{(x,y)}.
\end{equation*}
The proofs that the function $\mathsf{eq\usc{}pair}_{s,t}$ is indeed an inverse of $\mathsf{pair\usc{}eq}_{s,t}$ are also by induction.
\end{proof}

\begin{cor}
Let $B$ be a type family over $A$, and let $\pairr{x,y},\pairr{x',y'}:\sm{x:A}B(x)$. Then the map
\begin{equation*}
\mathsf{pair\usc{}eq}_{(x,y),(x',y')} : (\id{\pairr{x,y}}{\pairr{x',y'}})\to \Big(\sm{p:x=x'}\id{\mathsf{tr}_B(p,y)}{y'}\Big)
\end{equation*}
is an equivalence.
\end{cor}

\begin{defn}
We say that a type $A$ is \define{contractible}\index{contractible!type|textbf} if there is a term of type
\begin{equation*}
\iscontr(A) \defeq \sm{c:A}\prd{x:A}c=x.
\end{equation*}
Given a term $(c,C):\iscontr(A)$, we call $c:A$ the \define{center of contraction}\index{center of contraction|textbf} of $A$, and we call $C:\prd{x:A}a=x$ the \define{contraction}\index{contraction} of $A$.
\end{defn}

Suppose $A$ is a contractible type with center of contraction $c$ and contraction $C$. Then the type of $C$ is (judgmentally) equal to the type
\begin{equation*}
\mathsf{const}_c\htpy\idfunc[A].
\end{equation*}
In other words, the contraction $C$ is a \emph{homotopy} from the constant function to the identity function.

\begin{defn}
Consider a type $A$ with a base point $a:A$. We say that $A$ satisfies \define{singleton induction}\index{singleton induction|textbf} if for every type family $B$ over $A$, the map
\begin{equation*}
\mathsf{ev\usc{}pt}:\Big(\prd{x:A}B(x)\Big)\to B(a)
\end{equation*}
given by $f\mapsto f(a)$ has a section. In other words, we have a function and a homotopy
\begin{align*}
\mathsf{sing\usc{}ind}_{A,a} & : B(a)\to \prd{x:A}B(x) \\
\mathsf{sing\usc{}comp}_{A,a} & : \prd{b:B(a)} \mathsf{sing\usc{}ind}_{A,a}(b,a)=b.
\end{align*}
\end{defn}

\begin{prp}\label{thm:contractible}
A type $A$ is contractible if and only if it satisfies signleton induction.
\end{prp}

\begin{eg}
By definition the unit type\index{unit type!contractibility} $\unit$ satisfies singleton induction, so it is contractible.
\end{eg}

\begin{prp}[Lemma 3.11.8 in \cite{hottbook}]\label{thm:total_path}
For any $x:A$, the type
\begin{equation*}
\sm{y:A}x=y
\end{equation*}
is contractible.\index{identity type!contractibility of total space|textit}
\end{prp}

\begin{proof}
We have the term $(x,\refl{x}):\sm{y:A}x=y$, and both maps in the composite
\begin{equation*}
\begin{tikzcd}[column sep=large]
\prd{t:\sm{y:A}x=y}B(t) \arrow[r,"\mathsf{ev\usc{}pair}"] & \prd{y:A}{p:x=y}B((y,p)) \arrow[r,"\mathsf{ev\usc{}refl}"] & B((x,\refl{x}))
\end{tikzcd}
\end{equation*}
have sections, so the composite has a section. The composite is $\mathsf{ev\usc{}pt}$, so we see that the asserted type satisfies singleton induction.
\end{proof}

\begin{defn}
Let $f:A\to B$ be a function, and let $b:B$. The \define{fiber}\index{fiber|textbf}\index{homotopy fiber|see {fiber}} of $f$ at $b$ is defined to be the type
\begin{equation*}
\fib{f}{b}\defeq\sm{a:A}f(a)=b.
\end{equation*}
\end{defn}

In other words, the fiber of $f$ at $b$ is the type of $a:A$ that get mapped by $f$ to $b$.
One may think of the fiber as a type theoretic version of the pre-image\index{pre-image|see {fiber}} of a point.

\begin{eg}[Lemma 4.8.1 of \cite{hottbook}]\label{eg:fib_proj}
Consider a type family $B$ over $A$. Then the map
\begin{equation*}
B(a)\to \fib{\proj 1}{a}
\end{equation*}
given by $b\mapsto ((a,b),\refl{a})$ is an equivalence. In other words, the fibers of the projection function $\proj 1 : \big(\sm{x:A}B(x)\big)\to A$ are just the fibers of the family $B$.
\end{eg}

\begin{defn}
We say that a function $f:A\to B$ is \define{contractible}\index{contractible!map|textbf} if there is a term of type
\begin{equation*}
\iscontr(f)\defeq\prd{b:B}\iscontr(\fib{f}{b}).
\end{equation*}
\end{defn}

We cite Chapter 4 of \cite{hottbook} for the following result, although it is well-known that it can be proven directly and without the use of function extensionality.

\begin{prp}[Chapter 4 in \cite{hottbook}]\label{thm:contr_equiv}
A function is an equivalence if and only if it is contractible.\index{contractible!map!is an equivalence|textit}
\end{prp}

\begin{cor}\label{cor:contr_path}
Let $A$ be a type, and let $a:A$. Then the type
\begin{equation*}
\sm{x:A}x=a
\end{equation*}
is contractible.
\end{cor}

\begin{proof}
Since the identity function is an equivalence, the fibers of the identity function are contractible by \autoref{thm:contr_equiv}. Note that $\sm{x:A}x=a$ is exactly the fiber of $\idfunc[A]$ at $a:A$.
\end{proof}

\section{The fundamental theorem of identity types}
Consider a family
\begin{equation*}
f : \prd{x:A}B(x)\to C(x)
\end{equation*}
of maps. Such $f$ is also called a \define{fiberwise map} or \define{fiberwise transformation}.

\begin{defn}[Definition 4.7.5 of \cite{hottbook}]
We define the map
\begin{equation*}
\total{f}:\sm{x:A}B(x)\to\sm{x:A}C(x).
\end{equation*}
by $\lam{(x,y)}(x,f(x,y))$.
\end{defn}

\begin{lem}[Theorem 4.7.6 of \cite{hottbook}]\label{lem:fib_total}
For any fiberwise transformation $f:\prd{x:A}B(x)\to C(x)$, and any $a:A$ and $c:C(a)$, there is an equivalence
\begin{equation*}
\eqv{\fib{f(a)}{c}}{\fib{\total{f}}{\pairr{a,c}}}.
\end{equation*}
\end{lem}

\begin{eg}
There are equivalences
\begin{equation*}
\eqv{\fib{(\apfunc{f})_{x,y}}{q}}{\fib{\delta_f}{(x,y,q)}}
\end{equation*}
for any $q:f(x)=f(y)$, because the triangle
\begin{equation*}
\begin{tikzcd}
A \arrow[rr,"{\lam{x}(x,x,\refl{x})}"] \arrow[dr,swap,"\delta_f"] & & \sm{x,y:A}x=y \arrow[dl,"\total{\total{\apfunc{f}}}"] \\
& \sm{x,y:A}f(x)=f(y)
\end{tikzcd}
\end{equation*}
commutes, and the top map is an equivalence.
\end{eg}

\begin{prp}[Theorem 4.7.7 of \cite{hottbook}]\label{thm:fib_equiv}
Let $f:\prd{x:A}B(x)\to C(x)$ be a fiberwise transformation. The following are logically equivalent:
\begin{enumerate}
\item For each $x:A$, the map $f_x:B(x)\to C(x)$ is an equivalence. In this case we say that $f$ is a \define{fiberwise equivalence}.
\item The map $\total{f}:\sm{x:A}B(x)\to\sm{x:A}C(x)$ is an equivalence.
\end{enumerate}
\end{prp}

The following theorem is the key to many results about identity types, which we will use instead of the \emph{encode-decode method}\index{encode-decode method} of \cite{LicataShulman}. We refer to it as the \define{fundamental theorem of identity types}.

\begin{thm}[Theorem 5.8.2 of \cite{hottbook}]\label{thm:id_fundamental}
Let $A$ be a type with $a:A$, and let $B$ be be a type family over $A$ with $b:B(a)$.
Then  the following are logically equivalent:
\begin{enumerate}
\item The canonical family of maps
\begin{equation*}
\ind{a{=}}(b):\prd{x:A} (a=x)\to B(x)
\end{equation*}
is a fiberwise equivalence.
\item The total space
\begin{equation*}
\sm{x:A}B(x)
\end{equation*}
is contractible.
\end{enumerate}
\end{thm}

\begin{proof}
By \autoref{thm:fib_equiv} it follows that the fiberwise transformation $\ind{a{=}}(b)$ is a fiberwise equivalence if and only if it induces an equivalence
\begin{equation*}
\eqv{\Big(\sm{x:A}a=x\Big)}{\Big(\sm{x:A}B(x)\Big)}
\end{equation*}
on total spaces. We have that $\sm{x:A}a=x$ is contractible. Now it follows by the 3-for-2 property of equivalences, applied in the case
\begin{equation*}
\begin{tikzcd}
\sm{x:A}a=x \arrow[rr,"\total{\ind{a{=}}(b)}"] \arrow[dr,swap,"\eqvsym"] & & \sm{x:A}B(x) \arrow[dl] \\
& \unit
\end{tikzcd}
\end{equation*}
that $\total{\ind{a{=}}(b)}$ is an equivalence if and only if $\sm{x:A}B(x)$ is contractible.
\end{proof}

Observe that in the proof of \cref{thm:id_fundamental} we haven't used the actual definition of the fiberwise transformation. Indeed, for any fiberwise transformation
\begin{equation*}
f:\prd{x:A}(a=x)\to B(x)
\end{equation*}
we have that $f$ is a fiberwise equivalence if and only if the total space of $B$ is contractible.

Since retracts of contractible types are again contractible, it follows that the only retract of the identity type is the identity type itself:

\begin{cor}
Let $a:A$, and let $B$ be a type family over $A$. If each $B(x)$ is a retract of $\id{a}{x}$, then $B(x)$ is equivalent to $\id{a}{x}$ for every $x:A$.
\end{cor}

As a first application of the fundamental theorem we give a quick new proof that equivalences are embeddings. The proof of the corresponding theorem in \cite{hottbook} is more involved.

\begin{defn}
An \define{embedding}\index{embedding|textbf} is a map $f:A\to B$ satisfying the property that
\begin{equation*}
\apfunc{f}:(\id{x}{y})\to(\id{f(x)}{f(y)})
\end{equation*}
is an equivalence for every $x,y:A$. We write $\mathsf{is\usc{}emb}(f)$ for the type of witnesses that $f$ is an embedding.
\end{defn}

\begin{prp}[Theorem 2.11.1 in \cite{hottbook}]
\label{cor:emb_equiv} 
Any equivalence is an embedding.\index{embedding!equivalences are embeddings|textit}\index{equivalence!is an embedding|textit}
\end{prp}

\begin{proof}
Let $e:\eqv{A}{B}$ be an equivalence, and let $x:A$. By \autoref{thm:id_fundamental} it follows that
\begin{equation*}
\apfunc{e} : (\id{x}{y})\to (\id{e(x)}{e(y)})
\end{equation*}
is an equivalence for every $y:A$ if and only if the total space
\begin{equation*}
\sm{y:A}e(x)=e(y)
\end{equation*}
is contractible for every $y:A$. Now observe that $\sm{y:A}e(x)=e(y)$ is equivalent to the fiber $\fib{e}{e(x)}$, which is contractible by \cref{thm:contr_equiv}.
\end{proof}

\begin{defn}
A type $A$ is said to be a \define{proposition} if there is a term of type
\begin{equation*}
\isprop(A)\defeq\prd{x,y:A}\iscontr(x=y).
\end{equation*}
\end{defn}

We will often use either of the following characterizations of propositions.

\begin{lem}[Lemma 3.11.10 and Exercise 3.5 of \cite{hottbook}]\label{lem:prop_char}
For any type $A$ the following are equivalent:
\begin{enumerate}
\item $A$ is a proposition.
\item $A$ is \define{proof irrelevant} in the sense that $\prd{x,y:A}x=y$.
\item $A\to\iscontr(A)$. 
\end{enumerate}
\end{lem}

\begin{eg}\label{eg:prop_contr}
Any contractible type is a proposition. The empty type is an equivalence by a direct application of the induction principle of the empty type. Furthermore, any retract of a proposition is again a proposition. In particular, propositions are closed under equivalences.
\end{eg}

\begin{lem}\label{lem:id_fib}
Consider a function $f:A\to B$. Then the canonical map
\begin{equation*}
(s=t)\to \fib{\apfunc{f}}{\ct{\proj 2(s)}{\proj 2(t)^{-1}}}
\end{equation*}
is an equivalence for every $s,t:\fib{f}{b}$ and every $b:B$. 
\end{lem}

\begin{proof}
By \cref{thm:fib_equiv} it suffices to show that the type
\begin{equation*}
\sm{y:A}{q:f(y)=b}{p:\proj 1(s)=y} \ap{f}{p}=\ct{\proj 2(s)}{q^{-1}}
\end{equation*}
is contractible, which is immediate by two applications of \cref{thm:total_path}.
\end{proof}

\begin{prp}[Lemma 7.6.2 of \cite{hottbook}]\label{thm:prop_emb}
A map is an embedding if and only if its fibers are propositions.
\end{prp}

\begin{proof}
If $f$ is an embedding, then the fibers of $\apfunc{f}$ are contractible by \cref{thm:contr_equiv}. Thus it follows by \cref{lem:id_fib} that the fibers of $f$ are propositions.

Conversely, if the fibers of $f$ are propositions, then we have by \cref{lem:id_fib} an equivalence
\begin{equation*}
\eqv{((x,p)=(y,\refl{f(y)}))}{\fib{\apfunc{f}}{p}}
\end{equation*}
for any $p:f(x)=f(y)$, which shows that the fibers of $\apfunc{f}$ are contractible. Thus $f$ is an embedding by \cref{thm:contr_equiv}.
\end{proof}

\begin{defn}
A type family $B$ over $A$ is said to be a \define{subtype} of $A$ if for each $x:A$ the type $B(x)$ is a proposition.
\end{defn}

\begin{cor}\label{thm:subtype}
A type family $B$ over $A$ is a subtype if and only if the projection map
\begin{equation*}
\proj 1 : \Big(\sm{x:A}B(x)\Big)\to A
\end{equation*}
is an embedding.
\end{cor}

\begin{proof}
Immediate by \cref{eg:fib_proj}.
\end{proof}

\section{Function extensionality}
\begin{prp}[Theorem 4.9.5 of \cite{hottbook}]\label{thm:funext_wkfunext}
The following are equivalent:
\begin{enumerate}
%\item The principle of \define{homotopy induction}\index{homotopy induction}: for every $f:\prd{x:A}B(x)$ and every type family
%\begin{equation*}
%\Gamma,g:\prd{x:A}B(x), H:f\htpy g \vdash P(g,H)~\mathrm{type},
%\end{equation*}
%the map
%\begin{equation*}
%\Big(\prd{g:\prd{x:A}B(x)}{H:f\htpy g}P(g,H)\Big)\to P(f,\mathsf{htpy\usc{}refl}_f)
%\end{equation*}
%given by $s\mapsto s(f,\mathsf{htpy\usc{}refl}_f)$ has a section.
\item The \define{function extensionality principle}\index{function extensionality}: For every type family $B$ over $A$, and any two dependent functions $f,g:\prd{x:A}B(x)$, the canonical map\index{htpy_eq@{$\mathsf{htpy\usc{}eq}$}|textbf}
\begin{equation*}
\mathsf{htpy\usc{}eq}(f,g) : (\id{f}{g})\to (f\htpy g)
\end{equation*}
by path induction (sending $\refl{f}$ to $\lam{x}\refl{f(x)}$) is an equivalence. We will write $\mathsf{eq\usc{}htpy}$\index{eq_htpy@{$\mathsf{eq\usc{}htpy}$}} for its inverse.
\item The \define{weak function extensionality principle}\index{weak function extensionality} holds: For every type family $B$ over $A$ one has\index{contractible!weak function extensionality}
\begin{equation*}
\Big(\prd{x:A}\iscontr(B(x))\Big)\to\iscontr\Big(\prd{x:A}B(x)\Big).
\end{equation*}
\end{enumerate}
\end{prp}

From now on we will assume that function extensionality holds.

\begin{cor}[Theorem 7.1.9 of \cite{hottbook}]\label{thm:prop_pi}
For any type family $B$ over $A$ one has
\begin{equation*}
\Big(\prd{x:A}\isprop(B(x))\Big)\to \isprop\Big(\prd{x:A}B(x)\Big).
\end{equation*}
In particular, if $B$ is a proposition, then $A\to B$ is a proposition for any type $A$.
\end{cor}

\begin{thm}[Theorem 2.15.7 of \cite{hottbook}]\label{thm:choice}
Let $C(x,y)$ be a type in context $\Gamma,x:A,y:B(x)$. Then the map
\begin{equation*}
\varphi:\Big(\prd{x:A}\sm{y:B(x)}C(x,y)\Big)\to \Big(\sm{f:\prd{x:A}B(x)}\prd{x:A}C(x,f(x))\Big)
\end{equation*}
given by $\lam{h}(\lam{x}\proj 1(h(x)),\lam{x}\proj 2(h(x)))$ is an equivalence.
\end{thm}

\begin{cor}
For type $A$ and any type family $C$ over $B$, the map
\begin{equation*}
\Big(\sm{f:A\to B} \prd{x:A}C(f(x))\Big)\to\Big(A\to\sm{y:B}C(x)\Big)
\end{equation*}
given by $\lam{(f,g)}{x}(f(x),g(x))$ is an equivalence.
\end{cor}

\subsection{Composing with equivalences}

We show in this section that a map $f:A\to B$ is an equivalence if and only if for any type family $P$ over $B$, the precomposition map\marginpar{I do not find this\\ result in \cite{hottbook}}
\begin{equation*}
\blank\circ f: \Big(\prd{y:B}P(y)\Big)\to\Big(\prd{x:A}P(f(x))\Big)
\end{equation*}
is an equivalence. In the proof we use the notion of \emph{path-split} maps, which were introduced by Rijke, Shulman, and Spitters in \cite{RSS}.

\begin{defn}
We say that a map $f:A\to B$ is \define{path-split}\index{path-split|textbf} if $f$ has a section, and for each $x,y:A$ the map
\begin{equation*}
\apfunc{f}(x,y):(x=y)\to (f(x)=f(y))
\end{equation*}
also has a section. We write $\mathsf{path\usc{}split}(f)$\index{path_split(f)@{$\mathsf{path\usc{}split}(f)$}|textbf} for the type
\begin{equation*}
\mathsf{sec}(f)\times\prd{x,y:A}\mathsf{sec}(\apfunc{f}(x,y)).
\end{equation*}
\end{defn}

We will also use the notion of \emph{half-adjoint equivalences}, which were introduced in \cite{hottbook}.

\begin{defn}[Definition 4.2.1 of \cite{hottbook}]
We say that a map $f:A\to B$ is a \define{half-adjoint equivalence}\index{half-adjoint equivalence|textbf}, in the sense that there are
\begin{align*}
g & : B \to A\\
G & : f\circ g \htpy \idfunc[B] \\
H & : g\circ f \htpy \idfunc[A] \\
K & : G\cdot f \htpy f\cdot H.
\end{align*}
We write $\mathsf{half\usc{}adj}(f)$\index{half_adj(f)@{$\mathsf{half\usc{}adj}(f)$}|textbf} for the type of such quadruples $(g,G,H,K)$.
\end{defn}

\begin{prp}\label{ex:equiv_precomp}
For any map $f:A\to B$, the following are equivalent:
\begin{enumerate}
\item $f$ is an equivalence.
\item $f$ is path-split.
\item $f$ is a half-adjoint equivalence.
\item For any type family $P$ over $B$ the map
\begin{equation*}
\Big(\prd{y:B}P(y)\Big)\to\Big(\prd{x:A}P(f(x))\Big)
\end{equation*}
given by $s\mapsto s\circ f$ is an equivalence.
\item For any type $X$ the map
\begin{equation*}
(B\to X)\to (A\to X)
\end{equation*}
given by $g\mapsto g\circ f$ is an equivalence. 
\end{enumerate}
\end{prp}

\begin{proof}
To see that (i) implies (ii) we note that any equivalence has a section, and its action on paths is an equivalence by \cref{cor:emb_equiv} so again it has a section.

To show that (ii) implies (iii), assume that $f$ is path-split. Thus we have $(g,G):\mathsf{sec}(f)$, and the assumption that $\apfunc{f}:(x=y)\to (f(x)=f(y))$ has a section for every $x,y:A$ gives us a term of type
\begin{equation*}
\prd{x:A}\fib{\apfunc{f}}{G(f(x))}.
\end{equation*}
By \cref{thm:choice} this type is equivalent to
\begin{equation*}
\sm{H:\prd{x:A}g(f(x))=x}\prd{x:A}G(f(x))=\ap{f}{H(x)},
\end{equation*}
so we obtain $H:g\circ f\htpy \idfunc[A]$ and $K:G\cdot f\htpy f\cdot H$, showing that $f$ is a half-adjoint equivalence.

To show that (iii) implies (iv), suppose that $f$ comes equipped with $(g,G,H,K)$ witnessing that $f$ is a half-adjoint equivalence. Then we define the inverse of $\blank\circ f$ to be the map
\begin{equation*}
\varphi:\Big(\prd{x:A}P(f(x))\Big)\to\Big(\prd{y:B}P(y)\Big)
\end{equation*}
given by $s\mapsto \lam{y}\mathsf{tr}_P(G(y),sg(y))$. 

To see that $\varphi$ is a section of $\blank\circ f$, let $s:\prd{x:A}P(f(x))$. By function extensionality it suffices to construct a homotopy $\varphi(s)\circ f\htpy s$. In other words, we have to show that
\begin{equation*}
\mathsf{tr}_P(G(f(x)),s(g(f(x)))=s(x)
\end{equation*}
for any $x:A$. Now we use the additional homotopy $K$ from our assumption that $f$ is a half-adjoint equivalence. Since we have $K(x):G(f(x))=\ap{f}{H(x)}$ it suffices to show that
\begin{equation*}
\mathsf{tr}_P(\ap{f}{H(x)},sgf(x))=s(x).
\end{equation*}
A simple path-induction argument yields that
\begin{equation*}
\mathsf{tr}_P(\ap{f}{p})\htpy \mathsf{tr}_{P\circ f}(p)
\end{equation*}
for any path $p:x=y$ in $A$, so it suffices to construct an identification
\begin{equation*}
\mathsf{tr}_{P\circ f}(H(x),sgf(x))=s(x).
\end{equation*}
We have such an identification by $\apd{H(x)}{s}$.

To see that $\varphi$ is a retraction of $\blank\circ f$, let $s:\prd{y:B}P(y)$. By function extensionality it suffices to construct a homotopy $\varphi(s\circ f)\htpy s$. In other words, we have to show that
\begin{equation*}
\mathsf{tr}_P(G(y),sfg(y))=s(y)
\end{equation*}
for any $y:B$. We have such an identification by $\apd{G(y)}{s}$. This completes the proof that (iii) implies (iv).

Note that (v) is an immediate consequence of (iv), since we can just choose $P$ to be the constant family $X$.

It remains to show that (v) implies (i). Suppose that
\begin{equation*}
\blank\circ f:(B\to X)\to (A\to X)
\end{equation*}
is an equivalence for every type $X$. Then its fibers are contractible by \cref{thm:contr_equiv}. In particular, choosing $X\jdeq A$ we see that the fiber
\begin{equation*}
\fib{\blank\circ f}{\idfunc[A]}\jdeq \sm{h:B\to A}h\circ f=\idfunc[A]
\end{equation*}
is contractible. Thus we obtain a function $h:B\to A$ and a homotopy $H:h\circ f\htpy\idfunc[A]$ showing that $h$ is a retraction of $f$. We will show that $h$ is also a section of $f$. To see this, we use that the fiber
\begin{equation*}
\fib{\blank\circ f}{f}\jdeq \sm{i:B\to B} i\circ f=f
\end{equation*}
is contractible (choosing $X\jdeq B$). 
Of course we have $(\idfunc[B],\refl{f})$ in this fiber. However we claim that there also is an identification $p:(f\circ h)\circ f=f$, showing that $(f\circ h,p)$ is in this fiber, because
\begin{align*}
(f\circ h)\circ f & \jdeq f\circ (h\circ f) \\
& = f\circ \idfunc[A] \\
& \jdeq f
\end{align*}
Now we conclude by the contractibility of the fiber that there is an identification $(\idfunc[B],\refl{f})=(f\circ h,p)$. In particular we obtain that $\idfunc[B]=f\circ h$, showing that $h$ is a section of $f$.
\end{proof}

\subsection{Characterizing fiberwise transformations}

\begin{defn}
Consider two functions $f:A\to X$ and $g:B\to X$. We define the type 
\begin{equation*}
\mathrm{hom}_X(f,g)\defeq \sm{h:A\to B} f\htpy g\circ h.
\end{equation*}
\end{defn}

In other words, the type $\mathrm{hom}_X(f,g)$ is the type of functions $h:A\to B$ equipped with a homotopy witnessing that the triangle
\begin{equation*}
\begin{tikzcd}[column sep=tiny]
A \arrow[dr,swap,"f"] \arrow[rr,"h"] & & B \arrow[dl,"g"] \\
& X
\end{tikzcd}
\end{equation*}

\begin{prp}
Let $P$ and $Q$ be two type families over $X$, and write $\proj 1^P$ and $\proj 1^Q$ for their first projections, respectively. Then there is an equivalence
\begin{equation*}
\mathsf{tot\usc{}triangle}:\Big(\prd{x:X} P(x)\to Q(x)\Big)\to \mathrm{hom}_X(\proj 1^P,\proj 1^Q)
\end{equation*}
given by $\mathsf{tot\usc{}triangle}(f)\defeq (\total{f},\lam{(x,y)}\refl{x})$, is an equivalence.
\end{prp}

\begin{cor}\label{cor:fib_triangle}
For any two maps $f:A\to X$ and $g:B\to X$, the map
\begin{equation*}
\mathsf{fib\usc{}triangle} : \mathrm{hom}_X(f,g) \to \prd{x:X}\fib{f}{x}\to \fib{g}{x}
\end{equation*}
given by $\lam{x}{(a,p)}(h(a),\ct{H(a)^{-1}}{p})$ is an equivalence.
\end{cor}

\section{Homotopy pullbacks}
Suppose we are given a map $f:A\to B$, and type families $P$ over $A$, and $Q$ over $B$.
Then any fiberwise map
\begin{equation*}
g:\prd{x:A}P(x)\to Q(f(x))
\end{equation*}
gives rise to a commuting square
\begin{equation*}
\begin{tikzcd}[column sep=large]
\sm{x:A}P(x) \arrow[r,"{\total[f]{g}}"] \arrow[d,swap,"\proj 1"] & \sm{y:B}Q(y) \arrow[d,"\proj 1"] \\
A \arrow[r,swap,"f"] & B
\end{tikzcd}
\end{equation*}
where $\total[f]{g}$ is defined as $\lam{(x,p)}(f(x),g(x,y))$. 
We will show in \cref{thm:pb_fibequiv} that $g$ is a fiberwise equivalence\index{fiberwise equivalence} if and only if this square is a \emph{pullback square}\index{pullback square}. This generalization of \cref{thm:fib_equiv} is therefore abstracting away from the notion of fiberwise equivalence, and it serves as our motivating theorem to introduce pullbacks. The connection between pullbacks and fiberwise equivalences has an important role in the descent theorem\index{descent} in \cref{chap:descent}.

\subsection{Cartesian squares}

Recall that a square
\begin{equation*}
\begin{tikzcd}
C \arrow[r,"q"] \arrow[d,swap,"p"] & B \arrow[d,"g"] \\
A \arrow[r,swap,"g"] & X
\end{tikzcd}
\end{equation*}
is said to \define{commute}\index{commuting square|textbf} if there is a homotopy $H:f\circ p\htpy g\circ q$. 
The pullback property is a \emph{universal property} of the upper left corner of a commuting square (in our case $C$), characterizing the maps \emph{into} it.

To describe the universal property of pullbacks we first need to have a closer look at the \emph{anatomy} of commuting squares.

\begin{defn}\label{defn:cospan}
A commuting square
\begin{equation*}
\begin{tikzcd}
C \arrow[r,"q"] \arrow[d,swap,"p"] & B \arrow[d,"g"] \\
A \arrow[r,swap,"f"] & X
\end{tikzcd}
\end{equation*}
with $H:f\circ p\htpy g\circ q$ can be dissected into three parts, consisting of a \emph{cospan}, a type, and a \emph{cone}, where
\begin{enumerate}
\item A \define{cospan}\index{cospan|textbf} consists of three types $A$, $X$, and $B$, and maps $f:A\to X$ and $g:B\to X$.
\item Given a type $C$, a \define{cone}\index{cone!on a cospan|textbf} on the cospan $A \stackrel{f}{\rightarrow} X \stackrel{g}{\leftarrow} B$ with \define{vertex} $C$\index{vertex!of a cone|textbf} consists of maps $p:C\to A$, $q:C\to B$ and a homotopy $H:f\circ p\htpy g\circ q$. We write\index{cone(C)@{$\mathsf{cone}(\blank)$}|textbf}
\begin{equation*}
\mathsf{cone}(C)\defeq \sm{p:C\to A}{q:C\to B}f\circ p\htpy g\circ q
\end{equation*}
for the type of cones with vertex $C$.
\end{enumerate}
\end{defn}

Given a cone with vertex $C$ on a span $A\stackrel{f}{\rightarrow} X \stackrel{g}{\leftarrow} B$ and a map $h:C'\to C$, we construct a new cone with vertex $C'$ in the following definition.

\begin{defn}
For any cone $(p,q,H)$ with vertex $C$ and any type $C'$, we define a map\index{cone map@{$\mathsf{cone\usc{}map}$}|textbf}
\begin{equation*}
\mathsf{cone\usc{}map}(p,q,H):(C'\to C)\to\mathsf{cone}(C')
\end{equation*}
by $h\mapsto (p\circ h,q\circ h,H\circ h)$. 
\end{defn}

\begin{defn}
We say that a commuting square
\begin{equation*}
\begin{tikzcd}
C \arrow[r,"q"] \arrow[d,swap,"p"] & B \arrow[d,"g"] \\
A \arrow[r,swap,"f"] & X
\end{tikzcd}
\end{equation*}
with $H:f\circ p\htpy g\circ q$ is a \define{pullback square}\index{pullback square|textbf}, or that it is \define{cartesian}\index{cartesian square|textbf}, if it satisfies the \define{universal property} of pullbacks\index{universal property!of pullbacks}, which asserts that the map
\begin{equation*}
\mathsf{cone\usc{}map}(p,q,H):(C'\to C)\to\mathsf{cone}(C')
\end{equation*}
is an equivalence for every type $C'$. 
\end{defn}

We often indicate the universal property with a diagram as follows:
\begin{equation*}
\begin{tikzcd}
C' \arrow[drr,bend left=15,"{q'}"] \arrow[dr,densely dotted,"h"] \arrow[ddr,bend right=15,swap,"{p'}"] \\
& C \arrow[r,"q"] \arrow[d,swap,"p"] & B \arrow[d,"g"] \\
& A \arrow[r,swap,"f"] & X
\end{tikzcd}
\end{equation*}
since the universal property states that for every cone $(p',q',H')$ with vertex $C'$, the type of pairs $(h,\alpha)$ consisting of $h:C'\to C$ equipped with $\alpha:\mathsf{cone\usc{}map}((p,q,H),h)=(p',q',H')$ is contractible by \cref{thm:contr_equiv}.

In order to see what goes on in the universal property of pullbacks, we need to first characterize the identity type of $\mathsf{cone}(C)$, for any type $C$.

\begin{lem}\label{lem:id_cone}%
\index{identity type!of cone@{of $\mathsf{cone}(C)$}|textit}%
Let $(p,q,H)$ and $(p',q',H')$ be cones on a cospan $f:A\rightarrow X \leftarrow B:g$, both with vertex $C$. Then the type $(p,q,H)=(p',q',H')$ is equivalent to the type of triples $(K,L,M)$ consisting of
\begin{align*}
K & : p \htpy p' \\
L & : q \htpy q' \\
M & : \ct{H}{(g\cdot L)} \htpy \ct{(f\cdot K)}{H'}
\end{align*}
\end{lem}

\begin{rmk}
The homotopy $M$ witnesses that the square
\begin{equation*}
\begin{tikzcd}
f\circ p \arrow[r,"f\cdot K"] \arrow[d,swap,"H"] & f\circ p' \arrow[d,"{H'}"] \\
g\circ q \arrow[r,swap,"g\cdot L"] & g\circ q'
\end{tikzcd}
\end{equation*}
of homotopies commutes. Therefore $M$ is a homotopy of homotopies, and for each $z:C$ the identification $M(z)$ witnesses that the square of identifications
\begin{equation*}
\begin{tikzcd}[column sep=huge]
f(p(z)) \arrow[r,equals,"\ap{f}{K(z)}"] \arrow[d,equals,swap,"H(z)"] & f(p'(z)) \arrow[d,equals,"{H'(z)}"] \\
g(q(z)) \arrow[r,equals,swap,"\ap{g}{L(z)}"] & g(q'(z))
\end{tikzcd}
\end{equation*}
commutes. 
\end{rmk}

\begin{proof}[Proof of \cref{lem:id_cone}]
By the fundamental theorem of identity types (\cref{thm:id_fundamental}) and associativity of $\Sigma$-types (\cref{ex:sigma_assoc}) it suffices to show that the type
\begin{equation*}
\sm{p':C\to A}{q':C\to B}{H':f\circ p'\htpy g\circ q'}{K:p\htpy p'}{L:q\htpy q'} \ct{H}{(g\cdot L)} \htpy \ct{(f\cdot K)}{H'}
\end{equation*}
is contractible. Now we apply \cref{ex:sigma_swap} repeatedly to see that this type is equivalent to the type
\begin{equation*}
\sm{p':C\to A}{K: p\htpy p'}{q':C\to B}{L: q\htpy q'}{H':f\circ p'\htpy g\circ q'} \ct{H}{(g\cdot L)} \htpy \ct{(f\cdot K)}{H'}.
\end{equation*}
The types $\sm{p':C\to A} p\htpy p'$ and $\sm{q':C\to B} q\htpy q'$ are contractible by function extensionality, and  we have
\begin{samepage}
\begin{align*}
(p,\mathsf{htpy\usc{}refl}_p) & : \sm{p':C'\to A} p\htpy p' \\
(q,\mathsf{htpy\usc{}refl}_q) & : \sm{q':C'\to B} q\htpy q'.
\end{align*}%
\end{samepage}%
Thus we apply \cref{ex:contr_in_sigma} to see that the type of tuples $(p',K,q',L,H',M)$ is equivalent to the type
\begin{equation*}
\sm{H':f\circ p'\htpy g\circ q'} \ct{H}{\mathsf{htpy\usc{}refl}_{g\circ q}}\htpy \ct{\mathsf{htpy\usc{}refl}_{f\circ p}}{H'}.
\end{equation*}
Of course, the type $\ct{H}{\mathsf{htpy\usc{}refl}_{g\circ q}}\htpy \ct{\mathsf{htpy\usc{}refl}_{f\circ p}}{H'}$ is equivalent to the type $H\htpy H'$, and $\sm{H':f\circ p\htpy g\circ q} H\htpy H'$ is contractible.
\end{proof}

As a corollary we obtain the following characterization of the universal property of pullbacks.

\begin{thm}\label{thm:pullback_up}
Consider a commuting square
\begin{equation*}
\begin{tikzcd}
C \arrow[r,"q"] \arrow[d,swap,"p"] & B \arrow[d,"g"] \\
A \arrow[r,swap,"f"] & X
\end{tikzcd}
\end{equation*}
with $H:f\circ p\htpy g\circ q$
Then the following are equivalent:\index{universal property!of pullbacks (characterization)|textit}
\begin{enumerate}
\item The square is a pullback square.
\item For every type $C'$ and every cone $(p',q',H')$ with vertex $C'$, the type of quadruples $(h,K,L,M)$ consisting of
\begin{align*}
h & : C'\to C \\
K & : p\circ h \htpy p' \\
L & : q\circ h \htpy q' \\
M & : \ct{(H\cdot h)}{(g\cdot L)} \htpy \ct{(f\cdot K)}{H'}
\end{align*}
is contractible.
\end{enumerate}
\end{thm}

\begin{rmk}
The homotopy $M$ in \cref{thm:pullback_up} witnesses that the square
\begin{equation*}
\begin{tikzcd}
f\circ p\circ h \arrow[r,"f\cdot K"] \arrow[d,swap,"H\cdot h"] & f\circ p' \arrow[d,"{H'}"] \\
g\circ q\circ h \arrow[r,swap,"g\cdot L"] & g\circ q'
\end{tikzcd}
\end{equation*}
of homotopies commutes.
\end{rmk}

\subsection{The unique existence of pullbacks}

\begin{defn}
Let $f:A\to X$ and $B\to X$ be maps. Then we define
\begin{align*}
A\times_X B & \defeq \sm{x:A}{y:B}f(x)=g(y) \\
\pi_1 & \defeq \proj 1 & & : A\times_X B\to A \\
\pi_2 & \defeq \proj 1\circ\proj 2 & & : A\times_X B\to B\\
\pi_3 & \defeq \proj 2\circ\proj 2 & & : f\circ \pi_1 \htpy g\circ\pi_2.
\end{align*}
The type $A\times_X B$ is called the \define{canonical pullback}\index{canonical pullback|textbf} of $f$ and $g$.
\end{defn}

Note that $A\times_X B$ depends on $f$ and $g$, although this dependency is not visible in the notation.

\begin{thm}
Given maps $f:A\to X$ and $g:B\to X$, the commuting square\index{canonical pullback|textit}
\begin{equation*}
\begin{tikzcd}
A\times_X B \arrow[r,"\pi_2"] \arrow[d,swap,"\pi_1"] & B \arrow[d,"g"] \\
A \arrow[r,swap,"f"] & X,
\end{tikzcd}
\end{equation*}
is a pullback square.
\end{thm}

\begin{proof}
Let $C$ be a type. Our goal is to show that the map
\begin{equation*}
\mathsf{cone\usc{}map}(\pi_1,\pi_2,\pi_3): (C\to A\times_X B)\to \mathsf{cone}(C)
\end{equation*}
is an equivalence. 
By double application of \cref{thm:choice} we obtain equivalences
\begin{align*}
(C\to A\times_X B) & \jdeq C\to \sm{x:A}{y:B}f(x)=g(y) \\
& \eqvsym \sm{p:C\to A}\prd{z:C}\sm{y:B} f(p(z))= y \\
& \eqvsym \sm{p:C\to A}{q:C\to B}\prd{z:C} f(p(z))= g(q(z)) \\
& \jdeq \mathsf{cone}(C)
\end{align*}
The composite of these equivalences is the map
\begin{equation*}
\lam{f}(\lam{z}\proj 1(f(z)),\lam{z} \proj 1(\proj 2(f(z))),\lam{z}\proj 2(\proj 2(f(z)))),
\end{equation*}
which is \emph{exactly} the map $\mathsf{cone\usc{}map}(\pi_1,\pi_2,\pi_3)$, and since it is a composite of equivalences it follows that it is itself an equivalence.
\end{proof}

In the following lemma we establish the uniqueness of pullbacks up to equivalence via a \emph{3-for-2 property} for pullbacks.

\begin{lem}\label{lem:pb_3for2}\index{pullback!3-for-2 property|textit}\index{3-for-2 property!of pullbacks|textit}%
Consider the squares
\begin{equation*}
\begin{tikzcd}
C \arrow[r,"q"] \arrow[d,swap,"p"] & B \arrow[d,"g"] & {C'} \arrow[r,"{q'}"] \arrow[d,swap,"{p'}"] & B \arrow[d,"g"] \\
A \arrow[r,swap,"f"] & X & A \arrow[r,swap,"f"] & X
\end{tikzcd}
\end{equation*}
with homotopies $H:f\circ p \htpy g\circ q$ and $H':f\circ p'\htpy g\circ q'$.
Furthermore, suppose we have a map $h:C'\to C$ equipped with
\begin{align*}
K & : p\circ h \htpy p' \\
L & : q\circ h \htpy q' \\
M & : \ct{(H\cdot h)}{(g\cdot L)} \htpy \ct{(f\cdot K)}{H'}.
\end{align*}
If any two of the following three properties hold, so does the third:
\begin{samepage}%
\begin{enumerate}
\item $C$ is a pullback.
\item $C'$ is a pullback.
\item $h$ is an equivalence.
\end{enumerate}%
\end{samepage}%
\end{lem}

\begin{proof}
By the characterization of the identity type of $\mathsf{cone}(C')$ given in \cref{lem:id_cone} we obtain an identification
\begin{equation*}
\mathsf{cone\usc{}map}((p,q,H),h)=(p',q',H')
\end{equation*}
from the triple $(K,L,M)$. 
Let $D$ be a type, and let $k:D\to C'$ be a map. We observe that
\begin{align*}
\mathsf{cone\usc{}map}((p,q,H),(h\circ k)) & \jdeq (p\circ (h\circ k),q\circ (h\circ k),H\circ (h\circ k)) \\
& \jdeq ((p\circ h)\circ k,(q\circ h)\circ k, (H\circ h)\circ k) \\
& \jdeq \mathsf{cone\usc{}map}(\mathsf{cone\usc{}map}((p,q,H),h),k) \\
& = \mathsf{cone\usc{}map}((p',q',H'),k).
\end{align*}
Thus we see that the triangle 
\begin{equation*}
\begin{tikzcd}[column sep=-1em]
(D\to C') \arrow[rr,"{h\circ \blank}"] \arrow[dr,swap,"{\mathsf{cone\usc{}map}(p',q',H')}"] & & (D\to C) \arrow[dl,"{\mathsf{cone\usc{}map}(p,q,H)}"] \\
& \mathsf{cone}(D)
\end{tikzcd}
\end{equation*}
commutes. Therefore it follows from the 3-for-2 property of equivalences that if any two of the following properties hold, then so does the third:
\begin{enumerate}
\item The map $\mathsf{cone\usc{}map}(p,q,H):(D\to C)\to \mathsf{cone}(D)$ is an equivalence,
\item The map $\mathsf{cone\usc{}map}(p',q',H'):(D\to C')\to \mathsf{cone}(D)$ is an equivalence,
\item The map $h\circ\blank : (D\to C')\to (D\to C)$ is an equivalence.
\end{enumerate}
Thus the 3-for-2 property for pullbacks follows from the fact that $h$ is an equivalence if and only if $h\circ\blank : (D\to C')\to (D\to C)$ is an equivalence for any type $D$, which was established in \cref{lem:postcomp_equiv}.
\end{proof}

Pullbacks are not only unique in the sense that any two pullbacks of the same cospan are equivalent, they are \emph{uniquely unique}\index{uniquely uniqueness!of pullbacks} in the sense that the type of quadruples $(h,K,L,M)$ as in \cref{lem:pb_3for2} is contractible.

\begin{cor}
Suppose both commuting squares
\begin{equation*}
\begin{tikzcd}
C \arrow[r,"q"] \arrow[d,swap,"p"] & B \arrow[d,"g"] & {C'} \arrow[r,"{q'}"] \arrow[d,swap,"{p'}"] & B \arrow[d,"g"] \\
A \arrow[r,swap,"f"] & X & A \arrow[r,swap,"f"] & X
\end{tikzcd}
\end{equation*}
with homotopies $H:f\circ p \htpy g\circ q$ and $H':f\circ p'\htpy g\circ q'$ are pullback squares.
Then the type of quadruples $(e,K,L,M)$ consisting of an equivalence $e:\eqv{C'}{C}$ equipped with
\begin{align*}
K & : p\circ e \htpy p' \\
L & : q\circ e \htpy q' \\
M & : \ct{(g\cdot L)}{(H\cdot e)} \htpy \ct{(f\cdot K)}{H'}.
\end{align*}
is contractible.
\end{cor}

\begin{proof}
We have seen that the type of quadruples $(h,K,L,M)$ is equivalent to the fiber of $\mathsf{cone\usc{}map}(p,q,H)$ at $(p',q',H')$. By \cref{lem:pb_3for2} it follows that $h$ is an equivalence. Since $\isequiv(h)$ is a proposition (and hence contractible as soon as it is inhabited) it follows that the type of quadruples $(e,K,L,M)$ is contractible. 
\end{proof}

\begin{defn}
Given a commuting square
\begin{equation*}
\begin{tikzcd}
C \arrow[r,"q"] \arrow[d,"p"] & B \arrow[d,"g"] \\
A \arrow[r,swap,"f"] & X
\end{tikzcd}
\end{equation*}
with $H:f\circ p \htpy g \circ q$, we define the \define{gap map}\index{gap map|textbf}\index{pullback!gap map|textbf}
\begin{equation*}
\mathsf{gap}(p,q,H):C \to A\times_X B
\end{equation*}
by $\lam{z}(p(z),q(z),H(z))$. Furthermore, we will write\index{is_pullback@{$\mathsf{is\usc{}pullback}$}|textbf}
\begin{equation*}
\mathsf{is\usc{}pullback}(f,g,H)\defeq \isequiv(\mathsf{gap}(p,q,H)).
\end{equation*}
\end{defn}

\begin{thm}\label{thm:is_pullback}
Consider a commuting square
\begin{equation*}
\begin{tikzcd}
C \arrow[r,"q"] \arrow[d,"p"] & B \arrow[d,"g"] \\
A \arrow[r,swap,"f"] & X
\end{tikzcd}
\end{equation*}
with $H:f\circ p \htpy g \circ q$. The following are equivalent:
\begin{enumerate}
\item The square is a pullback square
\item There is a term of type
\begin{equation*}
\mathsf{is\usc{}pullback}(p,q,H)\defeq \isequiv(\mathsf{gap}(p,q,H)).
\end{equation*}
\end{enumerate}
\end{thm}

\begin{proof}
Note that there are homotopies
\begin{align*}
K & : \pi_1\circ \mathsf{gap}(p,q,H) \htpy p \\
L & : \pi_2\circ \mathsf{gap}(p,q,H) \htpy q \\
M & : \ct{(\pi_3\cdot \mathsf{gap}(p,q,H))}{(g\cdot L)} \htpy \ct{(f\cdot K)}{H}.
\end{align*}
given by 
\begin{align*}
K & \defeq \lam{z}\refl{p(z)} \\
L & \defeq \lam{z}\refl{q(z)} \\
M & \defeq \lam{z}\ct{\mathsf{right\usc{}unit}(H(z))}{\mathsf{left\usc{}unit}(H(z))^{-1}}.
\end{align*}
Therefore the claim follows by \cref{lem:pb_3for2}.
\end{proof}

\subsection{Fiberwise equivalences}

\begin{lem}\label{lem:pb_subst}
Let $f:A\to B$, and let $Q$ be a type family over $B$. Then the square
\begin{equation*}
\begin{tikzcd}[column sep=6em]
\sm{x:A}Q(f(x)) \arrow[r,"{\lam{(x,q)}(f(x),q)}"] \arrow[d,swap,"\proj 1"] & \sm{y:B}Q(b) \arrow[d,"\proj 1"] \\
A \arrow[r,swap,"f"] & B
\end{tikzcd}
\end{equation*}
commutes by $H\defeq \lam{(x,q)}\refl{f(x)}$. This is a pullback square.\index{substitution!as pullback|textit}
\end{lem}

\begin{proof}
By \cref{thm:is_pullback} it suffices to show that the gap map is an equivalence. The gap map is homotopic to the function
\begin{equation*}
\lam{(x,q)}(x,(f(x),q),\refl{f(x)}).
\end{equation*}
The inverse of this map is given by $\lam{(x,((y,q),p))}(x,\mathsf{tr}_Q(p^{-1},q))$, and it is straightforward to see that these maps are indeed mutual inverses.
\end{proof}

\begin{thm}\label{thm:pb_fibequiv}
Let $f:A\to B$, and let $g:\prd{a:A}P(a)\to Q(f(a))$ be a fiberwise transformation\index{fiberwise transformation|textit}. The following are equivalent:
\begin{enumerate}
\item The commuting square
\begin{equation*}
\begin{tikzcd}[column sep=large]
\sm{a:A}P(a) \arrow[r,"{\total[f]{g}}"] \arrow[d,->>] & \sm{b:B}Q(b) \arrow[d,->>] \\
A \arrow[r,swap,"f"] & B
\end{tikzcd}
\end{equation*}
is a pullback square.
\item $g$ is a fiberwise equivalence.\index{fiberwise equivalence|textit}
\end{enumerate}
\end{thm}

\begin{proof}
The gap map is homotopic to the composite
\begin{equation*}
\begin{tikzcd}[column sep=large]
\sm{x:A}P(x) \arrow[r,"\total{g}"] & \sm{x:A}Q(f(x)) \arrow[r,"{\mathsf{gap}'}"] & A \times_B \Big(\sm{y:B}Q(y)\Big)
\end{tikzcd}
\end{equation*}
where $\mathsf{gap}'$ is the gap map for the square in \cref{lem:pb_subst}. Since $\mathsf{gap}'$ is an equivalence, it follows by \cref{thm:fib_equiv} that the gap map is an equivalence if and only if $g$ is a fiberwise equivalence.
\end{proof}

\begin{lem}
Consider a commuting square
\begin{equation*}
\begin{tikzcd}
C \arrow[r,"q"] \arrow[d,swap,"p"] & B \arrow[d,"g"] \\
A \arrow[r,swap,"f"] & X
\end{tikzcd}
\end{equation*}
with $H:f\circ p\htpy g\circ q$, and consider the fiberwise transformation
\begin{equation*}
\fibf{(f,q,H)} : \prd{a:A} \fib{p}{a}\to \fib{g}{f(a)}
\end{equation*}
given by $\lam{a}{(c,u)}(q(c),\ct{H(c)^{-1}}{\ap{f}{u}})$. Then there is an equivalence
\begin{equation*}
\eqv{\fib{\mathsf{gap}(p,q,H)}{(a,b,\alpha)}}{\fib{\fibf{(f,q,H)}(a)}{(b,\alpha^{-1})}}
\end{equation*}
\end{lem}

\begin{proof}
To obtain an equivalence of the desired type we simply concatenate known equivalences:
\begin{align*}
\fib{h}{(a,b,\alpha)} & \jdeq \sm{z:C} (p(z),q(z),H(z))=(a,b,\alpha) \\
& \eqvsym \sm{z:C}{u:p(z)=a}{v:q(z)=b}\ct{H(z)}{\ap{g}{v}}=\ct{\ap{f}{u}}{\alpha} \\
& \eqvsym \sm{(z,u):\fib{p}{a}}{v:q(z)=b} \ct{H(z)^{-1}}{\ap{f}{u}}=\ct{\ap{g}{v}}{\alpha^{-1}} \\
& \eqvsym \fib{\varphi(a)}{(b,\alpha^{-1})}\qedhere
\end{align*}
\end{proof}

\begin{cor}\label{cor:pb_fibequiv}
Consider a commuting square
\begin{equation*}
\begin{tikzcd}
C \arrow[r,"q"] \arrow[d,swap,"p"] & B \arrow[d,"g"] \\
A \arrow[r,swap,"f"] & X
\end{tikzcd}
\end{equation*}
with $H:f\circ p\htpy g\circ q$. The following are equivalent:
\begin{enumerate}
\item The square is a pullback square.\index{pullback square!characterized by fiberwise equivalence|textit}
\item The induced map on fibers
\begin{equation*}
\lam{x}{(z,\alpha)}(q(z),\ct{H(z)^{-1}}{\ap{f}{\alpha}}):\prd{x:A}\fib{p}{x}\to \fib{g}{f(x)}
\end{equation*}
is a fiberwise equivalence.
\end{enumerate}
\end{cor}

\begin{cor}\label{cor:pb_trunc}
Consider a pullback square
\begin{equation*}
\begin{tikzcd}
C \arrow[r,"q"] \arrow[d,swap,"p"] & B \arrow[d,"g"] \\
A \arrow[r,swap,"f"] & X.
\end{tikzcd}
\end{equation*}
If $g$ is a $k$-truncated map, then so is $p$. In particular, if $g$ is an embedding then so is $p$.\index{truncated!map!pullbacks of truncated maps|textit}\index{embedding!pullbacks of embeddings|textit}
\end{cor}

\begin{proof}
Since the square is assumed to be a pullback square, it follows from \cref{cor:pb_fibequiv} that for each $x:A$, the fiber $\fib{p}{x}$ is equivalent to the fiber $\fib{g}{f(x)}$, which is $k$-truncated. Since $k$-truncated types are closed under equivalences by \cref{thm:ktype_eqv}, it follows that $p$ is a $k$-truncated map.
\end{proof}

\begin{cor}\label{cor:pb_equiv}
Consider a commuting square
\begin{equation*}
\begin{tikzcd}
C \arrow[r,"q"] \arrow[d,swap,"p"] & B \arrow[d,"g"] \\
A \arrow[r,swap,"f"] & X.
\end{tikzcd}
\end{equation*}
and suppose that $g$ is an equivalence. Then the following are equivalent:
\begin{enumerate}
\item The square is a pullback square.
\item The map $p:C\to A$ is an equivalence.\index{equivalence!pullback of|textit}
\end{enumerate}
\end{cor}

\begin{proof}
If the square is a pullback square, then by \cref{thm:pb_fibequiv} the fibers of $p$ are equivalent to the fibers of $g$, which are contractible by \cref{thm:contr_equiv}. Thus it follows that $p$ is a contractible map, and hence that $p$ is an equivalence.

If $p$ is an equivalence, then by \cref{thm:contr_equiv} both $\fib{p}{x}$ and $\fib{g}{f(x)}$ are contractible for any $x:X$. It follows by \cref{ex:contr_equiv} that the induced map $\fib{p}{x}\to\fib{g}{f(x)}$ is an equivalence. Thus we apply \cref{cor:pb_fibequiv} to conclude that the square is a pullback.
\end{proof}

\begin{thm}\label{thm:pb_fibequiv_complete}
Consider a diagram of the form
\begin{equation*}
\begin{tikzcd}
A \arrow[d,swap,"f"] & B \arrow[d,"g"] \\
X \arrow[r,swap,"h"] & Y.
\end{tikzcd}
\end{equation*}
Then the type of triples $(i,H,p)$ consisting of a map $i:A\to B$, a homotopy $H:h\circ f\htpy g\circ i$, and a term $p$ witnessing that the square
\begin{equation*}
\begin{tikzcd}
A \arrow[d,swap,"f"] \arrow[r,"i"] & B \arrow[d,"g"] \\
X \arrow[r,swap,"h"] & Y.
\end{tikzcd}
\end{equation*}
is a pullback square, is equivalent to the type of fiberwise equivalences
\begin{equation*}
\prd{x:X}\eqv{\fib{f}{x}}{\fib{g}{h(x)}}.
\end{equation*}
\end{thm}

\begin{cor}\label{cor:pb_fibequiv_complete}
Let $h:X\to Y$ be a map, and let $P$ and $Q$ be families over $X$ and $Y$, respectively.
Then the type of triples $(i,H,p)$ consisting of a map 
\begin{equation*}
i:\Big(\sm{x:X}P(x)\Big)\to \Big(\sm{y:Y}Q(y)\Big),
\end{equation*}
a homotopy $H:h\circ \proj 1\htpy \proj 1\circ i$, and a term $p$ witnessing that the square
\begin{equation*}
\begin{tikzcd}
\sm{x:X}P(x) \arrow[d,swap,"\proj 1"] \arrow[r,"i"] & \sm{y:Y}Q(y) \arrow[d,"\proj 1"] \\
X \arrow[r,swap,"h"] & Y.
\end{tikzcd}
\end{equation*}
is a pullback square, is equivalent to the type of fiberwise equivalences
\begin{equation*}
\prd{x:X}\eqv{P(x)}{Q(h(x))}.
\end{equation*}
\end{cor}

\subsection{Iterated pullbacks}

The following theorem is also called the \define{pasting property} of pullbacks.\index{pasting property!of pullbacks|textit}

\begin{thm}\label{thm:pb_pasting}
Consider a commuting diagram of the form
\begin{equation*}
\begin{tikzcd}
A \arrow[r,"k"] \arrow[d,swap,"f"] & B \arrow[r,"l"] \arrow[d,"g"] & C \arrow[d,"h"] \\
X \arrow[r,swap,"i"] & Y \arrow[r,swap,"j"] & Z
\end{tikzcd}
\end{equation*}
with homotopies $H:i\circ f\htpy g\circ k$ and $K:j\circ g\htpy h\circ l$, and the homotopy
\begin{equation*}
\ct{(j\cdot H)}{(K\cdot k)}:j\circ i\circ f\htpy h\circ l\circ k
\end{equation*}
witnessing that the outer rectangle commutes. Furthermore, suppose that the square on the right is a pullback square. Then the following are equivalent:
\begin{samepage}%
\begin{enumerate}
\item The square on the left is a pullback square.
\item The outer rectangle is a pullback square.
\end{enumerate}%
\end{samepage}%
\end{thm}

\begin{proof}
The commutativity of the two squares induces fiberwise transformations
\begin{align*}
& \prd{x:X}\fib{f}{x}\to \fib{g}{i(x)} \\
& \prd{y:Y}\fib{g}{y}\to \fib{h}{j(y)}.
\end{align*}
By the assumption that the square on the right is a pullback square, it follows from \cref{cor:pb_fibequiv} that the fiberwise transformation
\begin{equation*}
\prd{y:Y}\fib{g}{y}\to\fib{h}{j(y)}
\end{equation*}
is a fiberwise equivalence. Therefore it follows from 3-for-2 property of equivalences that the fiberwise transformation
\begin{equation*}
\prd{x:X}\fib{f}{x}\to\fib{g}{i(x)}
\end{equation*}
is a fiberwise equivalence if and only if the fiberwise transformation
\begin{equation*}
\prd{x:X}\fib{f}{x}\to\fib{h}{j(i(x))}
\end{equation*}
is a fiberwise equivalence. Now the claim follows from one more application of \cref{cor:pb_fibequiv}.
\end{proof}

\subsection{Fibers of squares}

\begin{thm}
Consider a commuting square
\begin{equation*}
\begin{tikzcd}
A \arrow[d,swap,"f"] \arrow[r,"i"] & B \arrow[d,"g"] \\
X \arrow[r,swap,"j"] & Y
\end{tikzcd}
\end{equation*}
with homotopy $H:j\circ f\htpy g\circ i$, and let $h:A \to X\times_Y B$ be the gap map.
Then the commuting square
\begin{equation*}
\begin{tikzcd}
\fib{h}{(x,b,p)} \arrow[d] \arrow[r] & \unit \arrow[d] \\
\fib{f}{x} \arrow[r] & \fib{g}{j(x)}
\end{tikzcd}
\end{equation*}
is a pullback square. We call $\fib{h}{(x,b,p)}$ the \define{fiber} of the square.
\end{thm}

\section{The univalence axiom}

The univalence axiom characterizes the identity type of the universe. Roughly speaking, it asserts that equivalent types are equal. It is considered to be an \emph{extensionality principle}\index{extensionality principle!types} for types. In the following theorem we introduce the univalence axiom and give two more equivalent ways of stating this.

\begin{thm}\label{thm:univalence}
The following are equivalent:
\begin{enumerate}
\item The \define{univalence axiom}\index{univalence axiom|textbf}: for any $A:\UU$ the map\index{equiv_eq@{$\equiveq$}|textbf}
\begin{equation*}
\equiveq\defeq \ind{A=}(\idfunc[A]) : \prd{B:\UU} (\id{A}{B})\to(\eqv{A}{B}).
\end{equation*}
is a fiberwise equivalence.\index{identity type!universe} If this is the case, we write
$\eqequiv$\index{eq equiv@{$\eqequiv$}|textbf}
for the inverse of $\equiveq$.
\item The type
\begin{equation*}
\sm{B:\UU}\eqv{A}{B}
\end{equation*}
is contractible for each $A:\UU$.
\item The principle of \define{equivalence induction}\index{equivalence induction}\index{induction principle!for equivalences}: for every $A:\UU$ and for every type family
\begin{equation*}
P:\prd{B:\UU} (\eqv{A}{B})\to \type,
\end{equation*}
the map
\begin{equation*}
\Big(\prd{B:\UU}{e:\eqv{A}{B}}P(B,e)\Big)\to P(A,\idfunc[A])
\end{equation*}
given by $f\mapsto f(A,\idfunc[A])$ has a section.\qedhere
\end{enumerate}
\end{thm}

It is a trivial observation, but nevertheless of fundamental importance, that by the univalence axiom the identity types of $\UU$ are equivalent to types in $\UU$, because it provides an equivalence $\eqv{(A=B)}{(\eqv{A}{B})}$, and the type $\eqv{A}{B}$ is in $\UU$ for any $A,B:\UU$. Since the identity types of $\UU$ are equivalent to types in $\UU$, we also say that the universe is \emph{locally small}.

\begin{defn}
\begin{enumerate}
\item A type $A$ is said to be \define{essentially small}\index{essentially small!type|textbf} if there is a type $X:\UU$ and an equivalence $\eqv{A}{X}$. We write\index{ess_small(A)@{$\esssmall(A)$}|textbf}
\begin{equation*}
\esssmall(A)\defeq\sm{X:\UU}\eqv{A}{X}.
\end{equation*}
\item A map $f:A\to B$ is said to be \define{essentially small}\index{essentially small!map|textbf} if for each $b:B$ the fiber $\fib{f}{b}$ is essentially small.
We write\index{ess_small(f)@{$\esssmall(f)$}|textbf}
\begin{equation*}
\esssmall(f)\defeq\prd{b:B}\esssmall(\fib{f}{b}).
\end{equation*}
\item A type $A$ is said to be \define{locally small}\index{locally small!type} if for every $x,y:A$ the identity type $x=y$ is essentially small.
We write\index{loc_small(A)@{$\locsmall(A)$}|textbf}
\begin{equation*}
\locsmall(A)\defeq \prd{x,y:A}\esssmall(x=y).
\end{equation*}
\end{enumerate}
\end{defn}

\begin{lem}\label{lem:isprop_ess_small}
The type $\esssmall(A)$ is a proposition for any type $A$.\index{essentially small!is a proposition|textit}
\end{lem}

\begin{proof}
Let $X$ be a type. Our goal is to show that the type
\begin{equation*}
\sm{Y:\UU}\eqv{X}{Y}
\end{equation*}
is a proposition. Suppose there is a type $X':\UU$ and an equivalence $e:\eqv{X}{X'}$, then the map
\begin{equation*}
(\eqv{X}{Y})\to (\eqv{X'}{Y})
\end{equation*}
given by precomposing with $e^{-1}$ is an equivalence. This induces an equivalence on total spaces
\begin{equation*}
\eqv{\Big(\sm{Y:\UU}\eqv{X}{Y}\Big)}{\Big(\sm{Y:\UU}\eqv{X'}{Y}\Big)}
\end{equation*}
However, the codomain of this equivalence is contractible by \cref{thm:univalence}. Thus it follows by \cref{cor:contr_prop} that the asserted type is a proposition.
\end{proof}

\begin{cor}
For each function $f:A\to B$, the type $\esssmall(f)$ is a proposition, and for each type $X$ the type $\locsmall(X)$ is a proposition.
\end{cor}

\begin{proof}
This follows from the fact that propositions are closed under dependent products, established in \cref{thm:trunc_pi}.
\end{proof}

\begin{thm}\label{thm:fam_proj}
For any small type $A:\UU$ there is an equivalence
\begin{equation*}
\eqv{(A\to \UU)}{\Big(\sm{X:\UU} X\to A\Big)}.
\end{equation*}
\end{thm}

\begin{proof}
Note that we have the function
\begin{equation*}
\varphi :\lam{B} \Big(\sm{x:A}B(x),\proj 1\Big) : (A\to \UU)\to \Big(\sm{X:\UU}X\to A\Big).
\end{equation*}
The fiber of this map at $(X,f)$ is by univalence and function extensionality equivalent to the type
\begin{equation*}
\sm{B:A\to \UU}{e:\eqv{(\sm{x:A}B(x))}{X}} \proj 1\htpy f\circ e.
\end{equation*}
By \cref{cor:fib_triangle} this type is equivalent to the type
\begin{equation*}
\sm{B:A\to \UU}\prd{a:A} \eqv{B(a)}{\fib{f}{a}},
\end{equation*}
and by `type theoretic choice', which was established in \cref{thm:choice}, this type is equivalent to
\begin{equation*}
\prd{a:A}\sm{X:\UU}\eqv{X}{\fib{f}{a}}.
\end{equation*}
We conclude that the fiber of $\varphi$ at $(X,f)$ is equivalent to the type $\esssmall(f)$. However, since $f:X\to A$ is a map between small types it is essentially small. Moreover, since being essentially small is a proposition by \cref{lem:isprop_ess_small}, it follows that $\fib{\varphi}{(X,f)}$ is contractible for every $f:X\to A$. In other words, $\varphi$ is a contractible map, and therefore it is an equivalence.
\end{proof}

\begin{rmk}
The inverse of the map
\begin{equation*}
\varphi : (A\to \UU)\to \Big(\sm{X:\UU}X\to A\Big).
\end{equation*}
constructed in \cref{thm:fam_proj} is the map $(X,f)\mapsto \fibf{f}$.
\end{rmk}

\begin{thm}\label{thm:classifier}
Let $f:A\to B$ be a map. Then there is an equivalence
\begin{equation*}
\eqv{\esssmall(f)}{\isclassified(f)},
\end{equation*}
where $\isclassified(f)$\index{is_classified(f)@{$\isclassified(f)$}|textbf} is the type of quadruples $(F,\tilde{F},H,p)$ consisting of maps
$F:B\to \UU$ and $\tilde{F}:A\to \sm{X:\UU}X$, a homotopy $H:F\circ f\htpy \proj 1\circ \tilde{F}$,  such that the commuting square
\begin{equation*}
\begin{tikzcd}
A \arrow[r,"\tilde{F}"] \arrow[d,swap,"f"] & \sm{X:\UU}X \arrow[d,"\proj 1"] \\
B \arrow[r,swap,"F"] & \UU
\end{tikzcd}
\end{equation*}
is a pullback square, as witnessed by $p$. If $f$ comes equipped with a term of type $\isclassified(f)$, we also say that $f$ is \define{classified}\index{classified by the universal family|textbf} by the universal family. 
\end{thm}

\begin{proof}
From \cref{ex:sq_fib} we obtain that the type of pairs $(\tilde{F},H)$ is equivalent to the type of fiberwise transformations
\begin{equation*}
\prd{b:B}\fib{f}{b}\to F(b).
\end{equation*}
By \cref{cor:pb_fibequiv} the square is a pullback square if and only if the induced map
\begin{equation*}
\prd{b:B}\fib{f}{b}\to F(b)
\end{equation*}
is a fiberwise equivalence. Thus the data $(F,\tilde{F},H,pb)$ is equivalent to the type of pairs $(F,e)$ where $e$ is a fiberwise equivalence from $\fibf{f}$ to $F$. By \cref{thm:choice} the type of pairs $(F,e)$ is equivalent to the type $\esssmall(f)$. 
\end{proof}

\begin{rmk}
For any type $A$ (not necessarily small), and any $B:A\to \UU$, the square\index{Sigma-type@{$\Sigma$-type}!as pullback of universal family|textit}
\begin{equation*}
\begin{tikzcd}[column sep=6em]
\sm{x:A}B(x) \arrow[d,swap,"\proj 1"] \arrow[r,"{\lam{(x,y)}(B(x),y)}"] & \sm{X:\UU}X \arrow[d,"\proj 1"] \\
A \arrow[r,swap,"B"] & \UU
\end{tikzcd}
\end{equation*}
is a pullback square. Therefore it follows that for any family $B:A\to\UU$ of small types, the projection map $\proj 1:\sm{x:A}B(x)\to A$ is an essentially small map.
To see that the claim is a direct consequence of \cref{lem:pb_subst} we write the asserted square in its rudimentary form:
\begin{equation*}
%\begin{gathered}[b]
\begin{tikzcd}[column sep=6em]
\sm{x:A}\mathrm{El}(B(x)) \arrow[d,swap,"\proj 1"] \arrow[r,"{\lam{(x,y)}(B(x),y)}"] & \sm{X:\UU}\mathrm{El}(X) \arrow[d,"\proj 1"] \\
A \arrow[r,swap,"B"] & \UU.
\end{tikzcd}%\\[-\dp\strutbox]\end{gathered}\qedhere
\end{equation*}
\end{rmk}

In the following theorem we show that a type is small if and only if its diagonal is classified by $\UU$.

\begin{thm}
Let $A$ be a type. The following are equivalent:
\begin{enumerate}
\item $A$ is locally small.\index{locally small|textit}
\item There are maps $I:A\times A\to\UU$ and $\tilde{I}:A\to\sm{X:\UU}X$, and a homotopy $H:I\circ \delta_A\htpy \proj 1\circ\tilde{I}$
such that the commuting square
\begin{equation*}
\begin{tikzcd}
A \arrow[r,"\tilde{I}"] \arrow[d,swap,"\delta_A"] & \sm{X:\UU}X \arrow[d,"\proj 1"] \\
A\times A \arrow[r,swap,"{I}"] & \UU
\end{tikzcd}
\end{equation*}
is a pullback square.\index{diagonal!of a type|textit}
\end{enumerate}
\end{thm}

\begin{proof}
In \cref{ex:diagonal} we have established that the identity type $x=y$ is the fiber of $\delta_A$ at $(x,y):A\times A$. Therefore it follows that $A$ is locally small if and only if the diagonal $\delta_A$ is essentially small.
Now the result follows from \cref{thm:classifier}.
\end{proof}

\section{Fiber sequences}

\begin{defn}
A \define{pointed type} is a pair $(X,x)$ consisting of a type $X$ equipped with a \define{base point} $x:X$. We will write $\UU_\ast$ for the type $\sm{X:\UU}X$ of all pointed types.
\end{defn}

In the following lemma we characterize the identity type of $\UU_\ast$. 

\begin{lem}\label{lem:equiv_of_ptdtype}
For any $(A,a),(B,b):\UU_\ast$ we have an equivalence
\begin{equation*}
\eqv{\Big(\pairr{A,a}=\pairr{B,b}\Big)}{\Big(\sm{e:\eqv{A}{B}}e(a)=b\Big)}.
\end{equation*}
\end{lem}

\begin{proof}[Construction]
By \cref{thm:eq_sigma} the type on the left hand side is
equivalent to the type $\sm{p:A=B}\tr_{\universalfam}({p},{a})=b$.
By the univalence axiom, the map 
\begin{equation*}
\equiveq_{A,B}:(A=B)\to (\eqv{A}{B})
\end{equation*}
is an equivalence for each $B:\UU$. 
Therefore, we have an equivalence of type
\begin{equation*}
\eqv{\Big(\sm{p:A=B}\tr_{\universalfam}({p},{a})=b\Big)}{\Big(\sm{e:\eqv{A}{B}}\tr_{\universalfam}({\eqequiv(e)},{a})=b\Big)}
\end{equation*} 
Moreover, by equivalence induction (the analogue of path induction for 
equivalences), we can compute the transport:
\begin{equation*}
\tr_{\universalfam}({\eqequiv(e)},{a})=e(a).
\end{equation*}
It follows that $\eqv{(\tr_{\universalfam}({\eqequiv(e)},{a})=b)}
{(e(a)=b)}$.
\end{proof}

\begin{defn}
\begin{enumerate}
\item Let $(X,\ast_X)$ be a pointed type. A \define{pointed family} over $(X,\ast_X)$ consists of a type family $P:X\to \UU$ equipped with a base point $\ast_P:P(\ast_X)$. 
\item Let $(P,\ast_P)$ be a pointed family over $(X,\ast_X)$. A \define{pointed section} of $(P,\ast_P)$ consists of a dependent function $f:\prd{x:X}P(x)$ and an identification $p:f(\ast_X)=\ast_P$. We define the \define{pointed $\Pi$-type} to be the type of pointed sections:
\begin{equation*}
\Pi^\ast_{(x:X)}P(x) \defeq \sm{f:\prd{x:X}P(x)}f(\ast_X)=\ast_P
\end{equation*}
In the case of two pointed types $X$ and $Y$, we may also view $Y$ as a pointed family over $X$. In this case we write $X\to_\ast Y$ for the type of pointed functions.
\item Given any two pointed sections $f$ and $g$ of a pointed family $P$ over $X$, we define the type of pointed homotopies
\begin{equation*}
f\htpy_\ast g \defeq \Pi^\ast_{(x:X)} f(x)=g(x),
\end{equation*}
where the family $x\mapsto f(x)=g(x)$ is equipped with the base point $\ct{p}{q^{-1}}$. 
\end{enumerate}
\end{defn}

\begin{defn}
\begin{enumerate}
\item For any pointed type $X$, we define the \define{pointed identity function} $\mathsf{id}^\ast_X\defeq (\idfunc[X],\refl{\ast})$. 
\item For any two pointed maps $f:X\to_\ast Y$ and $g:Y\to_\ast Z$, we define the \define{pointed composite}
\begin{equation*}
g\mathbin{\circ_\ast} f \defeq (g\circ f,\ct{\ap{g}{p_f}}{p_g}).
\end{equation*}
\end{enumerate}
\end{defn}

\begin{defn}
Let $X$ be a pointed type with base point $x$. We define the \define{loop space} $\loopspace{X,x}$ of $X$ at $x$ to be the pointed type $x=x$ with base point $\refl{x}$. 
\end{defn}

\begin{defn}
The loop space operation $\loopspacesym$ is \emph{functorial} in the sense that
\begin{enumerate}
\item For every pointed map $f:X\to_\ast Y$ there is a pointed map
\begin{equation*}
\loopspace{f}:\loopspace{X}\to_\ast \loopspace{Y},
\end{equation*}
defined by $\loopspace{f}(\omega)\defeq \ct{p_f}{\ap{f}{\omega}}{p_f^{-1}}$, which is base point preserving by $\mathsf{right\usc{}inv}(p_f)$. 
\item For every pointed type $X$ there is a pointed homotopy
\begin{equation*}
\loopspace{\mathsf{id}_X^\ast}\htpy_\ast \mathsf{id}^\ast_{\loopspace{X}}.
\end{equation*}
\item For any two pointed maps $f:X\to_\ast Y$ and $g:Y\to_\ast X$, there is a pointed homotopy witnessing that the triangle
\begin{equation*}
\begin{tikzcd}
& \loopspace{Y} \arrow[dr,"\loopspace{g}"] \\
\loopspace{X} \arrow[rr,swap,"\loopspace{g\circ_\ast f}"] \arrow[ur,"\loopspace{f}"] & & \loopspace{Z}
\end{tikzcd}
\end{equation*}
of pointed types commutes.
\end{enumerate}
\end{defn}

\begin{lem}\label{lem:equiv_of_ptdequiv}
For any $\pairr{e,p},\pairr{f,q}:\sm{e:\eqv{A}{B}}e(a)=b$, we have an equivalence of type
\begin{equation*}
\eqv{\Big(\pairr{e,p}=\pairr{f,q}\Big)}{\Big(\sm{h:e\htpy f} p=\ct{h(a)}{q}\Big)}.
\end{equation*}
\end{lem}

\begin{proof}[Construction]
The type $\pairr{e,p}=\pairr{f,q}$ is equivalent
to the type $\sm{h:e=f}\tr({h},{p})=q$.
Note that by the principle of function extensionality,
the map $\htpyeq:(e=f)\to(e\htpy f)$
is an equivalence. Furthermore, it follows by homotopy induction that for any 
$h:e\htpy f$ we have an equivalence of type
\begin{equation*}
\eqv{(\tr({\eqhtpy(h)},{p})=q)}
    {(p= \ct{h(a)}{q})}.\qedhere
\end{equation*}
\end{proof}

\begin{defn}
A \define{fiber sequence} $F\hookrightarrow E \twoheadrightarrow B$ consists of:
\begin{enumerate}
\item Pointed types $F$, $E$, and $B$, with base points $x_0$, $y_0$, and $b_0$ respectively, 
\item Base point preserving maps $i:F\to_\ast E$ and $p:E\to_\ast B$, with $\alpha:i(x_0)=y_0$ and $\beta:p(y_0)=b_0$,
\item A pointed homotopy $H:\mathsf{const}_{b_0}\htpy_\ast p\circ_\ast i$ witnessing that the square
\begin{equation*}
\begin{tikzcd}
F \arrow[r,"i"] \arrow[d] & E \arrow[d,"p"] \\
\unit \arrow[r,swap,"\mathsf{const}_{b_0}"] & B,
\end{tikzcd}
\end{equation*}
commutes and is a pullback square.
\end{enumerate}
We will write $\mathsf{Fib\usc{}Seq}$ for the type of all fiber sequences in $\UU$.
\end{defn}

\begin{thm}
The type of all fiber sequences is equivalent to the type
\begin{equation*}
\sm{(B,b):\UU_\ast}{P:B\to\UU}P(b).
\end{equation*}
\end{thm}
