\chapter{Localization at a prime}\label{chap:prime}

In this section we first show that for any family of maps between compact types in our setting of a univalent universe that is closed under pushouts, the subuniverse of $f$-local types is reflective. 

Then we proceed to study the $S$-local types for a set $S$ of integers. We say that a type $X$ is $S$-local if it is $\mathsf{deg}(k)$-local for any $k:S$. We show that for a simply connected type $X$, $S$-localization localizes the homotopy groups of$X$. 

\section{Localization away from sets of numbers}\label{section:localizationaway}

In this section, we focus on localization with respect to the \define{degree $k$
map} $\degg(k) : \sphere{1} \to \sphere{1}$, for $k : \N$, and with respect to
families of such maps.
The degree $k$ map is defined by circle induction by mapping $\base$ to $\base$
and $\lloop$ to $\lloop^k$.
Using suggestive language that mirrors the algebraic case, we call $L_{\degg(k)}$
``localization away from $k$'' and say that we are ``inverting $k$.''
We note that, for many applications, one considers localization away from $k$ for $k$ prime.
However, for the results of this section, $k$ can be any natural number.

As might be expected, $\degg(k)$-localizations can be combined to localize a type \emph{at}
a prime $p$, by localizing with respect to the family of maps $\degg(q)$ indexed by all
primes $q$ different from $p$.
With this case in mind, we study localization with respect to any family of degree $k$ maps
indexed by a map $S : A \to \N$ for some type $A$.
In other words, we localize with respect to the family $\degg\circ S : A \to \sphere{1} \to \sphere{1}$.
We denote the family $\degg\circ S$ by $\degg(S)$.

The main goal of this section is to show that, for simply connected types,
$\degg(S)$-localization localizes the homotopy groups away from $S$ (as long as $S$ indexed by $\N$),
as is true in the classical setting.

We begin with a discussion of localization of groups in \cref{ss:localizationofgroups}, which we need in order to describe the effect of localization on homotopy groups.

In \cref{ss:n-conn}, we give characterizations of $\degg(S)$-localization for highly connected types. In particular, we show that the localization of a simply connected type can be computed as the nullification with respect to $M_S$, the family of cofibers of the maps $\degg(S(a))$, or as the localization with respect to $\susp \degg(S)$, the family of suspensions of these maps.
We prove that the $\degg(S)$-localization of a pointed, $(n-1)$-connected type inverts $S$ in $\pi_n$,
and deduce from this that every group has a localization away from $S$.

These characterizations of $\degg(S)$-localization are of interest to us for two reasons. First, the observation that $\degg(S)$-localization can be computed via a nullification implies that, restricted to simply connected types, $\degg(S)$-localization is a modality and therefore is better behaved than an arbitrary localization. Second, the fact that $\degg(S)$-localization can be computed as $\susp \degg(S)$-localization allows us to deduce that, for simply connected types, $\degg(S)$-localization commutes with taking loop spaces, a fact we use in the next section.

In \cref{ss:localizingKgn}, we give a method for computing the $\degg(S)$-localization of a loop space via a mapping telescope construction. This allows us to show that the localization of an Eilenberg-Mac Lane space $K(G,n)$, for $n \geq 1$ and $G$ abelian, is $K(G',n)$, where $G'$ is the algebraic localization of $G$ away from $S$.

In \cref{ss:localization-of-homotopy-groups}, we combine the results of \cref{ss:localizingKgn} and observations about the interaction between $\degg(S)$-localization and truncation to show that localizing a simply connected type at $\degg(S)$ localizes all of the homotopy groups away from $S$.
The results of the last two sections assume that the family $S$ is indexed by the natural numbers.

Finally, in \cref{ss:abelian} we give a direct proof of the fact that
the localization of an abelian group in the category of groups coincides with
its localization in the category of abelian groups, which also follows from
\cref{theorem:localizationKgn}.
This section is not needed in the rest of the paper.

\subsection{Localization of groups}\label{ss:localizationofgroups}

We fix a family $S : A \to \N$ of natural numbers and consider the family
of maps $\degg(S) : A \to S^1 \to S^1$ sending $a$ to $\degg(S(a))$.

We begin by defining the collection of groups that will be the
``local'' groups for our algebraic localization.
We make the standard assumption that the underlying type of a group is a set.

\begin{defn}
    For $k : \N$, a group $G$ is \define{uniquely $k$-divisible} if the $k$-th power map
    $g \mapsto g^k$ is a bijection.
    A group $G$ is \define{uniquely $S$-divisible} if it is uniquely $S(a)$-divisible for every $a : A$.
\end{defn}

When the group is additive, the $k$-th power map is usually called the ``multiplication by $k$ map.''
Notice that for non-abelian groups the map is not a group homomorphism in general.

Our first goal is to show that the homotopy groups of a $\degg(S)$-local type
are uniquely $S$-divisible.
In order to do this, we begin by characterizing the $\degg(S)$-local types.

\begin{lem}
A type $X$ is $\degg(S)$-local if and only if, for each $x:X$ and $a:A$, the map
$S(a) : \loopspace{X,x} \to \loopspace{X,x}$ sending 
$\omega$ to $\omega^{S(a)}$ is an equivalence.
\end{lem}

\begin{proof}
We apply \cref{lemma:pointed}.
Let $x:X$ and $a:A$.
By the universal property of the circle, we have an equivalence
$(\sphere{1} \pto X) \eqvsym \loopspace{X,x}$,
and it is easy to see that the square
\[
  \begin{tikzcd}
    (\sphere{1} \pto X) \arrow[r,"\simeq"] \arrow[d,swap,"\precomp{\degg(S(a))}"] & \loopspace{X,x} \arrow[d,"S(a)"] \\
    (\sphere{1} \pto X) \arrow[r,swap,"\simeq"] & \loopspace{X,x}
  \end{tikzcd}
\]
commutes.
So we conclude that $\precomp{\degg(S(a))}$ is an equivalence if and only if $S(a)$ is an equivalence.
\end{proof}

\begin{prp}\label{prop:homotopygroupsoflocalarelocal}
    If $X$ is a pointed, $\degg(S)$-local type, then $\pi_{m+1}(X)$ is uniquely $S$-divisible
    for each $m : \N$.
    Conversely, if for some $n \geq 0$, $X$ is a pointed, simply connected $n$-type
    such that $\pi_{m+1}(X)$ is uniquely $S$-divisible for each $m : \N$,
    then $X$ is $\degg(S)$-local.
\end{prp}

For the converse, the requirement that $X$ be truncated is needed because
we need to use Whitehead's theorem.

\begin{proof}
    Let $X$ be a pointed, $\degg(S)$-local type and let $a : A$.
    In this proof, we write $\degg(k, Y)$ for the degree $k$ map on $\loopspacesym Y$,
    where $k = S(a)$.
    For every $m : \N$,
    the map $\degg(k, X) : \loopspacesym X \to \loopspacesym X$ induces
    the map $\degg(k, \loopspacesym^m X) : \loopspacesym^m \loopspacesym X \to \loopspacesym^m \loopspacesym X$,
    by the Eckmann-Hilton argument~\cite[Theorem~2.1.6]{hottbook}.
    That is, $\loopspacesym^m \degg(k, X) = \degg(k, \loopspacesym^m X)$.
    It follows that $\loopspacesym^m \degg(k, X)$ induces the $k$-th power map on $\pi_{m+1}( X )$.
    Since $\degg(k, X)$ is an equivalence, the $k$-th power map on $\pi_{m+1}( X )$
    must be an equivalence for each $m$.
    In other words, $\pi_{m+1}( X )$ is uniquely $S$-divisible for each $m$.

    Conversely, suppose that, for some $n \geq 0$, $X$ is a pointed,
    simply connected $n$-type such that $\pi_{m+1}(X)$ is uniquely $S$-divisible
    for each $m : \N$.
    Let $a : A$ and let $k = S(a)$.
    Since $X$ is pointed and connected, it suffices to show that
    $\degg(k, X) : \loopspacesym X \to \loopspacesym X$ is an equivalence.
    By the above argument, $\degg(k, X)$ induces the $k$-th power map on each
    $\pi_{m+1}(X) = \pi_m(\loopspacesym X)$.
    By assumption, these $k$-th power maps are bijections.
    Since $\loopspacesym X$ is a pointed, connected $(n-1)$-type, it follows from the
    truncated Whitehead theorem~\cite[Theorem~8.8.3]{hottbook} that 
    $\degg(k, X)$ is an equivalence.
    % This last step is where we use that X is simply connected.  Otherwise,
    % we'd need to study pi_m(\loopspacesym X) at other base points.
\end{proof}

Our next goal is to show that a group $G$ is uniquely $S$-divisible if and only
if the Eilenberg--Mac Lane space $K(G, 1)$ is $\degg(S)$-local.
For this, we use the following theorem.

%\note{Update Theorem number when BDR is published.}
\begin{thm}[{\cite[Theorem 4]{BuchholtzDoornRijke}}]\label{theorem:catofgroups}
    Given $n>1$, the pointed, $(n-1)$-connected, $n$-truncated types together with pointed maps
    form a univalent category, and this category is equivalent to the category of abelian groups
    and group homomorphisms.
    Similarly, when $n=1$ we have an equivalence between the category of pointed, $0$-connected,
    $1$-truncated types and the category of groups. \qed
\end{thm}

The equivalence in \cref{theorem:catofgroups} is given by mapping a pointed,
$(n-1)$-connected, $n$-truncated type $X$ to $\pi_n( X )$,
and the inverse is given by mapping a group $G$ to the Eilenberg--Mac Lane
space $K(G,n)$, as constructed in~\cite{FinsterLicata}.
Moreover,~\cite{BuchholtzDoornRijke} shows that the inverse equivalence maps
short exact sequences of groups to fiber sequences of types.

\begin{cor}\label{corollary:characterizationpdivisible}
    For any $n > 1$, an abelian group $G$ is uniquely $S$-divisible if and only if its classifying space $K(G,n)$
    is $\degg(S)$-local. An arbitrary group $G$ is uniquely $S$-divisible if and only if $K(G,1)$ is $\degg(S)$-local.
\end{cor}

\begin{proof}
    As argued in the proof of \cref{prop:homotopygroupsoflocalarelocal}, $\degg(k)$ induces the $k$-th power map under the equivalence of categories, so one morphism is an equivalence if and only if the other is.
\end{proof}

Note that the $n > 1$ case of \cref{corollary:characterizationpdivisible} also follows
immediately from \cref{prop:homotopygroupsoflocalarelocal}.

Now we define what it means to localize a group away from $S$.
This concept is needed to state our main result, \cref{theorem:localizationlocalizes}.

\begin{defn}\label{def:localizationofgroups}
    Given a group $G$, a homomorphism $\eta : G \to G'$ is a \define{localization of $G$ away from $S$
    in the category of groups} if $G'$ is uniquely $S$-divisible and for every group $H$
    which is uniquely $S$-divisible, the precomposition map $\precomp{\eta} : \Hom(G', H) \to \Hom(G, H)$ is an equivalence.
    If $G$ is abelian and $\eta : G \to G'$ has the universal property only with respect to uniquely
    $S$-divisible abelian groups, we say $\eta$ is a \define{localization of $G$ away from $S$ in the category
    of abelian groups}.
\end{defn}

\begin{rmk}\label{rmk:localizationofgroups}
It will follow from \cref{lemma:localizationlocalizesfirst} that every group $G$ has a localization $G \to G'$
away from $S$, obtained by applying $\pi_1$ to the $\deg(S)$-localization of $K(G, 1)$.
Similarly, if $G$ is abelian, its localization in the category of abelian groups is obtained
by applying $\pi_2$ to the $\deg(S)$-localization of $K(G,2)$.
Moreover, by \cref{theorem:localizationKgn}, these algebraic localizations agree
when restricted to abelian groups.
In \cref{ss:abelian}, we give an independent, purely algebraic construction of
the localization of an abelian group away from $S$ which also shows that the
two localizations agree.
\end{rmk}

We conclude this section with three examples.
The first one shows that $\degg(k)$-localization does not preserve fiber sequences in general.
The second one shows the stronger statement that $\degg(k)$-localization is not a modality,
by giving an example showing that uniquely $k$-divisible groups are not closed under extensions.
(It is an easy exercise that if a localization $L$ preserves fiber sequences,
then $L$ is a modality.)
The third one shows that the $\degg(k)$-local types are not the separated types
for any reflective subuniverse $L$.

\begin{eg}\label{example:notlex}
    We show that for $k > 1$, $\degg(k)$-localization
    does not preserve all fiber sequences.
    We first make some general observations.
    By \cref{lemma:setsarelocal}, we know that if $f$ is a map between pointed,
    connected types, then sets are $f$-local.
    It follows that if $L_f$ preserves the fiber sequence
    \[
        G \longhookrightarrow \unit \lra K(G,1),
    \]
    then $K(G,1)$ is $f$-local.
    Indeed, the $f$-localization map between the fiber sequences gives
\begin{equation*}
\begin{tikzcd}
G \arrow[d,swap,"\eqvsym"] \arrow[r,hookrightarrow] & \unit \arrow[d,swap,"\eqvsym"] \arrow[r] & K(G,1) \arrow[d] \\
L_f G \arrow[r,hookrightarrow] & L_f\unit \arrow[r] & L_f K(G,1).
\end{tikzcd}
\end{equation*}
    which implies that $K(G,1) \simeq L_f K(G,1)$, since $L_f K(G,1)$ is connected
    by \cref{cor:preserve-n-connected}.
    Now take $f$ to be the degree $k$ map.
    By \cref{corollary:characterizationpdivisible}, $K(G,1)$ is $\degg(k)$-local if and only if
    $G$ is uniquely $k$-divisible.
    In particular, if $G = \Z/k\Z$, then $K(G,1)$ is not $\degg(k)$-local.
    So $\degg(k)$-localization does not preserve all fiber sequences.
\end{eg}

\note{Added the following example. It is much simpler than the next one. Should we keep both?}
\note{Haven't modified the intro to these examples.}

\begin{eg}\label{example:nonlocalfib2}
    Consider the map $K(\Z,1) \to K(\Q,1)$, induced by the inclusion of groups $\Z \to \Q$.
    Using the long exact sequence of homotopy groups, one sees that the fiber over any point is
    (merely) equivalent to the set $\Q/\Z$. Recall that sets are $\degg(k)$-local for any $k : \N$,
    so as long as $k > 0$, the type $K(\Z,1)$ is the total space of a fibration with $\degg(k)$-local fibers and $\degg(k)$-local base.
    On the other hand, if $k > 1$, $K(\Z,1)$ is not $\degg(k)$-local, so we see that $\degg(k)$-localization
    is not a modality in general.
\end{eg}

\begin{eg}\label{example:nonlocalfib}
    Let $B$ be the subtype of $\Q \to \Q$ consisting of functions with bounded support.
    We can define this type constructively as the type of functions together with
    a mere bound for their support:
    \[
         B \defeq \sm{f : \Q \to \Q} \, \ttrunc{-1}{\sm{b : \Q} \prd{x : \Q} (x > b) \to (f(x) = 0) \times (f(-x) = 0)}.
    \]
    The group operation is given by $(f + g)(x) = f(x) + g(x)$.
    Notice that both $\Q$ and $B$ are uniquely $k$-divisible for any $k > 0$.

    Consider the semidirect product $P \defeq B \rtimes \Q$, where $\Q$ acts by translation:
    \[
        (r \cdot f)(x) = f(x+r).
    \]
    We claim that $(\delta_0, 2) : P$ does not have a square root. To see
    this, assume we have $(f,r)$ such that $(f,r)^2 = (f + r\cdot f,2r) = (\delta_0, 2)$.
    Then it must be the case that $r = 1$ and thus $f(x) + f(x+1) = \delta_0(x)$ for all $x : \Q$.
    But then $f(x) = - f(x+1)$ if $x \neq 0$.
    We can now contradict our assumption by proving that $f(x) = 0$ for all $x$.
    To do this, notice that this fact is a mere proposition, so we
    can use the mere bound for the support of $f$ to conclude that $f(x+1)$ is $0$ if $x>0$ is large enough,
    and then induct backwards using $f(x) = - f(x+1)$ if $x>0$. The case $x<0$ is analogous.
    Thus we have obtained a short exact sequence of groups
    \[
        1 \lra B \lra P \lra \Q \lra 1
    \]
    in which the kernel and cokernel are uniquely $2$-divisible, but the middle group is not.

    It follows that if we apply the functor $K(\blank, 1)$ to this short exact sequence of groups,
    we obtain a fiber sequence with connected, $\degg(2)$-local base and fiber,
    but for which the total space is not $\degg(2)$-local.
    Phrased differently, a dependent sum of $\degg(2)$-local types, indexed by 
    a $\degg(2)$-local type, is not necessarily $\degg(2)$-local.
    So $\degg(2)$-localization is not a modality.

    A similar argument shows that for any $k > 1$, $P$ is not uniquely $k$-divisible
    and therefore that $\degg(k)$-localization is also not a modality.
\end{eg}

\begin{eg}
As a final example in a similar spirit, we show that for $k > 1$, $L_{\degg(k)}$ is not
of the form $L'$ for any reflective subuniverse $L$.
For any $L$, the $L$-separated types are determined by their identity types.
However, the identity types of $K(\Q,1)$ and $K(\Z,1)$ are equivalent,
and the former is $L_{\degg(k)}$-local while the latter is not.
\end{eg}

\subsection{Localizing highly connected types}\label{ss:n-conn}

We continue to fix a family $S : A \to \N$ of natural numbers and the associated family
of maps $\degg(S) : A \to S^1 \to S^1$ sending $a$ to $\degg(S(a))$.
We write $M_S$ for the family of cofibers, which sends $a$ to the cofiber
of $\degg(S(a))$, a ``Moore space.''
We write $\susp{\degg(S)}$ for the family of suspensions of the maps $\degg(S(a))$,
and similarly consider $\suspsym^n M_S$ and $\suspsym^n \degg(S)$.

The main result of this section is:

\begin{thm}\label{theorem:localizationisnullification}
    Let $n\geq 1$.
    If $X$ is $n$-connected, then its
    $\degg(S)$-localization, its $\suspsym^{n-1} M_S$-nullification and
    its $\suspsym^n \degg(S)$-localization coincide.
\end{thm}

Note that this implies that the same is true for the other localizations ``between''
$\degg(S)$ and $\suspsym^n \degg(S)$.
% E.g. N_{\suspsym^i M_S} for various i.

\begin{proof}
By \cref{cor:preserve-n-connected},
$L_{\susp[n] \degg(S)} X$ is $n$-connected,
since $\trunc{n}{\suspsym^n \, \sphere{1}} \eqvsym 1$.
Thus, by \cref{theorem:characterizinglocalness}, $L_{\susp[n] \degg(S)} X$
is $\degg(S)$-local and $\suspsym^{n-1} M_S$-local.
So the natural maps
\[
  L_{\susp[n] \degg(S)} X \lra L_{\suspsym^{n-1} M_S} X \lra L_{\degg(S)} X
\]
are equivalences, by \cref{lemma:comparelocalization}(3).
\end{proof}

This theorem fails without the connectedness hypothesis, as the next example shows.

\begin{eg}
    The type $\sphere{1}$ is not $\degg(k)$-local for $k > 1$ but it is $M_k$-null.
    On the one hand, $\loopspacesym \sphere{1}$ is equivalent to the integers, and the $k$-fold map
    is multiplication by $k$, which is not an equivalence.
    On the other, mapping from the cofiber sequence $\sphere{1}\to \sphere{1} \to M_k$ into $\sphere{1}$ we
    obtain the fiber sequence:
    \[
        (M_k \to_\ast \sphere{1}) \longhookrightarrow (\sphere{1} \to_\ast \sphere{1}) \xrightarrow{k} (\sphere{1}\to_\ast \sphere{1}),
    \]
    and the multiplication by $k$ map on the integers is injective, so the pointed mapping space $M_k \to_\ast \sphere{1}$ is contractible. Therefore $\sphere{1}$ is $M_k$-null by \cref{cor:pointed_null}.
\end{eg}

As an application of \cref{theorem:localizationisnullification}, we have:

\begin{cor}\label{corollary:commutativitylooplocalizationsimpconn}
    For a pointed, simply connected type $X$, we have
    \[
        \loopspacesym L_{\degg(S)} X \simeq L_{\degg(S)} \loopspacesym X.
    \]
\end{cor}

\begin{proof}
This follows from \cref{remark:commutativitylooplocalization} and \cref{theorem:localizationisnullification}.
\end{proof}

\cref{example:notlex} shows that the assumption that $X$ is simply connected cannot be removed.
Note that we cannot necessarily iterate the interchange of $\loopspacesym$ and $\degg(S)$-localization,
since $\loopspacesym X$ might fail to be simply connected.

\medskip

It also follows that these localizations preserve $l$-connected types.

\begin{cor}\label{corollary:plocalizationpreservesconnectedness}
    For $l \geq -2$ and $n \geq 0$, if $X$ is $l$-connected, then $L_{\susp[n] \degg(S)} X$ is $l$-connected.
\end{cor}
% We really only need the case n = 0 of this result.
% This lemma can also be proved using lemma:truncationpreserveslocal,
% and can then be used to prove theorem:localizationisnullification.

\begin{proof}
    If $n \geq l$, this follows from \cref{cor:preserve-n-connected}.
    If $n < l$, then \cref{theorem:localizationisnullification} implies
    that $L_{\susp[n] \degg(S)} X \eqvsym L_{\susp[l] \degg(S)} X$, putting us
    in the situation where $n = l$.
%\note{Having to break into cases makes me think we still haven't
%captured the cleanest way of organizing this.}
\end{proof}

The following proposition implies that localizations of groups always exist
(see \cref{rmk:localizationofgroups})
and is also used to prove \cref{theorem:localizationKgn}.

\begin{prp}\label{lemma:localizationlocalizesfirst}
    For $n \geq 1$,
    the $\degg(S)$-localization of a pointed, $(n-1)$-connected type $X$ localizes $\pi_n(X)$ away from $S$.
    The algebraic localization takes place in the category of groups if $n = 1$ and
    abelian groups otherwise.
\end{prp}

\begin{proof}
    By \cref{prop:homotopygroupsoflocalarelocal},
    we know that $\pi_n(L_{\degg(S)} X)$ is uniquely $S$-divisible.
    So it remains to show that precomposition with $\pi_n(\eta) : \pi_n(X) \to \pi_n(L_{\degg(S)} X)$
    induces an equivalence
    \[
        \precomp{\pi_n(\eta)} : \Hom(\pi_n(L_{\degg(S)} X),  H) \lra \Hom(\pi_n(X), H)
    \]
    for every uniquely $S$-divisible group $H$ (where $H$ is assumed to be abelian if $n>1$).
    Notice that this map is equivalent to 
    \[
        \precomp{\pi_n(\ttrunc{n}{\eta})} : \Hom(\pi_n(\ttrunc{n}{L_{\degg(S)} X}),  H) \lra \Hom(\pi_n(\ttrunc{n}{X}), H)
    \]
    which in turn is equivalent to
    \[
        \precomp{\ttrunc{n}{\eta}} : \left( \ttrunc{n}{L_{\degg(S)}X} \pto K(H, n) \right) \lra \left( \ttrunc{n}{X} \pto K(H, n) \right) 
    \]
    by \cref{theorem:catofgroups}, using the fact that $\ttrunc{n}{L_{\degg(S)}X}$ is still $(n-1)$-connected (\cref{corollary:plocalizationpreservesconnectedness}).
    Finally, this last map is equivalent to
    \[
        \precomp{\eta} : \left( L_{\degg(S)}X \pto K(H, n) \right) \lra \left( X \pto K(H, n) \right),
    \]
    since $K(H,n)$ is $n$-truncated.
    And this map is an equivalence, since $K(H,n)$ is $\degg(S)$-local by \cref{corollary:characterizationpdivisible}.
\end{proof}

The results of this section also let us deduce the following lemma,
which will be used in \cref{ss:localization-of-homotopy-groups}.

\begin{lem}\label{lemma:lex}
    $\degg(S)$-localization maps fiber sequences of simply connected types to fiber sequences.
\end{lem}

Note that this lemma cannot be used to prove \cref{corollary:commutativitylooplocalizationsimpconn},
since the fiber that arises in that case is $\loopspacesym X$, which is not assumed to be
simply connected.
%\note{But maybe it's enough to assume that $E$ and $X$ are simply connected?}

\begin{proof}
    Let $F \to E \to X$ be a fiber sequence of simply connected types, with $F$ the fiber over $x_0$.
    By \cref{theorem:localizationisnullification}, the $L$- and $L'$-localizations
    of these types agree, where $L \defeq L_{\degg(S)}$.
    Applying \cref{proposition:preservationfibersequences},\marginnote{I don't know where this proposition went. Need to fix} we get a diagram
\begin{equation*}
\begin{tikzcd}
F \arrow[d,swap,"\eta"] \arrow[r,hookrightarrow] & E \arrow[d,swap,"l"] \arrow[r] & X \arrow[d,"\eta=\eta'"] \\
LF \arrow[r,hookrightarrow] & E' \arrow[r] & L'X \mathrlap{\,\,=\, L X ,}
\end{tikzcd}
\end{equation*}
    in which the rows are fiber sequences, with $L F$ the fiber over $\eta(x_0)$.
    To show that $l$ is an $L$-localization, it is enough to show that $E'$ is $L$-local.
    Since it is the total space of a fibration of simply connected spaces,
    it is simply connected. And since it is $L'$-local,
    by the same proposition, we deduce that it is $L$-local, as needed.
\end{proof}

\cref{example:nonlocalfib} shows that the $\degg(S)$-localization of a
fiber sequence is not a fiber sequence in general.

\medskip

We end this section with the following lemma, which is of a similar
flavor and which will be used in \cref{ss:localization-of-homotopy-groups}.

\begin{lem}\label{lemma:truncationpreserveslocal}
    For $l \geq -2$ and $n \geq 0$, the $l$-truncation of a $\suspsym^n \degg(S)$-local type is $\suspsym^n \degg(S)$-local.
\end{lem}
% I think we only need the n = 0 case of this later.

\begin{proof}
    Let $X$ be a $\suspsym^n \degg(S)$-local type.
    Fix $a : A$ and let $k = S(a)$.
    We must show that $k : \loopspacesym^{n+1} \ttrunc{l}{X} \to \loopspacesym^{n+1} \ttrunc{l}{X}$
    is an equivalence for each basepoint $x' : \ttrunc{l}{X}$.
    If $l - (n+1) \leq -2$, then $\loopspacesym^{n+1} \ttrunc{l}{X}$ is
    contractible, so this is clear.
    So assume that $l - (n+1) > -2$, and in particular that $l > -2$.
    Since being an equivalence is a mere proposition,
    we can assume that $x' = \tproj{l}{x}$ for some $x : X$.
    Recall that $l$-truncation
    is the subuniverse of separated types for $(l-1)$-truncation (\cref{example:truncationisseparated}).
    Applying \cref{thm:separation_characterization} $n+1$ times gives the equivalences
    in the square
\begin{equation*}
\begin{tikzcd}
\trunc{l-n-1}{\loopspacesym^{n+1} X} \arrow[d,swap,"\trunc{l-n-1}{k}"] \arrow[r,"\eqvsym"] & \loopspacesym^{n+1} \trunc{l}X \arrow[d,"k"] \\
\trunc{l-n-1}{\loopspacesym^{n+1} X}. \arrow[r,"\eqvsym"] & \loopspacesym^{n+1} \trunc{l} X
\end{tikzcd}
\end{equation*}
    To show that the square commutes, it suffices to check this after precomposing
    with the truncation map $\loopspacesym^{n+1} X \to \trunc{l-n-1}{\loopspacesym^{n+1} X}$.
    And this follows since the equivalences commute with the natural maps from
    $\loopspacesym^{n+1} X$ and
    both vertical maps commute with $k : \loopspacesym^{n+1} X \to \loopspacesym^{n+1} X$.
    Since $X$ is $\suspsym^n \degg(k)$-local, the map on the right hand side is an equivalence,
    so the result follows.
\end{proof}

\subsection{Localization of Eilenberg--Mac Lane spaces}\label{ss:localizingKgn}

In this section, we compute the localization of a loop space as a mapping telescope,
in a way that is familiar from classical topology.
This specializes to give a concrete description of the
localization of an Eilenberg--Mac Lane space for an abelian group.
This is the key ingredient in the proofs of the results in the next section.

For the remainder of \cref{section:localizationaway}, we assume that our family
$S$ is indexed by the natural numbers, i.e., we have $S : \N \to \N$ and
$\degg(S)$ sends $i$ to $\degg(S(i))$.

For example, if we want to invert a decidable subset of the natural numbers
$T : \N \to \bool$, we can define a family $S : \N \to \N$ by
\[
    S(k) \defeq \begin{cases} k, & \text{if $T(k)$} \\
                              1, & \text{otherwise.}
                \end{cases}
\]
It is easy to see that a type is $\degg(S)$-local if and only if it is $\degg(k)$-local for every $k$ such that $T(k)$ is true.
The most important example of this form is the case of localizing \emph{at} a prime $p$,
in which we take $T$ to be the subset of primes different from $p$.

\begin{lem}\label{lemma:pmapisorthogonal}
    Given a pointed, simply connected type $X$, the $k$-fold map $k : \loopspacesym X \to \loopspacesym X$
    is an $L_{\degg(k)}$-equivalence.
\end{lem}

\begin{proof}
    We have the usual square
\begin{equation*}
\begin{tikzcd}
\loopspacesym X \arrow[d,swap,"k"] \arrow[r,"\eta"] & L_{\degg(k)} \loopspacesym X \arrow[d,"L_{\degg(k)} k"] \\
\loopspacesym X \arrow[r,swap,"\eta"] & L_{\degg(k)} \loopspacesym X
\end{tikzcd}
\end{equation*}
    Recall that the right vertical map is the unique map that makes the square commute.
    Now, by \cref{corollary:commutativitylooplocalizationsimpconn}, we know that for pointed, simply connected types we have
    $L_{\degg(k)} \loopspacesym X \simeq \loopspacesym L_{\degg(k)} X$ so that,
    by the uniqueness of the right vertical map, the above square is equivalent to the square
\begin{equation*}
\begin{tikzcd}
\loopspacesym X \arrow[d,swap,"k"] \arrow[r,"\loopspacesym \eta"] & \loopspacesym L_{\degg(k)} X \arrow[d,"k"] \\
\loopspacesym X \arrow[r,swap,"\loopspacesym \eta"] & \loopspacesym L_{\degg(k)} X
\end{tikzcd}
\end{equation*}
    where the map on the right is the usual $k$-fold map.
    But in this square the right vertical map is an equivalence, since $L_{\degg(k)} X$
    is $\degg(k)$-local.
\end{proof}

We are almost ready to localize the loop space of a simply connected type away from $S : \N \to \N$.
Before doing this we need a general fact about sequential colimits.

\begin{rmk}\label{remark:equivalenceofseqcolim}
Given a sequential diagram $X$:
\[
    X_0 \xrightarrow{h_0} X_1 \xrightarrow{h_1} \cdots
\]
we can consider the shifted sequential diagram $X_1 \to X_2 \to \cdots$, which we denote $X[1]$.
Analogously, we can shift a natural transformation $f : X \to Y$ between sequential
diagrams to obtain a natural transformation $f[1] : X[1] \to Y[1]$.
Notice that the transition maps $h_n$ induce a natural transformation $h : X \to X[1]$, and that moreover, the
induced map $\colim\, h : \colim X \to \colim X[1]$ is an equivalence.
It then follows that if $f : X \to X$ is a natural transformation such that there exist $g, g' : X \to X[1]$
with $g \circ f = h$ and $f[1] \circ g' = h$, then $f$ induces an equivalence $\colim\, f : \colim X \xrightarrow{\sim} \colim X$.
\end{rmk}

\begin{thm}\label{theorem:localizationastelescope}
    Let $X$ be pointed and simply connected, and
    define $s : \N \to \N$ by $s(n) = \prod_{i=0}^n S(i)$.
    Then the $\degg(S)$-localization of $\loopspacesym X$ is equivalent to the colimit of the sequence
    \[\loopspacesym X \xrightarrow{s(0)}\loopspacesym X \xrightarrow{s(1)} \loopspacesym X \xrightarrow{s(2)} \cdots, \]
    where as usual a natural number $k$ is used to denote the $k$-fold map.
\end{thm}

\begin{proof}
    We must show that the colimit is $\degg(S)$-local and that
    the map $\loopspacesym X \to \colim\, \loopspacesym X$ is an $L_{\degg(S)}$-equivalence.
    % Being a deg(S)-equivalence is not the same as being a deg(S(k))-equivalence
    % for every k.
    
    To prove that $\loopspacesym X \to \colim\, \loopspacesym X$ is an $L_{\degg(S)}$-equivalence
    we use \cref{lemma:orthogonalcomposition}, so it is enough to show that all the maps in the
    diagram are $L_{\degg(S)}$-equivalences. To prove this last fact notice that \cref{lemma:pmapisorthogonal}
    implies that the $s(k)$-fold map is an $L_{\degg(s(k))}$-equivalence. But in particular this implies that it is also an
    $L_{\degg(S)}$-equivalence, since every $\degg(S)$-local type is $\degg(s(k))$-local for every $k$.
    % The map s(0) is not a deg(S(1))-equivalence in general, so we can't work
    % one k at a time above.  But it works to handle all k at once.

    Now we must show that the colimit is $\degg(S(k))$-local for each $k$.
    So fix $k$ and recall that
    the colimit is equivalent to the colimit of the sequence starting at $k$.
    That is, we consider the colimit of the sequential diagram with objects
    $C_n = \loopspacesym X$ and transition maps $h_n : C_n \to C_{n+1}$ given by $s(n+k)$.
    Then $\colim_n C_n$ is pointed and connected by \cite{DoornRijkeSojakova},
    so it is enough to check that the map $S(k) : \loopspacesym \colim_n C_n \to \loopspacesym \colim_n C_n$ is an equivalence.
    By the commutativity of loop spaces and sequential colimits \cite{DoornRijkeSojakova},
    it is then enough to show that the natural transformation $S(k) : \loopspacesym C_n \to \loopspacesym C_n$
    induces an equivalence in the colimit.

    By the Eckmann-Hilton argument~\cite[Theorem~2.1.6]{hottbook}, the transition map
    $\loopspacesym h_n : \loopspacesym C_n \to \loopspacesym C_{n+1}$ is homotopic to $s(n+k)$.
    We can then apply \cref{remark:equivalenceofseqcolim},
    taking $f$ to be the natural transformation given by
    $S(k)$ in every degree, and both $g$ and $g'$ to be the natural transformation given by
    $s(n + k)/S(k) : \loopspacesym C_n \to \loopspacesym C_{n+1}$ in degree $n$.
    We have that $g_n \circ f_n = f_{n+1} \circ g'_n = s(n + k)$, and moreover that
    $g \circ f = f[1] \circ g' = h$ as natural transformations.
    So $S(k) : \loopspacesym C_n \to \loopspacesym C_n$ induces an equivalence in the colimit,
    as needed.
\end{proof}

Using the fact that $K(G,n)$ is the loop space of a simply connected space
when $G$ is abelian we deduce:

\begin{cor}\label{corollary:localizationKgn}
    For $n \geq 1$ and $G$ abelian,
    the $\degg(S)$-localization of $K(G,n)$ is equivalent to the colimit of the sequence
    \[
      K(G,n) \xrightarrow{s(0)} K(G,n) \xrightarrow{s(1)} \cdots .  \tag*{\qed}
    \]
\end{cor}

In particular, it follows from \cite{DoornRijkeSojakova} that the
$\deg(S)$-localization of $K(G,n)$ is $n$-truncated and $(n-1)$-connected,
so it is again a $K(G',n)$ for some group $G'$.
By \cref{lemma:localizationlocalizesfirst}, $G'$ is the algebraic localization of $G$
away from $S$ (in the category of groups if $n = 1$ and abelian groups otherwise).
We deduce:

\begin{thm}\label{theorem:localizationKgn}
    Let $n\geq 1$ and let $G$ be any abelian group.
    Then the $\degg(S)$-localization map $K(G,n) \to L_{\degg(S)} K(G,n)$ is a map between Eilenberg--Mac Lane spaces and
    is induced by an algebraic localization of $G$ away from $S$.
    Moreover, the algebraic localization of $G$ is equivalent to the colimit
    \[
      G \xrightarrow{s(0)} G \xrightarrow{s(1)} \cdots
    \]
    and is therefore abelian, even when $n = 1$.  \qed
\end{thm}

It follows that the two types of algebraic localization agree for abelian groups,
so we do not need to distinguish between them in what follows.

\subsection{Localization of homotopy groups}\label{ss:localization-of-homotopy-groups}

In this section we prove the main theorem of the paper:
for simply connected types, $\degg(S)$-localization localizes all homotopy groups away from $S$.
We continue to assume that our family $S$ is indexed by the natural numbers,
as the essential ingredient is our result on the localization of Eilenberg--Mac Lane spaces from the previous section.
%We use this and \cref{lemma:lex} to induct over the Postnikov tower.
The other ingredient we need is that for simply connected types,
truncation commutes with $\degg(S)$-localization, which follows
from the next lemma.

\begin{lem}\label{lemma:locofscntrunc}
    The $\degg(S)$-localization of a simply connected, $n$-truncated type is $n$-truncated.
\end{lem}

\begin{proof}
    First, we show that it suffices to prove the statement for pointed types.
    Let us assume given a simply connected and $n$-truncated type $X$, and let us denote the statement
    of the theorem by $P(X)$. If we prove $X \to P(X)$ it follows that $\ttrunc{-1}{X} \to P(X)$,
    since being $n$-truncated is a mere proposition. But this is enough, for $X$ is
    simply connected, which implies that $\ttrunc{1}{X}$ is contractible, and hence that
    $\ttrunc{-1}{X}$ is inhabited.

    We now assume that $X$ is pointed and proceed by induction.
    If $X$ is $-2$, $-1$, $0$ or $1$-truncated, we are done, since $X$ is also simply connected, and thus contractible.
    If $X$ is $(n+1)$-truncated and $n > 0$, consider the fiber sequence
    \[ K(G,n+1) \longhookrightarrow X \lra \ttrunc{n}{X}. \]
    Since the types in the fiber sequence are simply connected, $\degg(S)$-localization
    preserves this fiber sequence, by \cref{lemma:lex}. So we obtain a fiber sequence
    \[L_{\degg(S)} K(G,n+1) \longhookrightarrow L_{\degg(S)}X \lra L_{\degg(S)} \ttrunc{n}{X}. \]
    The type $L_{\degg(S)} \ttrunc{n}{X}$ is $n$-truncated by the induction hypothesis
    and $L_{\degg(S)} K(G,n+1)$ is $(n+1)$-truncated by \cref{theorem:localizationKgn}.
    So $L_{\degg(S)} X$ must be $(n+1)$-truncated as well.
\end{proof}

The proof of \cref{lemma:locofscntrunc} shows that we can compute
the $\degg(S)$-localization of a simply connected $n$-type $X$ by
localizing the Postnikov tower of $X$.

From \cref{lemma:locofscntrunc}, \cref{lemma:truncationpreserveslocal} and
the argument used in \cref{lemma:commutelocalization}, but restricted to simply connected types,
we deduce:

\begin{cor}\label{corollary:localizationandtruncationcommute}
    For simply connected types, $\degg(S)$-localization and $n$-trunca\-tion commute.\qed
\end{cor}

We now give the main theorem of the paper.

\begin{thm}\label{theorem:localizationlocalizes}
    The $\degg(S)$-localization of a pointed, simply connected type $X$ localizes all of the
    homotopy groups away from $S$.
\end{thm}

\begin{proof}
    Since $\degg(S)$-localization preserves simply connectedness, it is immediate that $\pi_1(L_{\degg(S)} X)$ is trivial.
    Now fix $n \geq 1$ and consider the fiber sequence
    \[
        K(\pi_{n+1}(X),n+1) \longhookrightarrow \ttrunc{n+1}{X} \lra \ttrunc{n}{X}.
    \]
    Notice that all of the types in the fiber sequence are simply connected.
    Applying $\degg(S)$-localization we obtain a map of fiber sequences
\begin{equation*}
\begin{tikzcd}
K(\pi_{n+1}(X),n+1) \arrow[d] \arrow[r,hookrightarrow] & \ttrunc{n+1}{X} \arrow[d] \arrow[r] & \ttrunc{n}{X} \arrow[d] \\
L_{\degg(S)} K(\pi_{n+1}(X),n+1) \arrow[r,hookrightarrow] & \ttrunc{n+1}{L_{\degg(S)}X} \arrow[r] & \ttrunc{n}{L_{\degg(S)} X}
\end{tikzcd}
\end{equation*} 
    by \cref{lemma:lex} combined with the commutativity of truncation and
    $\degg(S)$-localization (\cref{corollary:localizationandtruncationcommute}).
    The result now follows from \cref{theorem:localizationKgn}.
\end{proof}

We also have a partial converse to \cref{theorem:localizationlocalizes}.

\begin{thm}\label{theorem:characterize-localization}
  Let $X$ and $X'$ be pointed, simply connected $n$-types, for some $n \geq 0$.
  If $f : X \to X'$ is a pointed map such that $\pi_m(f) : \pi_m(X) \to \pi_m(X')$ is
  an algebraic localization away from $S$ for each $m > 1$,
  then $f$ is a $\degg(S)$-localization of $X$.
\end{thm}

\begin{proof}
  By \cref{prop:homotopygroupsoflocalarelocal}, $X'$ is $\degg(S)$-local.
  Therefore, we have a commuting triangle
  \[
    \begin{tikzcd}
      X \arrow[r,"f"] \arrow[d,swap,"\eta"] & X' \\
      L_{\degg(S)} X \arrow[ur,dashed,swap,"f'"]
    \end{tikzcd}
  \]
  and it suffices to show that $f'$ is an equivalence.
  Since $L_{\degg(S)} X$ is also a pointed, simply connected $n$-type,
  if we show that $\pi_m(f')$ is a bijection for each $m > 1$,
  the truncated Whitehead theorem~\cite[Theorem~8.8.3]{hottbook} implies
  that $f'$ is an equivalence.
  By assumption, $\pi_m(f)$ is an algebraic localization of $\pi_m(X)$.
  By \cref{theorem:localizationlocalizes}, the same is true of $\pi_m(\eta)$.
  Since $\pi_m(f') \circ \pi_m(\eta) = \pi_m(f)$, it follows that
  $\pi_m(f')$ is a bijection.
\end{proof}

\subsection{Algebraic localization of abelian groups}\label{ss:abelian}

As observed in \cref{rmk:localizationofgroups},
\cref{theorem:localizationKgn} implies that the localization of an abelian group
away from a set $S$, in the category of groups, coincides with its localization in
the category of abelian groups. It moreover provides a construction of the localization.
Since we could not find these results in the literature, and since \cref{theorem:localizationKgn}
has an indirect, homotopical proof, in this section we give a short, independent, algebraic proof
of these results.

The following elementary lemma is the key ingredient.
 
\begin{lem}
    Let $H$ be a group, let $n,m : \N$, and let $x$, $y$, $\hat{x}$ and $\hat{y}$ be
    elements of $H$ with the property that
    $\hat{x}$ is the unique element of $H$ such that $\hat{x}^n = x$ and
    $\hat{y}$ is the unique element of $H$ such that $\hat{y}^m = y$.
    If $x$ and $y$ commute, then $\hat{x}$ and $\hat{y}$ commute.
\end{lem}

\begin{proof}
    We first show that $x$ and $\hat{y}$ commute.
    The $m$-th power of $x \hat{y} x^{-1}$ is $x \hat{y}^m x^{-1} = x y x^{-1} = y$,
    so by the uniqueness of $\hat{y}$, it must be the case that $x \hat{y} x^{-1} = \hat{y}$.

    Similarly, we have that
    the $n$-th power of $\hat{y} \hat{x} \hat{y}^{-1}$ is
    $\hat{y} \hat{x}^n \hat{y}^{-1} = \hat{y} x \hat{y}^{-1} = x$,
    so by the uniqueness of $\hat{x}$, we have $\hat{y} \hat{x} \hat{y}^{-1} = \hat{x}$.
\end{proof}

\begin{prp}
    Let $G$ be an abelian group, let $n : \N$, and let $n : G \to G$ be the multiplication by $n$ map.
    Then, for every uniquely $n$-divisible group $H$, the precomposition map
    \[
        n^* : \Hom(G,H) \lra \Hom(G,H)
    \]
    is an equivalence.
\end{prp}

\begin{proof}
    Given a map $f : G \to H$, we have to show that there exists a unique $\hat{f} : G \to H$ such that
    $\hat{f} \circ n = f$.
    Since $H$ is uniquely $n$-divisible, we have an inverse function $\phi : H \to H$ to the $n$-th power map $n : H \to H$.
    It follows that if $\hat{f}$ exists, then $\hat{f} = \phi \circ f$.
    So we only have to check that $\hat{f}$ is a group homomorphism.

    It is clear that $\hat{f}$ preserves the identity element, so it remains to show that it preserves
    the group operation. Take two elements $a,b : G$, and consider $f(a),f(b) : H$. These are two commuting
    elements that have unique $n$-th roots. So their $n$-th roots $\phi(f(a)), \phi(f(b)) : H$ must also
    commute. This implies that
    \[
        (\hat{f}(a) \cdot \hat{f}(b))^n = \hat{f}(a)^n \cdot \hat{f}(b)^n = f(a) \cdot f(b) = f(a + b) .
    \]
    So by the uniqueness of $n$-th roots, $\hat{f}(a) \cdot \hat{f}(b) = \phi(f(a + b)) = \hat{f}(a+b)$.
\end{proof}

Using this result, given an abelian group $G$,
it is straightforward to prove that the colimit of the sequence displayed in
\cref{theorem:localizationKgn} is a localization of $G$ away from the family $S$
in the category of groups.
Moreover, this colimit is abelian, so it is also the localization of $G$
in the category of abelian groups.
