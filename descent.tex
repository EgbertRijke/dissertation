\chapter{Type theoretic descent}\label{chap:descent}

In this chapter we study homotopy pushouts, which were established in homotopy type theory as higher inductive types in section 6.8 of \cite{hottbook}. From this chapter on, we will assume that universes are closed under homotopy pushouts. This is the last assumption that we will be making in the present work. In particular, we will not assume the existence of higher inductive types with some self-reference in the constructors (e.g.~the propositional truncation).

Our first main result is the descent theorem for homotopy pushouts (\cref{thm:descent,cor:descent_fib}), in which we establish that a cartesian transformation of spans
\begin{equation*}
\begin{tikzcd}
A' \arrow[d]  & S' \arrow[l] \arrow[r] \arrow[d] \arrow[dl,phantom,"\llcorner" very near start] \arrow[dr,phantom,"\lrcorner" very near start] & B' \arrow[d] \\
A & S \arrow[l] \arrow[r] & B
\end{tikzcd}
\end{equation*}
extends uniquely to a cartesian transformation of the pushout squares, i.e. a commuting cube
\begin{equation*}
\begin{tikzcd}
& S' \arrow[dl] \arrow[dr] \arrow[d] \\
A' \arrow[d] & S \arrow[dl] \arrow[dr] & B' \arrow[dl,crossing over] \arrow[d] \\
A \arrow[dr] & A'\sqcup^{S'}B' \arrow[d] \arrow[from=ul,crossing over] & B \arrow[dl] \\
& A\sqcup^S B
\end{tikzcd}
\end{equation*}
of which the vertical sides are pullback squares. Here we need univalence to establish uniqueness.

The second main theorem of this chapter, \cref{thm:cartesian_cube}, is an adaption to homotopy type theory of a theorem due to \cite{AnelBiedermanFinsterJoyal}. It is closely related to the descent theorem but can be stated without a universe: for any commuting cube
\begin{equation*}
\begin{tikzcd}
& S' \arrow[dl] \arrow[dr] \arrow[d] \\
A' \arrow[d] & S \arrow[dl] \arrow[dr] & B' \arrow[dl,crossing over] \arrow[d] \\
A \arrow[dr] & X' \arrow[d] \arrow[from=ul,crossing over] & B \arrow[dl] \\
& X
\end{tikzcd}
\end{equation*}
of which the two vertical back squares are pullback squares, the two vertical front squares are pullback squares if and only if the square
\begin{equation*}
\begin{tikzcd}
A' \sqcup^{S'} B' \arrow[r] \arrow[d] & X' \arrow[d] \\
A\sqcup^{S} B \arrow[r] & X.
\end{tikzcd}
\end{equation*}
is a pullback square. Even though this statement does not involve a universe, we use the univalence axiom in our proof that this square being pullback implies that the front two vertical squares of the cube are pullback squares. Function extensionality suffices for the converse direction.

\section{Homotopy pushouts}

\subsection{Pushouts as higher inductive types}

\begin{defn}
A \define{span} $\mathcal{S}$ from $A$ to $B$ is a triple $(S,f,g)$ consisting of a type $S$ and maps $f:S\to A$ and $g:S\to B$. We write $\mathsf{span}(A,B)$ for the type of small spans from $A$ to $B$. 
\end{defn}

\begin{defn}
Consider a span $\mathcal{S}\jdeq (S,f,g)$ from $A$ to $B$, and let $X$ be a type. A cocone with vertex $X$ on $\mathcal{S}$ is a triple $(i,j,H)$ consisting of maps $i:A\to X$, $j:B\to X$, and a homotopy $H:i\circ f\htpy j\circ g$ witnessing that the square
\begin{equation*}
\begin{tikzcd}
S \arrow[d,swap,"f"] \arrow[r,"g"] & B \arrow[d,"j"] \\
A \arrow[r,swap,"i"] & X
\end{tikzcd}
\end{equation*}
commutes. We write $\mathsf{cocone}_{\mathcal{S}}(X)$ for the type of cocones with vertex $X$ on $\mathcal{S}$. 
\end{defn}

\begin{defn}
Consider a commuting square
\begin{equation*}
\begin{tikzcd}
S \arrow[d,swap,"f"] \arrow[r,"g"] & B \arrow[d,"j"] \\
A \arrow[r,swap,"i"] & X,
\end{tikzcd}
\end{equation*}
with $H:i\circ f\htpy j\circ g$, and let $Y$ be a type. We define the operation
\begin{equation*}
\mathsf{cocone\usc{}map}((i,j,H),Y) \defeq (X\to Y) \to \mathsf{cocone}_{\mathcal{S}}(Y).
\end{equation*}
by $h\mapsto (h\circ i,h\circ j,h\cdot H)$. 
\end{defn}

\begin{defn}
A commuting square
\begin{equation*}
\begin{tikzcd}
S \arrow[r,"g"] \arrow[d,swap,"f"] & B \arrow[d,"j"] \\
A \arrow[r,swap,"i"] & X
\end{tikzcd}
\end{equation*}
with $H:i\circ f \htpy j\circ g$ is said to be a \define{(homotopy) pushout square}\index{pushout square} if the cocone $(i,j,H)$ with vertex $X$ on the span $\mathcal{S}\jdeq (S,f,g)$
satisfies the \define{universal property of pushouts}\index{universal property!of pushouts|textbf}, which asserts that the map
\begin{equation*}
\mathsf{cocone\usc{}map}(i,j,H):(X\to Y)\to \mathsf{cocone}(Y)
\end{equation*}
is an equivalence for any type $Y$. Sometimes pushout squares are also called \define{cocartesian squares}\index{cocartesian square|textbf}.
\end{defn}

\begin{defn}
Consider a pushout square
\begin{equation*}
\begin{tikzcd}
S \arrow[r,"g"] \arrow[d,swap,"f"] & B \arrow[d,"j"] \\
A \arrow[r,swap,"i"] & X
\end{tikzcd}
\end{equation*}
with $H:i\circ f \htpy j\circ g$, and consider a cocone $(i',j',H')$ with vertex $Y$ on the same span $\mathcal{S}\jdeq(S,f,g)$. Then the unique map $h:X\to Y$ such that 
\begin{equation*}
\mathsf{cocone\usc{}map}((i,j,H),Y,h)= (i',j',H')
\end{equation*}
is called the \define{cogap map} of $(i',j',H')$. We also write $\mathsf{cogap}(i',j',H')$ for the cogap map, and we write
\begin{align*}
\mathsf{left\usc{}comp}(i',j',H') & : i' \htpy \mathsf{cogap}(i',j',H')\circ \inl  \\
\mathsf{right\usc{}comp}(i',j',H') & : j' \htpy \mathsf{cogap}(i',j',H')\circ \inr  \\
\mathsf{coh\usc{}comp}(i',j',H') & : \ct{(\mathsf{left\usc{}comp}(i',j',H')\cdot g)}{H'} \htpy \ct{(\mathsf{cogap}(i',j',H')\cdot \glue)}{(\mathsf{right\usc{}comp}(i',j',H')\cdot f)}.
\end{align*}
for the homotopies determining the uniqueness of $\mathsf{cogap}(i',j',H')$.
\end{defn}

\begin{prp}\label{thm:pushout_up}
Consider a commuting square\index{universal property!of pushouts|textit}
\begin{equation*}
\begin{tikzcd}
S \arrow[r,"g"] \arrow[d,swap,"f"] & B \arrow[d,"j"] \\
A \arrow[r,swap,"i"] & X,
\end{tikzcd}
\end{equation*}
with $H:i\circ f\htpy j\circ g$. The following are equivalent:
\begin{enumerate}
\item The square is a pushout square.
\item The square
\begin{equation*}
\begin{tikzcd}
Y^X \arrow[r,"\blank\circ j"] \arrow[d,swap,"\blank\circ i"] & Y^B \arrow[d,"\blank\circ g"] \\
Y^A \arrow[r,swap,"\blank\circ f"] & Y^S,
\end{tikzcd}
\end{equation*}
which commutes by the homotopy
\begin{equation*}
\lam{h} \mathsf{eq\usc{}htpy}(h\cdot H),
\end{equation*}
is a pullback square, for every type $Y$.
\item For every type family $P$ over $X$, the square
\begin{equation*}
\begin{tikzcd}[column sep=9em]
\prd{x:X}P(x) \arrow[r,"\blank\circ j"] \arrow[d,swap,"\blank\circ i"] & \prd{b:B}P(j(b)) \arrow[d,"\blank\circ g"] \\
\prd{a:A}P(i(a)) \arrow[r,swap,"{\lam{h}{x} \tr_{P}(H(x),h(f(x)))}"] & \prd{x:S}P(j(g(x))),
\end{tikzcd}
\end{equation*}
which commutes by the homotopy
\begin{equation*}
\lam{h}\mathsf{eq\usc{}htpy}(\lam{x}\apd{h}{H(x)})
\end{equation*}
is a pullback square. This property is also called the \define{dependent universal property of pushouts}.
\item The gap map of the square
\begin{equation*}
\begin{tikzcd}
\prd{x:X}P(x) \arrow[r] \arrow[d] & \prd{b:B}P(j(b)) \arrow[d] \\
\prd{a:A}P(i(a)) \arrow[r] & \prd{x:S}P(j(g(x)))
\end{tikzcd}
\end{equation*}
has a section, for any type family $P$ over $X$. This property is also called the \define{induction principle of pushouts}.
\end{enumerate}
\end{prp}

\begin{defn}
From now on we will assume that any span has a pushout, and moreover that universes are closed under pushouts. We will write $A\sqcup^{\mathcal{S}} B$ for the pushout of the span $\mathcal{S}\jdeq(S,f,g)$ from $A$ to $B$. The type $A\sqcup^{\mathcal{S}} B$ comes equipped with a colimiting cocone $(\inl,\inr,\glue)$, as displayed in the pushout square
\begin{equation*}
\begin{tikzcd}
S \arrow[d,swap,"f"] \arrow[r,"g"] & B \arrow[d,"\inr"] \\
A \arrow[r,swap,"\inl"] & A \sqcup^{\mathcal{S}} B.
\end{tikzcd}
\end{equation*}
\end{defn}

\begin{rmk}
We note that if $\mathcal{S}\jdeq (S,f,g)$ is a span of \emph{pointed} and pointed maps between them, then the pushout $A\sqcup^{\mathcal{S}} B$ of $\mathcal{S}$ is again a pointed type. The cocone $(\inl,\inr,\glue)$ consists of two pointed maps and a pointed homotopy filling the square of pointed maps. Moreover, the pushout $A\sqcup^{\mathcal{S}} B$ satisfies a pointed version of the universal property: for any pointed type $Y$ the square
\begin{equation*}
\begin{tikzcd}
(A\sqcup^{\mathcal{S}} B \to_\ast Y) \arrow[r] \arrow[d] & (B\to_\ast Y) \arrow[d] \\
(A\to_\ast Y) \arrow[r] & (S\to_\ast Y)
\end{tikzcd}
\end{equation*}
is a pullback square. 
\end{rmk}

\begin{comment}
\begin{proof}
It is straightforward to verify that the triangle
\begin{equation*}
\begin{tikzcd}
\phantom{T^A \times_{T^S} T^B} & T^X \arrow[dl,swap,"{\mathsf{cocone\usc{}map}(i,j,H)}"] \arrow[dr,"{\mathsf{gap}(\blank\circ i,\blank\circ j, \mathsf{eq\usc{}htpy}(\blank\cdot H))}"] & \phantom{\mathsf{cocone}(T)} \\
\mathsf{cocone}(T) \arrow[rr,swap,"{\mathsf{gap}(i,j,\mathsf{eq\usc{}htpy}(H))}"] & & T^A \times_{T^S} T^B
\end{tikzcd}
\end{equation*}
commutes. Since the bottom map is an equivalence by \cref{lem:cocone_pb}, it follows that if either one of the remaining maps is an equivalence, so is the other. The claim now follows by \cref{thm:is_pullback}.
\end{proof}
\end{comment}

\subsection{Examples of pushouts}

\begin{defn}
Let $X$ be a type. We define the \define{suspension}\index{suspension|textbf} $\susp X$\index{SX@{$\susp X$}|textbf} of $X$ to be the pushout of the span
\begin{equation*}
\begin{tikzcd}
X \arrow[r] \arrow[d] & \unit \arrow[d,"\inr"] \\
\unit \arrow[r,swap,"\inl"] & \susp X 
\end{tikzcd}
\end{equation*}
We will write $\north\defeq\inl(\ttt)$ and $\south\defeq\inr(\ttt)$. 
\end{defn}

\begin{rmk}By the universal property it follows that the map
\begin{equation*}
(\susp X \to Y) \to \sm{y,y':Y}X\to (y=y')
\end{equation*}
given by $h\mapsto (h(\north),h(\south),h\cdot\glue)$ is an equivalence. 

Moreover, if $X$ is a pointed type, then the suspension is considered to be a pointed type with base point $\north$. By the universal property of $\susp X$ it follows that the square
\begin{equation*}
\begin{tikzcd}
(\susp X\to_\ast Y) \arrow[r] \arrow[d] & (\unit \to_\ast Y) \arrow[d] \\
(\unit\to_\ast Y) \arrow[r] & (X\to_\ast Y)
\end{tikzcd}
\end{equation*}
is a pullback square. Since $\unit\to_\ast Y$ is contractible, it follows that 
\begin{equation*}
(\susp X\to_\ast Y) \eqvsym \loopspace{X\to_\ast Y} \eqvsym X \to_\ast \loopspace Y
\end{equation*}
\end{rmk}

\begin{defn}
Given a map $f:A\to B$, we define the \define{cofiber}\index{cofiber|textbf} $\mathsf{cof}_f$\index{cofib_f@{$\mathsf{cof}_f$}|textbf} of $f$ as the pushout
\begin{equation*}
\begin{tikzcd}
A \arrow[r,"f"] \arrow[d] & B \arrow[d,"\inr"] \\
\unit \arrow[r,swap,"\inl"] & \mathsf{cof}_f. 
\end{tikzcd}
\end{equation*}
The cofiber of a map is sometimes also called the \define{mapping cone}\index{mapping cone|textbf}.
\end{defn}

\begin{defn}
We define the \define{join}\index{join} $\join{X}{Y}$\index{join X Y@{$\join{X}{Y}$}|textbf} of $X$ and $Y$ to be the pushout 
\begin{equation*}
\begin{tikzcd}
X\times Y \arrow[r,"\proj 2"] \arrow[d,swap,"\proj 1"] & Y \arrow[d,"\inr"] \\
X \arrow[r,swap,"\inl"] & X \ast Y. 
\end{tikzcd}
\end{equation*}
\end{defn}

\begin{defn}
We define the \define{$n$-sphere}\index{n-sphere@{$n$-sphere}|textbf} $\sphere{n}$\index{Sn@{$\sphere{n}$}|textbf} for any $n\geq -1$ by induction on $n$, by taking
\begin{align*}
\sphere{-1} & \defeq \emptyt \\
\sphere{0} & \defeq \bool \\
\sphere{n+1} & \defeq \join{\bool}{\sphere{n}}.
\end{align*}
\end{defn}


\begin{defn}
Suppose $A$ and $B$ are pointed types, with base points $a_0$ and $b_0$, respectively. The \define{(binary) wedge}\index{wedge@(binary) wedge|textbf} $A\vee B$ of $A$ and $B$ is defined as the pushout
\begin{equation*}
\begin{tikzcd}
\bool \arrow[r] \arrow[d] & A+B \arrow[d] \\
\unit \arrow[r] & A\vee B.
\end{tikzcd}
\end{equation*}
\end{defn}

\begin{defn}
Given a type $I$, and a family of pointed types $A$ over $i$, with base points $a_0(i)$. We define the \define{(indexed) wedge}\index{wedge@{(indexed) wedge}|textbf} $\bigvee_{(i:I)}A_i$ as the pushout
\begin{equation*}
\begin{tikzcd}[column sep=huge]
I \arrow[d] \arrow[r,"{\lam{i}(i,a_0(i))}"] & \sm{i:I}A_i \arrow[d] \\
\unit \arrow[r] & \bigvee_{(i:I)} A_i.
\end{tikzcd}
\end{equation*}
\end{defn}

\begin{defn}
Suppose $A$ and $B$ are pointed types. We define the \define{wedge inclusion} $\mathsf{wedge\usc{}in}:A\vee B\to A\times B$ to be the unique map obtained via the universal property of pushouts as indicated in the diagram
\begin{equation*}
\begin{tikzcd}
\unit \arrow[d] \arrow[r] &[1em] B \arrow[d] \arrow[ddr,bend left=15,"{\lam{b}(a_0,b)}"] \\
A \arrow[r] \arrow[drr,bend right=15,swap,"{\lam{a}(a,b_0)}"] & A\vee B \arrow[dr,swap,"\mathsf{wedge\usc{}in}" near start] \\
& & A \times B.
\end{tikzcd}
\end{equation*}
We define the \define{smash product} $A\wedge B$ of $A$ and $B$ as the cofiber of the wedge inclusion, i.e.~as a pushout
\begin{equation*}
\begin{tikzcd}[column sep=large]
A\vee B \arrow[r,"\mathsf{wedge\usc{}in}"] \arrow[d] & A\times B \arrow[d] \\
\unit \arrow[r] & A\wedge B.
\end{tikzcd}
\end{equation*}
\end{defn}

\subsection{Properties of iterated pushouts}
\begin{prp}\label{thm:pushout_pasting}
Consider the following configuration of commuting squares:\index{pushout!pasting property|textit}\index{pasting property!for pushouts|textit}
\begin{equation*}
\begin{tikzcd}
A \arrow[r,"i"] \arrow[d,swap,"f"] & B \arrow[r,"k"] \arrow[d,swap,"g"] & C \arrow[d,"h"] \\
X \arrow[r,swap,"j"] & Y \arrow[r,swap,"l"] & Z
\end{tikzcd}
\end{equation*}
with homotopies $H:j\circ f\htpy g\circ i$ and $K:l\circ g\htpy h\circ k$, and suppose that the square on the left is a pushout square. 
Then the square on the right is a pushout square if and only if the outer rectangle is a pushout square.
\end{prp}

\begin{proof}
Let $T$ be a type. Taking the exponent $T^{(\blank)}$ of the entire diagram of the statement of the theorem, we obtain the following commuting diagram
\begin{equation*}
\begin{tikzcd}
T^Z \arrow[r,"\blank\circ l"] \arrow[d,swap,"\blank\circ h"] & T^Y \arrow[d,swap,"\blank\circ g"] \arrow[r,"\blank\circ j"] & T^X \arrow[d,"\blank\circ f"] \\
T^C \arrow[r,swap,"\blank\circ k"] & T^B \arrow[r,swap,"\blank\circ i"] & T^A.
\end{tikzcd}
\end{equation*}
By the assumption that $Y$ is the pushout of $B\leftarrow A \rightarrow X$, it follows that the square on the right is a pullback square. It follows by \autoref{thm:pb_pasting} that the rectangle on the left is a pullback if and only if the outer rectangle is a pullback. Thus the statement follows by the second characterization in \autoref{thm:pushout_up}.
\end{proof}

\begin{lem}
Consider a map $f:A\to B$. Then the cofiber of the map $\inr:B\to \mathsf{cof}_f$ is equivalent to the suspension $\susp{A}$ of $A$. 
\end{lem}

\begin{prp}
Consider a commuting square
\begin{equation*}
\begin{tikzcd}
A \arrow[r,"i"] \arrow[d,swap,"f"] & B \arrow[d,"g"] \\
X \arrow[r,swap,"j"] & Y
\end{tikzcd}
\end{equation*}
and write $\mathsf{cogap}: X\sqcup^A B\to Y$ for the cogap map. 
Then the square
\begin{equation*}
\begin{tikzcd}
\mathsf{cof}_f \arrow[d] \arrow[r] & \mathsf{cof}_g \arrow[d] \\
\unit \arrow[r] & \mathsf{cof}_{\mathsf{cogap}}
\end{tikzcd}
\end{equation*}
is a pushout square.
\end{prp}

\section{Descent for pushouts}\label{sec:descent}

\subsection{Type families over pushouts}

\begin{defn}
Consider a commuting square
\begin{equation*}
\begin{tikzcd}
S \arrow[r,"g"] \arrow[d,swap,"f"] & B \arrow[d,"j"] \\
A \arrow[r,swap,"i"] & X.
\end{tikzcd}
\end{equation*}
with $H:i\circ f\htpy j\circ g$, where all types involved are in $\UU$. The type $\mathsf{Desc}(\mathcal{S})$\index{Desc@{$\mathsf{Desc}(\mathcal{S})$}|textbf} of \define{descent data}\index{descent data|textbf} for $X$, is defined to be the type of triples $(P_A,P_B,P_S)$ consisting of
\begin{align*}
P_A & : A \to \UU \\
P_B & : B \to \UU \\
P_S & : \prd{x:S} \eqv{P_A(f(x))}{P_B(g(x))}.
\end{align*}
Furthermore, we define the map\index{desc_fam@{$\mathsf{desc\usc{}fam}_{\mathcal{S}}$}|textbf}
\begin{equation*}
\mathsf{desc\usc{}fam}_{\mathcal{S}}(i,j,H) : (X\to \UU)\to \mathsf{Desc}(\mathcal{S})
\end{equation*}
by $P\mapsto (P\circ i,P\circ j,\lam{x}\mathsf{tr}_P(H(x)))$.
\end{defn}

\begin{prp}\label{thm:desc_fam}
Consider a commuting square
\begin{equation*}
\begin{tikzcd}
S \arrow[r,"g"] \arrow[d,swap,"f"] & B \arrow[d,"j"] \\
A \arrow[r,swap,"i"] & X.
\end{tikzcd}
\end{equation*}
with $H:i\circ f\htpy j\circ g$. If the square is a pushout square, then the function\index{desc_fam@{$\mathsf{desc\usc{}fam}_{\mathcal{S}}$}!is an equivalence|textit}
\begin{equation*}
\mathsf{desc\usc{}fam}_{\mathcal{S}}(i,j,H) : (X\to \UU)\to \mathsf{Desc}(\mathcal{S})
\end{equation*}
is an equivalence.
\end{prp}

\begin{proof}
By the 3-for-2 property of equivalences it suffices to construct an equivalence $\varphi:\mathsf{cocone}_{\mathcal{S}}(\UU)\to\mathsf{Desc}(\mathcal{S})$ such that the triangle
\begin{equation*}
\begin{tikzcd}
& \UU^X \arrow[dl,swap,"{\mathsf{cocone\usc{}map}_{\mathcal{S}}(i,j,H)}"] \arrow[dr,"{\mathsf{desc\usc{}fam}_{\mathcal{S}}(i,j,H)}"] & \phantom{\mathsf{cocone}_{\mathcal{S}}(\UU)} \\
\mathsf{cocone}_{\mathcal{S}}(\UU) \arrow[rr,densely dotted,"\eqvsym","\varphi"'] & & \mathsf{Desc}(\mathcal{S})
\end{tikzcd}
\end{equation*}
commutes.

Since we have equivalences
\begin{equation*}
\mathsf{equiv\usc{}eq}:\eqv{\Big(P_A(f(x))=P_B(g(x))\Big)}{\Big(\eqv{P_A(f(x))}{P_B(g(x))}\Big)}
\end{equation*}
for all $x:S$, we obtain an equivalence on the dependent products
\begin{equation*}
\eqv{\Big(\prd{x:S}P_A(f(x))=P_B(g(x))\Big)}{\Big(\prd{x:S}\eqv{P_A(f(x))}{P_B(g(x))}\Big)}.
\end{equation*}
by post-composing with the equivalences $\mathsf{equiv\usc{}eq}$. 
We define $\varphi$ to be the induced map on total spaces. Explicitly, we have
\begin{equation*}
\varphi\defeq \lam{(P_A,P_B,K)}(P_A,P_B,\lam{x}\mathsf{equiv\usc{}eq}(K(x))).
\end{equation*}
Then $\varphi$ is an equivalence by \cref{thm:fib_equiv}, and the triangle commutes because there is a homotopy
\begin{equation*}
\mathsf{equiv\usc{}eq}(\ap{P}{H(x)}) \htpy \mathsf{tr}_P(H(x)). \qedhere
\end{equation*}
\end{proof}

\begin{cor}\label{cor:desc_fam}
Consider descent data $(P_A,P_B,P_S)$ for a pushout square as in \cref{thm:desc_fam}.
Then the type of quadruples $(P,e_A,e_B,e_S)$ consisting of a family $P:X\to\UU$ equipped with fiberwise equivalences
\begin{samepage}
\begin{align*}
e_A & : \prd{a:A}\eqv{P_A(a)}{P(i(a))} \\
e_B & : \prd{b:B}\eqv{P_B(a)}{P(j(b))}
\end{align*}
\end{samepage}%
and a homotopy $e_S$ witnessing that the square
\begin{equation*}
\begin{tikzcd}[column sep=huge]
P_A(f(x)) \arrow[r,"e_A(f(x))"] \arrow[d,swap,"P_S(x)"] & P(i(f(x))) \arrow[d,"\mathsf{tr}_P(H(x))"] \\
P_B(g(x)) \arrow[r,swap,"e_B(g(x))"] & P(j(g(x)))
\end{tikzcd}
\end{equation*}
commutes, is contractible.
\end{cor}

\begin{proof}
The fiber of $\mathsf{desc\usc{}fam}_{\mathcal{S}}(i,j,H)$ map at $(P_A,P_B,P_S)$ is equivalent to the type of quadruples $(P,e_A,e_B,e_S)$ as described in the theorem, which are contractible by \cref{thm:contr_equiv}.
\end{proof}

For the remainder of this subsection we consider a pushout square
\begin{equation*}
\begin{tikzcd}
S \arrow[r,"g"] \arrow[d,swap,"f"] & B \arrow[d,"j"] \\
A \arrow[r,swap,"i"] & X.
\end{tikzcd}
\end{equation*}
with $H:i\circ f\htpy j\circ g$, descent data
\begin{align*}
P_A & : A \to \UU \\
P_B & : B \to \UU \\
P_S & : \prd{x:S} \eqv{P_A(f(x))}{P_B(g(x))},
\end{align*}
and a family $P:X\to\UU$ equipped with 
\begin{align*}
e_A & : \prd{a:A}\eqv{P_A(a)}{P(i(a))} \\
e_B & : \prd{b:B}\eqv{P_B(a)}{P(j(b))}
\end{align*}
and a homotopy $e_S$ witnessing that the square
\begin{equation*}
\begin{tikzcd}[column sep=huge]
P_A(f(x)) \arrow[r,"e_A(f(x))"] \arrow[d,swap,"P_S(x)"] & P(i(f(x))) \arrow[d,"\mathsf{tr}_P(H(x))"] \\
P_B(g(x)) \arrow[r,swap,"e_B(g(x))"] & P(j(g(x)))
\end{tikzcd}
\end{equation*}
commutes.

\begin{defn}
We define the commuting square
\begin{equation*}
\begin{tikzcd}[column sep=6em]
\sm{x:S}P_A(f(x)) \arrow[d,swap,"{f'\,\defeq\,\total[f]{\lam{x}\idfunc[P_A(f(x))]}}"] \arrow[r,"{g'\,\defeq\, \total[g]{e_S}}"] & \sm{b:B}P_B(b) \arrow[d,"{j'\,\defeq\, \total[j]{e_B}}"] \\
\sm{a:A}P_A(a) \arrow[r,swap,"{i'\, \defeq\, \total[i]{e_A}}"] & \sm{x:X}P(x)
\end{tikzcd}
\end{equation*}
with the homotopy $H':i'\circ f'\htpy j'\circ g'$ defined as
\begin{equation*}
\lam{(x,y)}\mathsf{eq\usc{}pair}(H(x),e_S(x,y)^{-1}).
\end{equation*}
Furthermore, we will write $\mathcal{S'}$ for the span
\begin{equation*}
\begin{tikzcd}
\sm{a:A}P_A(a) & \sm{x:S}P_A(f(x)) \arrow[l,swap,"{f'}"] \arrow[r,"{g'}"] & \sm{b:B}P_B(b).
\end{tikzcd}
\end{equation*}
\end{defn}

We now state the flattening lemma for pushouts, which should be compared to the flattening lemma for coequalizers, stated in Lemma 6.12.2 of \cite{hottbook}. We note that, using the dependent universal property of pushouts, our proof is substantially shorter.

\begin{lem}[The flattening lemma]\label{lem:flattening}
The commuting square\index{flattening lemma!for pushouts|textit}
\begin{equation*}
\begin{tikzcd}
\sm{x:S}P_A(f(x)) \arrow[d,swap,"{f'}"] \arrow[r,"{g'}"] & \sm{b:B}P_B(b) \arrow[d,"{j'}"] \\
\sm{a:A}P_A(a) \arrow[r,swap,"{i'}"] & \sm{x:X}P(x)
\end{tikzcd}
\end{equation*}
is a pushout square.
\end{lem}

\begin{proof}
Note that we have a commuting cube
\begin{equation*}
\begin{tikzcd}[row sep=large]
& Y^{\sm{x:X}P(x)} \arrow[dl] \arrow[d,"\mathsf{ev\usc{}pair}"] \arrow[dr] \\
Y^{\sm{a:A}P_A(a)} \arrow[d,swap,"\mathsf{ev\usc{}pair}"] & \prd{x:X}Y^{P(x)} \arrow[dl] \arrow[dr] & Y^{\sm{b:B}P_B(b)} \arrow[dl,crossing over] \arrow[d,"\mathsf{ev\usc{}pair}"] \\
\prd{a:A}Y^{P_A(a)} \arrow[dr] & Y^{\sm{x:S}P_A(f(x))} \arrow[from=ul,crossing over] \arrow[d,swap,"\mathsf{ev\usc{}pair}"] & \prd{b:B}Y^{P_B(b)} \arrow[dl] \\
\phantom{\prd{b:B}Y^{P_B(b)}} & \prd{x:S}Y^{P_A(f(x))} & \phantom{\prd{a:A}Y^{P_A(a)}}
\end{tikzcd}
\end{equation*}
for any type $Y$. In this cube, the bottom square is a pullback square by property (iii) of \cref{thm:pushout_up}. The vertical maps (of the form $\mathsf{ev\usc{}pair}$) are equivalences, so it follows that the top square is a pullback square. We conclude that $\sm{x:X}P(x)$ is a pushout.
\end{proof}

\subsection{The descent property for pushouts}

\begin{defn}
Consider a span $\mathcal{S}$ from $A$ to $B$, and a span $\mathcal{S}'$ from $A'$ to $B'$. A \define{cartesian transformation} of spans\index{cartesian transformation!of spans|textbf} from $\mathcal{S}'$ to $\mathcal{S}$ is a tuple
\begin{equation*}
(h_A,h_S,h_B,F,G,p_f,p_g)
\end{equation*}
consisting of maps $h_A:A'\to A$, $h_S:S'\to S$, and $h_B:B'\to B$, as indicated in the diagram
\begin{equation*}
\begin{tikzcd}
A' \arrow[d,swap,"h_A"]  & S' \arrow[l,swap,"{f'}"] \arrow[r,"{g'}"] \arrow[d,swap,"h_S"] & B' \arrow[d,"h_B"] \\
A & S \arrow[l,"f"] \arrow[r,swap,"g"] & B,
\end{tikzcd}
\end{equation*}
with homotopies $F:f\circ h_S\htpy h_A\circ f'$ and $G:g\circ h_S\htpy h_B\circ g'$, satisfying the conditions
\begin{align*}
p_f & : \mathsf{is\usc{}pullback}(h_S,f',F) \\
p_g & : \mathsf{is\usc{}pullback}(h_S,g',G)
\end{align*}
that both squares are pullback squares. We write $\mathsf{cart}(\mathcal{S}',\mathcal{S})$\index{cart(S,S')@{$\mathsf{cart}(\mathcal{S},\mathcal{S}')$}|textbf} for the type of cartesian transformations from $\mathcal{S}'$ to $\mathcal{S}$, and we write
\begin{equation*}
\mathsf{Cart}(\mathcal{S}) \defeq \sm{A',B':\UU}{\mathcal{S}':\mathsf{span}(A',B')}\mathsf{cart}(\mathcal{S}',\mathcal{S}).
\end{equation*}
\end{defn}

Given descent data $(P_A,P_B,P_S)$ on a span $\mathcal{S}$ from $A$ to $B$, we obtain a cartesian transformation
\begin{equation*}
\begin{tikzcd}[column sep=large]
\sm{a:A}P_A(a) \arrow[d,swap,"\proj 1"] & \sm{x:S}P_A(f(x)) \arrow[d,swap,"\proj 1"] \arrow[l,swap,"{\total[f]{\idfunc}}"] \arrow[r,"{\total[g]{P_S}}"] & \sm{b:B}P_B(b) \arrow[d,"\proj 1"] \\
A & S \arrow[l,"f"] \arrow[r,swap,"g"] & B
\end{tikzcd}
\end{equation*}
with the canonical homotopies witnessing that the squares commute. Note that both the left and right commuting squares are pullback squares by \cref{thm:pb_fibequiv}. Thus we obtain an operation
\begin{equation*}
\mathsf{cart\usc{}desc}_{\mathcal{S}}:\mathsf{Desc}(\mathcal{S})\to \mathsf{Cart}(\mathcal{S}).
\end{equation*}

\begin{lem}\label{lem:cart_desc}
For any span $\mathcal{S}$, the operation\index{cart_desc@{$\mathsf{cart\usc{}desc}_{\mathcal{S}}$}|textit}
\begin{equation*}
\mathsf{cart\usc{}desc}_{\mathcal{S}}:\mathsf{Desc}(\mathcal{S})\to \mathsf{Cart}(\mathcal{S})
\end{equation*}
is an equivalence.
\end{lem}

\begin{proof}
Note that by \cref{thm:pb_fibequiv_complete} it follows that the types of triples $(f',F,p_f)$ and $(g',G,p_g)$ are equivalent to the types of fiberwise equivalences
\begin{align*}
& \prd{x:S}\eqv{\fib{h_S}{x}}{\fib{h_A}{f(x)}} \\
& \prd{x:S}\eqv{\fib{h_S}{x}}{\fib{h_B}{g(x)}}
\end{align*} 
respectively. Furthermore, by \cref{thm:fam_proj} the types of pairs $(S',h_S)$, $(A',h_A)$, and $(B',h_B)$ are equivalent to the types $S\to \UU$, $A\to \UU$, and $B\to \UU$, respectively. Therefore it follows that the type $\mathsf{Cart}(\mathcal{S})$ is equivalent to the type of tuples $(Q,P_A,\varphi,P_B,P_S)$ consisting of
\begin{align*}
Q & : S\to \UU \\
P_A & : A \to \UU \\
P_B & : B \to \UU \\
\varphi & : \prd{x:S}\eqv{Q(x)}{P_A(f(x))} \\
P_S & : \prd{x:S}\eqv{Q(x)}{P_B(g(x))}.
\end{align*}
However, the type of $\varphi$ is equivalent to the type $P_A\circ f=Q$. Thus we see that the type of pairs $(Q,\varphi)$ is contractible, so our claim follows.
\end{proof}

\begin{defn}
Consider a commuting square
\begin{equation*}
\begin{tikzcd}
S \arrow[r,"g"] \arrow[d,swap,"f"] & B \arrow[d,"j"] \\
A \arrow[r,swap,"i"] & X
\end{tikzcd}
\end{equation*}
with $H:i\circ f\htpy j\circ g$. 
We define an operation\index{cart map!{$\mathsf{cart\usc{}map}_{\mathcal{S}}$}|textbf}
\begin{equation*}
\mathsf{cart\usc{}map}_{\mathcal{S}}:{\Big(\sm{X':\UU}X'\to X\Big)}\to \mathsf{Cart}(\mathcal{S}).
\end{equation*}
\end{defn}

\begin{proof}[Construction]
Let $X':\UU$ and $h_X:X'\to X$. Then we define $A'$, $B'$, and $S'$ as the pullbacks
\begin{align*}
A' & \defeq A\times_X X' \\
B' & \defeq B\times_X X' \\
S' & \defeq S\times_A A',
\end{align*}
resulting in a diagram of the form
\begin{equation*}
\begin{tikzcd}
& S' \arrow[dl] \arrow[dr,densely dotted] \arrow[d] \\
A' \arrow[d] & S \arrow[dl] \arrow[dr] & B' \arrow[dl,crossing over] \arrow[d] \\
A \arrow[dr] & X' \arrow[d] \arrow[from=ul,crossing over] & B \arrow[dl] \\
& X
\end{tikzcd}
\end{equation*}
By the universal property of $B'$ it follows that there is a unique map $g':S'\to B'$ making the cube commute. 
Moreover, since the two front squares and the back left squares are pullback squares by construction, it follows by \cref{thm:pb_pasting} that also the back right square is a pullback square. Thus we obtain a cartasian transformation of spans.
\end{proof}

The following theorem is analogous to \cref{thm:desc_fam}.

\begin{thm}[The descent theorem for pushouts]\label{thm:descent}\index{descent theorem!for pushouts|textit}
Consider a commuting square
\begin{equation*}
\begin{tikzcd}
S \arrow[r,"g"] \arrow[d,swap,"f"] & B \arrow[d,"j"] \\
A \arrow[r,swap,"i"] & X
\end{tikzcd}
\end{equation*}
with $H:i\circ f\htpy j\circ g$. 
If this square is a pushout square, then the operation $\mathsf{cart\usc{}map}_{\mathcal{S}}$\index{cart map!{$\mathsf{cart\usc{}map}_{\mathcal{S}}$}!is an equivalence|textit} is an equivalence
\begin{equation*}
\eqv{\Big(\sm{X':\UU}X'\to X\Big)}{\mathsf{Cart}(\mathcal{S})}
\end{equation*}
\end{thm}

\begin{proof}
It suffices to show that the square
\begin{equation*}
\begin{tikzcd}[column sep=huge]
\UU^X \arrow[r,"{\mathsf{desc\usc{}fam}_{\mathcal{S}}(i,j,H)}"] \arrow[d,swap,"\mathsf{map\usc{}fam}_X"] & \mathsf{Desc}(\mathcal{S}) \arrow[d,"\mathsf{cart\usc{}desc}_{\mathcal{S}}"] \\
\sm{X':\UU}X^{X'} \arrow[r,swap,"\mathsf{cart\usc{}map}_{\mathcal{S}}"] & \mathsf{Cart}(\mathcal{S})
\end{tikzcd}
\end{equation*}
commutes. To see that this suffices, note that the operation $\mathsf{map\usc{}fam}_X$ is an equivalence by \cref{thm:fam_proj}, the operation $\mathsf{desc\usc{}fam}_{\mathcal{S}}(i,j,H)$ is an equivalence by \cref{thm:desc_fam}, and the operation $\mathsf{cart\usc{}desc}_{\mathcal{S}}$ is an equivalence by \cref{lem:cart_desc}.

To see that the square commutes, note that the composite
\begin{equation*}
\mathsf{cart\usc{}map}_{\mathcal{S}}\circ \mathsf{map\usc{}fam}_X
\end{equation*}
takes a family $P:X\to \UU$ to the cartesian transformation of spans
\begin{equation*}
\begin{tikzcd}
A\times_X\tilde{P} \arrow[d,swap,"\pi_1"] & S\times_A\Big(A\times_X\tilde{P}\Big) \arrow[l] \arrow[r] \arrow[d,swap,"\pi_1"] & B\times_X\tilde{P} \arrow[d,"\pi_1"] \\
A & S \arrow[l] \arrow[r] & B,
\end{tikzcd}
\end{equation*}
where $\tilde{P}\defeq\sm{x:X}P(x)$.

The composite 
\begin{equation*}
\mathsf{cart\usc{}desc}_{\mathcal{S}}\circ \mathsf{desc\usc{}fam}_X
\end{equation*}
takes a family $P:X\to \UU$ to the cartesian transformation of spans
\begin{equation*}
\begin{tikzcd}
\sm{a:A}P(i(a)) \arrow[d] & \sm{s:S}P(i(f(s))) \arrow[l] \arrow[r] \arrow[d] & \sm{b:B}P(j(b)) \arrow[d] \\
A & S \arrow[l] \arrow[r] & B
\end{tikzcd}
\end{equation*}
These cartesian natural transformations are equal by \cref{thm:pb_fibequiv}.
\end{proof}

Since $\mathsf{cart\usc{}map}_{\mathcal{S}}$ is an equivalence it follows that its fibers are contractible. 

\begin{cor}\label{cor:descent_fib}
Consider a diagram of the form 
\begin{equation*}
\begin{tikzcd}
& S' \arrow[d,swap,"h_S"] \arrow[dl,swap,"{f'}"] \arrow[dr,"{g'}"] \\
A' \arrow[d,swap,"h_A"] & S \arrow[dl,swap,"f"] \arrow[dr,"g"] & B' \arrow[d,"{h_B}"] \\
A \arrow[dr,swap,"i"] & & B \arrow[dl,"j"] \\
& X
\end{tikzcd}
\end{equation*}
with homotopies
\begin{align*}
F & : f\circ h_S \htpy h_A\circ f' \\
G & : g\circ h_S \htpy h_B\circ g' \\
H & : i\circ f \htpy j\circ g,
\end{align*}
and suppose that the bottom square is a pushout square, and the top squares are pullback squares.
Then the type of tuples $((X',h_X),(i',I,p),(j',J,q),(H',C))$ consisting of
\begin{enumerate}
\item A type $X':\UU$ together with a morphism
\begin{equation*}
h_X : X'\to X,
\end{equation*}
\item A map $i':A'\to X'$, a homotopy $I:i\circ h_A\htpy h_X\circ i'$, and a term $p$ witnessing that the square
\begin{equation*}
\begin{tikzcd}
A' \arrow[d,swap,"h_A"] \arrow[r,"{i'}"] & X' \arrow[d,"h_X"] \\
A \arrow[r,swap,"i"] & X
\end{tikzcd}
\end{equation*}
is a pullback square.
\item A map $j':B'\to X'$, a homotopy $J:j\circ h_B\htpy h_X\circ j'$, and a term $q$ witnessing that the square
\begin{equation*}
\begin{tikzcd}
B' \arrow[d,swap,"h_B"] \arrow[r,"{j'}"] & X' \arrow[d,"h_X"] \\
B \arrow[r,swap,"j"] & X
\end{tikzcd}
\end{equation*}
is a pullback square,
\item A homotopy $H':i'\circ f'\htpy j'\circ g'$, and a homotopy
\begin{equation*}
C : \ct{(i\cdot F)}{(\ct{(I\cdot f')}{(h_X\cdot H')})} \htpy \ct{(H\cdot h_S)}{(\ct{(j\cdot G)}{(J\cdot g')})}
\end{equation*}
witnessing that the cube
\begin{equation*}
\begin{tikzcd}
& S' \arrow[dl] \arrow[dr] \arrow[d] \\
A' \arrow[d] & S \arrow[dl] \arrow[dr] & B' \arrow[dl,crossing over] \arrow[d] \\
A \arrow[dr] & X' \arrow[d] \arrow[from=ul,crossing over] & B \arrow[dl] \\
& X,
\end{tikzcd}
\end{equation*}
commutes,
\end{enumerate}
is contractible.
\end{cor}

The following theorem should be compared to the flattening lemma, \cref{lem:flattening}.\index{flattening lemma!for pushouts}

\begin{thm}\label{cor:descent}
Consider a commuting cube
\begin{equation*}
\begin{tikzcd}
& S' \arrow[dl,swap,"{f'}"] \arrow[dr,"{g'}"] \arrow[d,"h_S"] \\
A' \arrow[d,swap,"h_A"] & S \arrow[dl,swap,"f" near start] \arrow[dr,"g" near start] & B' \arrow[dl,crossing over,"{j'}" near end] \arrow[d,"h_B"] \\
A \arrow[dr,swap,"i"] & X' \arrow[d,"h_X" near start] \arrow[from=ul,crossing over,"{i'}"' near end] & B \arrow[dl,"j"] \\
& X
\end{tikzcd}
\end{equation*}
in which the bottom square is a pushout, and the two vertical squares in the back are pullbacks. Then the following are equivalent:
\begin{enumerate}
\item The two vertical squares in the front are pullback squares.
\item The top square is a pushout square.
\end{enumerate}
\end{thm}

\begin{proof}
By \cref{cor:pb_fibequiv} we have fiberwise equivalences
\begin{align*}
F & : \prd{x:S}\eqv{\fib{h_S}{x}}{\fib{h_A}{f(x)}} \\
G & : \prd{x:S}\eqv{\fib{h_S}{x}}{\fib{h_B}{g(x)}} \\
I & : \prd{a:A}\eqv{\fib{h_A}{a}}{\fib{h_X}{i(a)}} \\
J & : \prd{b:B}\eqv{\fib{h_B}{b}}{\fib{h_X}{j(b)}}. 
\end{align*}
Moreover, since the cube commutes we obtain a fiberwise homotopy
\begin{equation*}
K : \prd{x:S} I(f(x))\circ F(x) \htpy J(g(x))\circ G(x).
\end{equation*}
We define the descent data $(P_A,P_B,P_S)$ consisting of $P_A:A\to\UU$, $P_B:B\to\UU$, and $P_S:\prd{x:S}\eqv{P_A(f(x))}{P_B(g(x))}$ by
\begin{align*}
P_A(a) & \defeq \fib{h_A}{a} \\
P_B(b) & \defeq \fib{h_B}{b} \\
P_S(x) & \defeq G(x)\circ F(x)^{-1}.
\end{align*}
We have
\begin{align*}
P & \defeq \fibf{h_X} \\
e_A & \defeq I \\
e_B & \defeq J \\
e_S & \defeq K.
\end{align*}
Now consider the diagram
\begin{equation*}
\begin{tikzcd}
\sm{s:S}\fib{h_S}{s} \arrow[r] \arrow[d] & \sm{s:S}\fib{h_A}{f(s)} \arrow[r] \arrow[d] & \sm{b:B}\fib{h_B}{b} \arrow[d] \\
\sm{a:A}\fib{h_A}{a} \arrow[r] & \sm{a:A}\fib{h_A}{a} \arrow[r] & \sm{x:X}\fib{h_X}{x}
\end{tikzcd}
\end{equation*}
Since the top and bottom map in the left square are equivalences, we obtain that the left square is a pushout square. Moreover, the right square is a pushout by \cref{lem:flattening}. Therefore it follows by \cref{thm:pushout_pasting} that the outer rectangle is a pushout square.

Now consider the commuting cube
\begin{equation*}
\begin{tikzcd}
& \sm{s:S}\fib{h_S}{s} \arrow[dl] \arrow[dr] \arrow[d] \\
\sm{a:A}\fib{h_A}{a} \arrow[d] & S' \arrow[dl] \arrow[dr] & \sm{b:B}\fib{h_B}{b} \arrow[dl,crossing over] \arrow[d] \\
A' \arrow[dr,swap] & \sm{x:X}\fib{h_X}{x} \arrow[d] \arrow[from=ul,crossing over] & B' \arrow[dl] \\
& X'.
\end{tikzcd}
\end{equation*}
We have seen that the top square is a pushout. The vertical maps are all equivalences, so the vertical squares are all pushout squares. Thus it follows from one more application of \cref{thm:pushout_pasting} that the bottom square is a pushout.
\end{proof}

%\begin{cor}
%For any map $f:A\sqcup^S B\to X$, and any $x:X$, the square
%\begin{equation*}
%\begin{tikzcd}
%\fib{f_S}{x} \arrow[r] \arrow[d] & \fib{f_B}{x} \arrow[d] \\
%\fib{f_A}{x} \arrow[r] & \fib{f}{x}
%\end{tikzcd}
%\end{equation*}
%is a pushout square.
%\end{cor}

\begin{thm}\label{thm:cartesian_cube}
Consider a commuting cube of types 
\begin{equation*}\label{eq:cube}
\begin{tikzcd}
& S' \arrow[dl] \arrow[dr] \arrow[d] \\
A' \arrow[d] & S \arrow[dl] \arrow[dr] & B' \arrow[dl,crossing over] \arrow[d] \\
A \arrow[dr] & X' \arrow[d] \arrow[from=ul,crossing over] & B \arrow[dl] \\
& X,
\end{tikzcd}
\end{equation*}
and suppose the two vertical squares in the back are pullback squares. Then the following are equivalent:
\begin{enumerate}
\item The two vertical squares in the front are pullback squares.
\item The commuting square
\begin{equation*}
\begin{tikzcd}
A' \sqcup^{S'} B' \arrow[r] \arrow[d] & X' \arrow[d] \\
A\sqcup^{S} B \arrow[r] & X
\end{tikzcd}
\end{equation*}
is a pullback square.
\end{enumerate}
\end{thm}

\begin{proof}
To see that (i) implies (ii), it suffices to show that the pullback 
\begin{equation*}
(A\sqcup^{S} B)\times_{X}X'
\end{equation*}
has the universal property of the pushout. This follows by the descent theorem, since the vertical squares in the cube
\begin{equation*}
\begin{tikzcd}
& S' \arrow[dl] \arrow[dr] \arrow[d] \\
A' \arrow[d] & S \arrow[dl] \arrow[dr] & B' \arrow[dl,crossing over] \arrow[d] \\
A \arrow[dr] & (A\sqcup^{S} B)\times_{X}X' \arrow[d] \arrow[from=ul,crossing over] & B \arrow[dl] \\
& A\sqcup^{S} B
\end{tikzcd}
\end{equation*}
are pullback squares by \cref{thm:pb_pasting}.

To prove that (ii) implies (i), we note that in the cube
\begin{equation*}
\begin{tikzcd}
& S' \arrow[dl] \arrow[dr] \arrow[d] \\
A' \arrow[d] & S \arrow[dl] \arrow[dr] & B' \arrow[dl,crossing over] \arrow[d] \\
A \arrow[dr] & A'\sqcup^{S'}B' \arrow[d] \arrow[from=ul,crossing over] & B \arrow[dl] \\
& A\sqcup^S B,
\end{tikzcd}
\end{equation*}
the two back squares are pullback squares, and the top and bottom squares are pushout squares. Therefore it follows from \cref{cor:descent} that the two front squares are pullback squares. Now we obtain (i) from the pasting lemma for pushouts.
\end{proof}

\section{Applications of the descent theorem for pushouts}
\sectionmark{Applications of the descent theorem}

\subsection{Fiber sequences}

\begin{defn}
A \define{pointed type} is a pair $(X,x)$ consisting of a type $X$ equipped with a \define{base point} $x:X$. We will write $\UU_\ast$ for the type $\sm{X:\UU}X$ of all pointed types.
\end{defn}

In the following lemma we characterize the identity type of $\UU_\ast$. 

\begin{lem}\label{lem:equiv_of_ptdtype}
For any $(A,a),(B,b):\UU_\ast$ we have an equivalence
\begin{equation*}
\eqv{\Big(\pairr{A,a}=\pairr{B,b}\Big)}{\Big(\sm{e:\eqv{A}{B}}e(a)=b\Big)}.
\end{equation*}
\end{lem}

\begin{proof}[Construction]
By \cref{thm:eq_sigma} the type on the left hand side is
equivalent to the type $\sm{p:A=B}\tr_{\universalfam}({p},{a})=b$.
By the univalence axiom, the map 
\begin{equation*}
\equiveq_{A,B}:(A=B)\to (\eqv{A}{B})
\end{equation*}
is an equivalence for each $B:\UU$. 
Therefore, we have an equivalence of type
\begin{equation*}
\eqv{\Big(\sm{p:A=B}\tr_{\universalfam}({p},{a})=b\Big)}{\Big(\sm{e:\eqv{A}{B}}\tr_{\universalfam}({\eqequiv(e)},{a})=b\Big)}
\end{equation*} 
Moreover, by equivalence induction (the analogue of path induction for 
equivalences), we can compute the transport:
\begin{equation*}
\tr_{\universalfam}({\eqequiv(e)},{a})=e(a).
\end{equation*}
It follows that $\eqv{(\tr_{\universalfam}({\eqequiv(e)},{a})=b)}
{(e(a)=b)}$.
\end{proof}

\begin{defn}
\begin{enumerate}
\item Let $(X,\ast_X)$ be a pointed type. A \define{pointed family} over $(X,\ast_X)$ consists of a type family $P:X\to \UU$ equipped with a base point $\ast_P:P(\ast_X)$. 
\item Let $(P,\ast_P)$ be a pointed family over $(X,\ast_X)$. A \define{pointed section} of $(P,\ast_P)$ consists of a dependent function $f:\prd{x:X}P(x)$ and an identification $p:f(\ast_X)=\ast_P$. We define the \define{pointed $\Pi$-type} to be the type of pointed sections:
\begin{equation*}
\Pi^\ast_{(x:X)}P(x) \defeq \sm{f:\prd{x:X}P(x)}f(\ast_X)=\ast_P
\end{equation*}
In the case of two pointed types $X$ and $Y$, we may also view $Y$ as a pointed family over $X$. In this case we write $X\to_\ast Y$ for the type of pointed functions.
\item Given any two pointed sections $f$ and $g$ of a pointed family $P$ over $X$, we define the type of pointed homotopies
\begin{equation*}
f\htpy_\ast g \defeq \Pi^\ast_{(x:X)} f(x)=g(x),
\end{equation*}
where the family $x\mapsto f(x)=g(x)$ is equipped with the base point $\ct{p}{q^{-1}}$. 
\end{enumerate}
\end{defn}

\begin{defn}
\begin{enumerate}
\item For any pointed type $X$, we define the \define{pointed identity function} $\idp_X\defeq (\idfunc[X],\refl{\ast})$. 
\item For any two pointed maps $f:X\to_\ast Y$ and $g:Y\to_\ast Z$, we define the \define{pointed composite}
\begin{equation*}
g\mathbin{\circ_\ast} f \defeq (g\circ f,\ct{\ap{g}{p_f}}{p_g}).
\end{equation*}
\end{enumerate}
\end{defn}

\begin{defn}
Let $X$ be a pointed type with base point $x$. We define the \define{loop space} $\loopspace{X,x}$ of $X$ at $x$ to be the pointed type $x=x$ with base point $\refl{x}$. 
\end{defn}

\begin{defn}
The loop space operation $\loopspacesym$ is \emph{functorial} in the sense that
\begin{enumerate}
\item For every pointed map $f:X\to_\ast Y$ there is a pointed map
\begin{equation*}
\loopspace{f}:\loopspace{X}\to_\ast \loopspace{Y},
\end{equation*}
defined by $\loopspace{f}(\omega)\defeq \ct{p_f}{\ap{f}{\omega}}{p_f^{-1}}$, which is base point preserving by $\rightinv(p_f)$. 
\item For every pointed type $X$ there is a pointed homotopy
\begin{equation*}
\loopspace{\idp_X}\htpy_\ast \idp_{\loopspace{X}}.
\end{equation*}
\item For any two pointed maps $f:X\to_\ast Y$ and $g:Y\to_\ast X$, there is a pointed homotopy witnessing that the triangle
\begin{equation*}
\begin{tikzcd}
& \loopspace{Y} \arrow[dr,"\loopspace{g}"] \\
\loopspace{X} \arrow[rr,swap,"\loopspace{g\circ_\ast f}"] \arrow[ur,"\loopspace{f}"] & & \loopspace{Z}
\end{tikzcd}
\end{equation*}
of pointed types commutes.
\end{enumerate}
\end{defn}

\begin{lem}\label{lem:equiv_of_ptdequiv}
For any $\pairr{e,p},\pairr{f,q}:\sm{e:\eqv{A}{B}}e(a)=b$, we have an equivalence of type
\begin{equation*}
\eqv{\Big(\pairr{e,p}=\pairr{f,q}\Big)}{\Big(\sm{h:e\htpy f} p=\ct{h(a)}{q}\Big)}.
\end{equation*}
\end{lem}

\begin{proof}[Construction]
The type $\pairr{e,p}=\pairr{f,q}$ is equivalent
to the type $\sm{h:e=f}\tr({h},{p})=q$.
Note that by the principle of function extensionality,
the map $\htpyeq:(e=f)\to(e\htpy f)$
is an equivalence. Furthermore, it follows by homotopy induction that for any 
$h:e\htpy f$ we have an equivalence of type
\begin{equation*}
\eqv{(\tr({\eqhtpy(h)},{p})=q)}
    {(p= \ct{h(a)}{q})}.\qedhere
\end{equation*}
\end{proof}

\begin{defn}
A \define{fiber sequence} $F\hookrightarrow E \twoheadrightarrow B$ consists of:
\begin{enumerate}
\item Pointed types $F$, $E$, and $B$, with base points $x_0$, $y_0$, and $b_0$ respectively, 
\item Base point preserving maps $i:F\to_\ast E$ and $p:E\to_\ast B$, with $\alpha:i(x_0)=y_0$ and $\beta:p(y_0)=b_0$,
\item A pointed homotopy $H:\const_{b_0}\htpy_\ast p\circ_\ast i$ witnessing that the square
\begin{equation*}
\begin{tikzcd}
F \arrow[r,"i"] \arrow[d] & E \arrow[d,"p"] \\
\unit \arrow[r,swap,"\const_{b_0}"] & B,
\end{tikzcd}
\end{equation*}
commutes and is a pullback square.
\end{enumerate}
We will write $\FibSeq$ for the type of all fiber sequences in $\UU$.
\end{defn}

\begin{prp}
The type of all fiber sequences is equivalent to the type
\begin{equation*}
\sm{(B,b):\UU_\ast}{P:B\to\UU}P(b).
\end{equation*}
\end{prp}

\subsection{Fiber sequences obtained by the descent property}

\begin{defn}
Let $f:A\to B$ be a map. The \define{codiagonal}\index{codiagonal}\index{nabla@{$\nabla_f$}} $\nabla_f$ of $f$ is the map obtained from the universal property of the pushout, as indicated in the diagram
\begin{equation*}
\begin{tikzcd}
A \arrow[d,swap,"f"] \arrow[r,"f"] \arrow[dr, phantom, "\ulcorner", very near end] & B \arrow[d,"\inr"] \arrow[ddr,bend left=15,"{\idfunc[B]}"] \\
B \arrow[r,"\inl"] \arrow[drr,bend right=15,swap,"{\idfunc[B]}"] & B\sqcup^{A} B \arrow[dr,densely dotted,near start,swap,"\nabla_f"] \\
& & B
\end{tikzcd}
\end{equation*}
\end{defn}

\begin{prp}
For any map $f:A\to B$ and any $y:B$, there is an equivalence $\eqv{\fib{\nabla_f}{y}}{\susp(\fib{f}{y})}$. 
\end{prp}

\begin{proof}
For any $b:B$ we have the commuting cube 
\begin{equation*}
\begin{tikzcd}
& \fib{f}{b} \arrow[dl] \arrow[d] \arrow[dr] \\
\unit \arrow[d] & A \arrow[dl] \arrow[dr] & \unit \arrow[dl,crossing over] \arrow[d] \\
B \arrow[dr] & \unit \arrow[from=ul,crossing over] \arrow[d,swap,"b"] & B \arrow[dl] \\
& B
\end{tikzcd}
\end{equation*}
of which the vertical sides are pullback squares. Hence we obtain the pullback square
\begin{equation*}
\begin{tikzcd}
\susp{\fib{f}{b}} \arrow[r] \arrow[d] & \unit \arrow[d,"b"] \\
B\sqcup^{A} B \arrow[r] & B
\end{tikzcd}
\end{equation*}
from \cref{thm:cartesian_cube}, from which the claim follows.
\end{proof}

\begin{defn}
Consider two maps $f:A\to B$ and $g:C\to D$.
The \define{pushout-product}\index{pushout-product}
\begin{equation*}
f\square g : (A\times D)\sqcup^{A\times C} (B\times C)\to B\times D
\end{equation*}
of $f$ and $g$ is defined by the universal property of the pushout as the unique map rendering the diagram
\begin{equation*}
\begin{tikzcd}
A\times C \arrow[r,"{f\times \idfunc[C]}"] \arrow[d,swap,"{\idfunc[A]\times g}"] & B\times C \arrow[d,"\inr"] \arrow[ddr,bend left=15,"{\idfunc[B]\times g}"] \\
A\times D \arrow[r,"\inl"] \arrow[drr,bend right=15,swap,"{f\times\idfunc[D]}"] & (A\times D)\sqcup^{A\times C} (B\times C) \arrow[dr,densely dotted,swap,near start,"f\square g"] \\
& & B\times D
\end{tikzcd}
\end{equation*}
commutative.
\end{defn}

\begin{prp}
For any two maps $f:A\to B$ and $g:C\to D$, and any $(b,d):B\times D$, there is an equivalence
\begin{equation*}
\eqv{\fib{f\square g}{b,d}}{\join{\fib{f}{b}}{\fib{g}{d}}}.
\end{equation*}
\end{prp}

\begin{proof}
Let $b:B$ and $d:D$. Then we have the commuting cube 
\begin{equation*}
\begin{tikzcd}
& \fib{f}{b}\times \fib{g}{d} \arrow[dl] \arrow[d] \arrow[dr] \\
\fib{f}{b} \arrow[d] & A\times C \arrow[dl] \arrow[dr] & \fib{g}{d} \arrow[dl,crossing over] \arrow[d] \\
A\times D \arrow[dr] & \unit \arrow[from=ul,crossing over] \arrow[d] & B\times C \arrow[dl] \\
& B\times D
\end{tikzcd}
\end{equation*}
of which the vertical sides are pullback squares. Hence the claim follows from \cref{thm:cartesian_cube}.
\end{proof}

\begin{defn}\label{defn:fib_join}
Let $f:A\to X$ and $g:B\to X$ be maps into $X$. We define the \define{fiberwise join} $\join[X]{A}{B}$ and the \define{join}\footnote{\emph{Warning}: By $\join{f}{g}$ we do \emph{not} mean the functorial action of the
join, applied to $(f,g)$.} $\join{f}{g}:\join[X]{A}{B}\to X$ of
$f$ and $g$, as indicated in the following diagram:
\begin{equation*}
\begin{tikzcd}
A\times_X B \arrow[r,"\pi_2"] \arrow[d,swap,"\pi_1"] \arrow[dr, phantom, "{\ulcorner}", at end] & B \arrow[d,"\inr"] \arrow[ddr,bend left=15,"g"] \\
A \arrow[r,swap,"\inl"] \arrow[drr,bend right=15,swap,"f"] & \join[X]{A}{B} \arrow[dr,densely dotted,swap,near start,"\join{f}{g}" xshift=1ex] \\
& & X.
\end{tikzcd}
\end{equation*}
\end{defn}

\begin{thm}\label{defn:join-fiber}
Let $f:A\to X$ and $g:B\to X$ be maps into $X$, and let $x:X$. Then there is
an equivalence
\begin{equation*}
\eqv{\fib{\join{f}{g}}{x}}{\join{\fib{f}{x}}{\fib{g}{x}}}.
\end{equation*}
\end{thm}

\begin{proof}
We have the following commuting cube
\begin{equation*}
\begin{tikzcd}
& \fib{f}{x}\times\fib{g}{x} \arrow[dl] \arrow[d] \arrow[dr] \\
\fib{f}{x} \arrow[d] & A\times_X B \arrow[dl] \arrow[dr] & \fib{g}{x} \arrow[dl,crossing over] \arrow[d] \\
A \arrow[dr] & \unit \arrow[d] \arrow[from=ul,crossing over] & B \arrow[dl] \\
& X
\end{tikzcd}
\end{equation*}
in which the vertical squares are pullback squares. Therefore it follows by \cref{thm:cartesian_cube} that the square
\begin{equation*}
\begin{tikzcd}
\join{\fib{f}{x}}{\fib{g}{x}} \arrow[r] \arrow[d] & \unit \arrow[d] \\
\join[X]{A}{B} \arrow[r] & X
\end{tikzcd}
\end{equation*}
is a pullback square.
\end{proof}

\begin{rmk}
The join operation on maps with a common codomain is associative up to homotopy (this was formalized by Brunerie, see Proposition 1.8.6 of \cite{BruneriePhD}), and it is a commutative operation on the generalized elements of a type $X$. Furthermore, the unique map of type $\emptyt\to X$ is a unit for the join operation.
\end{rmk}

\begin{defn}
Let $A$ and $B$ be pointed types with base points $a_0:A$ and $b_0:B$. The \define{wedge inclusion}\index{wedge inclusion} is defined as follows by the universal property of the wedge:
\begin{equation*}
\begin{tikzcd}[column sep=huge]
\unit \arrow[r] \arrow[d] & B \arrow[d,"\inr"] \arrow[ddr,bend left=15,"{\lam{b}(a_0,b)}"] \\
A \arrow[r,"\inl"] \arrow[drr,bend right=15,swap,"{\lam{a}(a,b_0)}"] & A\vee B \arrow[dr,densely dotted,swap,"{\mathsf{wedge\usc{}in}_{A,B}}"{near start,xshift=1ex}] \\
& & A\times B
\end{tikzcd}
\end{equation*}
\end{defn}

\begin{prp}
There is a fiber sequence 
\begin{equation*}
\join{\loopspace{A}}{\loopspace{B}}\hookrightarrow A\vee B\twoheadrightarrow A\times B.
\end{equation*}
\end{prp}

\begin{proof}
We have the commuting cube 
\begin{equation*}
\begin{tikzcd}
& \loopspace{B}\times\loopspace{A} \arrow[dl] \arrow[d] \arrow[dr] \\
\loopspace{B} \arrow[d,swap,"\mathsf{const}_a"] & \unit \arrow[dl] \arrow[dr] & \loopspace{A} \arrow[dl,crossing over] \arrow[d,"\mathsf{const}_b"] \\
A \arrow[dr] & \unit \arrow[from=ul,crossing over] \arrow[d] & B \arrow[dl] \\
& A\times B
\end{tikzcd}
\end{equation*}
of which the vertical sides are pullback squares. Hence the claim follows from \cref{thm:cartesian_cube}.
\end{proof}

\begin{defn}
Consider a pointed type $X$. We define the map $\mathsf{fold}:X\vee X\to X$ by the universal property of the wedge as indicated in the diagram
\begin{equation*}
\begin{tikzcd}
\unit \arrow[d,swap,"x_0"] \arrow[r,"x_0"] \arrow[dr, phantom, "\ulcorner", very near end] & X \arrow[d,"\inr"] \arrow[ddr,bend left=15,"{\idfunc[X]}"] \\
X \arrow[r,"\inl"] \arrow[drr,bend right=15,swap,"{\idfunc[X]}"] & X\vee X \arrow[dr,densely dotted,near start,swap,"\mathsf{fold}"] \\
& & X.
\end{tikzcd}
\end{equation*}
\end{defn}

\begin{prp}
There is a fiber sequence
\begin{equation*}
\susp\loopspace{X} \hookrightarrow X\vee X \twoheadrightarrow X.
\end{equation*}
\end{prp}

\begin{proof}
We have the commuting cube 
\begin{equation*}
\begin{tikzcd}
& \loopspace{X} \arrow[dl] \arrow[d] \arrow[dr] \\
\unit \arrow[d] & \unit \arrow[dl] \arrow[dr] & \unit \arrow[dl,crossing over] \arrow[d] \\
X \arrow[dr] & \unit \arrow[from=ul,crossing over] \arrow[d] & X \arrow[dl] \\
& X
\end{tikzcd}
\end{equation*}
of which the vertical sides are pullback squares. Hence the claim follows from \cref{thm:cartesian_cube}.
\end{proof}

\begin{rmk}
As a corollary, there are fiber sequences
\begin{align*}
\sphere{1} \hookrightarrow \rprojective{\infty}\vee \rprojective{\infty} & \twoheadrightarrow \rprojective{\infty} \\
\sphere{2} \hookrightarrow \cprojective{\infty}\vee \cprojective{\infty} & \twoheadrightarrow \cprojective{\infty}.
\end{align*}
Here we take $\rprojective{\infty}\defeq K(\Z/2,1)$ and $\cprojective{\infty}\defeq K(\Z,2)$, where the Eilenberg-Mac Lane space $K(G,n)$ is defined in \cite{FinsterLicata}.
\end{rmk}

\begin{cor}
There is a fiber sequence
\begin{equation*}
(\susp\loopspace{X})^{\vee n} \hookrightarrow X^{\vee (n+1)} \twoheadrightarrow X.
\end{equation*}
\end{cor}

\begin{defn}\label{defn:coh_hspace}
A \define{coherent H-space} consists of a type $X$ equipped with a unit $1:X$, a multiplication operation $\mu:X \to (X \to X)$ such that the function $\mu(x,\blank)$ and $\mu(\blank,y)$ are equivalences for each $x:X$ and $y:X$, respectively, and \emph{coherent} unit laws
\begin{align*}
\mathsf{right\usc{}unit} & : \prd{x: X} \mu(x,1)= x \\
\mathsf{left\usc{}unit} & : \prd{y:X} \mu(1,y) = y \\
\mathsf{coh\usc{}unit} & : \mathsf{left\usc{}unit}(1)=\mathsf{right\usc{}unit}(1).
\end{align*}
\end{defn}

The following theorem is also known as the Hopf-construction.

\begin{thm}\label{thm:hopf_construction}
For any coherent H-space $X$ there is a fiber sequence
\begin{equation*}
X \hookrightarrow \join{X}{X} \twoheadrightarrow \susp X.
\end{equation*}
The map $\eta_X:\join{X}{X}\to \susp X$ is called the \define{Hopf fibration}.
\end{thm}

\begin{proof}
We have the commuting cube
\begin{equation*}
\begin{tikzcd}
& X \times X \arrow[dl,swap,"\pi_1"] \arrow[d,swap,"\mu"] \arrow[dr,"\pi_2"] \\
X \arrow[d] & X \arrow[dl] \arrow[dr] & X \arrow[dl,crossing over] \arrow[d] \\
\unit \arrow[dr] & \join{X}{X} \arrow[from=ul,crossing over] \arrow[d,densely dotted] & \unit \arrow[dl] \\
& \susp X,
\end{tikzcd}
\end{equation*}
where the front map is obtained by the universal property of the pushout.

In this cube, the two vertical squares in the back are pullback squares by \cref{thm:pb_fibequiv}, since $\mu(\blank,y)$ and $\mu(x,\blank)$ are equivalences for every $x:X$ and $y:X$, respectively. Since the bottom and top squares are pushout squares it follows by \cref{thm:cartesian_cube} that the front two squares are pullback squares.
\end{proof}
