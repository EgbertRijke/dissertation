\chapter{\texorpdfstring{$2$}{2}-simplicial types}\label{chap:simplicial}

We have observed in \cref{eg:rcoeq} that for any $f:A\to X$ we have a reflexive coequalizer diagram
\begin{equation*}
\begin{tikzcd}
\Big(\sm{x,y:A}f(x)=f(y)\Big) \arrow[r,yshift=1ex] \arrow[r,yshift=-1ex] & A \arrow[l] \arrow[r] & \join[X]{A}{A},
\end{tikzcd}
\end{equation*}
and for any coherent H-space $X$ we have a reflexive coequalizer diagram
\begin{equation*}
\begin{tikzcd}
X \arrow[r,yshift=1ex] \arrow[r,yshift=-1ex] & \unit \arrow[l] \arrow[r] & \susp X.
\end{tikzcd}
\end{equation*}
However, both the pre-kernel of a map and a coherent H-space come equipped with a multiplicative structure that satisfies unit laws. What spaces do we get when take not only the reflexive coequalizer, but also take the multiplicative structure and unit laws into account?

We answer this question by studying at `$2$-truncated simplicial types', by which we understand tiples $(A_0,A_1,A_2)$ consisting of
\begin{align*}
A_0 & : \UU \\
A_1 & : A_0 \to A_0 \to \UU\\
A_2 & : \prd{x,y,z:A_0} A_1(x,y) \to A_1(x,z)\to A_1(y,z) \to \UU,
\end{align*}
representing the types of $0$-, $1$-, and $2$-cells, equipped with a `unital' or `simplicial' structure consisting of a unit (the reflexivity term of $A_1$), a right and left unit law and a coherence between them.

In \cref{prp:2disc_emb} we show that the operation $\Delta:\UU\to \mathsf{sType}_2$, taking types to discrete $2$-simplicial types, is an embedding. Moreover, its action on morphisms
\begin{equation*}
(X\to Y) \to \mathsf{sType}_2(\Delta X,\Delta Y)
\end{equation*}
is an equivalence. As in the case of reflexive graphs, discrete $2$-simplicial types play a pivotal role in this chapter. We analyze the discrete $2$-simplicial types further by showing that they are right orthogonal to the morphisms between the walking $0$-, $1$-, and $2$-cells, also known as the standard $0$-, $1$-, and $2$-simplices. The standard simplices organize themselves in a cosimplicial object
\begin{equation*}
\begin{tikzcd}
\Delta[2] \arrow[r,yshift=1ex] \arrow[r,yshift=-1ex] & \Delta[1] \arrow[l,yshift=2ex] \arrow[l] \arrow[l,yshift=-2ex] \arrow[r] & \Delta[0] \arrow[l,yshift=1ex] \arrow[l,yshift=-1ex]
\end{tikzcd}
\end{equation*}
and a fibration is a morphism that is right orthogonal with respect to all morphisms in the above cosimplicial diagram. This condition is equivalent to being right orthogonal to the horn inclusions.

Analogous to \cref{thm:rcoeq_is_pushout}, we show that the geometric realization of a $2$-simplicial type $\mathcal{A}$ is simply the homotopy colimit of the diagram
\begin{equation*}
\begin{tikzcd}
& \tilde{A}_2 \arrow[dl] \arrow[d] \arrow[dr] \\
\tilde{A}_1 \arrow[d] & \tilde{A}_1 \arrow[dl] \arrow[dr] & \tilde{A}_1 \arrow[dl,crossing over] \arrow[d] \\
\tilde{A}_0 & \tilde{A}_0 \arrow[from=ul,crossing over] & \tilde{A}_0,
\end{tikzcd}
\end{equation*}
where $\tilde{A}_i$ is the total space of all $i$-cells.
This presentation of the geometric realization allows us to compute geometric realizations as iterated pushouts.

We will show in \cref{cor:2-pre-kernel} that the geometric realization of the $2$-pre-kernel of a map $f:A\to X$ is the triple fiberwise join $\join[X]{A}{\join[X]{A}{A}}$, and we will show in \cref{cor:geom_hspace} that the geometric realization of a coherent H-space is the cofiber of the Hopf fibration
\begin{equation*}
\begin{tikzcd}
\join{X}{X} \arrow[r,"\eta_X"] \arrow[d] & \susp X \arrow[d] \\
\unit \arrow[r] & \mathsf{cof}_{\eta_X}.
\end{tikzcd}
\end{equation*}
Of course, in this result we use the coherence of a coherent H-space.

We will also be able to use the material of this chapter to show that the join and smash product are associative operations. 

\section{Discrete and Segal \texorpdfstring{$2$}{2}-simplicial types}

\begin{defn}
A small \define{$2$-simplicial type}\marginnote{I'm not entirely sold on the terminology of $2$-simplicial type. Open to suggestions} $\mathcal{A}$ consists of:
\begin{enumerate}
\item A \define{$2$-semi-simplicial structure} $(A_0,A_1,A_2)$ consisting of
\begin{align*}
A_0 & : \UU \\
A_1 & : A_0 \to A_0 \to \UU \\
A_2 & : \prd{x_0,x_1,x_2:A_0} A_1(x_0,x_1)\to A_1(x_0,x_2)\to A_1(x_1,x_2)\to\UU.
\end{align*}
Terms of $A_0$ are also called \define{vertices} or \define{$0$-cells}. Terms of type $A_1(x,y)$ are also called \define{edges} or \define{$1$-cells}. Terms of type $A_2(p,r,q)$ are also called \define{$2$-cells}.
\item A \define{$2$-simplicial structure} or \define{unital structure}
\begin{equation*}
(\rfx{\mathcal{A}},\mathsf{right\usc{}unit}_{\mathcal{A}},\mathsf{left\usc{}unit}_{\mathcal{A}},\mathsf{coh\usc{}unit}_{\mathcal{A}})
\end{equation*}
consisting of
\begin{align*}
\rfx{\mathcal{A}} & : \prd{x:\pts{A}} \edg{A}(x,x) \\
\mathsf{right\usc{}unit}_{\mathcal{A}} & : \prd{x_0,x_1:A_0}{f:A_1(x_0,x_1)} A_2(f,f,\rfx{\mathcal{A}}(x_1)) \\
\mathsf{left\usc{}unit}_{\mathcal{A}} & : \prd{x_1,x_2:A_0}{g:A_1(x_1,x_2)} A_2(\rfx{\mathcal{A}}(x_1),g,g) \\
\mathsf{coh\usc{}unit}_{\mathcal{A}} & : \prd{x:A_0} \mathsf{left\usc{}unit}(\rfx{\mathcal{A}}(x))=\mathsf{right\usc{}unit}(\rfx{\mathcal{A}}(x)).
\end{align*}
Given a 2-semi-simplicial structure $\mathcal{A}$, we write $\mathsf{simp}(\mathcal{A})$ for the type of quadruples in the $2$-simplicial structre of $\mathcal{A}$.
\end{enumerate}
\end{defn}

\begin{eg}
Let $X$ be a type. The \define{discrete} $2$-simplicial type $\Delta X$ consists of
\begin{samepage}
\begin{align*}
\pts{\Delta X} & \defeq X \\
\edg{\Delta X}(x,y) & \defeq x=y \\
\Delta X_2(p,r,q) & \defeq \ct{p}{q}=r,
\end{align*}
\end{samepage}%
using that the identity type is reflexive and that composition satisfies the coherent unit laws. 
\end{eg}

\begin{eg}
Let $X$ be a type. The \define{indiscrete} $2$-simplicial type $\nabla X$ consists of
\begin{samepage}
\begin{align*}
\pts{\Delta X} & \defeq X \\
\edg{\Delta X}(x,y) & \defeq \unit \\
\Delta X_2(p,r,q) & \defeq \unit. \\
\end{align*}
\end{samepage}%
Since the types of $1$-cells and $2$-cells are always contractible, the indiscrete $2$-simplicial type $\nabla X$ possesses the unital structure in a unique way. 
\end{eg}

\begin{eg}
Let $f:A\to X$ be a map. The \define{$2$-pre-kernel} $k^2(f)$ consists of
\begin{samepage}
\begin{align*}
\pts{k^2(f)} & \defeq A \\
\edg{k^2(f)}(x,y) & \defeq f(x)=f(y) \\
k^2(f)_2(p,r,q) & \defeq \ct{p}{q}=r. \\
\end{align*}
\end{samepage}%
The unital structure of $k^2(f)$ comes from the unital structure of the identity type of $B$.
\end{eg}

\begin{eg}
Let $X$ be a coherent H-space, as specified in \cref{defn:coh_hspace}. Then we have the $2$-simplicial type $S^2X$ consisting of
\begin{samepage}
\begin{align*}
\pts{S^2X} & \defeq \unit \\
\edg{S^2X}(s,t) & \defeq X \\
S^2X_2(x,z,y) & \defeq \mu(x,y)=z \\
\end{align*}
\end{samepage}%
Since $X$ has a unit and the multiplication satisfies the coherent unit laws, we obtain the unital structure of $S^2 X$ in a canonical way.
\end{eg}

\begin{table}
\centering
\caption{\label{table:stype_eg}The specification of the $2$-simplicial types $\Delta X$, $\nabla X$, $k^2(f)$, and $S^2 X$.}
\begin{tabular}{lllll}
\toprule
$\mathcal{A}$ & $\Delta X$ & $\nabla X$ & $k^2(f)$ & $S^2X$ \\
\midrule
$\pts{A}$ & $X$ & $X$ & $A$ & $\unit$  \\
$\edg{A}(x,y)$ &  $x=y$ & $\unit$ & $f(x)=f(y)$ & $X$ \\
$\rfx{\mathcal{A}}$(x) & $\refl{x}$ & $\ttt$ & $\refl{f(x)}$ & $1$  \\
$A_2(p,r,q)$ & $\ct{p}{q}=r$ & $\unit$ & $\ct{p}{q}=r$ & $\mu(p,q)=r$ \\
$\mathsf{right\usc{}unit}_{\mathcal{A}}(p)$ & $\mathsf{right\usc{}unit}(p)$ & $\ttt$ & $\mathsf{right\usc{}unit}(p)$ & $\mathsf{right\usc{}unit}_{\mu}(p)$ \\ 
$\mathsf{left\usc{}unit}_{\mathcal{A}}(p)$ & $\mathsf{left\usc{}unit}(p)$ & $\ttt$ & $\mathsf{left\usc{}unit}(p)$ & $\mathsf{left\usc{}unit}_{\mu}(p)$ \\
$\mathsf{coh\usc{}unit}_{\mathcal{A}}(p)$ & $\mathsf{coh\usc{}unit}(p)$ & $\refl{\ttt}$ & $\mathsf{coh\usc{}unit}(p)$ & $\mathsf{coh\usc{}unit}_{\mu}(p)$ \\
\bottomrule
\end{tabular}
\end{table}

\begin{defn}
Let $\mathcal{A}$ be a $2$-simplicial type. We define
\begin{align*}
\tilde{A}_0 & \defeq A_0 \\
\tilde{A}_1 & \defeq \sm{x,y:A_0} A_1(x,y) \\
\tilde{A}_2 & \defeq \sm{x,y,z:A_0}{p:A_1(x,y)}{r:A_1(x,z)}{q:A_1(y,z)} A_2(p,r,q),
\end{align*}
We also define the maps
\begin{equation*}
\begin{tikzcd}
\tilde{A}_2 \arrow[r,yshift=2ex,"\pi_{01}"] \arrow[r,"\pi_{02}" description] \arrow[r,yshift=-2ex,swap,"\pi_{12}"] &[2em] \tilde{A}_1 \arrow[r,yshift=1ex,"\pi_0"] \arrow[r,yshift=-1ex,swap,"\pi_1"] & \tilde{A}_0
\end{tikzcd}
\end{equation*}
in the expected way.
\end{defn}

\begin{defn}
Consider
\begin{align*}
f_0 & : \pts{A}\to \pts{B} \\
f_1 & : \prd{x,y:A_0} A_1(x,y) \to B_1(f_0(x),f_0(y)) \\
f_2 & : \prd{x,y,z:A_0}{p:A_1(x,y)}{r:A_1(x,z)}{q:A_1(y,z)} \\
& \qquad\qquad\qquad A_2(p,r,q) \to B_2(f_1(p),f_1(r),f_1(q))
\end{align*}
We define the map $\mathsf{ev\usc{}simp}(f_0,f_1,f_2):\mathsf{simp}(\mathcal{A})\to\mathsf{simp}(\mathcal{B})$ by
\begin{equation*}
(\alpha,\beta_{10},\beta_{01},\beta_{00}) \mapsto
\left(\begin{array}{l}
\lam{x} f_1(\alpha(x)) \\
\lam{x}{y}{p} f_2(\beta_{10}(p)) \\
\lam{x}{y}{q} f_2(\beta_{01}(q)) \\
\lam{x} \ap{f_2}{\beta_{00}(x)}
\end{array}\right)
\end{equation*}
A \define{morphism of $2$-simplicial types} is defined to be a quadruple $(f_0,f_1,f_2,f_u)$ consisting of $(f_0,f_1,f_2)$ as before, and an identification
\begin{align*}
f_u & : \mathsf{ev\usc{}simp}(f_0,f_1,f_2,(\rfx{\mathcal{A}},\mathsf{right\usc{}unit}_{\mathcal{A}},\mathsf{left\usc{}unit}_{\mathcal{A}},\mathsf{coh\usc{}unit}_{\mathcal{A}})) \\
& \qquad\qquad = (\rfx{\mathcal{B}},\mathsf{right\usc{}unit}_{\mathcal{B}},\mathsf{left\usc{}unit}_{\mathcal{B}},\mathsf{coh\usc{}unit}_{\mathcal{B}}).
\end{align*}
We write $\mathsf{sType}_2(\mathcal{A},\mathcal{B})$ for the type of morphisms of $2$-simplicial types from $\mathcal{A}$ to $\mathcal{B}$.
\end{defn}

It is clear that for each $2$-simplicial type $\mathcal{A}$ there is an identity morphism, and that morphisms of $2$-simplicial types can be composed. Moreover, composition is associative and satisfies the unit laws.

\begin{eg}
For every map $f:A\to B$ there is a unique way to equip it with the structure of a morphism of $2$-simplicial types from $\Delta(A)$ to $\Delta(B)$. Indeed, we have seen in \cref{eg:rgraph_morphism} that $\apfunc{f}$ is the unique action on edges such that $\ap{f}{\refl{x}}=\refl{f(x)}$ for each $x:A$. Similarly, it follows by \cref{lem:yoneda} that the type of fiberwise transformations
\begin{equation*}
f_2 : \prd{x,y,z:A}{p:x=y,\ q:y=z,\ r:x=z} (\ct{p}{q}=r)\to (\ct{\ap{f}{p}}{\ap{f}{q}}=\ap{f}{r})
\end{equation*}
satisfying a left unit law, is contractible. Furthermore, the type of 
\begin{equation*}
\mathsf{\beta_{10}} : \prd{x,y:A}{p:x=y} f_2(\mathsf{right\usc{}unit}(p))= \mathsf{right\usc{}unit}(\ap{f}{p})
\end{equation*}
equipped with a coherence with the left unit law is again contractible. We conclude that the discrete functor $\Delta$ is an embedding in the sense that its action on morphisms
\begin{equation*}
(A\to B)\to \mathsf{sType}_2(\Delta(A),\Delta(B))
\end{equation*}
is an equivalence. We note that our proof that $\Delta$ is an embedding uses the coherence of the unit laws.
\end{eg}

\begin{eg}
The \define{standard $0$-simplex} $\Delta[0]$ is the $2$-simplicial type with just one vertex. The universal property of $\Delta[0]$ is that the function 
\begin{equation*}
\mathsf{sType}_2(\Delta[0],\mathcal{A}) \to \tilde{A}_0
\end{equation*}
given by $f\mapsto \pts{f}(\ttt)$ is an equivalence, for every $2$-simplicial type $\mathcal{A}$.

The \define{standard $1$-simplex} $\Delta[1]$ is the $2$-simplicial type with two vertices $0$ and $1$, and a non-degenerate edge $s$ from $0$ to $1$. The universal property of the stardard $1$-simplex is that the function
\begin{equation*}
\mathsf{sType}_2(\Delta[1],\mathcal{A}) \to \tilde{A}_1
\end{equation*}
is an equvialence for every $2$-simplicial type $\mathcal{A}$.

The \define{standard $2$-simplex} $\Delta[2]$ is the $2$-simplicial type consisting of three vertices, three non-degenerate arrows, and a $2$-cell filling the triangle, as indicated in the diagram
\begin{equation*}
\begin{tikzcd}[column sep=tiny]
& 1 \arrow[dr] \\
0 \arrow[ur] \arrow[rr] & & 2.
\end{tikzcd}
\end{equation*}
The function
\begin{equation*}
\mathsf{sType}_2(\Delta[2],\mathcal{A}) \to \tilde{A}_2
\end{equation*}
that evaluates $f$ at the non-degenerate $2$-cell, is an equvialence. This is the universal property of the standard $2$-simplex.

We obtain a diagram of the form
\begin{equation*}
\begin{tikzcd}
\Delta[2] \arrow[r,yshift=1ex] \arrow[r,yshift=-1ex] & \Delta[1] \arrow[l,yshift=2ex] \arrow[l] \arrow[l,yshift=-2ex] \arrow[r] & \Delta[0] \arrow[l,yshift=1ex] \arrow[l,yshift=-1ex]
\end{tikzcd}
\end{equation*}
of morphisms of $2$-simplicial types, where the three arrows $\Delta[1]\to\Delta[2]$ are characterized by the three non-degenerate edges of $\Delta[2]$, the two arrows $\Delta[0]\to\Delta[1]$ are characterized by the two vertices of $\Delta[1]$, the downward arrows $\Delta[2]\to\Delta[1]$ are obtained from the unit laws, and the arrow $\Delta[1]\to \Delta[0]$ is by reflexivity\footnote{There is a `$2$-cosimplicial object of $2$-simplicial types', although we have not fully specified it here, because we will only make use of the morphisms}.
\end{eg}

\begin{eg}
The horns $\Lambda_0[2]$, $\Lambda_1[2]$, and $\Lambda_2[2]$ are the $2$-simplicial types indicated by the diagrams
\begin{equation*}
\begin{tikzcd}[column sep=tiny]
& 1 & &[2em] & 1 \arrow[dr] & &[2em] & 1 \arrow[dr] \\
0 \arrow[ur] \arrow[rr] & & 2, & 0 \arrow[ur] & & 2, & 0 \arrow[rr] & & 2,
\end{tikzcd}
\end{equation*}
respectively. More formally, $\Lambda_0[2]$ is the 2-simplicial type consisting of
\begin{align*}
\pts{\Lambda_0[2]} & \defeq \{0,1,2\} \\
\edg{\Lambda_0[2]}(i,j) & \defeq (i\leq j)\land (i=0) \\
\Lambda_0[2]_2(x_0,x_1,x_2,p,r,q) & \defeq \Big(\sm{\alpha:x_0=x_1}\mathsf{tr}(\alpha,r)=q\Big)\vee\Big(\sm{\alpha:x_1=x_2}\mathsf{tr}(\alpha^{-1},r)=p\Big),
\end{align*}
and there is only one way to equip this 2-semi-simplical structure with a simplical structure.
The horns $\Lambda_1[2]$ and $\Lambda_2[2]$ can be defined similarly. We note that the horns can also be presented as pushouts (of $2$-simplicial types) as follows:
\begin{equation*}
\begin{tikzcd}
\Delta[0] \arrow[r,"0"] \arrow[d,swap,"0"] \arrow[dr,phantom,"\ulcorner" at end] & \Delta[1] \arrow[d] & \Delta[0] \arrow[r,"1"] \arrow[d,swap,"0"] \arrow[dr,phantom,"\ulcorner" at end] & \Delta[1] \arrow[d] & \Delta[0] \arrow[r,"1"] \arrow[d,swap,"1"] \arrow[dr,phantom,"\ulcorner" at end] & \Delta[1] \arrow[d] \\
\Delta[1] \arrow[r] & \Lambda_0[2] & \Delta[1] \arrow[r] & \Lambda_1[2] & \Delta[1] \arrow[r] & \Lambda_2[2]
\end{tikzcd}
\end{equation*}
In particular, the horn inclusions $\mathsf{horn\usc{}in}_i:\Lambda_i[2]\to \Delta[2]$ can be obtained by the universal property of the pushout.
\end{eg}

\begin{defn}
Let $\mathcal{A}$ be a $2$-simplicial type. We say that $\mathcal{A}$ satisfies the \define{Segal condition} if the square
\begin{equation*}
\begin{tikzcd}
\tilde{A}_2 \arrow[r,"\pi_{12}"] \arrow[d,swap,"\pi_{01}"] & \tilde{A}_1 \arrow[d,"\pi_0"] \\
\tilde{A}_1 \arrow[r,swap,"\pi_1"] & \tilde{A}_0
\end{tikzcd}
\end{equation*}
is a pullback square.
\end{defn}

\begin{prp}
Let $\mathcal{A}$ be a $2$-simplicial type. The following are equivalent:
\begin{enumerate}
\item $\mathcal{A}$ satisfies the Segal condition.
\item $\mathcal{A}$ is $\mathsf{horn\usc{}in}_1$-local, where
\begin{equation*}
\mathsf{horn\usc{}in}_1 : \mathsf{sType}_2(\Lambda_1[2],\Delta[2])
\end{equation*}
is the horn inclusion. In other words, the map
\begin{equation*}
\mathsf{sType}_2(\Delta[2],\mathcal{A}) \to \mathsf{sType}_2(\Lambda_1[2],\mathcal{A})
\end{equation*}
is an equivalence.
\item For every $f:A_1(x,y)$ and $g:A_1(y,z)$ the type
\begin{equation*}
\sm{h:A_1(x,z)}A_2(f,h,g)
\end{equation*}
is contractible.
\end{enumerate}
\end{prp}

\begin{proof}
Note that the map $\mathsf{sType}_2(\Delta[2],\mathcal{A}) \to \mathsf{sType}_2(\Lambda_1[2],\mathcal{A})$ is the gap map of the square
\begin{equation*}
\begin{tikzcd}
\mathsf{sType}_2(\Delta[2],\mathcal{A}) \arrow[r] \arrow[d] & \mathsf{sType}_2(\Delta[1],\mathcal{A}) \arrow[d] \\
\mathsf{sType}_2(\Delta[1],\mathcal{A}) \arrow[r]  & \mathsf{sType}_2(\Delta[0],\mathcal{A}),
\end{tikzcd}
\end{equation*}
so it follows by the universal property of $\Lambda_1[2]$ as a pushout that condition (ii) holds if and only if this square is a pullback square. Now we also see that (i) holds if and only if (ii) holds, since we have a commuting cube
\begin{equation*}
\begin{tikzcd}
& \mathsf{sType}_2(\Delta[2],\mathcal{A}) \arrow[dl] \arrow[d] \arrow[dr] \\
\mathsf{sType}_2(\Delta[1],\mathcal{A}) \arrow[d] & \tilde{A}_2 \arrow[dl] \arrow[dr] & \mathsf{sType}_2(\Delta[1],\mathcal{A}) \arrow[d] \arrow[dl,crossing over] \\
\tilde{A}_1 \arrow[dr] & \mathsf{sType}_2(\Delta[0],\mathcal{A}) \arrow[from=ul,crossing over] \arrow[d] & \tilde{A}_1 \arrow[dl] \\
& \tilde{A}_0
\end{tikzcd}
\end{equation*}
in which the vertical maps are equivalences.\marginnote{Still have to write down equivalence with (iii)}
\end{proof}

\begin{defn}
Let $\mathcal{A}$ be a $2$-simplicial type. For $k=0,1,2$, we say that $\mathcal{A}$ is \define{$k$-discrete} if it is $\Delta[k]$-null, in the sense that the map
\begin{equation*}
\mathsf{sType}_2(\Delta[0],\mathcal{A})\to\mathsf{sType}_2(\Delta[k],\mathcal{A})
\end{equation*}
is an equivalence. We say that $\mathcal{A}$ is \define{discrete} if $\mathcal{A}$ is $k$-discrete for all $k\in\{0,1,2\}$.
\end{defn}

\begin{prp}
Let $\mathcal{A}$ be a $2$-simplicial type. The following are equivalent:\marginnote{In a presheaf $\infty$-category, is it the case that a morphism is $X$-null for every representable object $X$, if and only if it is $f$-local for every morphism between representables, or does such a relation depend on the shape of the indexing category (since we're using that the terminal object is representable)?}
\begin{enumerate}
\item $\mathcal{A}$ is discrete.
\item $\mathcal{A}$ is right orthogonal to every morphism in the diagram
\begin{equation*}
\begin{tikzcd}
\Delta[2] \arrow[r,yshift=1ex] \arrow[r,yshift=-1ex] & \Delta[1] \arrow[l,yshift=2ex] \arrow[l] \arrow[l,yshift=-2ex] \arrow[r] & \Delta[0]. \arrow[l,yshift=1ex] \arrow[l,yshift=-1ex]
\end{tikzcd}
\end{equation*}
\item $\mathcal{A}$ is $2$-discrete.
\item $\mathcal{A}$ is $1$-discrete and satisfies the Segal condition.
\end{enumerate}
\end{prp}

\begin{proof}
If $\mathcal{A}$ is $\Delta[0]$-, $\Delta[1]$-, and $\Delta[2]$-null, then it is $f$-local for any $f:\Delta[i]\to\Delta[j]$, for $i,j\in\{0,1,2\}$. Thus it is immediate that (i) holds if and only if (ii) holds. Note that (i) holds if and only if (iii) holds, because $\Delta[1]$ is a retract of $\Delta[2]$. To see that (i) holds if and only if (iv) holds, note that for $\mathcal{A}$ $1$-discrete the square
\begin{equation*}
\begin{tikzcd}
\tilde{A}_2 \arrow[r] \arrow[d] & \tilde{A}_1 \arrow[d] \\
\tilde{A}_1 \arrow[r] & \tilde{A}_0
\end{tikzcd}
\end{equation*}
is a pullback square if and only if both morphisms $\tilde{A}_2\to\tilde{A}_1$ are equivalences, which is by the 3-for-2 property equivalent to the condition that the composite $\tilde{A}_2\to \tilde{A}_0$ is an equivalence, i.e.~that $\mathcal{A}$ is $\Delta[2]$-null.
\end{proof}

In \cref{thm:emb_disc} we have established that every type can be equipped with the structure of a discrete reflexive graph in a unique way. Now we extend this observation to see that every type can be equipped with the structure of a discrete $2$-simplicial type in a unique way.

\begin{prp}\label{prp:2disc_emb}
The operation $\Delta:\UU\to \mathsf{sType}_2$ is an embedding.
\end{prp}

\begin{proof}
By the Yoneda lemma.
\end{proof}

\begin{defn}
A cone on a $2$-simplicial type $\mathcal{A}$ with vertex $X$ is simply a morphism
\begin{equation*}
\mathsf{sType}_2(\mathcal{A},\Delta X).
\end{equation*}
A morphism $f:\mathsf{sType}_2(\mathcal{A},\Delta X)$ is said to be a \define{geometric realization}\marginnote{If geometric realization is not good terminology, could use `$2$-simplicial coequalizer' or something along those lines} of $\mathcal{A}$ if the function
\begin{equation*}
\Delta(\blank)\circ f : (X\to Y) \to \mathsf{sType}_2(\mathcal{A},\Delta Y)
\end{equation*}
is an equivalence for every $Y$. 
\end{defn}

Our goal for the remainder of this chapter is to study geometric realizations of $2$-simplicial types. In particular, we want to establish that geometric realizations can be obtained as iterated pushouts, and derive from this fact some useful relations between pushouts that we know. However, we need some further machinery to get there.

\section{\texorpdfstring{$2$}{2}-cospans and \texorpdfstring{$2$}{2}-pullbacks}

A $2$-cospan is a diagram of the fom
\begin{equation*}
\begin{tikzcd}
A_{110} \arrow[d] & A_{101} \arrow[dl] \arrow[dr] & A_{011} \arrow[d] \\
A_{100} \arrow[dr] & A_{010} \arrow[d] \arrow[from=ul,crossing over] \arrow[from=ur,crossing over] & A_{001} \arrow[dl] \\
& A_{000}
\end{tikzcd}
\end{equation*}
More formally:

\begin{defn}
A \define{$2$-cospan} consists of 
\begin{enumerate}
\item types
\begin{equation*}
A_{000},A_{001},A_{010},A_{100},A_{011},A_{101},A_{110},
\end{equation*}
\item maps
\begin{align*}
f_{00\check{1}} & : A_{001}\to A_{000} \\
f_{0\check{1}0} & : A_{010}\to A_{000} \\
f_{\check{1}00} & : A_{100}\to A_{000} \\
f_{0\check{1}1} & : A_{011}\to A_{001} \\
f_{01\check{1}} & : A_{011}\to A_{010} \\
f_{\check{1}01} & : A_{101}\to A_{001} \\
f_{10\check{1}} & : A_{101}\to A_{100} \\
f_{\check{1}10} & : A_{110}\to A_{010} \\
f_{1\check{1}0} & : A_{110}\to A_{100},
\end{align*}
\item and homotopies
\begin{align*}
H_{0\check{1}\check{1}} & : f_{00\check{1}}\circ f_{0\check{1}1} \htpy f_{0\check{1}0}\circ f_{01\check{1}} \\
H_{\check{1}0\check{1}} & : f_{00\check{1}}\circ f_{\check{1}01} \htpy f_{\check{1}00}\circ f_{10\check{1}} \\
H_{\check{1}\check{1}0} & : f_{0\check{1}0}\circ f_{\check{1}10} \htpy f_{\check{1}00}\circ f_{1\check{1}0}.
\end{align*}
\end{enumerate}
\end{defn}

\begin{defn}
A \define{cone} with vertex $X$ on a $2$-cospan $\mathcal{A}$ consists of
\begin{enumerate}
\item maps
\begin{align*}
p_{011} & : X \to A_{011} \\
p_{101} & : X \to A_{101} \\
p_{110} & : X \to A_{110},
\end{align*}
\item homotopies
\begin{align*}
K_{001} & : f_{0\check{1}1}\circ p_{011} \htpy f_{\check{1}01}\circ p_{101} \\
K_{010} & : f_{01\check{1}}\circ p_{011} \htpy f_{\check{1}10}\circ p_{110} \\
K_{100} & : f_{10\check{1}}\circ p_{101} \htpy f_{1\check{1}0}\circ p_{110},
\end{align*}
\item and a homotopy
\begin{align*}
L_{000} & : \ct{(f_{\check{1}00}\cdot K_{100})}{(H_{\check{1}0\check{1}}\cdot p_{101})}{(f_{00\check{1}}\cdot K_{001})} \\
& \qquad = \ct{(H_{\check{1}\check{1}0}\cdot p_{110})}{(f_{0\check{1}0}\cdot K_{010})}{(H_{0\check{1}\check{1}}\cdot p_{011})}
\end{align*}
\end{enumerate}
We write $\mathsf{cone}(X)$ for the type of cones with vertex $X$ on $\mathcal{A}$.
\end{defn}

\begin{defn}
Consider a $2$-cospan $\mathcal{A}$ and a cone $\mathcal{C}$ with vertex $X$ on $\mathcal{A}$. We define for every type $Y$ a map
\begin{equation*}
\mathsf{cone\underline{~}map}(\mathcal{C}) : (Y\to X)\to \mathsf{cone}(X)
\end{equation*}
A cone $\mathcal{C}$ is said to be \define{limiting} if the map $\mathsf{cone\underline{~}map}(\mathcal{C})$ is an equivalence for every type $Y$.
\end{defn}

\begin{defn}
A \define{global section} of a $2$-span $\mathcal{A}$ consists of
\begin{enumerate}
\item terms $\gamma_{011}:A_{011}$, $\gamma_{101}:A_{101}$, and $\gamma_{110}:A_{110}$,
\item paths
\begin{align*}
\gamma_{001} & : f_{0\check{1}1}(\gamma_{011}) = f_{\check{1}01}(\gamma_{101}) \\
\gamma_{010} & : f_{01\check{1}}(\gamma_{011}) = f_{\check{1}10}(\gamma_{110}) \\
\gamma_{100} & : f_{10\check{1}}(\gamma_{101}) = f_{1\check{1}0}(\gamma_{110}),
\end{align*}
\item and a path
\begin{align*}
\gamma_{000} & : \ct{\mathsf{ap}_{f_{\check{1}00}}(\gamma_{100})}{H_{\check{1}0\check{1}}(\gamma_{101})}{\mathsf{ap}_{f_{00\check{1}}}(\gamma_{001})} \\
& \qquad = \ct{H_{\check{1}\check{1}0}(\gamma_{110})}{\mathsf{ap}_{f_{0\check{1}0}}(\gamma_{010})}{H_{0\check{1}\check{1}}(\gamma_{011})}
\end{align*}
\end{enumerate}
We write $\Gamma(\mathcal{A})$ for the type of global sections of $\mathcal{A}$.
\end{defn}

\begin{prp}
For any $2$-span $\mathcal{A}$ the type $\Gamma(\mathcal{A})$ is the vertex of a limiting cone.
\end{prp}

\begin{proof}
First we define a cone on the $2$-span $\mathcal{A}$ by taking
\begin{align*}
p_{011}(\gamma) & \defeq \gamma_{011} \\
p_{101}(\gamma) & \defeq \gamma_{101} \\
p_{110}(\gamma) & \defeq \gamma_{110} \\
K_{001}(\gamma) & \defeq \gamma_{001} \\
K_{010}(\gamma) & \defeq \gamma_{010} \\
K_{100}(\gamma) & \defeq \gamma_{100} \\
L_{000}(\gamma) & \defeq \gamma_{000}.
\end{align*}
for any global section $\gamma$ of $\mathcal{A}$. Now it is easy to verify that this cone is limiting. The proof is analogous to the proof showing that the canonical pullback is a pullback, by iterated applications of \cref{thm:choice}. 
\end{proof}

\begin{thm}
Consider a commuting cube
\begin{equation*}
\begin{tikzcd}
& A_{111} \arrow[dl] \arrow[d] \arrow[dr] \\
A_{110} \arrow[d] & A_{101} \arrow[dl] \arrow[dr] & A_{011} \arrow[d] \\
A_{100} \arrow[dr] & A_{010} \arrow[d] \arrow[from=ul,crossing over] \arrow[from=ur,crossing over] & A_{001} \arrow[dl] \\
& A_{000}.
\end{tikzcd}
\end{equation*}
The following are equivalent:
\begin{enumerate}
\item The cube is cartesian, in the sense that the cone with vertex $A_{111}$ is limiting.
\item The the square
\begin{equation*}
\begin{tikzcd}
A_{111} \arrow[r] \arrow[d] & A_{101}\times_{A_{001}} A_{011} \arrow[d] \\
A_{110} \arrow[r] & A_{100} \times_{A_{000}} A_{010}
\end{tikzcd}
\end{equation*}
is cartesian.
\end{enumerate}
\end{thm}

\begin{proof}
Note that the type of global sections $\Gamma(\mathcal{A})$ of the $2$-cospan
\begin{equation*}
\begin{tikzcd}
A_{110} \arrow[d] & A_{101} \arrow[dl] \arrow[dr] & A_{011} \arrow[d] \\
A_{100} \arrow[dr] & A_{010} \arrow[d] \arrow[from=ul,crossing over] \arrow[from=ur,crossing over] & A_{001} \arrow[dl] \\
& A_{000}.
\end{tikzcd}
\end{equation*}
is also the following pullback:
\begin{equation*}
\begin{tikzcd}
\Gamma(\mathcal{A}) \arrow[r] \arrow[d] & A_{101}\times_{A_{001}} A_{011} \arrow[d] \\
A_{110} \arrow[r] & A_{100} \times_{A_{000}} A_{010}.
\end{tikzcd}
\end{equation*}
Thus the claim follows.
\end{proof}

\begin{cor}
The $2$-pullback of the $2$-cospan
\begin{equation*}
\begin{tikzcd}
X \arrow[d] & Y \arrow[dl] \arrow[dr] & Z \arrow[d] \\
\unit \arrow[dr] & \unit \arrow[d] \arrow[from=ul,crossing over] \arrow[from=ur,crossing over] & \unit \arrow[dl] \\
& \unit
\end{tikzcd}
\end{equation*}
is the product $X\times Y\times Z$. 
\end{cor}

\begin{cor}
The $2$-pullback of the $2$-cospan
\begin{equation*}
\begin{tikzcd}
X\times Y \arrow[d] & X\times Z \arrow[dl] \arrow[dr] & Y\times Z \arrow[d] \\
X \arrow[dr] & Y \arrow[d] \arrow[from=ul,crossing over] \arrow[from=ur,crossing over] & Z \arrow[dl] \\
& \unit
\end{tikzcd}
\end{equation*}
is the product $X\times Y\times Z$. 
\end{cor}

\begin{cor}
The $2$-pullback of the $2$-cospan
\begin{equation*}
\begin{tikzcd}
\unit \arrow[d] & \unit \arrow[dl] \arrow[dr] & \unit \arrow[d] \\
\unit \arrow[dr] & \unit \arrow[d] \arrow[from=ul,crossing over] \arrow[from=ur,crossing over] & \unit \arrow[dl] \\
& X
\end{tikzcd}
\end{equation*}
is the double loop space $\loopspace[2]{X}$. 
\end{cor}

\section{\texorpdfstring{$2$}{2}-spans and \texorpdfstring{$2$}-pushouts}

A $2$-span is a diagram of the fom
\begin{equation*}
\begin{tikzcd}
& A_{111} \arrow[dl] \arrow[d] \arrow[dr] \\
A_{110} \arrow[d] & A_{101} \arrow[dl] \arrow[dr] & A_{011} \arrow[d] \\
A_{100} & A_{010} \arrow[from=ul,crossing over] \arrow[from=ur,crossing over] & A_{001}
\end{tikzcd}
\end{equation*}
More formally:

\begin{defn}
A \define{$2$-span} consists of 
\begin{enumerate}
\item types
\begin{equation*}
A_{111},A_{110},A_{101},A_{011},A_{100},A_{010},A_{001},
\end{equation*}
\item maps
\begin{align*}
f_{11\check{1}} & : A_{111} \to A_{110}\\
f_{1\check{1}1} & : A_{111} \to A_{101}\\
f_{\check{1}11} & : A_{111} \to A_{011}\\
f_{1\check{1}0} & : A_{110} \to A_{100}\\
f_{10\check{1}} & : A_{101} \to A_{100}\\
f_{\check{1}10} & : A_{110} \to A_{010}\\
f_{01\check{1}} & : A_{011} \to A_{010}\\
f_{\check{1}01} & : A_{101} \to A_{001}\\
f_{0\check{1}1} & : A_{011} \to A_{001},
\end{align*}
\item and homotopies
\begin{align*}
H_{\check{1}\check{1}1} & : f_{\check{1}01}\circ f_{1\check{1}1} \htpy f_{0\check{1}1}\circ f_{\check{1}11} \\
H_{\check{1}1\check{1}} & : f_{\check{1}10}\circ f_{11\check{1}} \htpy f_{01\check{1}}\circ f_{\check{1}11} \\
H_{1\check{1}\check{1}} & : f_{1\check{1}0}\circ f_{11\check{1}} \htpy f_{10\check{1}}\circ f_{1\check{1}1}.
\end{align*}
\end{enumerate}
\end{defn}

\begin{defn}
A \define{cocone} with vertex $X$ on a $2$-span $\mathcal{A}$ consists of
\begin{enumerate}
\item maps
\begin{align*}
i_{100} & : A_{100} \to X\\
i_{010} & : A_{010} \to X\\
i_{001} & : A_{001} \to X,
\end{align*}
\item homotopies
\begin{align*}
K_{110} & : i_{100}\circ f_{1\check{1}0} \htpy i_{010}\circ f_{\check{1}10} \\
K_{101} & : i_{100}\circ f_{10\check{1}} \htpy i_{001}\circ f_{\check{1}01} \\
K_{011} & : i_{010}\circ f_{01\check{1}} \htpy i_{001}\circ f_{0\check{1}1},
\end{align*}
\item and a homotopy
\begin{align*}
L_{111} & : \ct{(K_{011}\cdot f_{\check{1}11})}{(i_{010}\cdot H_{\check{1}1\check{1}})}{(K_{110}\cdot f_{11\check{1}})} \\
& \qquad = \ct{(i_{001}\cdot H_{\check{1}\check{1}1})}{(K_{101}\cdot f_{1\check{1}1})}{(i_{100}\cdot H_{1\check{1}\check{1}})}
\end{align*}
\end{enumerate}
We write $\mathsf{cocone}(X)$ for the type of cocones with vertex $X$ on $\mathcal{A}$.
\end{defn}

\begin{defn}
Consider a $2$-span $\mathcal{A}$ and a cocone $\mathcal{C}$ with vertex $X$ on $\mathcal{A}$. We define for every type $Y$ a map
\begin{equation*}
\mathsf{cocone\underline{~}map}(\mathcal{C}) : (X\to Y)\to \mathsf{cocone}(Y)
\end{equation*}
A cocone $\mathcal{C}$ is said to be \define{colimiting} if the map $\mathsf{cocone\underline{~}map}(\mathcal{C})$ is an equivalence for every type $Y$.
\end{defn}

\begin{defn}
Consider a $2$-span $\mathcal{A}$ and a cocone $\mathcal{C}$ with vertex $X$ on $\mathcal{A}$. A \define{dependent cocone structure} on a dependent type $P:X\to\UU$ over $\mathcal{C}$ consists of
\begin{enumerate}
\item dependent functions
\begin{align*}
j_{100} & : \prd{x:A_{100}}P(i_{100}(x))\\
j_{010} & : \prd{x:A_{010}}P(i_{010}(x))\\
j_{001} & : \prd{x:A_{001}}P(i_{001}(x)),
\end{align*}
\item homotopies
\begin{align*}
j_{110} & : \prd{x:A_{110}} \dpath{P}{K_{110}(x)}{i_{100}(f_{1\check{1}0}(x))}{i_{010}(f_{\check{1}10}(x))} \\
j_{101} & : \prd{x:A_{101}} \dpath{P}{K_{101}(x)}{i_{100}(f_{10\check{1}}(x))}{i_{001}(f_{\check{1}01}(x))} \\
j_{011} & : \prd{x:A_{011}} \dpath{P}{K_{011}(x)}{i_{010}(f_{01\check{1}}(x))}{i_{001}(f_{0\check{1}1}(x))},
\end{align*}
\item and a homotopy
\begin{align*}
j_{111} & : \ct{(K_{011}\cdot f_{\check{1}11})}{(i_{010}\cdot H_{\check{1}1\check{1}})}{(K_{110}\cdot f_{11\check{1}})} \\
& \qquad = \ct{(i_{001}\cdot H_{\check{1}\check{1}1})}{(K_{101}\cdot f_{1\check{1}1})}{(i_{100}\cdot H_{1\check{1}\check{1}})}
\end{align*}
\end{enumerate}
We write $\mathsf{cocone}(X)$ for the type of cocones with vertex $X$ on $\mathcal{A}$.
\end{defn}

\begin{thm}\label{thm:2pushout_iterated}
Consider a cube
\begin{equation*}
\begin{tikzcd}
& A_{111} \arrow[dl] \arrow[d] \arrow[dr] \\
A_{110} \arrow[d] & A_{101} \arrow[dl] \arrow[dr] & A_{011} \arrow[d] \\
A_{100} \arrow[dr] & A_{010} \arrow[d] \arrow[from=ul,crossing over] \arrow[from=ur,crossing over] & A_{001} \arrow[dl] \\
& A_{000}.
\end{tikzcd}
\end{equation*}
The following are equivalent:
\begin{enumerate}
\item The cube is cocartesian.
\item The the square
\begin{equation*}
\begin{tikzcd}
A_{011}\sqcup^{A_{111}} A_{101} \arrow[r] \arrow[d] & A_{001} \arrow[d] \\
A_{010}\sqcup^{A_{110}} A_{100} \arrow[r] & A_{000}
\end{tikzcd}
\end{equation*}
is cocartesian.
\end{enumerate}
\end{thm}

\begin{proof}
The cube $\mathcal{A}$ is cocartesian if and only if the cube $X^{\mathcal{A}}$ is cartesian for every type $X$, which happens if and only if the square
\begin{equation*}
\begin{tikzcd}
X^{A_{000}} \arrow[r] \arrow[d] & X^{A_{010}}\times_{X^{A_{110}}} X^{A_{100}} \arrow[d] \\
X^{A_{001}} \arrow[r] & X^{A_{011}}\times_{X^{A_{111}}} X^{A_{101}}
\end{tikzcd}
\end{equation*}
is cartesian. By the universal property of pushouts, this is equivalent to the condition that the square
\begin{equation*}
\begin{tikzcd}
X^{A_{000}} \arrow[r] \arrow[d] & X^{A_{010}\sqcup^{X^{A_{110}}}A_{100}} \arrow[d] \\
X^{A_{001}} \arrow[r] & X^{A_{011}\sqcup^{X^{A_{111}}}A_{101}}
\end{tikzcd}
\end{equation*}
is cartesian, for every type $X$. By another application of the universal property of pushouts, this is equivalent to the condition that the square
\begin{equation*}
\begin{tikzcd}
A_{011}\sqcup^{A_{111}} A_{101} \arrow[r] \arrow[d] & A_{001} \arrow[d] \\
A_{010}\sqcup^{A_{110}} A_{100} \arrow[r] & A_{000}
\end{tikzcd}
\end{equation*}
is cocartesian.
\end{proof}

\begin{cor}
The cube
\begin{equation*}
\begin{tikzcd}
& X \arrow[dl] \arrow[d] \arrow[dr] \\
\unit \arrow[d] & \unit \arrow[dl] \arrow[dr] & \unit \arrow[d] \\
\unit \arrow[dr] & \unit \arrow[d] \arrow[from=ul,crossing over] \arrow[from=ur,crossing over] & \unit \arrow[dl] \\
& \Sigma^2 X.
\end{tikzcd}
\end{equation*}
is cocartesian.
\end{cor}

\begin{cor}
The cube
\begin{equation*}
\begin{tikzcd}
& \unit \arrow[dl] \arrow[d] \arrow[dr] \\
\unit \arrow[d] & \unit \arrow[dl] \arrow[dr] & \unit \arrow[d] \\
X \arrow[dr] & Y \arrow[d] \arrow[from=ul,crossing over] \arrow[from=ur,crossing over] & Z \arrow[dl] \\
& X\vee Y\vee Z.
\end{tikzcd}
\end{equation*}
is cocartesian.
\end{cor}

\begin{cor}
The cube
\begin{equation*}
\begin{tikzcd}
& A\times B \times C \arrow[dl] \arrow[d] \arrow[dr] \\
A\times B \arrow[d] & A\times C \arrow[dl] \arrow[dr] & B\times C \arrow[dl,crossing over] \arrow[d] \\
A \arrow[dr] & B \arrow[d] \arrow[from=ul,crossing over] & C  \arrow[dl] \\
& \join{A}{\join{B}{C}}
\end{tikzcd}
\end{equation*}
is cocartesian.
\end{cor}

\begin{cor}
Consider an $H$-space $X$. Then the $2$-pushout of the $2$-span
\begin{equation*}
\begin{tikzcd}
& X \times X \arrow[dl,swap,"\proj 1"] \arrow[d,swap,"\mu"] \arrow[dr,"\pi_2"] \\
X \arrow[d] & X \arrow[dl] \arrow[dr] & X \arrow[dl,crossing over] \arrow[d] \\
\unit & \unit \arrow[from=ul,crossing over] & \unit
\end{tikzcd}
\end{equation*}
is the cofiber of the Hopf fibration $X \hookrightarrow \join{X}{X} \twoheadrightarrow \susp X$. 
\end{cor}

\begin{defn}
Let $X$, $Y$, and $Z$ be pointed types. We define the \define{fat wedge} $X\blacktriangledown Y\blacktriangledown Z$ to be the $2$-pushout
\begin{equation*}
\begin{tikzcd}
& \unit \arrow[dl] \arrow[d] \arrow[dr] \\
X \arrow[d] & Y \arrow[dl] \arrow[dr] & Z \arrow[d] \\
X\times Y \arrow[dr] & X\times Z \arrow[d] \arrow[from=ul,crossing over] \arrow[from=ur,crossing over] & Y\times Z \arrow[dl] \\
& X\blacktriangledown Y\blacktriangledown Z.
\end{tikzcd}
\end{equation*}
\end{defn}

\begin{thm}
(Needs results about the total cofiber.) The square
\begin{equation*}
\begin{tikzcd}
X\blacktriangledown Y\blacktriangledown Z \arrow[r] \arrow[d] & X\times Y\times Z \arrow[d] \\
\unit \arrow[r] & X\wedge Y\wedge Z
\end{tikzcd}
\end{equation*}
is cocartesian.
\end{thm}

\begin{prp}
Consider a cube
\begin{equation*}
\begin{tikzcd}
& A_{111} \arrow[dl] \arrow[d] \arrow[dr] \\
A_{110} \arrow[d] & A_{101} \arrow[dl] \arrow[dr] & A_{011} \arrow[d] \\
A_{100} \arrow[dr] & A_{010} \arrow[d] \arrow[from=ul,crossing over] \arrow[from=ur,crossing over] & A_{001} \arrow[dl] \\
& A_{000}.
\end{tikzcd}
\end{equation*}
The following are equivalent:
\begin{enumerate}
\item The cube is $n$-cocartesian.
\item The the square
\begin{equation*}
\begin{tikzcd}
A_{011}\sqcup^{A_{111}} A_{101} \arrow[r] \arrow[d] & A_{001} \arrow[d] \\
A_{010}\sqcup^{A_{110}} A_{100} \arrow[r] & A_{000}
\end{tikzcd}
\end{equation*}
is $n$-cocartesian.
\end{enumerate}
\end{prp}

\section{Geometric realizations as iterated pushouts}

\begin{defn}
Every $2$-simplicial type $\mathcal{A}$ determines a $2$-span $\tilde{\mathcal{A}}$ given by
\begin{equation*}
\begin{tikzcd}[column sep=large,row sep=large]
& \tilde A_2 \arrow[dl,swap,"\pi_{01}"] \arrow[d,"\pi_{02}"] \arrow[dr,"\pi_{12}"] \\
\tilde A_1 \arrow[d,swap,"\pi_0"] & \tilde A_1 \arrow[dl,"\pi_0" near end] \arrow[dr,swap,"\pi_1" near end] & \tilde A_1 \arrow[dl,crossing over,swap,"\pi_0" near start] \arrow[d,"\pi_1"] \\
\tilde A_0 & \tilde A_0 \arrow[from=ul,crossing over,"\pi_1" near start] & \tilde A_0. 
\end{tikzcd}
\end{equation*}
The maps range over the projections
\begin{align*}
\pi_0 & \defeq \lam{(x,y,p)}x\\
\pi_1 & \defeq \lam{(x,y,p)}y\\
\pi_{01} & \defeq \lam{(x,y,z,p,r,q,s)}(x,y,p) \\
\pi_{02} & \defeq \lam{(x,y,z,p,r,q,s)}(x,z,r) \\
\pi_{12} & \defeq \lam{(x,y,q,p,r,q,s)}(y,z,q),
\end{align*}
and the squares commute by the homotopies
\begin{align*}
H_{00} & \defeq \lam{(x,y,z,p,r,q,s)}\refl{x} & & : \pi_0\circ\pi_{01}\htpy \pi_0\circ \pi_{02} \\
H_{10} & \defeq \lam{(x,y,z,p,r,q,s)}\refl{y} & & : \pi_1\circ\pi_{01}\htpy \pi_0\circ \pi_{12} \\
H_{11} & \defeq \lam{(x,y,z,p,r,q,s)}\refl{z} & & : \pi_1\circ\pi_{02}\htpy \pi_1\circ \pi_{12}.
\end{align*}
Furthermore, for any $f:\mathsf{sType}_2(\mathcal{A},\Delta X)$ we define the cocone
\begin{equation*}
\tilde{f}\defeq (\tilde{f}_0,\tilde{f}_0,\tilde{f}_0,\tilde{f}_1,\tilde{f}_1,\tilde{f}_1,\tilde{f}_2)
\end{equation*}
on the $2$-span $\tilde{\mathcal{A}}$ with vertex $X$, where
\begin{align*}
\tilde{f}_0 & \defeq f_0 \\
\tilde{f}_1 & \defeq \lam{(x,y,p)}f_1(x,y,p),
\end{align*}
and the homotopy $\tilde{f}_2:\ct{(\tilde{f}_1\cdot \pi_{01})}{(\ct{(\tilde{f}_0\cdot H_{10})}{(\tilde{f}_1\cdot \pi_{12})})}\htpy \ct{(\tilde{f}_0\cdot H_{00})}{(\ct{(\tilde{f}_1\cdot\pi_{02})}{(\tilde{f}_0\cdot H_{11})})}$ takes $(x,y,z,p,r,q,s)$ to the composite
\begin{align*}
\ct{f_1(p)}{(\ct{\ap{f_0}{\refl{y}}}{f_1(q)})} & = \ct{f_1(p)}{f_1(q)} \\
& = f_1(r) \\
& = \ct{f_1(r)}{\ap{f_0}{\refl{z}}} \\
& = \ct{\ap{f_0}{\refl{x}}}{(\ct{f_1(r)}{\ap{f_0}{\refl{z}}})}.
\end{align*}
\end{defn}

\begin{comment}
\begin{lem}
Consider a $2$-semi-simplicial type $\mathcal{A}$
\begin{align*}
A_0 & : \UU \\
A_1 & : A_0\to A_0\to \UU \\
A_2 & : \prd{x,y,z:A_0} A_1(x,y)\to A_1(x,z)\to A_1(y,z)\to \UU,
\end{align*}
and consider
\begin{align*}
\alpha & : \prd{x,y:A_0} A_1(x,y)\to (x=y) \\
\beta & : \prd{x,y,z:A_0}{p:A_1(x,y)}{r:A_1(x,z)}{q:A_1(y,z)} A_2(p,r,q)\to (\ct{\alpha(p)}{\alpha(q)}=\alpha(r)). 
\end{align*}
Then the function $(x,y,z,p,q,r,s)\mapsto(x,x,x,y,y,y,z)$ of type
\begin{align*}
\cdots \to \tilde{A}_2
\end{align*}
is an equivalence. 
\end{lem}
\end{comment}

The following theorem, which states that $f:\mathcal{A}\to \Delta X$ is a geometric realization if and only if the cube
\begin{equation*}
\begin{tikzcd}
& \tilde A_2 \arrow[dl] \arrow[d] \arrow[dr] \\
\tilde A_1 \arrow[d] & \tilde A_1 \arrow[dl] \arrow[dr] & \tilde A_1 \arrow[dl,crossing over] \arrow[d] \\
\tilde A_0 \arrow[dr,swap,"f_0"] & \tilde A_0 \arrow[from=ul,crossing over] \arrow[d,swap,"f_0" near start] & \tilde A_0  \arrow[dl,"f_0"] \\
& X
\end{tikzcd}
\end{equation*}
is cocartesian, is analogous to \cref{thm:rcoeq_is_pushout}.

\begin{thm}
Consider a morphism $f:\mathsf{sType}_2(\mathcal{A},\Delta X)$. Then $f$ is a geometric realization if and only if the cocone $\tilde{f}$ on the $2$-span $\tilde{A}$ with vertex $X$ is colimiting.
\end{thm}

\begin{proof}
Let $Y$ be a type. Then we have the commuting triangle
\begin{equation*}
\begin{tikzcd}[column sep=tiny]
\phantom{\mathsf{cocone}_{\tilde{\mathcal{A}}}(Y)} & Y^X \arrow[dl,swap,"\Delta(\blank)\circ f"] \arrow[dr,"\mathsf{cocone\usc{}map}(\tilde{f})"] & \phantom{\mathsf{sType}_2(\mathcal{A},\Delta Y)} \\
\mathsf{sType}_2(\mathcal{A},\Delta Y) \arrow[rr,swap,"\varphi"] & & \mathsf{cocone}_{\tilde{\mathcal{A}}}(Y),
\end{tikzcd}
\end{equation*}
where the bottom map is given by $((f_0,f_1,f_2),\alpha)\mapsto (\tilde f_0,\tilde f_0,\tilde f_0,\tilde f_1,\tilde f_1,\tilde f_1,\tilde f_2)$. Therefore it suffices to show that $\varphi$ is an equivalence.

Note that the map $\varphi$ factors through the type of triples $(f_0,f_1,f_2)$.\marginnote{This proof is horrible to write down. I need to set myself to it, but I postpone because I hope to think of something smart.}
\end{proof}

\begin{cor}
The geometric realization of the discrete $2$-simplicial type $\Delta(A)$ is equivalent to $A$.
\end{cor}

\begin{cor}
The geometric realization of the indiscrete $2$-simplicial type $\nabla(A)$ is equivalent to the triple join $\join{A}{\join{A}{A}}$. 
\end{cor}

\begin{cor}\label{cor:2-pre-kernel}
The geometric realization of the $2$-pre-kernel of a map $f:A\to B$ is the type $\join[X]{A}{\join[X]{A}{A}}$. 
\end{cor}

\begin{cor}\label{cor:geom_hspace}
Given a coherent $H$-space $X$, the geometric realization of the $2$-simplicial type $S^2 X$ is the cofiber of the Hopf fibration $\eta_X: \join{X}{X} \to \susp X$.
\end{cor}

\begin{cor}
The geometric realization of the standard simplices $\Delta[0]$, $\Delta[1]$, and $\Delta[2]$ are contractible.
\end{cor}

\section{A speculative note on simplicial spaces}

\marginnote{It might be better to write something about simplicial spaces in \cref{chap:equifibrant}, or in the introduction}
We expect a simplicial space to be discrete, in the sense that the unit $\eta:\mathcal{A}\to |\mathcal{A}|$ of the geometric realization is an equivalence, if and only if it is $\Delta[k]$-null for all $k\in\mathbb{N}$, which happens if and only if $\mathcal{A}$ is $f$-local for all morphisms $f$ between representables. Furthermore, we expect that a morphism $f:\mathcal{A}\to\mathcal{B}$ of simplicial spaces is \'etale in the sense that the naturality square of the unit of geometric realization is a homotopy pullback square, if and only if $f$ is right orthogonal to all morphisms between representables. This can only be the case if the naturality squares (since $f$ is a morphism between presheaves valued in Kan-simplicial sets) are all pullback squares, i.e.~$f$ is cartesian. 
